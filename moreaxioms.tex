\clearpage

\section{More Axioms}

We give another set of axioms:

\begin{enumerate}
  \item[TC1] $(\mathcal{T}_1, \mathcal{F}_1)$ and $(\mathcal{T}_2,\mathcal{F}_2)$ are two respectively
  left functorial and right functorial torsion pairs in $\mathcal{X}$.
  \item[TC2] $\mathcal{T}_2\subseteq\mathcal{T}_1$ (equivalently $\mathcal{F}_1\subseteq\mathcal{F}_2$).
  \item[TC3] For any morphism $g:T_1 \to T_1'$ in $\mathcal{T}_1$ has a pseudocokernel in $\mathcal{T}_1$ which
  completes diagrams in a unique way wrt $\mathcal{F}_2$.
  \item[TC3$^*$] Dual of \textbf{TC3}.
  \item[TC4]\todo{explain this axiom}
  \begin{equation*}
    \begin{tikzcd}
      F_1\arrow{r}{f^K}
        & H_1\arrow{r}{^\forall f}\arrow[dashed]{d}
          & H_2 \arrow{r}{f^C}\arrow[equal]{d}
            & T_1 \arrow{d}\\
      i_1t_1(F_2)\arrow[dashed]{ur}\arrow{r}{\varepsilon}
        & F\arrow{r}
          & H_2\arrow{r}
            & j_2f_2(T_1)
    \end{tikzcd}
  \end{equation*}
  \item[TC4$^*$] Dual of \textbf{TC4}.
\end{enumerate}

\textbf{EXAMPLES}\todo{add examples from page 2, 9/11/16}

\begin{examples}
  \begin{enumerate}
    \item[2] If $(\mathcal{U},\mathcal{V})$ is a cotorsion pair in a triangulated
    category (as in Nakaoka's work \todo{Add reference}) produces an example.
    \item[3] Let $\mathcal{D}$ be a triangulated category with two $t$-structures
    $(\mathcal{U}_1,\mathcal{U}_1^\perp)$ and $(\mathcal{U}_2,\mathcal{U}_2^\perp)$ such that
    $\mathcal{U}_1[1]\subseteq\mathcal{U}_2\subseteq\mathcal{U}_1$. Then, these satisfy
    axioms \textbf{TC1}-\textbf{TC3},textbf{TC3$^*$}, hence $\mathcal{H}=\mathcal{U}_1\cap\mathcal{U}_2^\perp$ has
    kernels and cokernels. Moreover, TFAE:
    \begin{enumerate}
      \item \textbf{TC4} holds.
      \item If $V_1\to H_1\nto{f}\to H_2\nto{+}$ is a distinguished triangle such
      that $H_1, H_2\in\mathcal{H}$ and $V_1\in\mathcal{U}_1^\perp$, then $V_1\in\mathcal{U}_2^\perp[-1]$.
    \end{enumerate}
    And, dually, there is an equivalence of the following:
    \begin{enumerate}
      \item \textbf{TC4$^*$} holds.
      \item\label{ax:eqa} If $H_1\nto{f}H_2\to U_2\nto{+}$ is a distinguished triangle such
      that $H_1,H_2\in\mathcal{H}$ and $U_2\in\mathcal{U}_2$, then $U_2\in\mathcal{U}_2[1]$.\todo{These last two are
      not dual. Check which is the correct version.}
    \end{enumerate}
  \end{enumerate}
\end{examples}

We can see a special case of example 3 in the case of the derived category of a ring.
Let $R$ be a commutative ring, consider the t-structure $(\mathcal{U}_1,\mathcal{U}_1^\perp)=(\mathcal{D}^{\leq 0}(R), \mathcal{D}^{>0}(R))$
in $\mathcal{D}(R)$. Given an idempotent ideal $I=I^2\lhd R$, it defines three classes of modules
\begin{align*}
  \mathcal{C}_I &= \{ C\in\Mod R | IC=C\}\\
  \mathcal{T}_I &= \{ T\in\Mod R | IT=0\} \cong \Mod\frac{R}{I}\\
  \mathcal{F}_I &= \{ F\in\Mod R | Ix \neq 0 \forall x\in F\setminus \{0\} \}
\end{align*}
such that $(\mathcal{C}_I, \mathcal{T}_I)$ and $(\mathcal{T}_I,\mathcal{F}_I)$ make two torsion pairs.
We call the triple $(\mathcal{C}_I,\mathcal{T}_I,\mathcal{F}_I)$ a TTP triple.

We define the t-structure $(\mathcal{U}_2,\mathcal{U}_2^\perp)$ as the Happel-Reiten-Smalo t-structure
associated to the torsion pair $(\mathcal{C}_I,\mathcal{T}_I)$ in $\Mod R$:
\begin{align*}
  \mathcal{U}_2 &= \{ U_2\in\mathcal{D}^{\leq 0}(R) | H^0(U_2)\in\mathcal{C}_I \} \\
  \mathcal{U}_2^\perp &= \{ V_2\in\mathcal{D}^{\geq 0}(R) | H^0(V_2)\in\mathcal{T}_I \}.
\end{align*}

In this case we can check that condition \cref{ax:eqa} holds. In fact, let $\mathcal{H}$ be the heart
\begin{align*}
  \mathcal{U}_1\cap\mathcal{U}_2^\perp &= \mathcal{D}^{\leq 0}(R)\cap \mathcal{U}_2^\perp \\
  &= \{ T[0] | T\in\mathcal{T}_I \} \cong \Mod\frac{R}{I}.
\end{align*}
Hence, $\mathcal{H}$ is abelian.

Now, consider $V_1\in\mathcal{U}_1^\perp$ such that there is an exact triangle
\begin{equation*}
  V_1\to T_1[0]\nto{f[0]}T_2[0]\nto{+}
\end{equation*}
with $T_1,T_2\in\mathcal{H}$. Of course, $V_1 = \mathrm{Cone}(f)[-1]$, i.e.
\begin{equation*}
  V_1 = \cdots \to0\to\,\oset[2ex]{\small 0}{T_1}\,\nto{f}\,\oset[2ex]{\small 1}{T_2}\,\to 0\to \cdots
\end{equation*}
where the numbers over $T_1$ and $T_2$ represent their cohomological degree.

The fact that $V_1\in\mathcal{U}_1^\perp = \mathcal{D}^{> 0}(R)$ implies that $H^0(V_1)=0$, i.e. $f$ is mono.
To prove that $V_1\in\mathcal{U}_2^\perp[-1]$ we would need to show that $\mathrm{Coker}(f)=H^1(V_1)$ belongs to $\mathcal{T}_I$,
but this follows from the fact that $f$ is a mono in $\mathcal{T}_I$ which is a torsion class.
