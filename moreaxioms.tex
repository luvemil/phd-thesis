\clearpage

\section{Second approach to axiomatization}

We give another set of axioms:

\begin{enumerate}
  \item[TC1] $(\mathcal{T}_1, \mathcal{F}_1)$ and $(\mathcal{T}_2,\mathcal{F}_2)$ are two respectively
  left functorial and right functorial torsion pairs in $\mathcal{X}$.
  \item[TC2] $\mathcal{T}_2\subseteq\mathcal{T}_1$ (equivalently $\mathcal{F}_1\subseteq\mathcal{F}_2$).
  \item[TC3] For any morphism $g:T_1 \to T_1'$ in $\mathcal{T}_1$ has a pseudocokernel in $\mathcal{T}_1$ which
  completes diagrams in a unique way wrt $\mathcal{F}_2$.
  \item[TC3$^*$] Dual of \textbf{TC3}.
  \item[TC4]\todo{explain this axiom}
  \begin{equation*}
    \begin{tikzcd}
      F_1\arrow{r}{f^K}
        & H_1\arrow{r}{^\forall f}\arrow[dashed]{d}
          & H_2 \arrow{r}{f^C}\arrow[equal]{d}
            & T_1 \arrow{d}\\
      i_1t_1(F_2)\arrow[dashed]{ur}\arrow{r}{\varepsilon}
        & F\arrow{r}
          & H_2\arrow{r}
            & j_2f_2(T_1)
    \end{tikzcd}
  \end{equation*}
  \item[TC4$^*$] Dual of \textbf{TC4}.
\end{enumerate}

\textbf{EXAMPLES}\todo{add examples from page 2, 9/11/16}

\begin{enumerate}
  \item[2] If $(\mathcal{U},\mathcal{V})$ is a cotorsion pair in a triangulated
  category (as in Nakaoka's work \todo{Add reference}) produces an example.
  \item[3] Let $\mathcal{D}$ be a triangulated category with two $t$-structures
  $(\mathcal{U}_1,\mathcal{U}_1^\perp)$ and $(\mathcal{U}_2,\mathcal{U}_2^\perp)$ such that
  $\mathcal{U}_1[1]\subseteq\mathcal{U}_2\subseteq\mathcal{U}_1$. Then, these satisfy
  axioms \textbf{TC1}-\textbf{TC3},\textbf{TC3$^*$}, hence $\mathcal{H}=\mathcal{U}_1\cap\mathcal{U}_2^\perp$ has
  kernels and cokernels. Moreover, TFAE:
  \begin{enumerate}
    \item[1.a] \textbf{TC4} holds.
    \item[1.b] If $V_1\to H_1\nto{f}\to H_2\nto{+}$ is a distinguished triangle such
    that $H_1, H_2\in\mathcal{H}$ and $V_1\in\mathcal{U}_1^\perp$, then $V_1\in\mathcal{U}_2^\perp[-1]$.
  \end{enumerate}
  And, dually, there is an equivalence of the following:
  \begin{enumerate}
    \item[2.a] \textbf{TC4$^*$} holds.
    \item[2.b]\label{ax:eqa} If $H_1\nto{f}H_2\to U_2\nto{+}$ is a distinguished triangle such
    that $H_1,H_2\in\mathcal{H}$ and $U_2\in\mathcal{U}_2$, then $U_2\in\mathcal{U}_1[1]$.
  \end{enumerate}
\end{enumerate}

\begin{proof}[Proof of the equivalences in example 3]
  Let's $\mathcal{D}$ be a triangulated category with two t-structures as in example 3. The pseudocokernel
  of a morphism in $\mathcal{U}_1$ can be computed by taking the cone in $\mathcal{D}$, i.e. given a morphism
  $f:U_1\to U_1'$ in $\mathcal{U}_1$ we can compute a pseudocokernel in $\mathcal{U}_1$ by completing $f$
  to a triangle
  \begin{equation*}
    U_1\nto{f}U_1'\to \mathrm{Cone}(f)\nto{+}.
  \end{equation*}
  Moreover, this pseudocokernel satisfies \textbf{TC3}.

  Now, assume that \textbf{TC1}-\textbf{TC3},\textbf{TC3*} are satisfied together with axiom
  \textbf{1.b}, and consider the solid part of the diagram as in \textbf{TC4}:
  \begin{equation*}
    \begin{tikzcd}
      \mathrm{Cone}(f)[-1]\arrow{r}{f^K}
        & H_1\arrow{r}{f}\arrow{d}{\alpha}
          & H_1\arrow{r}{f^C}\arrow[equal]{d}
            & \mathrm{Cone}(f)\arrow{d}{\lambda}\\
      \tau_{\mathcal{U}_1}(V_2)\arrow{r}{\varepsilon}\arrow[dotted]{ur}{\beta}
        & V_2\arrow{r}
          & H_2\arrow{r}
            & \tau^{\mathcal{U}_2^\perp}\mathrm{Cone}(f)
    \end{tikzcd}
  \end{equation*}
  with $\mathrm{Cone}(f)[-1]\in\mathcal{U}_1^\perp$ and where the upper row is a
  distinguished triangle. By \textbf{1.b} then it belongs
  to $\mathcal{U}_2^\perp[-1]$, i.e. $\mathrm{Cone}(f)\in\mathcal{U}_2^\perp$, so
  $\lambda$ is an iso, conseguently $\alpha$ is an iso and so is $\varepsilon$, so
  there exist a map $\beta=\alpha^{-1}\circ\varepsilon$ making the diagram commute, that
  is \textbf{TC4} holds.

  Conversely, assume that \textbf{TC1}-\textbf{TC3},\textbf{TC3*} are satisfied together with
  \textbf{TC4}. Consider again the solid part of the diagram
  \begin{equation*}
    \begin{tikzcd}
      \mathrm{Cone}(f)[-1]\arrow{r}{f^K}\arrow{d}{\lambda[-1]}
        & H_1\arrow{r}{f}\arrow{d}{\alpha}
          & H_1\arrow{r}{f^C}\arrow[equal]{d}
            & \mathrm{Cone}(f)\arrow{d}{\lambda}\\
      \tau_{\mathcal{U}_2^\perp}(\mathrm{Cone}(f))[-1]\arrow{r}
        & V_2\arrow{r}
          & H_2\arrow{r}
            & \tau^{\mathcal{U}_2^\perp}\mathrm{Cone}(f)\\
        & \tau_{\mathcal{U}_1}(V_2)\arrow{u}{\varepsilon}\arrow[dotted, crossing over, bend left=70]{uu}[near end,']{\beta}
        & &
    \end{tikzcd}
  \end{equation*}
  with $\mathrm{Cone}(f)[-1]\in\mathcal{U}_1^\perp$. Neeman \todo{Add reference} guarantees
  that $\alpha$ can be taken so that the square on the left is a pullback. Axiom \textbf{TC4}
  gives the existence of $\beta:\tau_{\mathcal{U}_1}(V_2)\to H_1$ such that $\alpha\circ\beta=\varepsilon$.

  Since $\tau_{\mathcal{U}_1}$ is a functor, there is also a morphism
  $\tau_{\mathcal{U}_1}(\alpha):\tau_{\mathcal{U}_1}(H_1)=H_1\to \tau_{\mathcal{U}_1}(V_2)$ such that
  $\varepsilon\circ\tau_{\mathcal{U}_1}(\alpha)=\alpha$, hence
  $\varepsilon\circ\tau_{\mathcal{U}_1}(\alpha)\circ\beta = \varepsilon$. By the functoriality of
  the torsion pair $(\mathcal{U}_1,\mathcal{U}_1^\perp)$, this means that
  $\tau_{\mathcal{U}_1}(\alpha)\circ\beta=1_{\tau_{\mathcal{U}_1}(V_2)}$. Then, $\mathcal{\beta}$ is a section.

  Hence, we can write $\tau_{\mathcal{U}_1}(\alpha):H_1\to \tau_{\mathcal{U}_1}(V_2)$
  as
  \begin{equation*}
    \begin{tikzcd}
      \tau_{\mathcal{U}_1}(\alpha):\tau_{\mathcal{U}_1}(V_2)\oplus H_1'
      \arrow{r}{
          \begin{psmallmatrix}
            \ast \amsamp 0
          \end{psmallmatrix}
          }
        & \tau_{\mathcal{U}_1}(V_2)
    \end{tikzcd}
  \end{equation*}
  for some $H_1'\underset{\oplus}{<} H_1$ such that
  $\alpha$ vanishes on $H_1'$. If we consider the solid part of the diagram
  \begin{equation*}
    \begin{tikzcd}
      & H_1'\arrow[dashed]{dl}
      \arrow{d}{
        \begin{psmallmatrix}
          1 \\ 0
        \end{psmallmatrix}
      }\arrow{dr}{0}
        & & &\\
      \mathrm{Cone}(\tau_{\mathcal{U}_1}(\alpha))[-1]\arrow{r}
        & H_1\arrow{r}{\tau_{\mathcal{U}_1}(\alpha)}
          & \tau_{\mathcal{U}_1}(V_2)\arrow[dashed]{r}{+}
            & {}
    \end{tikzcd}
  \end{equation*}
  we can construct the dashed arrow, and the fact that the triangle commutes means
  that $H_1'\underset{\oplus}{<}\mathrm{Cone}(\tau_{\mathcal{U}_1}(\alpha))[-1]$.

  Observe that $\mathrm{Cone}(\alpha)=\mathrm{Cone}(\lambda)[-1]$, since
  the square
  \begin{equation*}
    \begin{tikzcd}
      \mathrm{Cone}(f)[-1]\arrow{r}{f^K}\arrow{d}{\lambda[-1]}
        & H_1\arrow{d}{\alpha} \\
      \tau_{\mathcal{U}_2^\perp}(\mathrm{Cone}(f))[-1]\arrow{r}
        & V_2
    \end{tikzcd}
  \end{equation*}
  is a pullback. Moreover, $\mathrm{Cone}(\lambda)[-1] = (\tau_{\mathcal{U}_2}(\mathrm{Cone}(f))[1])[-1]
  =\tau_{\mathcal{U}_2}(\mathrm{Cone}(f))$. Hence, $\mathrm{Cone}(\alpha)\in \mathcal{U}_2$ and
  $\tau^{\mathcal{U}_1^\perp}(\mathrm{Cone}(\alpha))=0$, that is,
  $\mathrm{Cone}(\alpha)\in\mathcal{U}_1$, and since there is a distinguished triangle
  \begin{equation*}
    H_1\nto{\alpha} V_2\to \mathrm{Cone}(\alpha)\nto{+}
  \end{equation*}
  with $H_1,\mathrm{Cone}(\alpha)\in\mathcal{U}_1$ it follows that
  $V_2\in\mathcal{U}_1$. Hence, $\tau_{\mathcal{U}_1}(V_2)\cong V_2$.

  We can then write $V_2\underset{\oplus}{<}H_1$ and consider the commutative diagram
  \begin{equation*}
    \begin{tikzcd}
      H_1 \cong H_1'\oplus V_2\arrow{r}{
        \begin{psmallmatrix}
          f' \amsamp \tilde{f}
        \end{psmallmatrix}
      }\arrow{d}{
        \begin{psmallmatrix}
          0 \amsamp 1
        \end{psmallmatrix}
      }
        & H_2\arrow[equal]{d}\\
      V_2\arrow{r}
        & H_2
    \end{tikzcd}
  \end{equation*}
  so $f'=0$. Hence, the inclusion $
  \begin{psmallmatrix}
    1 \\ 0
  \end{psmallmatrix}:H_1'\to H_1'\oplus V_2$ can be lifted to
  $\mathrm{Cone}(f)[-1]$ and $H_1'\underset{\oplus}{<}\mathrm{Cone}(f)[-1]$.
  Since $\mathrm{Cone}(f)[-1]\in \mathcal{U}_1^\perp$, so does $H_1'$. Similarly,
  $H_1'\in\mathcal{U}_1$ because $H_1\in\mathcal{U}_1$. Hence, $H_1'=0$ and $\alpha:H_1\to V_2$ is an iso.
  The same follows for $\lambda$. Therefore, $\mathrm{Cone}(f)\in\mathcal{U}_2^\perp$ which
  proves \textbf{1.b}.
\end{proof}

We can see a special case of example 3 in the case of the derived category of a ring.
Let $R$ be a commutative ring, consider the t-structure $(\mathcal{U}_1,\mathcal{U}_1^\perp)=(\mathcal{D}^{\leq 0}(R), \mathcal{D}^{>0}(R))$
in $\mathcal{D}(R)$. Given an idempotent ideal $I=I^2\lhd R$, it defines three classes of modules
\begin{align*}
  \mathcal{C}_I &= \{ C\in\Mod R | IC=C\}\\
  \mathcal{T}_I &= \{ T\in\Mod R | IT=0\} \cong \Mod\frac{R}{I}\\
  \mathcal{F}_I &= \{ F\in\Mod R | Ix \neq 0 \forall x\in F\setminus \{0\} \}
\end{align*}
such that $(\mathcal{C}_I, \mathcal{T}_I)$ and $(\mathcal{T}_I,\mathcal{F}_I)$ make two torsion pairs.
We call the triple $(\mathcal{C}_I,\mathcal{T}_I,\mathcal{F}_I)$ a TTP triple.

We define the t-structure $(\mathcal{U}_2,\mathcal{U}_2^\perp)$ as the Happel-Reiten-Smalo t-structure
associated to the torsion pair $(\mathcal{C}_I,\mathcal{T}_I)$ in $\Mod R$:
\begin{align*}
  \mathcal{U}_2 &= \{ U_2\in\mathcal{D}^{\leq 0}(R) | H^0(U_2)\in\mathcal{C}_I \} \\
  \mathcal{U}_2^\perp &= \{ V_2\in\mathcal{D}^{\geq 0}(R) | H^0(V_2)\in\mathcal{T}_I \}.
\end{align*}

In this case we can check that condition \cref{ax:eqa} holds. In fact, let $\mathcal{H}$ be the heart
\begin{align*}
  \mathcal{U}_1\cap\mathcal{U}_2^\perp &= \mathcal{D}^{\leq 0}(R)\cap \mathcal{U}_2^\perp \\
  &= \{ T[0] | T\in\mathcal{T}_I \} \cong \Mod\frac{R}{I}.
\end{align*}
Hence, $\mathcal{H}$ is abelian.

Now, consider $V_1\in\mathcal{U}_1^\perp$ such that there is an exact triangle
\begin{equation*}
  V_1\to T_1[0]\nto{f[0]}T_2[0]\nto{+}
\end{equation*}
with $T_1,T_2\in\mathcal{H}$. Of course, $V_1 = \mathrm{Cone}(f)[-1]$, i.e.
\begin{equation*}
  V_1 = \cdots \to0\to\,\oset[2ex]{\small 0}{T_1}\,\nto{f}\,\oset[2ex]{\small 1}{T_2}\,\to 0\to \cdots
\end{equation*}
where the numbers over $T_1$ and $T_2$ represent their cohomological degree.

The fact that $V_1\in\mathcal{U}_1^\perp = \mathcal{D}^{> 0}(R)$ implies that $H^0(V_1)=0$, i.e. $f$ is mono.
To prove that $V_1\in\mathcal{U}_2^\perp[-1]$ we would need to show that $\mathrm{Coker}(f)=H^1(V_1)$ belongs to $\mathcal{T}_I$,
but this follows from the fact that $f$ is a mono in $\mathcal{T}_I$ which is a torsion class.
