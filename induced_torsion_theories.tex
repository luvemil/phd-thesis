\clearpage

\section{Induced torsion theories}

\begin{lemma}
  Let $\mathcal{A}$ be a (nice) category with a torsion pair $(^\perp\mathcal{F}, \mathcal{F})$ and a precovering class
  $\mathcal{W}\subseteq \mathcal{F}$ such that for any $F\in\mathcal{F}$ there is an admissible
  sequence
  \begin{equation*}
    F'\to W\to F
  \end{equation*}
  such that $F'\in\mathcal{F}$.

  Then the torsion pair $(^\perp(\underline{\mathcal{F}}),\underline{\mathcal{F}})$ is left functorial.
\end{lemma}

Recall that the truncation $t:\underline{\mathcal{A}}\to {^\perp(\underline{\mathcal{F}})}$ is given by the following construction.

Let $M\in\mathcal{A}$ be any object, take an admissible sequence
\begin{equation*}
  T_M\to M\to F^M
\end{equation*}
with $T_M\in{^\perp\mathcal{F}}$ and $F^M\in\mathcal{F}$. Moreover, consider $W_F\to F^M$ with $W_M\in\mathcal{W}$
as before, and take the admissible pullback:
\begin{equation*}
  \begin{tikzcd}
    t(M)\arrow{d}\arrow{r}&W_F\arrow{d}\\
    M\arrow{r}&F
  \end{tikzcd}
\end{equation*}

Then, $t$ restricts to a functor $\underline{t}:\underline{\mathcal{A}}\to {^\perp\underline{\mathcal{F}}}$.

In order to prove that $(^\perp(\underline{\mathcal{F}}),\underline{\mathcal{F}})$ is left functorial we
need to show that $\underline{t}$ admits a right adjoint.

\begin{proof}
  Let $M\in\mathcal{A}$ and consider $M\to F^M$ and $W_F\to F^M$ as above.
  For any $T\in{^\perp\mathcal{F}\ast \mathcal{W}}$ consider any morphism
  $f:T\to M$. Since $T\to M\to F^M$ is 0 in $\underline{\mathcal{A}}$ we that the solid part of the
  following diagram commutes.
  \begin{equation*}
    \begin{tikzcd}
      T\arrow[bend right]{ddr}[']{f}\arrow{rrr}\arrow[dotted]{dr}{\psi}
        & & & W\arrow[dashed]{ld}[']{\phi}\arrow[bend left]{ddl}
              \\
        & t(M)\arrow{r}\arrow{d}
          & W_F\arrow{d}
            & \\
        & M\arrow{r}
          & F^M
            &
    \end{tikzcd}
  \end{equation*}
\end{proof}

Since $W_F\to F^M$ is a precover there is a morphism $\phi:W\to W_F$ making the diagram commute, and
since the square is an admissible pullback there is a morphism $\psi:T\to t(M)$ making the diagram commutative.

Hence, $\mathcal{A}(T,t(M))\to \mathcal{A}(T,M)$ is surjective. To conclude the proof we need to show
that when restricted to $\underline{\mathcal{A}}$ it becomes an iso. Assume that there are two morphisms
$\underline{\psi}$ and $\underline{\psi}'$ in $\underline{\mathcal{A}}$ such that the following commutes:
\begin{equation*}
  \begin{tikzcd}
    & t(M)\arrow{d}\\
    T\arrow[shift left=-1]{ur}[near end,']{\underline{\psi}'}
    \arrow[shift left=1]{ur}[near end]{\underline{\psi}}
    \arrow{r}[']{\underline{f}}
      & M
  \end{tikzcd}
\end{equation*}

So, if we call $h=\psi-\psi'$ in $\mathcal{A}$, we have that $T\nto{h} t(M) \to M$ factors through $W$
so that we have that the solid part of the following diagram commutes:
\begin{equation*}
  \begin{tikzcd}
    W\arrow[bend right]{dddrr}\arrow[bend left,dashed]{rrrdd}{\eta}
      \arrow[bend left,dotted]{ddrr}{\gamma}
      & & &\\
      & T\arrow{ul}[near start,']{p}\arrow{ddr}[']{g}\arrow{dr}[near start]{h}
        & & \\
      & & t(M)\arrow{d}\arrow{r}
          & W_F\arrow{d}\\
      & & M\arrow{r}
          & F^M
  \end{tikzcd}
\end{equation*}
where $\eta:W\to W_F$ comes from the fact that $W_F\to F^M$ is a precover, and $\gamma$ from the fact that
the square is an admissible pullback, and they make the complete diagram commute.

Let's call $h'=\gamma\circ p$, then composing both $h$ and $h'$ with $\rho:t(M)\to M$ gives the same morphism $g$.
Hence, $\rho\circ(h-h')=0$. But since $F'\nto{i} t(M) \to M$ is an admissible sequence we have the following exact
sequence of abelian groups:
\begin{equation*}
  \begin{tikzcd}
    \mathcal{A}(T,F)\arrow{r}
      & \mathcal{A}(T,t(M))\arrow{r}
        & \mathcal{A}(T,M)\\
      & h-h'\arrow[mapsto]{r}
        & 0
  \end{tikzcd}
\end{equation*}
So there is a map $k:T\to F'$ such that $i\circ k = h-h'$, but $\underline{k}=0$ so
$\underline{h}=\underline{h'}=0$, hence $\underline{\psi} - \underline{\psi}' =0$ which proves that
$\underline{\mathcal{A}}(T,t(M))\cong \underline{\mathcal{A}}(T,M)$.

The naturality of the isomorphism in $T$ and $M$ is clear.
