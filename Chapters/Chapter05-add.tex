













\section{Nakaoka contexts in special categories}
\todo{Add intro}
\subsection{Nakaoka contexts in Abelian categories}

Let's consider the case $\mathscr{C}=\mathscr{A}$ of an Abelian category with
two torsion pairs $\mathbbm{t}_i=(\mathcal{T}_i,\mathcal{F}_i)$ for $i=1,2$.
Consider $\mathbbm{t}=(\mathbbm{t}_1,\mathbbm{t}_2)$.

\begin{lemma}\label{rmk:2.2}
For an Abelian category $\mathscr{A}$, any Nakaoka context is integral.
\end{lemma}
\begin{proof}
  Let $\mathcal{T}_2\subseteq\mathcal{T}_1$, we need to show that \ref{ax:ct1},
  \ref{ax:ct3} and \ref{ax:ct3op} hold.
  \begin{torsionaxioms}
    \item It is well known that any torsion pair in an abelian category is functorial.
    \setcounter{enumi}{2}
    \item Let $g:T_1\to T_1'$ be a morphism in $\mathcal{T}_1$. Consider the cokernel morphism
    of $g$ in $\mathscr{A}$
    \begin{equation*}
      \Coker_\mathscr{A}(T_1\nto{g}T_1')=(T_1'\nto{c_g}\Coker(g)).
    \end{equation*}
    Since $\mathcal{T}_1$ is closed under quotient objects, we get that $\Coker(g)\in\mathcal{T}_1$.
    Therefore, we can choose $c_g:T_1'\to\Coker(g)$ as $g^C:T_1'\to\PCok_\mathscr{A}(g)$.\todo{Is there an explicit choice
    that we made somewhere before when we talk about $\PCok_\mathscr{A}$?}
    \varitem{^\ast} Anolagous to the previous.
  \end{torsionaxioms}

\bigskip\bigskip
Let us now verify the axiom (CT.4.1). Indeed, consider two morphisms $f\colon H\to H'$ and $g\colon K\to H'$ in $\mathcal{H}$ such that $f$ has a pseudo-cokernel $f^C\colon H'\to C$ with $C\in \T_2$ and consider the following pullback diagram in $\mathscr{A}$
\[
\xymatrix{
X\ar@{}[dr]|{P.B.}\ar@{.>}[d]_{x_H}\ar@{.>}[r]^{x_K}&K\ar[r]^{x_K^C}\ar[d]^{g}&C\ar@{=}[d]\\
H\ar[r]_f&H'\ar[r]_{f^C}&C
}
\]
where $x_K^C\colon K\to C$ is a pseudo-cokernel of $x_K$. Clearly $X\leq H\oplus K$, so that $X\in \F_2$.
\[
\xymatrix{
t_1X\ar@{.>}[rrd]\ar@{.>}[ddr]\\
&X\ar@{}[dr]|{P.B.}\ar@{.>}[d]_{x_H}\ar@{.>}[r]^{x_K}&K\ar[r]^{x_K^C}\ar[d]^{g}&C\ar@{=}[d]\\
&H\ar[r]_f&H'\ar[r]_{f^C}&C
}
\]
\end{proof}

\begin{corollary}\label{cor:2.3}
  Let $\mathbbm{t}=(\mathbbm{t}_1,\mathbbm{t}_2)$ Nakaoka context in $\mathscr{A}$.
  Then, for $f:H_1\to H_2$ in $\mathcal{H}$, the following statements hold:
  \begin{enumerate}[label=(\alph*),ref=(\alph*)]
    \item\label{cor:2.3:a} the cokernel of $f$ in $\mathcal{H}$ is the composition of the morphisms
      \begin{equation*}
        \begin{tikzcd}
          H_2\arrow{r}{c_f}&\Coker(f)\arrow{r}{\lambda_{2,\Coker(f)}}&f_2(\Coker(f));
        \end{tikzcd}
      \end{equation*}
    \item\label{cor:2.3:b} the kernel of $f$ in $\mathcal{H}$ is the composition of the morphisms
      \begin{equation*}
        \begin{tikzcd}
          t_1(\Ker(f))\arrow{r}{\varepsilon_{1,\Ker(f)}}&\Ker(f)\arrow{r}{k_f}&H_1;
        \end{tikzcd}
      \end{equation*}
    \item\label{cor:2.3:c} $f$ is an epimorphism in $\mathcal{H}$ if and only if $\Coker(f)\in\mathcal{T}_2$;
    \item\label{cor:2.3:d} $f$ is a monomorphism in $\mathcal{H}$ if and only if $\Ker(f)\in\mathcal{F}_1$.
  \end{enumerate}
\end{corollary}

\begin{proof}
  \ref{cor:2.3:a} and \ref{cor:2.3:b} follow from the proof of \ref{rmk:2.2}\todo{this should be a remark, fix it}
  and \ref{5:lem:3}.

  \ref{cor:2.3:c}
  \begin{enumerate}
    \item[$\Leftarrow$] is trivial.
    \item[$\Rightarrow$] By \ref{prop:1.4}\ref{prop:1.4:b} there exists $f^C:H_2\to T_2$, where
      $T_2=\Coker_{T_1}(f)\in\mathcal{T}_2$. Then, we have

      \begin{minipage}[b]{0.45\linewidth}
        \begin{equation*}
          \begin{tikzcd}[column sep=tiny]
            H_2\arrow{rr}{f^C}\arrow{dr}[']{c_f}
              & & T_2 \arrow[dashed, shift right]{ld}[']{u}\\
              & \Coker(f)\arrow[dashed, shift right]{ur}[']{v}
          \end{tikzcd}
        \end{equation*}
      \end{minipage}
      \begin{minipage}[b]{0.45\linewidth}
        \begin{equation*}
          \text{such that }
          \left\{
          \begin{array}{c}
            uf^C = c_f,\\
            vc_f = f^C.
          \end{array}
          \right.
        \end{equation*}
      \end{minipage}

      Hence, $uvc_f=c_f$, but $c_f$ is epi, therefore $uv=1$. Hence, $\Coker(f)$ is a direct
      summand of $T_2\in\mathcal{T}_2$ so $\Coker(f)\in\mathcal{T}_2$.
  \end{enumerate}

  \ref{cor:2.3:d} Similar to the previous proof.
\end{proof}

\begin{thm}\label{thm_2_4}
  Let $\mathbbm{t}=(\mathbbm{t}_1,\mathbbm{t}_2)$ Nakaoka context in an abelian category
  $\mathscr{A}$. Then, for
  $\mathcal{H}:=\mathcal{T}_1\cap\mathcal{F}_2$ the following statements are equivalent:
  \begin{enumerate}[label=(\alph*)]
    \item $\mathcal{H}$ is an abelian category.
    \item The following conditions hold:
      \begin{enumerate}[label=(\alph{enumi}\arabic*)]
        \item For any $f:H\to H'$ in $\mathcal{H}$, with $\Ker(f)\in\mathcal{F}_1$,
        we have that $\Ker(f)=0$.
        \item For any $f:H\to H'$ in $\mathcal{H}$, with $\Coker(f)\in\mathcal{T}_2$,
        we have that $\Coker(f)=0$.
        \item $\mathcal{H}$ is closed under kernels (resp. cokernels) of epimorphisms
        (resp. monomorphisms) in $\mathscr{A}$.
      \end{enumerate}
    \item $\mathcal{H}$ is closed under kernels and cokernels in $\mathscr{A}$.
  \end{enumerate}
\end{thm}

\begin{proof}
  ($(a)\Rightarrow (b1),(b2)$) Assume that $\clH$ is an abelian category. By \ref{cor:2.3}\ref{cor:2.3:d}, $(b1)$ holds if and only if any monomorphism in $\clH$ is a monomorphism in $\mathscr{A}$.

  Observe that for any $f:H\to H'$, we have that $\Ker_\clH(f)\nto{\tilde{f}^H}H$ is a monomorphism in $\mathscr{A}$. Indeed, by \ref{cor:2.3}\ref{cor:2.3:b}, we know that $\tilde{f}^H$ is the composition of
  \begin{equation*}
    \begin{tikzcd}
      t_1(\Ker(f))\arrow{r}{\varepsilon_{1,\Ker(f)}}
        & \Ker(f)\arrow{r}{k_f}
          & H;
    \end{tikzcd}
  \end{equation*}
  and, since $\varepsilon_{1,\Ker(f)}$ and $k_f$ are monomorphism in $\mathscr{A}$, so is $\tilde{f}^H$.

  Furthermore, if $f:H\to H'$ is a monomorphism in $\clH$, then $f=\Ker_\clH(\Coker_\clH(f))$ since $\clH$ is abelian. Thus, $f$ is a monomorphism in $\mathscr{A}$.

  A dual argument can be used to prove $(b2)$.

  ($(a)\Rightarrow (b3)$) Let $f:H\to H'$ in $\clH$ be a monomorphism in $\mathscr{A}$. We want to prove that $\Coker(f)\in\clH$.

  Observe that the diagram
  \begin{equation}\label{thm_2_4_eq_1}
    \begin{tikzcd}
      0\arrow{r}
      & H\arrow{r}{f}\arrow{d}{\alpha}
        & H'\arrow{r}{c_f}\arrow[equal]{d}
          & \Coker(f)\arrow{d}{\lambda_2}\arrow{r}
            & 0\\
      0 \arrow{r}
      & \Ker(g)\arrow{r}{k_g}
        & H' \arrow{r}{g\colon=\lambda_2c_f}
          & f_2(\Coker(f)) \arrow{r}
            & 0
    \end{tikzcd}
  \end{equation}
  is commutative and has exact rows. Then, by Snake's lemma we get that $\Ker(\alpha)=0$ and $\Coker(\alpha)\cong\Ker(\lambda_2)=t_2(\Coker(f))\in\T_2$.

  Since $\clH$ is abelian, we know that $f=\Ker_\clH(\Coker_\clH(f))$. Therefore, by $\ref{cor:2.3}$ and \eqref{thm_2_4_eq_1}, we get that the dashed morphism
  in the following commutative diagram exists
  \begin{equation*}
    \begin{tikzcd}
      & H\arrow{r}{f}\arrow[dashed]{dl}[']{\exists\varphi}
        & H'\arrow[equal]{d}\\
      t_1(\Ker(g))\arrow{r}{\varepsilon_{1,\Ker(g)}}
      & \Ker(g)\arrow{r}{k_g}
        & H'
    \end{tikzcd}
  \end{equation*}
  and it is an isomorphism. But, $k_g\varepsilon_{1,\Ker(g)}\varphi = f= k_g\alpha$ and $k_g$ is mono, hence $\varepsilon_{1,\Ker(g)}\varphi = \alpha$. Therefore,
  \begin{equation*}
    \Coker(\alpha)\cong \Coker(\varepsilon_{1,\Ker(g)})=\frac{\Ker(g)}{t_1(\Ker(g))}= f_1(\Ker(g))\in\F_1.
  \end{equation*}

  Thus, $\Coker(\alpha)\in\T_2\cap\F_1=0$ and so $\alpha$ is an isomorphism. By \eqref{thm_2_4_eq_1} is follows that $\Coker(f)\cong f_2(\Coker(f))\in\F_2$ and, using the fact that $T_1$ is closed under quotients, we conclude that $\Coker(f)\in\T_1\cap\F_2=\clH$. A dual argument shows that $\clH$ is closed under kernels of epimorphism in $\mathscr{A}$.

  ($(b)\Rightarrow (a)$) We already know that $\clH$ is preabelian. On order to prove that $\clH$ is abelian we need to show that any monomorphism (resp. epimorphism) in $\clH$ is a kernel (resp. cokernel).

  Let $f:H\to H'$ be a monomorphism in $\clH$. By \ref{cor:2.3} and $(b1)$, it follows that $f$ is a monomorphism in $\mathscr{A}$. Thus, we have a short exact sequence in $\mathscr{A}$
  \begin{equation*}
    \begin{tikzcd}
      0 \arrow{r}
      & H\arrow{r}{f}
        & H'\arrow{r}{c_f}
          & \Coker(f) \arrow{r}
            & 0.
    \end{tikzcd}
  \end{equation*}

  Furthermore, from $(b3)$ we know that $\Coker(f)\in\clH$. Then, $\Coker(f)=\Coker_\clH(f)$ and $\Ker(c_f)=\Ker_\clH(c_f)$. Therefore, $f=\Ker(\Coker(f))=\Ker_\clH(\Coker_\clH(f))$.

  ($(a),(b)\Rightarrow (c)$) Let $f:H\to H'$ be a morphism in $\clH$. We want to show that $\Ker(f)\in\clH$. Consider the exact sequence
  \begin{equation*}
    \begin{tikzcd}
      0 \arrow{r}
      & \Ker(f)\arrow{r}
        & H\arrow{r}
          & \Img(f) \arrow{r}
            & 0,
    \end{tikzcd}
  \end{equation*}
  by $(b3)$ it is enough to show that $\Img(f)\in\clH$. Since $\clH$ is abelian, we have the canonical factorization of $f$ in $\clH$
  \begin{equation*}
    \begin{tikzcd}
      H\arrow{rr}{f}\arrow[two heads]{rd}[']{f'}
      & & H'\\
      & \Img_\clH(f)\arrow[hook]{ur}[']{f''}
        &
    \end{tikzcd}
  \end{equation*}
  where $f'$ is an epi and $f''$ is a mono (in $\clH$). Then, by $(b1)$ and $(b2)$ we have that $f'$ and $f''$ are respectively an epi and a mono in $\mathscr{A}$. Therefore, $\Img(f)\cong \Img_\clH(f)\in\clH$, and by the exact sequence
  \begin{equation*}
    \begin{tikzcd}
      0 \arrow{r}
      & \Img(f)\arrow{r}
        & H'\arrow{r}
          & \Coker(f) \arrow{r}
            & 0,
    \end{tikzcd}
  \end{equation*}
  and $(b3)$ we conclude that $\Coker(f)\in\clH$.

  $(c)\Rightarrow (a)$ is clear.

\end{proof}

\todo{Add some conclusion}

\subsection{Nakaoka contexts in triangulated categories}

Let $\mathscr{C}=\mathcal{T}$ be a triangulated category on which idempotents split.
We start by recalling the definition of a t-structure in $\mathcal{T}$.

\begin{definition}
  A pair $(\mathcal{A},\mathcal{B})$ of full subcategories of $\mathcal{T}$ is a t-structure
  in $\mathcal{T}$ if
  \begin{enumerate}[label=(\alph*)]
    \item $\mathbbm{t}=(\mathcal{A},\mathcal{B}[-1])$ is a torsion pair in $\mathcal{T}$, and
    \item $\mathcal{A}[1]\subseteq \mathcal{A}$.
  \end{enumerate}
\end{definition}

\begin{rmk}
  It is well known that any t-structure $(\mathcal{A},\mathcal{B})$ in $\mathcal{T}$
  gives a functional torsion pair $\mathbbm{t}=(\mathcal{A},\mathcal[-1])$ and
  $\mathcal{B}[-1]\subseteq\mathcal{B}$. Furthermore, $\mathcal{A}$ and
  $\mathcal{B}$ are closed under extensions and direct summands. Note that the t-structure
  $(\mathcal{A},\mathcal{B})$ depends only on $\mathcal{A}$, since $\mathcal{B}=\mathcal{A}^\perp[1]$.
\end{rmk}

\begin{definition}
  A \emph{related} torsion pair $\mathbbm{t}=(\mathbbm{t}_1,\mathbbm{t}_2)$ in triangulated category
  $\mathcal{T}$ consists of the torsion pairs $\mathbbm{t}_1=(\mathcal{T}_1,\mathcal{F}_1)$
  and $\mathbbm{t}=(\mathcal{T}_2,\mathcal{F}_2)$ in $\mathcal{T}$ such that
  $\mathcal{T}_1[1]\subseteq \mathcal{T}_2\subseteq\mathcal{T}_1$.
\end{definition}

\begin{prop}\label{prop:2.5}
  Let $\mathbbm{t}=(\mathbbm{t}_1,\mathbbm{t}_2)$ be a related torsion pair in $\mathcal{T}$. Then
  \begin{enumerate}[label=(\alph*)]
    \item $(\mathcal{T}_1,\mathcal{F}_1[1])$ and $(\mathcal{T}_2,\mathcal{F}_2[1])$
    are t-structures in $\mathcal{T}$;
  \item $\mathbbm{t}$ is a pre-Abelian Nakaoka context in $\mathcal{T}$;
    \item the heart $\mathcal{H}_\mathbbm{t} := \mathcal{T}_1\cap\mathcal{F}_2[1]$ is a preabelian category.
  \end{enumerate}
\end{prop}

\begin{proof}
  (a) follows by the inclusions $\T_1[1]\subseteq \T_2\subseteq \T_2$ and, assuming (b), (c) is a consequence of Theorem~\ref{pre_abelian_theorem}. It remains to prove (b), i.e. to show that axioms \ref{ax:ct1}, \ref{ax:ct2}, \ref{ax:ct3}, and \ref{ax:ct3op}.

  \begin{itemize}
    \item[\ref{ax:ct1}] Follows from (a).
    \item[\ref{ax:ct2}] Is in the hypothesis.
    \item[\ref{ax:ct3}] Let $g:H\to H'$ in $\clH$ and complete it to a triangle:
      \begin{equation*}
        \begin{tikzcd}
          H\arrow{r}{g}&H'\arrow{r}{g^C}&C\arrow{r}&H[1].
        \end{tikzcd}
      \end{equation*}
      Since $\clH\subseteq \T_1$, we have that $H[1]\in\T_1[1]\subseteq \T_1$. Moreover, since $\T_1$ is closed under extensions, $C$ belongs to $\T_1$. Hence, $g^C:H'\to C$ is a pseusocokernel of $g$ in $\T_1$.\todo{It is a pseudocokernel because axioms of triangulated categories}

      Take $F\in\F_2$. Applying the functor $(-,F)\colon=\Hom(-,F)$ to the triangle above yields an exact sequence of abelian groups
      \begin{equation*}
        \begin{tikzcd}
          (H[1],F)\arrow{r}&(C,F)\arrow{r}&(H',F)\arrow{r}&(H,F).
        \end{tikzcd}
      \end{equation*}
      Observe that $\F_2=\T_2\rhperp \subseteq \T_1\rhperp[1]$, i.e. $\Hom(H[1],F)=0$. This concludes the proof.
    \item[\ref{ax:ct3op}] Dual to the previous.
  \end{itemize}
\end{proof}

\begin{definition}
  A related torsion pair $\mathbbm{t}=(\mathbbm{t}_1,\mathbbm{t}_2)$ in the triangulated category
  $\mathcal{T}$ is \emph{strong} if for any morphism $f:H_1\to H_2$, in $\mathcal{H}:=\mathcal{T}_1\cap\mathcal{F}_2$,
  and a distinguished triangle $Z\to H_1\nto{f}H_2\to Z[1]$, the following conditions
  hold true
  \begin{relatedtorsion}
    \item $Z\in\mathcal{F}_1$ if and only if $Z\in\mathcal{F}_2[-1]$;
    \item $Z[1]\in\mathcal{T}_2$ if and only if $Z\in\mathcal{T}_1$.
  \end{relatedtorsion}
\end{definition}

\begin{thm}\label{thm:2.6}
  Let $\mathbbm{t}=(\mathbbm{t}_1,\mathbbm{t}_2)$ be a strongly related torsion pair in
  the triangulated category $\mathcal{T}$. Then, the heart $\mathcal{H}=\mathcal{H}_\mathbbm{t}$
  is an abelian category.
\end{thm}

\begin{proof}
  By Proposition~\ref{prop:2.5} $\t$ is a pre-Abelian Nakaoka context. Therefore, by Theorem~\ref{thm:2.5} it is enough to check that \ref{ax:ct4} and \ref{ax:ct4op} hold.

  \begin{itemize}
    \item[\ref{ax:ct4}]
  \end{itemize}
\end{proof}

In the next example we use the previous result to recover the classical result, that is the heart of any t-structure in a triangulated category is abelian.

\begin{example}
  Let $(\mathcal{A},\mathcal{B})$ be a t-structure in $\mathcal{T}$. Consider
  $\mathbbm{t}_1:=(\mathcal{A},\mathcal{B}[-1])$ and $\mathbbm{t}_2 :=(\mathcal{A}[1],\mathcal{B})$.
  It is not hard to see that $\mathbbm{t}=(\mathbbm{t}_1,\mathbbm{t}_2)$ is a strongly related
  torsion pair in $\mathcal{T}$. In this case, by \ref{thm:2.6}, we get that $\mathcal{H}=\mathcal{A}\cap\mathcal{B}$
  is an abelian category (BBD theorem).
\end{example}

\begin{example}\label{example_tria}
  Let $\mathcal{D}$ be a triangulated category with two $t$-structures
  $(\mathcal{U}_1,\mathcal{U}_1^\perp)$ and $(\mathcal{U}_2,\mathcal{U}_2^\perp)$ such that
  $\mathcal{U}_1[1]\subseteq\mathcal{U}_2\subseteq\mathcal{U}_1$. Observe that for any
  morphism $f$ in $\mathcal{D}$, the cone $\mathrm{Cone}(f)$ of $f$ in $\mathcal{D}$ is
  always a pseudocokernel $f$, and similarly the shifted cone $\mathrm{Cone}(f)[-1]$ is always
  a pseudokernel of $f$.

  If, moreover, $\mathrm{Cone}(f)\in\mathcal{U}_1$\todo{This is actually always true} for any morphism
  $f$ in $\mathcal{U}_1$ and $\mathrm{Cone}(f)[-1]\in\mathcal{U}_2^\perp$, then
  $\mathbbm{t}=((\mathcal{U}_1,\mathcal{U}_1^\perp),(\mathcal{U}_2,\mathcal{U}_2^\perp))$ satisfy
  axioms \ref{ax:ct1}-\ref{ax:ct3},\ref{ax:ct3op}, hence $\mathcal{H}=\mathcal{U}_1\cap\mathcal{U}_2^\perp$ has
  kernels and cokernels.

  In this case, moreover, the following are equivalent:
  \begin{enumerate}
    \item[1.a] \ref{ax:ct4} holds.
    \item[1.b]\label{ax:eqb} If $V_1\to H_1\nto{f} H_2\nto{+}$ is a distinguished triangle such
    that $H_1, H_2\in\mathcal{H}$ and $V_1\in\mathcal{U}_1^\perp$, then $V_1\in\mathcal{U}_2^\perp[-1]$.
  \end{enumerate}
  And, dually, there is an equivalence of the following:
  \begin{enumerate}
    \item[2.a] \ref{ax:ct4op} holds.
    \item[2.b]\label{ax:eqa} If $H_1\nto{f}H_2\to U_2\nto{+}$ is a distinguished triangle such
    that $H_1,H_2\in\mathcal{H}$ and $U_2\in\mathcal{U}_2$, then $U_2\in\mathcal{U}_1[1]$.
  \end{enumerate}
\end{example}

\begin{proof}[Proof of the equivalences in example 2]
  Let $\mathcal{D}$ be a triangulated category with two t-structures as in example 3. The pseudocokernel
  of a morphism in $\mathcal{U}_1$ can be computed by taking the cone in $\mathcal{D}$, i.e. given a morphism
  $f:U_1\to U_1'$ in $\mathcal{U}_1$ we can compute a pseudocokernel in $\mathcal{U}_1$ by completing $f$
  to a triangle
  \begin{equation*}
    U_1\nto{f}U_1'\to \mathrm{Cone}(f)\nto{+}.
  \end{equation*}
  Moreover, this pseudocokernel satisfies \ref{ax:ct3}.

  Now, assume that \ref{ax:ct1}-\ref{ax:ct3},\ref{ax:ct3op} are satisfied together with axiom
  \textbf{1.b}, and consider the solid part of the diagram as in \ref{ax:ct4}:
  \begin{equation*}
    \begin{tikzcd}
      \mathrm{Cone}(f)[-1]\arrow{r}{f^K}
        & H_1\arrow{r}{f}\arrow{d}{\alpha}
          & H_1\arrow{r}{f^C}\arrow[equal]{d}
            & \mathrm{Cone}(f)\arrow{d}{\lambda}\\
      \tau_{\mathcal{U}_1}(V_2)\arrow{r}{\varepsilon}\arrow[dotted]{ur}{\beta}
        & V_2\arrow{r}
          & H_2\arrow{r}
            & \tau^{\mathcal{U}_2^\perp}\mathrm{Cone}(f)
    \end{tikzcd}
  \end{equation*}
  with $\mathrm{Cone}(f)[-1]\in\mathcal{U}_1^\perp$ and where the upper row is a
  distinguished triangle. By \textbf{1.b} then it belongs
  to $\mathcal{U}_2^\perp[-1]$, i.e. $\mathrm{Cone}(f)\in\mathcal{U}_2^\perp$, so
  $\lambda$ is an iso, conseguently $\alpha$ is an iso and so is $\varepsilon$, so
  there exist a map $\beta=\alpha^{-1}\circ\varepsilon$ making the diagram commute, that
  is \ref{ax:ct4} holds.

  Conversely, assume that \ref{ax:ct1}-\ref{ax:ct3},\ref{ax:ct3op} are satisfied together with
  \ref{ax:ct4}. Consider again the solid part of the diagram
  \begin{equation*}
    \begin{tikzcd}
      \mathrm{Cone}(f)[-1]\arrow{r}{f^K}\arrow{d}{\lambda[-1]}
        & H_1\arrow{r}{f}\arrow{d}{\alpha}
          & H_1\arrow{r}{f^C}\arrow[equal]{d}
            & \mathrm{Cone}(f)\arrow{d}{\lambda}\\
      \tau_{\mathcal{U}_2^\perp}(\mathrm{Cone}(f))[-1]\arrow{r}
        & V_2\arrow{r}
          & H_2\arrow{r}
            & \tau^{\mathcal{U}_2^\perp}\mathrm{Cone}(f)\\
        & \tau_{\mathcal{U}_1}(V_2)\arrow{u}{\varepsilon}\arrow[dotted, crossing over, bend left=70]{uu}[near end,']{\beta}
        & &
    \end{tikzcd}
  \end{equation*}
  with $\mathrm{Cone}(f)[-1]\in\mathcal{U}_1^\perp$. Neeman \cite[Lemma~1.4.3]{Nee01} guarantees
  that $\alpha$ can be taken so that the square on the left is a pullback. Axiom \ref{ax:ct4}
  gives the existence of $\beta:\tau_{\mathcal{U}_1}(V_2)\to H_1$ such that $\alpha\circ\beta=\varepsilon$.

  Since $\tau_{\mathcal{U}_1}$ is a functor, there is also a morphism
  $\tau_{\mathcal{U}_1}(\alpha):\tau_{\mathcal{U}_1}(H_1)=H_1\to \tau_{\mathcal{U}_1}(V_2)$ such that
  $\varepsilon\circ\tau_{\mathcal{U}_1}(\alpha)=\alpha$, hence
  $\varepsilon\circ\tau_{\mathcal{U}_1}(\alpha)\circ\beta = \varepsilon$. By the functoriality of
  the torsion pair $(\mathcal{U}_1,\mathcal{U}_1^\perp)$, this means that
  $\tau_{\mathcal{U}_1}(\alpha)\circ\beta=1_{\tau_{\mathcal{U}_1}(V_2)}$. Then, $\beta$ is a section.

  Hence, we can write $\tau_{\mathcal{U}_1}(\alpha):H_1\to \tau_{\mathcal{U}_1}(V_2)$
  as
  \begin{equation*}
    \begin{tikzcd}
      \tau_{\mathcal{U}_1}(\alpha):\tau_{\mathcal{U}_1}(V_2)\oplus H_1'
      \arrow{r}{
          \begin{psmallmatrix}
            \ast \amsamp 0
          \end{psmallmatrix}
          }
        & \tau_{\mathcal{U}_1}(V_2)
    \end{tikzcd}
  \end{equation*}
  for some $H_1'\underset{\oplus}{<} H_1$ such that
  $\alpha$ vanishes on $H_1'$. If we consider the solid part of the diagram
  \begin{equation*}
    \begin{tikzcd}
      & H_1'\arrow[dashed]{dl}
      \arrow{d}{
        \begin{psmallmatrix}
          1 \\ 0
        \end{psmallmatrix}
      }\arrow{dr}{0}
        & & &\\
      \mathrm{Cone}(\tau_{\mathcal{U}_1}(\alpha))[-1]\arrow{r}
        & H_1\arrow{r}{\tau_{\mathcal{U}_1}(\alpha)}
          & \tau_{\mathcal{U}_1}(V_2)\arrow[dashed]{r}{+}
            & {}
    \end{tikzcd}
  \end{equation*}
  we can construct the dashed arrow, and the fact that the triangle commutes means
  that $H_1'\underset{\oplus}{<}\mathrm{Cone}(\tau_{\mathcal{U}_1}(\alpha))[-1]$.

  Observe that $\mathrm{Cone}(\alpha)=\mathrm{Cone}(\lambda)[-1]$, since
  the square
  \begin{equation*}
    \begin{tikzcd}
      \mathrm{Cone}(f)[-1]\arrow{r}{f^K}\arrow{d}{\lambda[-1]}
        & H_1\arrow{d}{\alpha} \\
      \tau_{\mathcal{U}_2^\perp}(\mathrm{Cone}(f))[-1]\arrow{r}
        & V_2
    \end{tikzcd}
  \end{equation*}
  is a pullback. Moreover, $\mathrm{Cone}(\lambda)[-1] = (\tau_{\mathcal{U}_2}(\mathrm{Cone}(f))[1])[-1]
  =\tau_{\mathcal{U}_2}(\mathrm{Cone}(f))$. Hence, $\mathrm{Cone}(\alpha)\in \mathcal{U}_2$ and
  $\tau^{\mathcal{U}_1^\perp}(\mathrm{Cone}(\alpha))=0$, that is,
  $\mathrm{Cone}(\alpha)\in\mathcal{U}_1$, and since there is a distinguished triangle
  \begin{equation*}
    H_1\nto{\alpha} V_2\to \mathrm{Cone}(\alpha)\nto{+}
  \end{equation*}
  with $H_1,\mathrm{Cone}(\alpha)\in\mathcal{U}_1$ it follows that
  $V_2\in\mathcal{U}_1$. Hence, $\tau_{\mathcal{U}_1}(V_2)\cong V_2$.

  We can then write $V_2\underset{\oplus}{<}H_1$ and consider the commutative diagram
  \begin{equation*}
    \begin{tikzcd}
      H_1 \cong H_1'\oplus V_2\arrow{r}{
        \begin{psmallmatrix}
          f' \amsamp \tilde{f}
        \end{psmallmatrix}
      }\arrow{d}{
        \begin{psmallmatrix}
          0 \amsamp 1
        \end{psmallmatrix}
      }
        & H_2\arrow[equal]{d}\\
      V_2\arrow{r}
        & H_2
    \end{tikzcd}
  \end{equation*}
  so $f'=0$. Hence, the inclusion $
  \begin{psmallmatrix}
    1 \\ 0
  \end{psmallmatrix}:H_1'\to H_1'\oplus V_2$ can be lifted to
  $\mathrm{Cone}(f)[-1]$ and $H_1'\underset{\oplus}{<}\mathrm{Cone}(f)[-1]$.
  Since $\mathrm{Cone}(f)[-1]\in \mathcal{U}_1^\perp$, so does $H_1'$. Similarly,
  $H_1'\in\mathcal{U}_1$ because $H_1\in\mathcal{U}_1$. Hence, $H_1'=0$ and $\alpha:H_1\to V_2$ is an iso.
  The same follows for $\lambda$. Therefore, $\mathrm{Cone}(f)\in\mathcal{U}_2^\perp$ which
  proves \triangb.
\end{proof}

\begin{lemma}\label{lemma_tria}
\textcolor{blue}{  In the case of \ref{example_tria}, assume that \ref{ax:ct4} and \ref{ax:ct4op} hold, then:
  \begin{enumerate}
    \item\label{lemma_tria_a} For any $H,H'\in\clH_\mathbbm{t}$, $(H,H'[-1])=0$,
    \item\label{lemma_tria_b} $A\to B\to C$ is a short exact sequence in $\mathcal{H}_\mathbbm{t}$ iff $A\to B\to C\nto{+}$ is
    a triangle in $\X$.
  \end{enumerate}}
\end{lemma}

\begin{proof}
  \ref{lemma_tria_a} follows observing that
  $\clH[-1]=\mathcal{U}_1[-1]\cap\mathcal{U}_2^\perp[-1]\subseteq \mathcal{U}_2^\perp[-1]\subseteq(\mathcal{U}_1^\perp[1])[-1]=\mathcal{U}_1^\perp$.

  To prove \ref{lemma_tria_b} first consider a short exact sequence $A\nto{f} B\to C$ in $\mathcal{H}$, we want to
  prove that it is a triangle. Consider the triangle
  \begin{equation*}
    \mathrm{Cone}(f)[-1]\to A\nto{f} B \to \mathrm{Cone}(f),
  \end{equation*}
  since $f$ is mono, its pseudocokernel $\mathrm{Cone}(f)[-1]$ belongs to $\mathcal{U}_1^\perp$, so
  $\mathrm{Cone}(f)\in\mathcal{U}_2^\perp[-1]$ by \triangb, hence $\mathrm{Cone}(f)\in\mathcal{U}_2^\perp$.
  This implies that $\mathrm{Cone}(f)$ is the cokernel of $f$ and so $A\nto{f} B\to C\nto{+}$ is a triangle.

  Conversely, assume that $A\nto{f} B\nto{g} C\nto{+}$ is a triangle. Then, $\mathrm{Cone}(f)\cong C\in\clH=\mathcal{U}_1\cap\mathcal{U}_2^\perp$ and
  so $\mathrm{Cone}(f)[-1]\in\mathcal{U}_2^\perp[-1]\subseteq \mathcal{U}_1^\perp$. Hence, $f$ is mono. A dual argument
  shows that $g$ is epi. Therefore, $A\nto{f} B\nto{g} C$ is a short exact sequence in $\mathcal{H}$.
\end{proof}

\begin{example}
  Let $R$ be any (associative with 1) ring. Consider the triangulated category $\mathcal{T}:=\mathcal{D}(R)$.
  The derived category $\mathcal{D}(R)$ has the so called natural t-structure
  $(\mathcal{D}^{\leq 0}(R),\mathcal{D}^{\geq}(R))$ where
  \begin{align*}
    \mathcal{D}^{\leq 0}(R) &:= \{ X\in\mathcal{D}(R) \,|\,H^i(X)=0\text{ for } i>0\},\\
    \mathcal{D}^{\geq 0}(R) &:= \{ X\in\mathcal{D}(R) \,|\,H^i(x)=0\text{ for } i<0\}.
  \end{align*}

  For any ideal $I\trianglelefteq R$, we have the TTF-triple $(\mathcal{C}_I,\mathcal{T}_I,\mathcal{F}_I)$
  associated to $I$, where
  \begin{align*}
    \mathcal{C}_I &:= \{M\in\Mod{R}\,|\,IM=M\},\\
    \mathcal{T}_I &:= \{M\in\Mod{R}\,|\,IM=0\}\cong \Mod{\frac{R}{I}},\\
    \mathcal{F}_I &:= \{M\in\Mod{R}\,|\,Ix=0\text{ and }x\in M\Rightarrow x=0\}.
  \end{align*}

  Consider the t-structure (Happel-Reiten-Smalo) $(\mathcal{D}^{\leq 0}_{t_I}(R), \mathcal{D}^{\geq 0}_{t_I}(R))$
  associated to the torsion pair $t_I=(\mathcal{C}_I,\mathcal{T}_I)$, where
  \begin{align*}
    \mathcal{D}^{\leq 0}_{t_I}(R) &:= \{ X\in\mathcal{D}^{\leq 0}(R)\,|\, H^0(X)\in\mathcal{C}_I \},\\
    \mathcal{D}^{\geq 0}_{t_I}(R) &:= \{ X\in\mathcal{D}^{\geq 0}(R)\,|\, H^0(X)\in\mathcal{T}_I \}.
  \end{align*}

  It can be seen that $\mathbbm{t}=(\mathbbm{t}_1,\mathbbm{t}_2)$ where
  $\mathbbm{t}_1:=(\mathcal{D}^{\leq 0}(R),\mathcal{D}^{\geq 1}(R))$ and
  $\mathbbm{t}_2:=(\mathcal{D}^{\leq 0}_{t_I}(R),\mathcal{D}^{\geq 1}_{t_I}(R))$, is a strongly
  related torsion pair in $\mathcal{T}=\mathcal{D}(R)$.
\end{example}

\subsection{Polishchuk correspondence}

We recall the following bijection given by A. Polishchuk, and in order to do that,
for a t-structure $(\mathcal{U}_1,\mathcal{U}_1^\perp[1])$ in $\mathcal{T}$, we have the cohomological
functor $H^0_1:\mathcal{T}\to \mathcal{H}_1:=\mathcal{U}_1\cap\mathcal{U}_1^\perp[1]$
($\mathcal{H}_1$ is an abelian category).

\begin{prop}[Polishchuk]\label{prop:2.7}
  Let $(\mathcal{U}_1,\mathcal{U}_1^\perp[1])$ be a t-structure in a triangulated category.
  Then we have a bijection (Polishchuk's bijection)
  \begin{equation*}
    \begin{tikzcd}
      \left\{
      \begin{array}{c}
        \text{torsion pairs in} \\
        \mathcal{H}_1=\mathcal{U}_1\cap \mathcal{U}_1^\perp[1]
      \end{array}
      \right\}
      \arrow[leftrightarrow]{r}{\mathrm{Pol}_{\mathcal{H}_1}}
        &
        \left\{
          \begin{array}{c}
            \text{t-structures }
            (\mathcal{U}_2,\mathcal{U}_2^\perp) \\
            \text{ in } \mathcal{D}
            \text{ satisfying } \\ \mathcal{U}_1[1]\subseteq \mathcal{U}_2\subseteq\mathcal{U}_1
          \end{array}
        \right\}\\
      (\mathscr{C},\mathcal{Y})\arrow[mapsto]{r}
        & (\mathcal{U}_2,\mathcal{U}_2^\perp[1])\\
      (\mathcal{U}_2\cap\mathcal{H}_1,\mathcal{U}_2^\perp\cap\mathcal{H}_1)\arrow[mapsfrom]{r}
        & (\mathcal{U}_2,\mathcal{U}_2^\perp[1])
    \end{tikzcd}
  \end{equation*}
  where
  \begin{align*}
    \mathcal{U}_2 = \{X\in\mathcal{U}_1\,|\,H^0_1(X)\in\mathscr{C}\}\\
    \mathcal{U}_2^\perp = \{Y\in\mathcal{U}_1^\perp\,|\,H^0_1(Y)\in\mathcal{Y}\}.
  \end{align*}
\end{prop}

\begin{rmk}\label{rmk:2.8}
  \begin{enumerate}[label=(\arabic*)]
    \item\label{rmk:2.8:1} Note that $\mathrm{Pol}^{-1}_{\mathcal{H}_1}(\mathcal{U}_2,\mathcal{U}^\perp_2[1])=
      (\mathcal{U}_2\cap\mathcal{U}_1^\perp[1],\mathcal{H})$, where
      $\mathcal{H}:=\mathcal{U}_1\cap\mathcal{U}_2^\perp$.

    \item By \ref{rmk:2.8:1}, it follows that $\mathcal{H}$ is a torsion free class in the abelian category
      $\mathcal{H}_1:=\mathcal{U}_1\cap\mathcal{U}_1^\perp[1]$.
  \end{enumerate}
\end{rmk}

\begin{thm}\label{thm:2.9}
  Let $\mathbbm{t}=(\mathbbm{t}_1,\mathbbm{t}_2)$ be a related torsion pair in a triangulated
  category $\mathcal{T}$. Then, the following statements are equivalent.
  \begin{enumerate}[label=(\alph*)]
    \item For any distinguished triangle $V\to H_1\nto{f}H_2\nto{+}$, with
      $f$ a morphism in $\mathcal{H}=\mathcal{H}_\mathbbm{t}:=\mathcal{T}_1\cap\mathcal{F}_2$,
      we have that
      \begin{equation*}
        V\in\mathcal{F}_1 \Rightarrow V[1]\in\mathcal{F}_2.
      \end{equation*}
    \item For any monomorphism $\alpha:H_1\into H_2$, in the abelian category
      $\mathcal{H}_1:=\mathcal{T}_1\cap\mathcal{F}_1[1]$, with $H_1,H_2\in\mathcal{H}$,
      we have that $\Coker_{\mathcal{H}_1}(\alpha)\in\mathcal{H}$.
    \item $\mathcal{H}$ is closed under kernels and cokernels in the abelian category
      $\mathcal{H}_1$
    \item $\mathcal{H}$ is an abelian category.
    \item For any epimorphism $H\onto X$ in $\mathcal{H}_1$, with $H\in\mathcal{H}$,
      we have that $X\in\mathcal{H}$ (i.e. $\mathcal{H}$ is closed under quotients in $\mathcal{H}_1$).
  \end{enumerate}
\end{thm}

Let $t=(\mathcal{A},\mathcal{B})$ be a pair of full subcategories of the triangulated
category $\mathcal{T}$. We will use the following notation:
\begin{align*}
  t[1] &:= (\mathcal{A}[1],\mathcal{B}[1]),\\
  \overline{t} &:= (\mathcal{A},\mathcal{B}[1]).
\end{align*}

Note that $\overline{t}$ is a t-structure in $\mathcal{T}$ if and only if $t$ is a torsion
pair $\mathcal{T}$ such that $\mathcal{A}[1]\subseteq\mathcal{A}$.

\begin{rmk}
  Consider $\mathbbm{t}=(\mathbbm{t}_1,\mathbbm{t}_2)$, where $\mathbbm{t}_i:=(\mathcal{U}_i,\mathcal{U}_i^\perp)$ for
  $i=1,2$. We have
  \begin{enumerate}
    \item $\mathcal{H}_\mathbbm{t}:=\mathcal{U}_1\cap\mathcal{U}_2^\perp$,
      $\mathcal{H}_i:=\mathcal{U}_i\cap\mathcal{U}_i^\perp[1]$,
    \item $\mathbbm{t}' :=(\mathbbm{t}_2,\mathbbm{t}_1[1])$
  \end{enumerate}

  Note that
  \begin{enumerate}
    \setcounter{enumi}{2}
    \item $\mathbbm{t}=(\mathbbm{t}_1,\mathbbm{t}_2)$ is a related torsion pair in $\mathcal{T}$
      \begin{align*}
        &\Leftrightarrow \mathcal{U}_1[1]\subseteq\mathcal{U}_2\subseteq\mathcal{U}_1\\
        &\Leftrightarrow \mathcal{U}_2[1]\subseteq\mathcal{U}_1[1]\subseteq\mathcal{U}_2\\
        &\Leftrightarrow \mathbbm{t}'=(\mathbbm{t}_2,\mathbbm{t}_1[1]) \text{ is a related torsion pair in }\mathcal{T}.
      \end{align*}
    \item Let $\mathbbm{t}=(\mathbbm{t}_1,\mathbbm{t}_2)$ is a related torsion pair in $\mathcal{T}$. In
      this case, we have
      \begin{align*}
        &\mathcal{H}_\mathbbm{t} = \mathcal{U}_1\cap\mathcal{U}_2^\perp,~
        \mathcal{H}_{\mathbbm{t}'}=\mathcal{U}_2\cap \mathcal{U}_1^\perp[1],\\
        & \mathrm{Pol}_{\mathcal{H}_1}^{-1}(\overline{\mathbbm{t}}_2)=
        \mathrm{Pol}_{\mathcal{H}_1}^{-1}(\mathcal{U}_2,\mathcal{U}_2^\perp[1])=
        (\mathcal{H}_{\mathbbm{t}'},\mathcal{H}_{\mathbbm{t}}),\\
        &\mathrm{Pol}_{\mathcal{H}_2}^{-1}(\overline{\mathbbm{t}}_1[1])=
        \mathrm{Pol}_{\mathcal{H}_1}^{-1}(\mathcal{U}_1[1],\mathcal{U}_1^\perp[2])=
        (\mathcal{H}_{\mathbbm{t}}[1],\mathcal{H}_{\mathbbm{t}'}).
      \end{align*}

      Thus, $(\mathcal{H}_{\mathbbm{t}'},\mathcal{H}_\mathbbm{t})$ is a torsion pair in the abelian
      category $\mathcal{H}_1$,
      $(\mathcal{H}_\mathbbm{t}[1],\mathcal{H}_{\mathbbm{t}'})$ is a torsion pair in the abelian category
      $\mathcal{H}_2$.
  \end{enumerate}
\end{rmk}

\begin{corollary}
  Let $\mathbbm{t}=(\mathbbm{t}_1,\mathbbm{t}_2)$ be a related torsion pair in a triangulated category
  $\mathcal{T}$. Then, the following statements are equivalent:
  \begin{enumerate}[label=(\alph*)]
    \item For any distinguished triangle $V\to H_1\nto{f}H_2\nto{+}$, with
    $f$ a morphism in $\mathcal{H}_{\mathbbm{t}'}=\mathcal{T}_2\cap\mathcal{F}_1[1]$,
    we have that $V\in\mathcal{F}_2$ implies $V\in\mathcal{F}_1$.
    \item For any monomorphism $\alpha:H_1\into H_2$, in the abelian category
    $\mathcal{H}_2:=\mathcal{T}_2\cap\mathcal{F}_2[1]$, with $H_1,H_2\in\mathcal{H}_{\mathbbm{t}'}$,
    we have that $\Coker_{\mathcal{H}_2}(\alpha)\in\mathcal{H}_{\mathbbm{t}'}$.
    \item $\mathcal{H}_{\mathbbm{t}'}$ is closed under kernels and cokernels in
    the abelian category $\mathcal{H}_2$.
    \item $\mathcal{H}_{\mathbbm{t}'}$ is an abelian category.
    \item $\mathcal{H}_{\mathbbm{t}'}$ is closed under quotients in $\mathcal{H}_2$.
  \end{enumerate}
\end{corollary}

We recall that a torsion pair $(\mathcal{T},\mathcal{F})$ in an abelian category
$\mathcal{A}$ is cohereditary if the class $\mathcal{F}$ is closed under quotients
in $\mathcal{A}$.

\begin{definition}
  For a triangulated category $\mathcal{T}$, we consider the following classes:
  \begin{enumerate}
    \item $RtAb(\mathcal{T}):=\{\text{related torsion pairs }\mathbbm{t}=(\mathbbm{t}_1,\mathbbm{t}_2)
      \text{ in }\mathcal{T}\text{ s.t. }\mathcal{H}_\mathbbm{t}\text{ is abelian}\}$;
    \item
      \begin{align*}
        \mathrm{t-}stCoh(\mathcal{T}) &:=
        \left\{
        \begin{array}{c}
          \text{pairs }(\overline{\mathbbm{t}}_1,\tau)\text{ s.t. }
          \overline{\mathbbm{t}}_1\text{ is a t-structure in }\mathcal{T}\text{ and }
          \tau\text{ is a}\\
          \text{ cohereditary torsion pair in the abelian category } \\
          \mathcal{H}_1:=\mathcal{U}_1\cap\mathcal{U}_1^\perp[1]
        \end{array}
        \right\};
      \end{align*}
    \item[1\rlap{$^\prime$}.] $RtAb^\prime(\mathcal{T}):=\{\text{related torsion pairs }\mathbbm{t}=(\mathbbm{t}_1,\mathbbm{t}_2)
      \text{ in }\mathcal{T}\text{ s.t. }\mathcal{H}_{\mathbbm{t}'}\text{ is abelian}\}$;
    \item[2\rlap{$^\prime$}.]
      \begin{align*}
        \mathrm{t-}stCoh^\prime(\mathcal{T}) &:=
        \left\{
        \begin{array}{c}
          \text{pairs }(\overline{\mathbbm{t}}_2,\tau)\text{ s.t. }
          \overline{\mathbbm{t}}_2\text{ is a t-structure in }\mathcal{T}\text{ and }
          \tau\text{ is a}\\
          \text{ cohereditary torsion pair in the abelian category } \\
          \mathcal{H}_2:=\mathcal{U}_2\cap\mathcal{U}_2^\perp[1]
        \end{array}
        \right\}.
      \end{align*}
  \end{enumerate}
\end{definition}

\begin{thm}\label{thm:2.11}
  For a triangulated category $\mathcal{T}$, the following statements hold true.
  \begin{enumerate}[label=(\alph*)]
    \item There is a bijective correspondence
      \begin{equation*}
        \begin{tikzcd}[row sep=tiny]
          RtAb(\mathcal{T})\arrow[leftrightarrow]{r}{\alpha}
            & \mathrm{t}-stCoh(\mathcal{T})\\
          \mathbbm{t}\arrow[mapsto]{r}
            & (\overline{\mathbbm{t}}_1,\mathrm{Pol}^{-1}_{\mathcal{H}_1}(\overline{\mathbbm{t}}_2))\\
          (\mathbbm{t}_1,\mathbbm{t}_2)\arrow[mapsfrom]{r}
            & (\overline{\mathbbm{t}}_1,\tau)
        \end{tikzcd}
      \end{equation*}
      where $\overline{\mathbbm{t}}_2=\mathrm{Pol}_{\mathcal{H}_1}(\tau)$.
    \item There is a bijective correspondence
      \begin{equation*}
        \begin{tikzcd}
          RtAb^\prime(\mathcal{T})\arrow[leftrightarrow]{r}{\alpha'}
            & \mathrm{t}-stCoh^\prime(\mathcal{T})\\
          \mathbbm{t}\arrow[mapsto]{r}
            & (\overline{\mathbbm{t}}_2,\mathrm{Pol}^{-1}_{\mathcal{H}_2}(\overline{\mathbbm{t}}_1[1]))\\
          (\mathbbm{t}_1,\mathbbm{t}_2)\arrow[mapsfrom]{r}
            & (\overline{\mathbbm{t}}_2,\tau)
        \end{tikzcd}
      \end{equation*}
      where $\overline{\mathbbm{t}}_1=\mathrm{Pol}_{\mathcal{H}_2}(\tau)[-1]$.
  \end{enumerate}
\end{thm}

% \begin{lemma}\label{sec2:lem6}
%   The heart $\mathcal{H}=\mathcal{T}_1\cap\mathcal{F}_2$ has kernels and cokernels.
% \end{lemma}
%
% \begin{rmk}
%   If $\mathscr{C}$ is abelian, then:
%   \begin{enumerate}
%     \item $f:H\to H'$ is mono in $\mathcal{H}$ iff $\Ker_\mathscr{C}(f)\in\mathcal{F}_1$
%     \item $f:H\to H'$ is epi in $\mathcal{H}$ iff $\Coker_\mathscr{C}(f)\in\mathcal{T}_2$
%   \end{enumerate}
% \end{rmk}
%
% \begin{prop}\label{sec2:prop1}
%   Let $(\mathcal{T}_1,\mathcal{F}_1)$ and $(\mathcal{T}_2,\mathcal{F}_2)$ be two torsion pairs
%   in the abelian category $\mathscr{A}$. TFAE:
%   \begin{enumerate}
%     \item $\mathcal{H}$ is abelian.
%     \item We have:
%     \begin{enumerate}
%       \item If $f:H\to H'$ is a morphism in $\mathcal{H}$ such that
%       $\Ker_\mathscr{A}(f)\in\mathcal{F}_1$, then $\Ker_\mathscr{A}(f)=0$.
%       \item If $f:H\to H'$ is a morphism in $\mathcal{H}$ such that
%       $\Coker_\mathscr{A}(f)\in\mathcal{T}_2$, then $\Coker_\mathscr{A}(f)=0$.
%       \item $\mathcal{H}$ is closed by kernels of epics and cokernels of monics.
%     \end{enumerate}
%     \item $\mathcal{H}$ is an exact abelian subcategory of $\mathscr{A}$.\todo{?}
%   \end{enumerate}
% \end{prop}
