%************************************************
\chapter{Nakaoka contexts with abelian hearts}\label{ch:nakaoka} % $\mathbb{ZNR}$
%************************************************

Let $(\T^{\leq 0},\T^{\geq 0})$ be a $t$-structure in a triangulated category $\mathscr{C}$. It is well known (see \cite{bbd82}) that its heart $\clH=\T^{\leq 0}\cap\T^{\geq 0}$ is an abelian category.

A different result but in the same spirit arises for a cluster tilting subcategory $\mathscr{T}$ of a triangulated category $\mathscr{C}$. It was showed by Koenig and Zhu in \cite{Koenig2008} that in this case $\mathscr{C}/\mathscr{T}$ carries an abelian structure.

In both cases the construction of the abelian structure, i.e. of the kernels and cokernels, relies heavily on the triangulated structure of $\mathscr{C}$. In \cite{Nakaokaa}, Nakaoka proved that starting from any torsion pair $(\U,\V)$ in a triangulated category $\mathscr{C}$ it is possible to construct an abelian heart $\underline{\clH}$ in the quotient category $\underline{\mathscr{C}}=\mathscr{C}/\W$, where $\W=\U\cap\V[1]$. \todo{Check that $(\U\cap\V)$ is the exact one, since here I am referring to torsion pairs and Nakaoka uses cotorsion pairs in his article}
This result recovers both cases above giving a unified way of constructing kernels and cokernels in the heart.

Keeping this setting in mind, we will provide a set of axioms for a pair of torsion pairs $\mathbbm{t}_1=(\T_1,\F_1)$ and $\mathbbm{t}_2=(\T_2,\F_2)$ in a (sufficiently nice) \todo{Give a better specification of what this means}
additive category in such a way that $\clH = \T_1\cap\F_2$ becomes an abelian category. In a sense, our objective is to axiomatize $\underline{\mathscr{C}}$ and the pairs which are referred to as $(\underline{\C}^-,\underline{\V})$ and $(\underline{\U},\underline{\C}^+)$ in \cite{Nakaokaa}.

Since we don't have a triangulated structure to rely on, the axioms will impose some requirements on $\mathbbm{t}_1$ and $\mathbbm{t}_2$ to compensate. Moreover, we will require that idempotents split in $\mathscr{C}$.\todo{check this hypothesis}

The results in this chapter are part of a work in preparation by the author in conjunction with Manuel Saor\'in and Simone Virili.

%*****************************************
\section{Torsion pairs}
%*****************************************

Let $\mathscr{C}$ be an additive category and $\A\subseteq \mathscr{C}$ a class of objects. We introduce the following notation:
\begin{itemize}
  \item $\A\rhperp = \{X\in\mathscr{C}| \Hom_{\mathscr{C}}(\A,X)=0\}$
  \item $\lhperp{\A}=\{ X\in \mathscr{C}| \Hom_{\mathscr{C}}(X,\A)=0\}$.
\end{itemize}

Given a map $\varphi:X\to Y$, recall that a \emph{pseudocokernel} of $\varphi$ is a map $\psi:Y\to Z$ such that $\psi\varphi=0$ and any other $\psi':Y\to Z$ satisfying $\psi'\varphi=0$ factors (not necessarily uniquely) through $\psi$. There an obvious dual notion of \emph{pseudokernel}

\begin{definition}
  A pair $\mathbbm{t}=(\T,\F)$ of classes of objects of $\mathscr{C}$ is a \emph{torsion pair} if:
  \begin{enumerate}
    \item $\F=\T\rhperp$ and $\T = \lhperp{\F}$,
    \item given $X\in \mathscr{C}$ there is a pseudokernel-pseudocokernel sequence
      \begin{equation*}
        T_X\nto{\varepsilon_X} X\nto{\lambda_X} F^X
      \end{equation*}
      where $T_X\in\T$ and $F^X\in\F$.
  \end{enumerate}
  Moreover, a torsion pair $\mathbbm{t}=(\T,\F)$ is called \emph{left} (resp. \emph{right}) \emph{functorial} if $\T$ (resp. $\F$) is a coreflective (resp. reflective) subcategory of $\mathscr{C}$. If $\mathbbm{t}$ is both left and right functorial it is called just \emph{functorial}.
\end{definition}

It is a classical result that torsion pairs in Abelian categories are automatically functorial. Similarly, $t$-structures in triangulated categories are examples of functorial torsion pairs.

Test

%*****************************************
%*****************************************
%*****************************************
