%************************************************
\chapter{Nakaoka contexts with abelian hearts}\label{ch:nakaoka} % $\mathbb{ZNR}$
%************************************************

Let $(\T^{\leq 0},\T^{\geq 0})$ be a $t$-structure in a triangulated category $\mathscr{C}$. It is well known (see \cite{bbd82}) that its heart $\clH=\T^{\leq 0}\cap\T^{\geq 0}$ is an abelian category.

A different result but in the same spirit arises for a cluster tilting subcategory $\mathscr{T}$ of a triangulated category $\mathscr{C}$. It was showed by Koenig and Zhu in \cite{Koenig2008} that in this case $\mathscr{C}/\mathscr{T}$ carries an abelian structure.

In both cases the construction of the abelian structure, i.e. of the kernels and cokernels, relies heavily on the triangulated structure of $\mathscr{C}$. In \cite{Nakaokaa}, Nakaoka proved that starting from any torsion pair $(\U,\V)$ in a triangulated category $\mathscr{C}$ it is possible to construct an abelian heart $\underline{\clH}$ in the quotient category $\underline{\mathscr{C}}=\mathscr{C}/\W$, where $\W=\U\cap\V[1]$. \todo{Check that $(\U\cap\V)$ is the exact one, since here I am referring to torsion pairs and Nakaoka uses cotorsion pairs in his article}
This result recovers both cases above giving a unified way of constructing kernels and cokernels in the heart.

Keeping this setting in mind, we will provide a set of axioms for a pair of torsion pairs $\t_1=(\T_1,\F_1)$ and $\t_2=(\T_2,\F_2)$ in a (sufficiently nice) \todo{Give a better specification of what this means}
additive category in such a way that $\clH = \T_1\cap\F_2$ becomes an abelian category. In a sense, our objective is to axiomatize $\underline{\mathscr{C}}$ and the pairs which are referred to as $(\underline{\C}^-,\underline{\V})$ and $(\underline{\U},\underline{\C}^+)$ in \cite{Nakaokaa}. \todo{Add example of remove}

Since we don't have a triangulated structure to rely on, the axioms will impose some requirements on $\t_1$ and $\t_2$ to compensate. Moreover, we will require that idempotents split in $\mathscr{C}$.\todo{check this hypothesis}

The results in this chapter are part of a work in preparation by the author in conjunction with Manuel Saor\'in, Simone Virili and Octavio Mendoza.

%*****************************************
\section{Torsion pairs}
%*****************************************

Let $\mathscr{C}$ be an additive category and $\A\subseteq \mathscr{C}$ a class of objects. We introduce the following notation:
\begin{itemize}
  \item $\A\rhperp = \{X\in\mathscr{C}| \Hom_{\mathscr{C}}(\A,X)=0\}$
  \item $\lhperp{\A}=\{ X\in \mathscr{C}| \Hom_{\mathscr{C}}(X,\A)=0\}$.
\end{itemize}

Given a map $\varphi:X\to Y$, recall that a \emph{pseudocokernel} of $\varphi$ is a map $\psi:Y\to Z$ such that $\psi\varphi=0$ and any other $\psi':Y\to Z$ satisfying $\psi'\varphi=0$ factors (not necessarily uniquely) through $\psi$. There an obvious dual notion of \emph{pseudokernel}.

Recall that a full subcategory $\T$ of $\mathscr{C}$ is \emph{coreflective} (resp. \emph{reflective}) if the inclusion functor $i:\T\to\mathscr{C}$ has a right (resp. left) adjoint $t$.
\begin{definition}\label{5:def:torsion_pair}
  A pair $\t=(\T,\F)$ of classes of objects of $\mathscr{C}$ is a \emph{torsion pair} if:
  \begin{enumerate}
    \item $\F=\T\rhperp$ and $\T = \lhperp{\F}$,
    \item given $X\in \mathscr{C}$ there is a pseudokernel-pseudocokernel sequence
      \begin{equation*}
        T_X\nto{\varepsilon_X} X\nto{\lambda_X} F^X
      \end{equation*}
      where $T_X\in\T$ and $F^X\in\F$.
  \end{enumerate}
  Moreover, a torsion pair $\t=(\T,\F)$ is called \emph{left} (resp. \emph{right}) \emph{functorial} if $\T$ (resp. $\F$) is a coreflective
  \footnote{Recall that a full subcategory $\T$ of $\mathscr{C}$ is called coreflective if the inclusion functor $i\colon\T\to \mathscr{C}$ has a right adjoint $t$.}
  (resp. reflective) subcategory of $\mathscr{C}$. If $\t$ is both left and right functorial it is called just \emph{functorial}.
\end{definition}

\begin{rmk}
  Our definition of a torsion pair in additive categories requires the existence of a pseudokernel-pseudocokernel sequence, while the usual definition of a torsion pair in abelian categories requires a short exact sequence instead. However, as we'll show in the following lemma, the two definitions actually coincide in abelian categories.
\end{rmk}

\begin{lemma}
  Let $\mathscr{C}$ be an abelian category and $(\T,\F)$ be a torsion pair according to Definition \ref{5:def:torsion_pair}. Then, for any $X\in\mathscr{C}$ there is a short exact sequence
  \begin{equation*}
    \begin{tikzcd}
      0\arrow{r}&T_X\arrow{r}&X\arrow{r}&F^X\arrow{r}&0
    \end{tikzcd}
  \end{equation*}
  with $T_X\in\T$ and $F^X\in\F$.
\end{lemma}

\begin{proof}
  Consider a pseudokernel-pseudocokernel sequence $T_X\nto{\varepsilon_X} X\nto{\lambda^X} F^X$ with $T_X\in \T$ and $F^X\in\F$, and let $K=\Ker(\lambda^X)$. Then, we have the following diagram:
  \begin{equation*}
    \begin{tikzcd}[column sep=large,row sep=large]
      & K\arrow[tail]{d}{k}\arrow[dashed,shift left]{dl}{v}
        &\\
      T_X\arrow{r}{\varepsilon_X}\arrow[dashed,shift left]{ur}{u}
      & X\arrow{r}{\lambda^X}
        & F^X
    \end{tikzcd}
  \end{equation*}
  where $u$ and $v$ are given by the (pseudo)kernel properties of $K$ and $T_X$ respectively, i.e. $k\circ u=\varepsilon_X$ and $\varepsilon_X\circ v = k$. Combining the two we get that $k\circ u\circ v = \varepsilon_X\circ v = k$, so $u\circ v=1_K$ (since $k$ is mono). Hence, $u$ is a section and $K {< \atop \oplus} T_X$.
  % NOTE: can also use $\underset{\oplus}{<}$

  This implies that $\Hom_\mathscr{C}(K,F)=0$ for all $F\in\F$, i.e. $K\in\T$. Moreover, $\lambda^X$ is a pseudocokernel of $k$. Thus, for any $X\in\mathscr{C}$ there is a short exact sequence $0\to K\nto{k} X\nto{\lambda^X} F^X$ with $K\in\T$ and $F^X\in\F$ such that $\lambda^X$ is a pseudocokernel of $k$.

  To conclude the proof, consider the cokernel of $k$ and use a dual argument to get the desired short exact sequence.
\end{proof}


It is a classical result that torsion pairs in Abelian categories are automatically functorial. Similarly, if $(\T^{\leq 0},\T^{\geq 0})$ is a $t$-structure in a triangulated category, $(\T^{\leq 0}[1],\T^{\geq 0})$ is an example of functorial torsion pair.

The following lemma gives basic properties of left functorial torsion pairs. The proof is omitted, since it is a straightforward application of the definitions. The dual obiously hold for right functorial torsion pairs.

\begin{lemma}\label{rmk:1.1}
  Let $\t=(\T,\F)$ be a left functorial torsion pair in $\mathscr{C}$. Then
\begin{enumerate}[label=(\alph*)]
\item\label{rmk:1.1a} for any $M\in\mathscr{C}$, $T'\in\T$ and $\alpha\in\Hom_\mathscr{C}(T',M)$ there is a unique $\alpha'\in\Hom_\mathscr{C}(T',t(M))$ such that $\varepsilon_M\circ\alpha'=\alpha$;

\item\label{rmk:1.1b} for a morphis $g:T_1\to T_2$ in $\T$, any pseudocokernel $g^C:T_2\to C$ in $\T$ is also a pseudocokernel of $g$ in $\mathscr{C}$.
\end{enumerate}
\end{lemma}

\section{Nakaoka contexts in additive categories}
\todo{Add intro}
\subsection{Nakaoka contexts and the heart construction}

\begin{definition}
  A \emph{Nakaoka context} is a pair $\t=(\t_1,\t_2)$ of torsion pairs in $\mathscr{C}$, satisfying the following axioms:
  \begin{torsionaxioms}
    \item\label{ax:ct1} $\t_1=(\T_1,\F_1)$ and $\t_2=(\T_2,\F_2)$ are respectively a left functorial and a right functorial torsion pair;
    \item\label{ax:ct2} $\T_2\subseteq \T_1$ (equiv. $\F_1\subseteq\F_2$).
  \end{torsionaxioms}
\end {definition}

\begin{notation}
  If $\t=(\t_1,\t_2)$ is a Nakaoka context, we will always use the notation $\t_i = (\T_i,\F_i)$ for $i=1,2$ and the coreflection and reflection will be indicated as:

\begin{equation*}
\begin{tikzcd}
(i_1,t_1)\colon\T_1\arrow[hook, shift left]{r}{i_1}
& \mathscr{C}\arrow[shift left]{l}{t_1}
& \mathrm{and}
& (f_2,j_2)\colon\F_2\arrow[hook, shift right]{r}[']{j_2}
& \mathscr{C}\arrow[shift right]{l}[']{f_2}.
\end{tikzcd}
\end{equation*}

The counit of the adjunction $(i_1,t_1)$ will be denoted by $\varepsilon_1\colon t_1i_1\to \id_{\T_1}$, while the unit of $(f_2,j_2)$ will be denoted by $\lambda_2\colon \id_{\F_2}\to f_2j_2$.
\end{notation}

\begin{definition}
  The \emph{heart} of a Nakaoka context $\t=(\t_1,\t_2)$ is $\clH=\clH_\t:=\T_1\cap\F_2$.
\end{definition}

In the following lemma we explicitly state two general observations about Nakaoka contexts.

\begin{lemma}\label{5:lem:2}
Let $\t=(\t_1,\t_2)$ be a Nakaoka context. Then, the followings hold true:
\begin{enumerate}[label=(\alph*)]
\item $\F_1\cap\T_2=0$;
\item $f_2(\T_1)\subseteq\clH$ and $t_1(\F_2)\subseteq\clH$.
\end{enumerate}
\end{lemma}
\begin{proof}
(a) Since $\T_2\subseteq\T_1$, we have $\F_1\cap\T_2\subseteq \F_1\cap\T_1=0$.

\smallskip\noindent
(b) Let $T_2\in\T_2$ and $F_2\in\F_2$. Then, $\Hom_\mathscr{C}(T_2,t_1(F_2))\cong \Hom_\mathscr{C}(T_2,F_2)=0$.
Hence, $t_1(F_2)\in\T_2^\perp = \F_2$. An analogous proof shows that $f_2(\T_1)\subseteq\T_1$.
\end{proof}

In the following lemma we introduce a technical condition under which we can easily construct kernels of morphisms in the heart of a given Nakaoka context.

\begin{lemma}\label{5:lem:3}
Let $\t=(\t_1,\t_2)$ be a Nakaoka context and let $f\colon H\to H'$ be a morphism in the heart $\clH=\clH_\t$. If there is a morphism $f^K\colon K\to H$, with $K\in \F_2$, such that the following sequence is exact in $\Func(\T_1, \Ab)$

\begin{equation*}
(*)\qquad 0\to (-,K)\restriction_{\T_1}\to (-,H)\restriction_{\T_1}\to (-,H')\restriction_{\T_1},
\end{equation*}
then the composition $f^K\circ \varepsilon_{1,K}\colon t_1K\to K\to H$ is a kernel for $f$ in $\clH$.
\end{lemma}
\begin{proof}
Consider the exact sequence in $(*)$ and notice that it gives, by restriction of the functors, an exact sequence of the form:
\begin{equation*}
0\to (-,K)\restriction_{\clH}\to (-,H)\restriction_{\clH}\to (-,H')\restriction_{\clH}.
\end{equation*}
The map $\varepsilon_{1,K}\colon t_1K\to K$, induces a natural isomorphism $(-,t_1K)\restriction_\clH \to (-,K)\restriction_\clH$, so we get an exact sequence
\begin{equation*}
\begin{tikzcd}
  0\arrow{r}
  & (-,t_1K)\restriction_{\clH}\arrow{r}{k\circ \varepsilon_{1,K}\circ -}
  & (-,H)\restriction_{\clH}\arrow{r}
  & (-,H')\restriction_{\clH}.
\end{tikzcd}
\end{equation*}
To conclude one notices that, since $t_1K\in\clH$ by Lemma \ref{5:lem:2}, the above exact sequence means exactly that $f^K\circ \varepsilon_{1,K}$ is a kernel of $f$.
\end{proof}


\subsection{Pre-abelian Nakaoka contexts}
Recall that an additive category is  \emph{pre-Abelian} if any morphism has a kernel and a cokernel. In view of Lemma~\ref{5:lem:3}, it is natural to introduce the following definition:\todo{Clarify wrt cot pair notation}

\begin{definition}
A Nakaoka context is said to be \emph{pre-Abelian} if it satisfies the following axioms:
\begin{torsionaxioms}\setcounter{enumi}{2}
\item\label{ax:ct3} any  $g\colon H\to H'$ in $\clH(\subseteq \T_1)$ admits a pseudocokernel $g^C\colon H'\to C$ in $\T_1$, such that
\begin{equation*}
\begin{tikzcd}
0\arrow{r} &(C,-)_{|\F_2}\arrow{r}{(g^C,-)} & (H',-)_{|\F_2}\arrow{r}{(g,-)}& (H,-)_{|\F_2}
\end{tikzcd}
\end{equation*}
is an exact sequence in $\Func(\F_2,\Ab)$.
\varitem{^\ast}\label{ax:ct3op} Dual of \ref{ax:ct3}.
\end{torsionaxioms}
\end{definition}

\begin{thm}\label{pre_abelian_theorem}
For a pre-Abelian Nakaoka context $\t=(\t_1,\t_2)$, the heart $\clH=\clH_\t$ is a pre-Abelian category.
\end{thm}
\begin{proof}
This is a consequence of the axioms, Lemma \ref{5:lem:3} and its dual.
\end{proof}

In the following proposition we give a characterization of those morphisms that are monomorphisms in the heart. For this, remember that, in a pre-Abelian category, a morphism is mono if and only if its kernel is trivial.

\begin{prop}\label{prop:1.4}
  The following are equivalent for a morphism $f\colon H\to H'$ in the heart $\clH=\clH_\t$ of a pre-Abelian Nakaoka context $\t=(\t_1,\t_2)$:
  \begin{enumerate}[label=(\alph*)]
    \item $f$ is a monomorphism (in $\clH$);
    \item\label{prop:1.4:b} there is a pseudokernel $f^K\colon K\to H$ of $f$ in $\F_2$ such that $K\in \F_1$.
  \end{enumerate}
\end{prop}
\begin{proof}
  For any morphism $f\colon H\to H'$ in $\clH$, by the axiom (CT.$3^*$), we can consider a diagram as follows
  \begin{equation*}
    \begin{tikzcd}
      F_2\arrow{r}{f^K} & H\arrow{r}{f} & H'\\
      t_1{F_2}\arrow{u}{\varepsilon_{1,F_2}}\arrow{ur}[']{\tilde{f}^K}
    \end{tikzcd}
  \end{equation*}
  where $F_2\in \F_2$ is a pseudo-kernel of $f$ in $\F_2$ and, by Lemma \ref{5:lem:3}, $t_1F_2\to H$ is the kernel of $f$ in $\clH$.

  \smallskip\noindent
  (a)$\Rightarrow$(b). Since $f$ is a monomorphism in $\clH$, its kernel is trivial, that is, $t_1F_2=0$, i.e. $F_2\in \F_1$.

  \smallskip\noindent
  (b)$\Rightarrow$(a). If  $f^K\colon K\to H$ is a pseudokernel of $f$ in $\F_2$ such that $K\in \F_1$, then the kernel $0=t_1K\to H$ of $f$ in $\clH$ is trivial, that is, $f$ is a monomorphism.
\end{proof}

\subsection{Integral Nakaoka contexts}

Our final objective is to find conditions that make $\clH_\t$ into an Abelian category. For a pre-Abelian category, being Abelian amounts to saying that every mono is a kernel and every epi is a cokernel. However, we can consider a middle step before abelianity, namely \emph{integra categories}.

\begin{definition}
  A pre-Abelian category is called \emph{integral} if pullbacks of epis are epi and pushout of monos are mono.
\end{definition}

An integral category is Abelian if any morphism that is both epi and mono is an isomorphism. In this section and the next we will introduce axioms that guarantee the integrality and abelianity, respectively, of $\clH$.

\begin{definition}
A pre-Abelian Nakaoka context $\t=(\t_1,\t_2)$ is said to be \emph{integral} provided the following axioms hold:
\begin{enumerate}
\item[(CT.4.1)] Given two morphisms $f\colon H\to H'$ and $g\colon K\to H'$ in $\clH$ such that $f$ has a pseudo-cokernel $f^C\colon H'\to C$ with $C\in \T_2$, we can construct a commutative diagram
\[
\xymatrix{
X\ar@{.>}[d]_{x_H}\ar@{.>}[r]^{x_K}&K\ar[r]^{x_K^C}\ar[d]^{g}&C'\\
H\ar[r]_f&H'\ar[r]_{f^C}&C
}
\]
where $x_K^C\colon K\to C$ is a pseudo-cokernel of $x_K$, $X\in \clH$ and $C'\in \T_2$.
\item[(CT.4.1$^*$)]  Dual to (CT.4.1).
\end{enumerate}
\end{definition}

\begin{thm}\label{5:thm:integral}
The following are equivalent for a pre-Abelian Nakaoka context $\t=(\t_1,\t_2)$:
\begin{enumerate}[label=(\alph*)]
\item $\t$ is integral;
\item the heart $\clH=\clH_\t$ is an integral pre-Abelian category.
\end{enumerate}
\end{thm}
\begin{proof}
(a)$\Rightarrow$(b). Consider a pullback diagram in $\clH$:
\[
\xymatrix{
A\ar@{}[dr]|{PB}\ar[r]^{f'}\ar[d]_{g'}&B\ar[d]^g\\
C\ar[r]_{f}&D
}
\]
and suppose that $f$ is an epi in $\clH$. By (the dual of) Proposition \ref{prop:1.4}, we can find a pseudo-cokernel $f^C\colon D\to T_2$ of $f$, with $T_2\in \T_2$. By the axiom (CT.4.1), we obtain a commutative diagram as follows
\[
\xymatrix{
X\ar@{.>}[d]_{x_C}\ar@{.>}[r]^{x_B}&B\ar[r]^{f''}\ar[d]^{g}&T'_2\\
C\ar[r]_f&D\ar[r]_{f^C}&T_2
}
\]
where $f''\colon B\to T_2$ is a pseudo-cokernel of $x_B$, and $X\in \clH$. Since the square we started with is a pullback, there is a unique morphism $\alpha\colon X\to A$, making the following diagram commute:
\[
\xymatrix{
X\ar@/_15pt/[ddr]_{x_C}\ar@/_-15pt/[rrd]^{x_B}\ar@{.>}[dr]|\alpha\\
&A\ar@{}[dr]|{PB}\ar[r]^{f'}\ar[d]_{g'}&B\ar[d]^g\\
&C\ar[r]_{f}&D
}
\]
Now, since $x_B$ is a morphism in $\clH$ that admits a pseudo-cokernel (in $\T_1$) which belongs to $\T_2$, $x_B$ is an epimorphism in $\clH$ by Proposition \ref{prop:1.4}. Furthermore, since $x_B=f'\circ\alpha$, also $f'$ is an epi. One can prove dually that pushouts of monomorphisms are mono, so $\clH$ is integral.

\smallskip\noindent
(b)$\Rightarrow$(a). Suppose that $\clH$ is an integral category and let us show how this implies the axiom (CT.4.1), one can verify (CT.4.1$^*$) dually. Given two morphisms $f\colon H\to H'$ and $g\colon K\to H'$ in $\clH$ such that $f$ has a pseudo-cokernel $f^C\colon H'\to C$ with $C\in \T_2$, we can take the following pullback diagram in $\clH$
\[
\xymatrix{
X\ar[d]_{x_H}\ar[r]^{x_K}\ar@{}[dr]|{PB}&K\ar[d]^{g}\\
H\ar[r]_f&H'
}
\]
Now, by the integrality of $\clH$, $x_K$ is an epimorphism in $\clH$ and so, by Proposition \ref{prop:1.4}, it admits a pseudo-cockernel $K\to C'$ (in $\T_1$) such that $C'\in \T_1$.
\end{proof}

\subsection{Abelian Nakaoka contexts}


\begin{definition}
An integral Nakaoka context $\t=(\t_1,\t_2)$ in $\mathscr{C}$ is said to be \emph{Abelian} provided the following axioms hold:
\begin{enumerate}
\item[(CT.4.2)] A morphism $f\colon H\to H'$ in $\clH$ is an isomorphism if and only if it fits in a sequence:
\[
\xymatrix{
K\ar[r]^{f^K}&H\ar[r]^f&H'\ar[r]^{f^C}&C
}
\]
where $f^K$ is a pseudo-kernel of $f$ in $\F_2$, $f^C$ is a pseudo-cokernel of $f$ in $\T_1$, $K\in \F_1$ and $C\in \T_2$;
\item[(CT.4.2$^*$)] dual to (CT.4.2).
\end{enumerate}
\end{definition}


\begin{thm}\label{thm:2.6}
The following are equivalent for a pre-Abelian Nakaoka context $\t=(\t_1,\t_2)$:
\begin{enumerate}[label=(\alph*)]
\item the heart $\clH=\clH_\t$ is an Abelian category;
\item $\t$ is Abelian (that is, it satisfies (CT.4.1), (CT.4.2) and their duals);
\item $\t$ satisfies the following axioms:
  \begin{torsionaxioms}\setcounter{enumii}{3}
  \item\label{ax:ct4} given a morphism $f\colon H\to H'$ in $\clH$ that admits a pseudo-kernel $f^K\colon H''\to H$ in $\F_2$, such that $H''\in \F_1$, and the commutative diagram
\begin{equation*}
\xymatrix{
 & H\ar[d]^{a}\ar[r]^{f} & H'\ar@{=}[d]\ar[r]^{f^C} & T_1\ar[d]^{\lambda_{2,T_1}}\\
t_1F_2\ar@{.>}[ur]^{b}\ar[r]^{\varepsilon_{1,F_2}} & F_2\ar[r]^{g^K} & H'\ar[r]^{g} & f_2T_1
}
\end{equation*}
where $f^C$  is a pseudo-cokernel of $f$ in $\T_1$ and $g^K$ is a pseudo-kernel of $g$ in $\F_2$, there exists a morphism $b\colon t_1F_2\to H_1$ such that $ab=\varepsilon_{1,F_2}$;
\setcounter{enumi}{4}
\varitem{\ast}\label{ax:ct4op} dual to (CT.4).
\end{torsionaxioms}
\end{enumerate}
\end{thm}
\begin{proof}
  (a)$\Leftrightarrow$(b). This is almost tautological since, by Theorem \ref{5:thm:integral}, (CT.4.1) is equivalent to saying that $\clH$ is integral while, by Proposition \ref{prop:1.4}, (CT.4.2) is equivalent to saying that, in $\clH$, morphisms that are both epi and mono are isomorphisms; to conclude it is enough to notice that Abelian categories are exactly those pre-Abelian categories which are integral and where morphisms that are both epi and mono are isomorphisms.


\smallskip\noindent
(a)$\Rightarrow$(c).
Let $f\colon H\to H'$ be a morphism in $\clH$ that admits a pseudo-kernel $f^K\colon K\to H$ in $\F_2$, such that $K\in \F_1$, and the commutative diagram
\begin{equation*}
\xymatrix{
 & H\ar@{.>}[dl]_{\beta}\ar[d]^{a}\ar[r]^{f} & H'\ar@{=}[d]\ar[r]^{f^C} & T_1\ar[d]^{\lambda_{2,T_1}}\\
t_1F_2\ar[r]^{\varepsilon_{1,F_2}} & F_2\ar[r]^{g^K} & H'\ar[r]^{g} & f_2T_1
}
\end{equation*}
where $f^C$  is a pseudo-cokernel of $f$ in $\T_1$ and $g^K$ is a pseudo-kernel of $g$ in $\F_2$. Since $f$ is a monomorphism in $\clH$ by Proposition \ref{prop:1.4} (a), and $\clH$ is Abelian, we have that $f=\Ker_\clH \Coker_\clH (f)$. On the other hand, by Lemma \ref{rmk:1.1} (a), there is $\beta\colon H_1\to t_1F_2$ such that $\varepsilon_{1,F_2}\beta = a$ completing the diagram above. Since $f=\Ker \Coker(f)$, it follows that $\beta$ is an isomorphism and (CT.4) follows by setting $b:=\beta^{-1}$. (CT.4$^*$) can be verified by a dual argument.



\smallskip\noindent
(c)$\Rightarrow$(a). By Theorem \ref{pre_abelian_theorem}, $\clH$ is pre-Abelian. To prove that $\clH$ is Abelian, we just need to show that any monomorphism (resp. epimorphism) is a kernel (resp. cokernel).
Hence, let $f\colon H\to H'$ be a monomorphism in $\clH$; by Proposition~\ref{prop:1.4}\todo{Check this reference, the original refers to a lemma}
we know that $f$ admits a pseudo-kernel in $\F_2$ that belongs in $\F_1$ and thus, by (CT.4), we get a commutative diagram as follows
\begin{equation*}
\begin{tikzcd}
 & H\arrow{r}{f}\arrow{d}{a} & H'\arrow{r}{f^C}\arrow[equal]{d} & T_1\arrow{d}{\lambda_{2,T_1}}\\
t_1F_2\arrow{ur}{b}\arrow{r}{\varepsilon_{1,F_2}} & F_2\arrow{r}{g^K} & H'\arrow{r}{g} & f_2T_1
\end{tikzcd}
\end{equation*}
where $f^C$  is a pseudo-cokernel of $f$ in $\T_1$ and $g^K$ is a pseudo-kernel of $g$ in $\F_2$. Let $\alpha\colon H\to H'$ be a morphism such that $g\alpha =0$. Since $F_2$ is a pseudo-kernel of $g$, there is a morphism $\alpha'\colon H\to F_2$ such that $g^K\alpha'=\alpha$. By Lemma \ref{rmk:1.1}, $\alpha'$ factors through  $\varepsilon_{1,F_2}$, as in the following diagram:
\begin{equation*}
\begin{tikzcd}
& H_1\arrow{r}{f}\arrow{d}{a} & H_2\arrow{r}{f^C}\arrow[equal]{d}& T_1\arrow{d}{\lambda_{2,T_1}}\\
t_1F_2\arrow{ur}{b}\arrow{r}{\varepsilon_{1,F_2}} & F_2\arrow{r}{g^K} & H_2\arrow{r}{g} & f_2T_1\\
& H.\arrow{ur}{\alpha}\arrow[dotted]{u}[description]{\alpha'}\arrow[dotted]{ul}[description]{\alpha''} & &
\end{tikzcd}
\end{equation*}
By setting $\alpha''' := b\alpha''\colon H\to H$, we get $f\alpha''' = g^K ab\alpha'' = g^K\varepsilon_{1,F_2}\alpha''=g^K\alpha' = \alpha$. Thus, any morphism $\alpha\colon H \to H'$ such that $g\alpha =0$ factors through $f$.
\end{proof}



\clearpage














\section{Nakaoka contexts in special categories}
\subsection{Nakaoka contexts in Abelian categories}

Let's consider the case $\mathcal{X}=\mathcal{A}$ of an Abelian category with
two torsion pairs $\mathbb{t}_i=(\mathcal{T}_i,\mathcal{F}_i)$ for $i=1,2$.
Consider $\mathbb{t}=(\mathbb{t}_1,\mathbb{t}_2)$.

\begin{lemma}\label{rmk:2.2}
For an Abelian category $\A$, any Nakaoka context is integral.
\end{lemma}
\begin{proof}
  Let $\mathcal{T}_2\subseteq\mathcal{T}_1$, we need to show that \cref{ax:ct1},
  \cref{ax:ct3} and \cref{ax:ct3op} hold.
  \begin{torsionaxioms}
    \item It is well known that any torsion pair in an abelian category is functorial.
    \setcounter{enumi}{2}
    \item Let $g:T_1\to T_1'$ be a morphism in $\mathcal{T}_1$. Consider the cokernel morphism
    of $g$ in $\mathcal{A}$
    \begin{equation*}
      \Coker_\mathcal{A}(T_1\nto{g}T_1')=(T_1'\nto{c_g}\Coker(g)).
    \end{equation*}
    Since $\mathcal{T}_1$ is closed under quotient objects, we get that $\Coker(g)\in\mathcal{T}_1$.
    Therefore, we can choose $c_g:T_1'\to\Coker(g)$ as $g^C:T_1'\to\PCok_\mathcal{A}(g)$.\todo{Is there an explicit choice
    that we made somewhere before when we talk about $\PCok_\mathcal{A}$?}
    \varitem{^\ast} Anologous to the previous.
  \end{torsionaxioms}

\bigskip\bigskip
Let us now verify the axiom (CT.4.1). Indeed, consider two morphisms $f\colon H\to H'$ and $g\colon K\to H'$ in $\mathcal{H}$ such that $f$ has a pseudo-cokernel $f^C\colon H'\to C$ with $C\in \T_2$ and consider the following pullback diagram in $\A$
\[
\xymatrix{
X\ar@{}[dr]|{P.B.}\ar@{.>}[d]_{x_H}\ar@{.>}[r]^{x_K}&K\ar[r]^{x_K^C}\ar[d]|{g}&C\ar@{=}[d]\\
H\ar[r]_f&H'\ar[r]_{f^C}&C
}
\]
where $x_K^C\colon K\to C$ is a pseudo-cokernel of $x_K$. Clearly $X\leq H\oplus K$, so that $X\in \F_2$.
\[
\xymatrix{
t_1X\ar@{.>}[rrd]\ar@{.>}[ddr]\\
&X\ar@{}[dr]|{P.B.}\ar@{.>}[d]_{x_H}\ar@{.>}[r]^{x_K}&K\ar[r]^{x_K^C}\ar[d]|{g}&C\ar@{=}[d]\\
&H\ar[r]_f&H'\ar[r]_{f^C}&C
}
\]
\end{proof}

\begin{corollary}\label{cor:2.3}
  Let $\mathbb{t}=(\mathbb{t}_1,\mathbb{t}_2)$ be a torsion pair in $\mathcal{A}$ with $\mathcal{T}_2\subseteq\mathcal{T}_1$.
  Then, for $f:H_1\to H_2$ in $\mathcal{H}$, the following statements hold:
  \begin{enumerate}[label=(\alph*),ref=(\alph*)]
    \item\label{cor:2.3:a} the cokernel of $f$ in $\mathcal{H}$ is the composition of the morphisms
      \begin{equation*}
        \begin{tikzcd}
          H_2\arrow{r}{c_f}&\Coker(f)\arrow{r}{\lambda_{2,\Coker(f)}}&f_2(\Coker(f));
        \end{tikzcd}
      \end{equation*}
    \item\label{cor:2.3:b} the kernel of $f$ in $\mathcal{H}$ is the composition of the morphisms
      \begin{equation*}
        \begin{tikzcd}
          t_1(\Ker(f))\arrow{r}{\varepsilon_{1,\Ker(f)}}&\Ker(f)\arrow{r}{k_f}&H_1;
        \end{tikzcd}
      \end{equation*}
    \item\label{cor:2.3:c} $f$ is an epimorphism in $\mathcal{H}$ if and only if $\Coker(f)\in\mathcal{T}_2$;
    \item\label{cor:2.3:d} $f$ is a monomorphism in $\mathcal{H}$ if and only if $\Ker(f)\in\mathcal{F}_1$.
  \end{enumerate}
\end{corollary}

\begin{proof}
  \ref{cor:2.3:a} and \ref{cor:2.3:b} follow from the proof of \cref{rmk:2.2}\todo{this should be a remark, fix it}
  and \cref{prop:1.3}.

  \ref{cor:2.3:c}
  \begin{enumerate}
    \item[$\Leftarrow$] is trivial.
    \item[$\Rightarrow$] By \cref{prop:1.4} \ref{prop:1.4:b} there exists $f^C:H_2\to T_2$, where
      $T_2=\Coker_{T_1}(f)\in\mathcal{T}_2$. Then, we have

      \begin{minipage}[b]{0.45\linewidth}
        \begin{equation*}
          \begin{tikzcd}[column sep=tiny]
            H_2\arrow{rr}{f^C}\arrow{dr}[']{c_f}
              & & T_2 \arrow[dashed, shift right]{ld}[']{u}\\
              & \Coker(f)\arrow[dashed, shift right]{ur}[']{v}
          \end{tikzcd}
        \end{equation*}
      \end{minipage}
      \begin{minipage}[b]{0.45\linewidth}
        \begin{equation*}
          \text{such that }
          \left\{
          \begin{array}{c}
            uf^C = c_f,\\
            vc_f = f^C.
          \end{array}
          \right.
        \end{equation*}
      \end{minipage}

      Hence, $uvc_f=c_f$, but $c_f$ is epi, therefore $uv=1$. Hence, $\Coker(f)$ is a direct
      summand of $T_2\in\mathcal{T}_2$ so $\Coker(f)\in\mathcal{T}_2$.
  \end{enumerate}

  \ref{cor:2.3:d} Similar to the previous proof.
\end{proof}

\begin{thm}\label{thm_2_4}
  Let $\mathbb{t}_i=(\mathcal{T}_i,\mathcal{F}_i)$ be a torsion pair in an abelian category
  $\mathcal{A}$, for $i=1,2$, such that $\mathcal{T}_2\subseteq \mathcal{T}_1$. Then, for
  $\mathcal{H}:=\mathcal{T}_1\cap\mathcal{F}_2$ the following statements are equivalent:
  \begin{enumerate}[label=(\alph*)]
    \item $\mathcal{H}$ is an abelian category.
    \item The following conditions hold:
      \begin{enumerate}[label=(\alph{enumi}\arabic*)]
        \item For any $f:H\to H'$ in $\mathcal{H}$, with $\Ker(f)\in\mathcal{F}_1$,
        we have that $\Ker(f)=0$.
        \item For any $f:H\to H'$ in $\mathcal{H}$, with $\Coker(f)\in\mathcal{T}_2$,
        we have that $\Coker(f)=0$.
        \item $\mathcal{H}$ is closed under kernels (resp. cokernels) of epimorphisms
        (resp. monomorphisms) in $\mathcal{A}$.
      \end{enumerate}
    \item $\mathcal{H}$ is closed under kernels and cokernels in $\mathcal{A}$.
  \end{enumerate}
\end{thm}

\begin{proof}
  ($(a)\Rightarrow (b1),(b2)$) Assume that $\H$ is an abelian category. By \cref{cor:2.3}\ref{cor:2.3:d}, $(b1)$ holds if and only if any monomorphism in $\H$ is a monomorphism in $\A$.

  Observe that for any $f:H\to H'$, we have that $\Ker_\H(f)\nto{\tilde{f}^H}H$ is a monomorphism in $\A$. Indeed, by \cref{cor:2.3}\ref{cor:2.3:b}, we know that $\tilde{f}^H$ is the composition of
  \begin{equation*}
    \begin{tikzcd}
      t_1(\Ker(f))\arrow{r}{\varepsilon_{1,\Ker(f)}}
        & \Ker(f)\arrow{r}{k_f}
          & H;
    \end{tikzcd}
  \end{equation*}
  and, since $\varepsilon_{1,\Ker(f)}$ and $k_f$ are monomorphism in $\A$, so is $\tilde{f}^H$.

  Furthermore, if $f:H\to H'$ is a monomorphism in $\H$, then $f=\Ker_\H(\Coker_\H(f))$ since $\H$ is abelian. Thus, $f$ is a monomorphism in $\A$.

  A dual argument can be used to prove $(b2)$.

  ($(a)\Rightarrow (b3)$) Let $f:H\to H'$ in $\H$ be a monomorphism in $\A$. We want to prove that $\Coker(f)\in\H$.

  Observe that the diagram
  \begin{equation}\label{thm_2_4_eq_1}
    \begin{tikzcd}
      0\arrow{r}
      & H\arrow{r}{f}\arrow{d}{\alpha}
        & H'\arrow{r}{c_f}\arrow[equal]{d}
          & \Coker(f)\arrow{d}{\lambda_2}\arrow{r}
            & 0\\
      0 \arrow{r}
      & \Ker(g)\arrow{r}{k_g}
        & H' \arrow{r}{g\colon=\lambda_2c_f}
          & f_2(\Coker(f)) \arrow{r}
            & 0
    \end{tikzcd}
  \end{equation}
  is commutative and has exact rows. Then, by Snake's lemma we get that $\Ker(\alpha)=0$ and $\Coker(\alpha)\cong\Ker(\lambda_2)=t_2(\Coker(f))\in\T_2$.

  Since $\H$ is abelian, we know that $f=\Ker_\H(\Coker_\H(f))$. Therefore, by $\cref{cor:2.3}$ and \eqref{thm_2_4_eq_1}, we get that the dashed morphism
  in the following commutative diagram exists
  \begin{equation*}
    \begin{tikzcd}
      & H\arrow{r}{f}\arrow[dashed]{dl}[']{\exists\varphi}
        & H'\arrow[equal]{d}\\
      t_1(\Ker(g))\arrow{r}{\varepsilon_{1,\Ker(g)}}
      & \Ker(g)\arrow{r}{k_g}
        & H'
    \end{tikzcd}
  \end{equation*}
  and it is an isomorphism. But, $k_g\varepsilon_{1,\Ker(g)}\varphi = f= k_g\alpha$ and $k_g$ is mono, hence $\varepsilon_{1,\Ker(g)}\varphi = \alpha$. Therefore,
  \begin{equation*}
    \Coker(\alpha)\cong \Coker(\varepsilon_{1,\Ker(g)})=\frac{\Ker(g)}{t_1(\Ker(g))}= f_1(\Ker(g))\in\F_1.
  \end{equation*}

  Thus, $\Coker(\alpha)\in\T_2\cap\F_1=0$ and so $\alpha$ is an isomorphism. By \eqref{thm_2_4_eq_1} is follows that $\Coker(f)\cong f_2(\Coker(f))\in\F_2$ and, using the fact that $T_1$ is closed under quotients, we conclude that $\Coker(f)\in\T_1\cap\F_2=\H$. A dual argument shows that $\H$ is closed under kernels of epimorphism in $\A$.

  ($(b)\Rightarrow (a)$) We already know that $\H$ is preabelian. On order to prove that $\H$ is abelian we need to show that any monomorphism (resp. epimorphism) in $\H$ is a kernel (resp. cokernel).

  Let $f:H\to H'$ be a monomorphism in $\H$. By \cref{cor:2.3} and $(b1)$, it follows that $f$ is a monomorphism in $\A$. Thus, we have a short exact sequence in $\A$
  \begin{equation*}
    \begin{tikzcd}
      0 \arrow{r}
      & H\arrow{r}{f}
        & H'\arrow{r}{c_f}
          & \Coker(f) \arrow{r}
            & 0.
    \end{tikzcd}
  \end{equation*}

  Furthermore, from $(b3)$ we know that $\Coker(f)\in\H$. Then, $\Coker(f)=\Coker_\H(f)$ and $\Ker(c_f)=\Ker_\H(c_f)$. Therefore, $f=\Ker(\Coker(f))=\Ker_\H(\Coker_\H(f))$.

  ($(a),(b)\Rightarrow (c)$) Let $f:H\to H'$ be a morphism in $\H$. We want to show that $\Ker(f)\in\H$. Consider the exact sequence
  \begin{equation*}
    \begin{tikzcd}
      0 \arrow{r}
      & \Ker(f)\arrow{r}
        & H\arrow{r}
          & \Img(f) \arrow{r}
            & 0,
    \end{tikzcd}
  \end{equation*}
  by $(b3)$ it is enough to show that $\Img(f)\in\H$. Since $\H$ is abelian, we have the canonical factorization of $f$ in $\H$
  \begin{equation*}
    \begin{tikzcd}
      H\arrow{rr}{f}\arrow[two heads]{rd}[']{f'}
      & & H'\\
      & \Img_\H(f)\arrow[hook]{ur}[']{f''}
        &
    \end{tikzcd}
  \end{equation*}
  where $f'$ is an epi and $f''$ is a mono (in $\H$). Then, by $(b1)$ and $(b2)$ we have that $f'$ and $f''$ are respectively an epi and a mono in $\A$. Therefore, $\Img(f)\cong \Img_\H(f)\in\H$, and by the exact sequence
  \begin{equation*}
    \begin{tikzcd}
      0 \arrow{r}
      & \Img(f)\arrow{r}
        & H'\arrow{r}
          & \Coker(f) \arrow{r}
            & 0,
    \end{tikzcd}
  \end{equation*}
  and $(b3)$ we conclude that $\Coker(f)\in\H$.

  $(c)\Rightarrow (a)$ is clear.

\end{proof}

\subsection{Nakaoka contexts in triangulated categories}

Let $\mathcal{X}=\mathcal{T}$ be a triangulated category on which idempotents split.
We start by recalling the definition of a t-structure in $\mathcal{T}$.

\begin{definition}
  A pair $(\mathcal{A},\mathcal{B})$ of full subcategories of $\mathcal{T}$ is a t-structure
  in $\mathcal{T}$ if
  \begin{enumerate}[label=(\alph*)]
    \item $\mathbb{t}=(\mathcal{A},\mathcal{B}[-1])$ is a torsion pair in $\mathcal{T}$, and
    \item $\mathcal{A}[1]\subseteq \mathcal{A}$.
  \end{enumerate}
\end{definition}

\begin{rmk}
  It is well known that any t-structure $(\mathcal{A},\mathcal{B})$ in $\mathcal{T}$
  gives a functional torsion pair $\mathbb{t}=(\mathcal{A},\mathcal[-1])$ and
  $\mathbb{B}[-1]\subseteq\mathcal{B}$. Furthermore, $\mathcal{A}$ and
  $\mathcal{B}$ are closed under extensions and direct summands. Note that the t-structure
  $(\mathcal{A},\mathcal{B})$ depends only on $\mathcal{A}$, since $\mathcal{B}=\mathcal{A}^\perp[1]$.
\end{rmk}

\begin{definition}
  A \emph{related} torsion pair $\mathbb{t}=(\mathbb{t}_1,\mathbb{t}_2)$ in triangulated category
  $\mathcal{T}$ consists of the torsion pairs $\mathbb{t}_1=(\mathcal{T}_1,\mathcal{F}_1)$
  and $\mathbb{t}=(\mathcal{T}_2,\mathcal{F}_2)$ in $\mathcal{T}$ such that
  $\mathcal{T}_1[1]\subseteq \mathcal{T}_2\subseteq\mathcal{T}_1$.
\end{definition}

\begin{prop}
  Let $\mathbb{t}=(\mathbb{t}_1,\mathbb{t}_2)$ be a related torsion pair in $\mathcal{T}$. Then
  \begin{enumerate}[label=(\alph*)]
    \item $(\mathcal{T}_1,\mathcal{F}_1[1])$ and $(\mathcal{T}_2,\mathcal{F}_2[1])$
    are t-structures in $\mathcal{T}$;
    \item $\mathbb{t}$ is a compatible torsion pair in $\mathcal{T}$;
    \item the heart $\mathcal{H}_\mathbb{t} := \mathcal{T}_1\cap\mathcal{F}_2[1]$ is a preabelian category.
  \end{enumerate}
\end{prop}

\begin{definition}
  A related torsion pair $\mathbb{t}=(\mathbb{t}_1,\mathbb{t}_2)$ in the triangulated category
  $\mathcal{T}$ is \emph{strong} if for any morphism $f:H_1\to H_2$, in $\mathcal{H}:=\mathcal{T}_1\cap\mathcal{F}_2$,
  and a distinguished triangle $Z\to H_1\nto{f}H_2\to Z[1]$, the following conditions
  hold true
  \begin{relatedtorsion}
    \item $Z\in\mathcal{F}_1$ if and only if $Z\in\mathcal{F}_2[-1]$;
    \item $Z[1]\in\mathcal{T}_2$ if and only if $Z\in\mathcal{T}_1$.
  \end{relatedtorsion}
\end{definition}

\begin{thm}\label{thm:2.6}
  Let $\mathbb{t}=(\mathbb{t}_1,\mathbb{t}_2)$ be a strongly related torsion pair in
  the triangulated category $\mathcal{T}$. Then, the heart $\mathcal{H}=\mathcal{H}_\mathbb{t}$
  is an abelian category.
\end{thm}

\begin{example}
  Let $(\mathcal{A},\mathcal{B})$ be a t-structure in $\mathcal{T}$. Consider
  $\mathbb{t}_1:=(\mathcal{A},\mathcal{B}[-1])$ and $\mathbb{t}_2 :=(\mathcal{A}[1],\mathcal{B})$.
  It is not hard to see that $\mathbb{t}=(\mathbb{t}_1,\mathbb{t}_2)$ is a strongly related
  torsion pair in $\mathcal{T}$. In this case, by \cref{thm:2.6}, we get that $\mathcal{H}=\mathcal{A}\cap\mathcal{B}$
  is an abelian category (BBD theorem).
\end{example}

\begin{example}\label{example_tria}
  Let $\mathcal{D}$ be a triangulated category with two $t$-structures
  $(\mathcal{U}_1,\mathcal{U}_1^\perp)$ and $(\mathcal{U}_2,\mathcal{U}_2^\perp)$ such that
  $\mathcal{U}_1[1]\subseteq\mathcal{U}_2\subseteq\mathcal{U}_1$. Observe that for any
  morphism $f$ in $\mathcal{D}$, the cone $\mathrm{Cone}(f)$ of $f$ in $\mathcal{D}$ is
  always a pseudocokernel $f$, and similarly the shifted cone $\mathrm{Cone}(f)[-1]$ is always
  a pseudokernel of $f$.

  If, moreover, $\mathrm{Cone}(f)\in\mathcal{U}_1$ for any morphism
  $f$ in $\mathcal{U}_1$ and $\mathrm{Cone}(f)[-1]\in\mathcal{U}_2^\perp$, then
  $\mathbb{t}=((\mathcal{U}_1,\mathcal{U}_1^\perp),(\mathcal{U}_2,\mathcal{U}_2^\perp))$ satisfy
  axioms \ref{ax:ct1}-\ref{ax:ct3},\ref{ax:ct3op}, hence $\mathcal{H}=\mathcal{U}_1\cap\mathcal{U}_2^\perp$ has
  kernels and cokernels.

  In this case, moreover, the following are equivalent:
  \begin{enumerate}
    \item[1.a] \ref{ax:ct4} holds.
    \item[1.b]\label{ax:eqb} If $V_1\to H_1\nto{f} H_2\nto{+}$ is a distinguished triangle such
    that $H_1, H_2\in\mathcal{H}$ and $V_1\in\mathcal{U}_1^\perp$, then $V_1\in\mathcal{U}_2^\perp[-1]$.
  \end{enumerate}
  And, dually, there is an equivalence of the following:
  \begin{enumerate}
    \item[2.a] \ref{ax:ct4op} holds.
    \item[2.b]\label{ax:eqa} If $H_1\nto{f}H_2\to U_2\nto{+}$ is a distinguished triangle such
    that $H_1,H_2\in\mathcal{H}$ and $U_2\in\mathcal{U}_2$, then $U_2\in\mathcal{U}_1[1]$.
  \end{enumerate}
\end{example}

\begin{proof}[Proof of the equivalences in example 2]
  Let $\mathcal{D}$ be a triangulated category with two t-structures as in example 3. The pseudocokernel
  of a morphism in $\mathcal{U}_1$ can be computed by taking the cone in $\mathcal{D}$, i.e. given a morphism
  $f:U_1\to U_1'$ in $\mathcal{U}_1$ we can compute a pseudocokernel in $\mathcal{U}_1$ by completing $f$
  to a triangle
  \begin{equation*}
    U_1\nto{f}U_1'\to \mathrm{Cone}(f)\nto{+}.
  \end{equation*}
  Moreover, this pseudocokernel satisfies \ref{ax:ct3}.

  Now, assume that \ref{ax:ct1}-\ref{ax:ct3},\ref{ax:ct3op} are satisfied together with axiom
  \textbf{1.b}, and consider the solid part of the diagram as in \ref{ax:ct4}:
  \begin{equation*}
    \begin{tikzcd}
      \mathrm{Cone}(f)[-1]\arrow{r}{f^K}
        & H_1\arrow{r}{f}\arrow{d}{\alpha}
          & H_1\arrow{r}{f^C}\arrow[equal]{d}
            & \mathrm{Cone}(f)\arrow{d}{\lambda}\\
      \tau_{\mathcal{U}_1}(V_2)\arrow{r}{\varepsilon}\arrow[dotted]{ur}{\beta}
        & V_2\arrow{r}
          & H_2\arrow{r}
            & \tau^{\mathcal{U}_2^\perp}\mathrm{Cone}(f)
    \end{tikzcd}
  \end{equation*}
  with $\mathrm{Cone}(f)[-1]\in\mathcal{U}_1^\perp$ and where the upper row is a
  distinguished triangle. By \textbf{1.b} then it belongs
  to $\mathcal{U}_2^\perp[-1]$, i.e. $\mathrm{Cone}(f)\in\mathcal{U}_2^\perp$, so
  $\lambda$ is an iso, conseguently $\alpha$ is an iso and so is $\varepsilon$, so
  there exist a map $\beta=\alpha^{-1}\circ\varepsilon$ making the diagram commute, that
  is \ref{ax:ct4} holds.

  Conversely, assume that \ref{ax:ct1}-\ref{ax:ct3},\ref{ax:ct3op} are satisfied together with
  \ref{ax:ct4}. Consider again the solid part of the diagram
  \begin{equation*}
    \begin{tikzcd}
      \mathrm{Cone}(f)[-1]\arrow{r}{f^K}\arrow{d}{\lambda[-1]}
        & H_1\arrow{r}{f}\arrow{d}{\alpha}
          & H_1\arrow{r}{f^C}\arrow[equal]{d}
            & \mathrm{Cone}(f)\arrow{d}{\lambda}\\
      \tau_{\mathcal{U}_2^\perp}(\mathrm{Cone}(f))[-1]\arrow{r}
        & V_2\arrow{r}
          & H_2\arrow{r}
            & \tau^{\mathcal{U}_2^\perp}\mathrm{Cone}(f)\\
        & \tau_{\mathcal{U}_1}(V_2)\arrow{u}{\varepsilon}\arrow[dotted, crossing over, bend left=70]{uu}[near end,']{\beta}
        & &
    \end{tikzcd}
  \end{equation*}
  with $\mathrm{Cone}(f)[-1]\in\mathcal{U}_1^\perp$. Neeman \todo{Add reference} guarantees
  that $\alpha$ can be taken so that the square on the left is a pullback. Axiom \ref{ax:ct4}
  gives the existence of $\beta:\tau_{\mathcal{U}_1}(V_2)\to H_1$ such that $\alpha\circ\beta=\varepsilon$.

  Since $\tau_{\mathcal{U}_1}$ is a functor, there is also a morphism
  $\tau_{\mathcal{U}_1}(\alpha):\tau_{\mathcal{U}_1}(H_1)=H_1\to \tau_{\mathcal{U}_1}(V_2)$ such that
  $\varepsilon\circ\tau_{\mathcal{U}_1}(\alpha)=\alpha$, hence
  $\varepsilon\circ\tau_{\mathcal{U}_1}(\alpha)\circ\beta = \varepsilon$. By the functoriality of
  the torsion pair $(\mathcal{U}_1,\mathcal{U}_1^\perp)$, this means that
  $\tau_{\mathcal{U}_1}(\alpha)\circ\beta=1_{\tau_{\mathcal{U}_1}(V_2)}$. Then, $\mathcal{\beta}$ is a section.

  Hence, we can write $\tau_{\mathcal{U}_1}(\alpha):H_1\to \tau_{\mathcal{U}_1}(V_2)$
  as
  \begin{equation*}
    \begin{tikzcd}
      \tau_{\mathcal{U}_1}(\alpha):\tau_{\mathcal{U}_1}(V_2)\oplus H_1'
      \arrow{r}{
          \begin{psmallmatrix}
            \ast \amsamp 0
          \end{psmallmatrix}
          }
        & \tau_{\mathcal{U}_1}(V_2)
    \end{tikzcd}
  \end{equation*}
  for some $H_1'\underset{\oplus}{<} H_1$ such that
  $\alpha$ vanishes on $H_1'$. If we consider the solid part of the diagram
  \begin{equation*}
    \begin{tikzcd}
      & H_1'\arrow[dashed]{dl}
      \arrow{d}{
        \begin{psmallmatrix}
          1 \\ 0
        \end{psmallmatrix}
      }\arrow{dr}{0}
        & & &\\
      \mathrm{Cone}(\tau_{\mathcal{U}_1}(\alpha))[-1]\arrow{r}
        & H_1\arrow{r}{\tau_{\mathcal{U}_1}(\alpha)}
          & \tau_{\mathcal{U}_1}(V_2)\arrow[dashed]{r}{+}
            & {}
    \end{tikzcd}
  \end{equation*}
  we can construct the dashed arrow, and the fact that the triangle commutes means
  that $H_1'\underset{\oplus}{<}\mathrm{Cone}(\tau_{\mathcal{U}_1}(\alpha))[-1]$.

  Observe that $\mathrm{Cone}(\alpha)=\mathrm{Cone}(\lambda)[-1]$, since
  the square
  \begin{equation*}
    \begin{tikzcd}
      \mathrm{Cone}(f)[-1]\arrow{r}{f^K}\arrow{d}{\lambda[-1]}
        & H_1\arrow{d}{\alpha} \\
      \tau_{\mathcal{U}_2^\perp}(\mathrm{Cone}(f))[-1]\arrow{r}
        & V_2
    \end{tikzcd}
  \end{equation*}
  is a pullback. Moreover, $\mathrm{Cone}(\lambda)[-1] = (\tau_{\mathcal{U}_2}(\mathrm{Cone}(f))[1])[-1]
  =\tau_{\mathcal{U}_2}(\mathrm{Cone}(f))$. Hence, $\mathrm{Cone}(\alpha)\in \mathcal{U}_2$ and
  $\tau^{\mathcal{U}_1^\perp}(\mathrm{Cone}(\alpha))=0$, that is,
  $\mathrm{Cone}(\alpha)\in\mathcal{U}_1$, and since there is a distinguished triangle
  \begin{equation*}
    H_1\nto{\alpha} V_2\to \mathrm{Cone}(\alpha)\nto{+}
  \end{equation*}
  with $H_1,\mathrm{Cone}(\alpha)\in\mathcal{U}_1$ it follows that
  $V_2\in\mathcal{U}_1$. Hence, $\tau_{\mathcal{U}_1}(V_2)\cong V_2$.

  We can then write $V_2\underset{\oplus}{<}H_1$ and consider the commutative diagram
  \begin{equation*}
    \begin{tikzcd}
      H_1 \cong H_1'\oplus V_2\arrow{r}{
        \begin{psmallmatrix}
          f' \amsamp \tilde{f}
        \end{psmallmatrix}
      }\arrow{d}{
        \begin{psmallmatrix}
          0 \amsamp 1
        \end{psmallmatrix}
      }
        & H_2\arrow[equal]{d}\\
      V_2\arrow{r}
        & H_2
    \end{tikzcd}
  \end{equation*}
  so $f'=0$. Hence, the inclusion $
  \begin{psmallmatrix}
    1 \\ 0
  \end{psmallmatrix}:H_1'\to H_1'\oplus V_2$ can be lifted to
  $\mathrm{Cone}(f)[-1]$ and $H_1'\underset{\oplus}{<}\mathrm{Cone}(f)[-1]$.
  Since $\mathrm{Cone}(f)[-1]\in \mathcal{U}_1^\perp$, so does $H_1'$. Similarly,
  $H_1'\in\mathcal{U}_1$ because $H_1\in\mathcal{U}_1$. Hence, $H_1'=0$ and $\alpha:H_1\to V_2$ is an iso.
  The same follows for $\lambda$. Therefore, $\mathrm{Cone}(f)\in\mathcal{U}_2^\perp$ which
  proves \triangb.
\end{proof}

\begin{lemma}\label{lemma_tria}
\textcolor{blue}{  In the case of \cref{example_tria}, assume that \ref{ax:ct4} and \ref{ax:ct4op} hold, then:
  \begin{enumerate}
    \item\label{lemma_tria_a} For any $H,H'\in\H_\mathbb{t}$, $(H,H'[-1])=0$,
    \item\label{lemma_tria_b} $A\to B\to C$ is a short exact sequence in $\mathcal{H}_\mathbb{t}$ iff $A\to B\to C\nto{+}$ is
    a triangle in $\X$.
  \end{enumerate}}
\end{lemma}

\begin{proof}
  \ref{lemma_tria_a} follows observing that
  $\H[-1]=\mathcal{U}_1[-1]\cap\mathcal{U}_2^\perp[-1]\subseteq \mathcal{U}_2^\perp[-1]\subseteq(\mathcal{U}_1^\perp[1])[-1]=\mathcal{U}_1^\perp$.

  To prove \ref{lemma_tria_b} first consider a short exact sequence $A\nto{f} B\to C$ in $\mathcal{H}$, we want to
  prove that it is a triangle. Consider the triangle
  \begin{equation*}
    \mathrm{Cone}(f)[-1]\to A\nto{f} B \to \mathrm{Cone}(f),
  \end{equation*}
  since $f$ is mono, its pseudocokernel $\mathrm{Cone}(f)[-1]$ belongs to $\mathcal{U}_1^\perp$, so
  $\mathrm{Cone}(f)\in\mathcal{U}_2^\perp[-1]$ by \triangb, hence $\mathrm{Cone}(f)\in\mathcal{U}_2^\perp$.
  This implies that $\mathrm{Cone}(f)$ is the cokernel of $f$ and so $A\nto{f} B\to C\nto{+}$ is a triangle.

  Conversely, assume that $A\nto{f} B\nto{g} C\nto{+}$ is a triangle. Then, $\mathrm{Cone}(f)\cong C\in\H=\mathcal{U}_1\cap\mathcal{U}_2^\perp$ and
  so $\mathrm{Cone}(f)[-1]\in\mathcal{U}_2^\perp[-1]\subseteq \mathcal{U}_1^\perp$. Hence, $f$ is mono. A dual argument
  shows that $g$ is epi. Therefore, $A\nto{f} B\nto{g} C$ is a short exact sequence in $\mathcal{H}$.
\end{proof}

\begin{example}
  Let $R$ be any (associative with 1) ring. Consider the triangulated category $\mathcal{T}:=\mathcal{D}(R)$.
  The derived category $\mathcal{D}(R)$ has the so called natural t-structure
  $(\mathcal{D}^{\leq 0}(R),\mathcal{D}^{\geq}(R))$ where
  \begin{align*}
    \mathcal{D}^{\leq 0}(R) &:= \{ X\in\mathcal{D}(R) \,|\,H^i(X)=0\text{ for } i>0\},\\
    \mathcal{D}^{\geq 0}(R) &:= \{ X\in\mathcal{D}(R) \,|\,H^i(x)=0\text{ for } i<0\}.
  \end{align*}

  For any ideal $I\trianglelefteq R$, we have the TTF-triple $(\mathcal{C}_I,\mathcal{T}_I,\mathcal{F}_I)$
  associated to $I$, where
  \begin{align*}
    \mathcal{C}_I &:= \{M\in\Mod{R}\,|\,IM=M\},\\
    \mathcal{T}_I &:= \{M\in\Mod{R}\,|\,IM=0\}\cong \Mod{\frac{R}{I}},\\
    \mathcal{F}_I &:= \{M\in\Mod{R}\,|\,Ix=0\text{ and }x\in M\Rightarrow x=0\}.
  \end{align*}

  Consider the t-structure (Happel-Reiten-Smalo) $(\mathcal{D}^{\leq 0}_{t_I}(R), \mathcal{D}^{\geq 0}_{t_I}(R))$
  associated to the torsion pair $t_I=(\mathcal{C}_I,\mathcal{T}_I)$, where
  \begin{align*}
    \mathcal{D}^{\leq 0}_{t_I}(R) &:= \{ X\in\mathcal{D}^{\leq 0}(R)\,|\, H^0(X)\in\mathcal{C}_I \},\\
    \mathcal{D}^{\geq 0}_{t_I}(R) &:= \{ X\in\mathcal{D}^{\geq 0}(R)\,|\, H^0(X)\in\mathcal{T}_I \}.
  \end{align*}

  It can be seen that $\mathbb{t}=(\mathbb{t}_1,\mathbb{t}_2)$ where
  $\mathbb{t}_1:=(\mathcal{D}^{\leq 0}(R),\mathcal{D}^{\geq 1}(R))$ and
  $\mathbb{T}_2:=(\mathcal{D}^{\leq 0}_{t_I}(R),\mathcal{D}^{\geq 1}_{t_I}(R))$, is a strongly
  related torsion pair in $\mathcal{T}=\mathcal{D}(R)$.
\end{example}

\subsection{Polishchuk correspondence}

We recall the following bijection given by A. Polishchuk, and in order to do that,
for a t-structure $(\mathcal{U}_1,\mathcal{U}_1^\perp[1])$ in $\mathcal{T}$, we have the cohomological
functor $H^0_1:\mathcal{T}\to \mathcal{H}_1:=\mathcal{U}_1\cap\mathcal{U}_1^\perp[1]$
($\mathcal{H}_1$ is an abelian category).

\begin{prop}[Polishchuk]\label{prop:2.7}
  Let $(\mathcal{U}_1,\mathcal{U}_1^\perp[1])$ be a t-structure in a triangulated category.
  Then we have a bijection (Polishchuk's bijection)
  \begin{equation*}
    \begin{tikzcd}
      \left\{
      \begin{array}{c}
        \text{torsion pairs in} \\
        \mathcal{H}_1=\mathcal{U}_1\cap \mathcal{U}_1^\perp[1]
      \end{array}
      \right\}
      \arrow[leftrightarrow]{r}{\mathrm{Pol}_{\mathcal{H}_1}}
        &
        \left\{
          \begin{array}{c}
            \text{t-structures }
            (\mathcal{U}_2,\mathcal{U}_2^\perp) \\
            \text{ in } \mathcal{D}
            \text{ satisfying } \\ \mathcal{U}_1[1]\subseteq \mathcal{U}_2\subseteq\mathcal{U}_1
          \end{array}
        \right\}\\
      (\mathcal{X},\mathcal{Y})\arrow[mapsto]{r}
        & (\mathcal{U}_2,\mathcal{U}_2^\perp[1])\\
      (\mathcal{U}_2\cap\mathcal{H}_1,\mathcal{U}_2^\perp\cap\mathcal{H}_1)\arrow[mapsfrom]{r}
        & (\mathcal{U}_2,\mathcal{U}_2^\perp[1])
    \end{tikzcd}
  \end{equation*}
  where
  \begin{align*}
    \mathcal{U}_2 = \{X\in\mathcal{U}_1\,|\,H^0_1(X)\in\mathcal{X}\}\\
    \mathcal{U}_2^\perp = \{Y\in\mathcal{U}_1^\perp\,|\,H^0_1(Y)\in\mathcal{Y}\}.
  \end{align*}
\end{prop}

\begin{rmk}\label{rmk:2.8}
  \begin{enumerate}[label=(\arabic*)]
    \item\label{rmk:2.8:1} Note that $\mathrm{Pol}^{-1}_{\mathcal{H}_1}(\mathcal{U}_2,\mathcal{U}^\perp_2[1])=
      (\mathcal{U}_2\cap\mathcal{U}_1^\perp[1],\mathcal{H})$, where
      $\mathcal{H}:=\mathcal{U}_1\cap\mathcal{U}_2^\perp$.

    \item By \ref{rmk:2.8:1}, it follows that $\mathcal{H}$ is a torsion free class in the abelian category
      $\mathcal{H}_1:=\mathcal{U}_1\cap\mathcal{U}_1^\perp[1]$.
  \end{enumerate}
\end{rmk}

\begin{thm}\label{thm:2.9}
  Let $\mathbb{t}=(\mathbb{t}_1,\mathbb{t}_2)$ be a related torsion pair in a triangulated
  category $\mathcal{T}$. Then, the following statements are equivalent.
  \begin{enumerate}[label=(\alph*)]
    \item For any distinguished triangle $V\to H_1\nto{f}H_2\nto{+}$, with
      $f$ a morphism in $\mathcal{H}=\mathcal{H}_\mathbb{t}:=\mathcal{T}_1\cap\mathcal{F}_2$,
      we have that
      \begin{equation*}
        V\in\mathcal{F}_1 \Rightarrow V[1]\in\mathcal{F}_2.
      \end{equation*}
    \item For any monomorphism $\alpha:H_1\into H_2$, in the abelian category
      $\mathcal{H}_1:=\mathcal{T}_1\cap\mathcal{F}_1[1]$, with $H_1,H_2\in\mathcal{H}$,
      we have that $\Coker_{\mathcal{H}_1}(\alpha)\in\mathcal{H}$.
    \item $\mathcal{H}$ is closed under kernels and cokernels in the abelian category
      $\mathcal{H}_1$
    \item $\mathcal{H}$ is an abelian category.
    \item For any epimorphism $H\onto X$ in $\mathcal{H}_1$, with $H\in\mathcal{H}$,
      we have that $X\in\mathcal{H}$ (i.e. $\mathcal{H}$ is closed under quotients in $\mathcal{H}_1$).
  \end{enumerate}
\end{thm}

Let $t=(\mathcal{A},\mathcal{B})$ be a pair of full subcategories of the triangulated
category $\mathcal{T}$. We will use the following notation:
\begin{align*}
  t[1] &:= (\mathcal{A}[1],\mathcal{B}[1]),\\
  \overline{t} &:= (\mathcal{A},\mathcal{B}[1]).
\end{align*}

Note that $\overline{t}$ is a t-structure in $\mathcal{T}$ if and only if $t$ is a torsion
pair $\mathcal{T}$ such that $\mathcal{A}[1]\subseteq\mathcal{A}$.

\begin{rmk}
  Consider $\mathbb{t}=(\mathbb{t}_1,\mathbb{t}_2)$, where $\mathbb{t}_i:=(\mathcal{U}_i,\mathcal{U}_i^\perp)$ for
  $i=1,2$. We have
  \begin{enumerate}
    \item $\mathcal{H}_\mathbb{t}:=\mathcal{U}_1\cap\mathcal{U}_2^\perp$,
      $\mathcal{H}_i:=\mathcal{U}_i\cap\mathcal{U}_i^\perp[1]$,
    \item $\mathbb{t}' :=(\mathbb{t}_2,\mathbb{t}_1[1])$
  \end{enumerate}

  Note that
  \begin{enumerate}
    \setcounter{enumi}{2}
    \item $\mathbb{t}=(\mathbb{t}_1,\mathbb{t}_2)$ is a related torsion pair in $\mathcal{T}$
      \begin{align*}
        &\Leftrightarrow \mathcal{U}_1[1]\subseteq\mathcal{U}_2\subseteq\mathcal{U}_1\\
        &\Leftrightarrow \mathcal{U}_2[1]\subseteq\mathcal{U}_1[1]\subseteq\mathcal{U}_2\\
        &\Leftrightarrow \mathbb{t}'=(\mathbb{t}_2,\mathbb{t}_1[1]) \text{ is a related torsion pair in }\mathcal{T}.
      \end{align*}
    \item Let $\mathbb{t}=(\mathbb{t}_1,\mathbb{t}_2)$ is a related torsion pair in $\mathcal{T}$. In
      this case, we have
      \begin{align*}
        &\mathcal{H}_\mathbb{t} = \mathcal{U}_1\cap\mathcal{U}_2^\perp,~
        \mathcal{H}_{\mathbb{t}'}=\mathcal{U}_2\cap \mathcal{U}_1^\perp[1],\\
        & \mathrm{Pol}_{\mathcal{H}_1}^{-1}(\overline{\mathbb{t}}_2)=
        \mathrm{Pol}_{\mathcal{H}_1}^{-1}(\mathcal{U}_2,\mathcal{U}_2^\perp[1])=
        (\mathcal{H}_{\mathbb{t}'},\mathcal{H}_{\mathbb{t}}),\\
        &\mathrm{Pol}_{\mathcal{H}_2}^{-1}(\overline{\mathbb{t}}_1[1])=
        \mathrm{Pol}_{\mathcal{H}_1}^{-1}(\mathcal{U}_1[1],\mathcal{U}_1^\perp[2])=
        (\mathcal{H}_{\mathbb{t}}[1],\mathcal{H}_{\mathbb{t}'}).
      \end{align*}

      Thus, $(\mathcal{H}_{\mathbb{t}'},\mathcal{H}_\mathbb{t})$ is a torsion pair in the abelian
      category $\mathcal{H}_1$,
      $(\mathcal{H}_\mathbb{t}[1],\mathcal{H}_{\mathbb{t}'})$ is a torsion pair in the abelian category
      $\mathcal{H}_2$.
  \end{enumerate}
\end{rmk}

\begin{corollary}
  Let $\mathbb{t}=(\mathbb{t}_1,\mathbb{t}_2)$ be a related torsion pair in a triangulated category
  $\mathcal{T}$. Then, the following statements are equivalent:
  \begin{enumerate}[label=(\alph*)]
    \item For any distinguished triangle $V\to H_1\nto{f}H_2\nto{+}$, with
    $f$ a morphism in $\mathcal{H}_{\mathbb{t}'}=\mathcal{T}_2\cap\mathcal{F}_1[1]$,
    we have that $V\in\mathcal{F}_2$ implies $V\in\mathcal{F}_1$.
    \item For any monomorphism $\alpha:H_1\into H_2$, in the abelian category
    $\mathcal{H}_2:=\mathcal{T}_2\cap\mathcal{F}_2[1]$, with $H_1,H_2\in\mathcal{H}_{\mathbb{t}'}$,
    we have that $\Coker_{\mathcal{H}_2}(\alpha)\in\mathcal{H}_{\mathbb{t}'}$.
    \item $\mathcal{H}_{\mathbb{t}'}$ is closed under kernels and cokernels in
    the abelian category $\mathcal{H}_2$.
    \item $\mathcal{H}_{\mathbb{t}'}$ is an abelian category.
    \item $\mathcal{H}_{\mathbb{t}'}$ is closed under quotients in $\mathcal{H}_2$.
  \end{enumerate}
\end{corollary}

We recall that a torsion pair $(\mathcal{T},\mathcal{F})$ in an abelian category
$\mathcal{A}$ is cohereditary if the class $\mathcal{F}$ is closed under quotients
in $\mathcal{A}$.

\begin{definition}
  For a triangulated category $\mathcal{T}$, we consider the following classes:
  \begin{enumerate}
    \item $RtAb(\mathcal{T}):=\{\text{related torsion pairs }\mathbb{t}=(\mathbb{t}_1,\mathbb{t}_2)
      \text{ in }\mathcal{T}\text{ s.t. }\mathcal{H}_\mathbb{t}\text{ is abelian}\}$;
    \item
      \begin{align*}
        \mathrm{t-}stCoh(\mathcal{T}) &:=
        \left\{
        \begin{array}{c}
          \text{pairs }(\overline{\mathbb{t}}_1,\tau)\text{ s.t. }
          \overline{\mathbb{t}}_1\text{ is a t-structure in }\mathcal{T}\text{ and }
          \tau\text{ is a}\\
          \text{ cohereditary torsion pair in the abelian category } \\
          \mathcal{H}_1:=\mathcal{U}_1\cap\mathcal{U}_1^\perp[1]
        \end{array}
        \right\};
      \end{align*}
    \item[1\rlap{$^\prime$}.] $RtAb^\prime(\mathcal{T}):=\{\text{related torsion pairs }\mathbb{t}=(\mathbb{t}_1,\mathbb{t}_2)
      \text{ in }\mathcal{T}\text{ s.t. }\mathcal{H}_{\mathbb{t}'}\text{ is abelian}\}$;
    \item[2\rlap{$^\prime$}.]
      \begin{align*}
        \mathrm{t-}stCoh^\prime(\mathcal{T}) &:=
        \left\{
        \begin{array}{c}
          \text{pairs }(\overline{\mathbb{t}}_2,\tau)\text{ s.t. }
          \overline{\mathbb{t}}_2\text{ is a t-structure in }\mathcal{T}\text{ and }
          \tau\text{ is a}\\
          \text{ cohereditary torsion pair in the abelian category } \\
          \mathcal{H}_2:=\mathcal{U}_2\cap\mathcal{U}_2^\perp[1]
        \end{array}
        \right\}.
      \end{align*}
  \end{enumerate}
\end{definition}

\begin{thm}\label{thm:2.11}
  For a triangulated category $\mathcal{T}$, the following statements hold true.
  \begin{enumerate}[label=(\alph*)]
    \item There is a bijective correspondence
      \begin{equation*}
        \begin{tikzcd}[row sep=tiny]
          RtAb(\mathcal{T})\arrow[leftrightarrow]{r}{\alpha}
            & \mathrm{t}-stCoh(\mathcal{T})\\
          \mathbb{t}\arrow[mapsto]{r}
            & (\overline{\mathbb{t}}_1,\mathrm{Pol}^{-1}_{\mathcal{H}_1}(\overline{\mathbb{t}}_2))\\
          (\mathbb{t}_1,\mathbb{t}_2)\arrow[mapsfrom]{r}
            & (\overline{\mathbb{t}}_1,\tau)
        \end{tikzcd}
      \end{equation*}
      where $\overline{\mathbb{t}}_2=\mathrm{Pol}_{\mathcal{H}_1}(\tau)$.
    \item There is a bijective correspondence
      \begin{equation*}
        \begin{tikzcd}
          RtAb^\prime(\mathcal{T})\arrow[leftrightarrow]{r}{\alpha'}
            & \mathrm{t}-stCoh^\prime(\mathcal{T})\\
          \mathbb{t}\arrow[mapsto]{r}
            & (\overline{\mathbb{t}}_2,\mathrm{Pol}^{-1}_{\mathcal{H}_2}(\overline{\mathbb{t}}_1[1]))\\
          (\mathbb{t}_1,\mathbb{t}_2)\arrow[mapsfrom]{r}
            & (\overline{\mathbb{t}}_2,\tau)
        \end{tikzcd}
      \end{equation*}
      where $\overline{\mathbb{t}}_1=\mathrm{Pol}_{\mathcal{H}_2}(\tau)[-1]$.
  \end{enumerate}
\end{thm}

% \begin{lemma}\label{sec2:lem6}
%   The heart $\mathcal{H}=\mathcal{T}_1\cap\mathcal{F}_2$ has kernels and cokernels.
% \end{lemma}
%
% \begin{rmk}
%   If $\mathcal{X}$ is abelian, then:
%   \begin{enumerate}
%     \item $f:H\to H'$ is mono in $\mathcal{H}$ iff $\Ker_\mathcal{X}(f)\in\mathcal{F}_1$
%     \item $f:H\to H'$ is epi in $\mathcal{H}$ iff $\Coker_\mathcal{X}(f)\in\mathcal{T}_2$
%   \end{enumerate}
% \end{rmk}
%
% \begin{prop}\label{sec2:prop1}
%   Let $(\mathcal{T}_1,\mathcal{F}_1)$ and $(\mathcal{T}_2,\mathcal{F}_2)$ be two torsion pairs
%   in the abelian category $\mathcal{A}$. TFAE:
%   \begin{enumerate}
%     \item $\mathcal{H}$ is abelian.
%     \item We have:
%     \begin{enumerate}
%       \item If $f:H\to H'$ is a morphism in $\mathcal{H}$ such that
%       $\Ker_\mathcal{A}(f)\in\mathcal{F}_1$, then $\Ker_\mathcal{A}(f)=0$.
%       \item If $f:H\to H'$ is a morphism in $\mathcal{H}$ such that
%       $\Coker_\mathcal{A}(f)\in\mathcal{T}_2$, then $\Coker_\mathcal{A}(f)=0$.
%       \item $\mathcal{H}$ is closed by kernels of epics and cokernels of monics.
%     \end{enumerate}
%     \item $\mathcal{H}$ is an exact abelian subcategory of $\mathcal{A}$.\todo{?}
%   \end{enumerate}
% \end{prop}




%*****************************************
%*****************************************
%*****************************************
