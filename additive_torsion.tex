\clearpage

\section{Torsion pairs}

We fix an additive category $\mathcal{X}$ with pseudokernels and pseudocokernels
on which idempotents split.

\begin{definition}
  A pair $\mathbb{t}=(\mathcal{T},\mathcal{F})$ of full subcategories of $\mathcal{X}$
  is a torsion pair in $\mathcal{X}$ if:
  \begin{enumerate}
    \item
    \begin{align*}
      \mathcal{F}&=\mathcal{T}^\perp = \{X\in\mathcal{X}\,|\,\mathcal{X}(\mathcal{T},X)=0\},\\
      \mathcal{T}&={^\perp\mathcal{F}} = \{X\in\mathcal{X}\,|\,\mathcal{X}(X,\mathcal{F})=0\};
    \end{align*}
    \item FOr each $M\in\mathcal{X}$ there is a pseudokernel-pseudocokernel sequence
    \begin{equation*}
      \begin{tikzcd}
        T_M\arrow{r}{\varepsilon_M}& M\arrow{r}{\lambda_M}&F^M
      \end{tikzcd}
    \end{equation*}
    where $T_M\in\mathcal{T}$ and $F^M\in\mathcal{F}$.
  \end{enumerate}

  If in addition the assignment $M\mapsto t(M) :=T_M$ (resp. $M\mapsto f(M):=F^M)$ is
  functorial and defines an adjoint pair $(i,t)$ (resp. $(f,j)$), where $i:\mathcal{T}\into \mathcal{X}$
  (resp. $j:\mathcal{F}\into \mathcal{X}$) is the inclusion functor, then we say that
  $\mathbb{t}$ is left (resp. right) functorial. In such a case,
  $\varepsilon$ (resp. $\lambda$) is the counit (resp. unit) of the given adjoint pair. We say
  that $\mathbb{t}$ is functorial if it is right and left functorial.
\end{definition}

\begin{rmk}\label{rmk:1.1}
  Let $\mathbb{t}=(\mathcal{T},\mathcal{F})$ be a left functorial torsion pair in $\mathcal{X}$.
  Then
  \begin{enumerate}
    \item[(a)] For any $M\in\mathcal{X}, T'\in\mathcal{T}$ and $\alpha\in\mathcal{X}(T',M)$ there
    is a unique $\alpha'\in\mathcal{X}(T',t(M))$ such that $\varepsilon_M\circ\alpha'=\alpha$, i.e.
    \begin{equation*}
      \begin{tikzcd}
        & T'\arrow[dashed]{ld}[']{\exists!\,\alpha'}\arrow{d}{\alpha}\\
        t(M)\arrow{r}{\varepsilon_M}&M
      \end{tikzcd}
    \end{equation*}
    \item[(b)] Let $g:T_1\to T_2$ be a morphism in $\mathcal{T}$, which admits a pseudocokernel
    $g^C:T_2\to\PCok_\mathcal{T}(g)$ in $\mathcal{T}$. Then $g^C$ is a pseudocokernel of $g$
    in $\mathcal{X}$.
  \end{enumerate}
\end{rmk}

\begin{proof}
  \begin{enumerate}[label=(\alph*)]
    \item\label{rmk:1.1:proof:a}  Since $(i,t)$ is an adjoint pair, we have a functorial isomorphism
      \begin{equation*}
        \Theta:\myhom{\mathcal{T}}{T',t(M)}\nto{\sim}\myhom{\mathcal{X}}{i(T'),M}=\myhom{\mathcal{X}}{T',M}.
      \end{equation*}

      Let $\alpha':=\Theta^{-1}(\alpha)$. Then, $\varepsilon_M\circ \alpha'=\Theta(\alpha')=\alpha$.
    \item Let $X\in\mathcal{X}$ and $h:T_2\to X$ such that $hg=0$. Consider the commutative diagram
    \begin{equation*}
      \begin{tikzcd}
        T_1\arrow{r}{g}
          & T_2\arrow{r}{g^C}\arrow{d}{h}\arrow[dashed]{dr}{\exists h'}
            & \PCok_\mathcal{T}(g)\arrow[dashed]{d}{\exists h''}\\
          & X
            & t(X)\arrow{l}{\varepsilon_X}.
      \end{tikzcd}
    \end{equation*}
    Then, $h'\circ g=0$. In fact, $0=h\circ g=\varepsilon_X \circ h' \circ g$
    and $\varepsilon_X\circ 0=0$, so, by \ref{rmk:1.1:proof:a}, $h'\circ g=0$.

    Since $h'\circ g=0$, it follows that there is a map $h'':\PCok_\mathcal{T}(g)\to t(X)$
    such that $h''\circ g^C=h'$. Finally, $h=(\varepsilon_X\circ h'')\circ g^C$.
  \end{enumerate}
\end{proof}

The dual also holds.
