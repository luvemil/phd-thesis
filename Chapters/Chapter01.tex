%************************************************
\phantomsection
\manualmark
\markboth{\spacedlowsmallcaps{Introduction}}{\spacedlowsmallcaps{Introduction}}%
\addtocontents{toc}{\protect\vspace{\beforebibskip}}%
\addcontentsline{toc}{chapter}{\tocEntry{Introduction}}%
\chapter*{Introduction}\label{ch:introduction}
%************************************************

Model categories were introduced by Quillen for the first time in \cite{Q} as a formal way of dealing with the notion of homotopy in a category. The main problem is the following: we have a class of morphisms, called weak equivalences, and we want to treat them as if they were isomorphisms. Formally inverting these weak equivalences does not necessarily produce nice results, since we have to resort to some kind of calculus of fractions to construct the morphisms in the quotient category, and these methods are not guaranteed to yield a \emph{set} of homomorphisms. This, however, happens whenever the weak equivalences are part of a model structure on a model category.

In \cite{Hov02} Hovey proved that there is a bijective correspondence between abelian model structures on abelian categories and certain couples of complete cotorsion pairs. Namely, if $\clQ$, $\W$, and $\R$ are the classes of cofibrant, trivial, and fibrant objects respectively, then $(\clQ,\W\cap\R)$ and $(\clQ\cap\W,\R)$ are complete cotorsion pairs, and conversely given three such classes forming two complete cotorsion pair, there is an abelian model structure, indicated by $(\clQ,\W,\R)$, whose cofibrant, trivial and fibrant objects are exactly those classes.

This correspondence allows to look for models structures simply by looking for complete cotorsion pairs, for example given a ring $R$ there is a model structure (and, in fact, many of them) on the category $\Ch(R)$ of unbounded cochain complexes of $R$-modules whose homotopy category is the derived category of $R$. Moreover, if several model structures on a category satisfy certain inclusions, it is possible to find a recollement between their homotopy categories.

Subsequently, the same techniques have been applied to cotorsion pairs in exact categories, showing that Hovey's correspondence holds in the case of exact model structures in exact categories. This has greatly helped the theory of localization of model structures, since it applies easily to the case of extension closed subcategories of Grothendieck categories.

The first part of this thesis follows this idea and our objective will be to construct model structures in exact subcategories of the category $\Ch(\G)$ of unbounded cochain complexes on a Grothendieck category $\G$, in order to prove the existence of interesting recollements between their homotopy categories. More precisely, we will consider a complete hereditary cotorsion pair $(\A,\B)$ in a Grothendieck category $\G$ and construct model structures on $\Ch(\B)$ in order to find recollements involving $K(\B)$, $D(\B)$ together with $D(\G)$.

The second part of this thesis is devoted to a different topic of research, although our applications will still be mainly in triangulated categories.

In \cite{bbd82} Beilinson, Bernstein and Deligne introduced the notion of t-structure in a triangulated category in order to build perverse sheaves. They also proved that to any t-structure is associated an abelian category, called its heart. In \cite{Nakaoka}, Nakaoka proved that starting from any torsion pair $(\U,\V)$ in a triangulated category $\mathscr{C}$ it is possible to find an abelian heart $\underline{\clH}$ in the quotient category $\underline{\mathscr{C}}:=\mathscr{C}/\W$, where $\W=\U[1]\cap\V$. This result motivates the second part of this thesis, where we use torsion pairs in additive categories to build what we call Nakaoka contexts, and show that if they satisfy certain axioms they have an abelian heart.

We will then investigate the behaviour of Nakaoka contexts in Abelian and triangulated categories, with particular interest to the case of t-structures.

\medskip

This thesis is articulated as follows:
\begin{itemize}
  \item In Chapter~\ref{ch:acycexact} we introduce the notions of cotorsion pairs, model structures, and recollements, giving the statements of the main known results in the field and constructing the important cotorsion pairs that will be used later. This chapter is based on the first half of the joint work \cite{bazzoni2018recollements} with Silvana Bazzoni.
  \item In Chapter~\ref{ch:recoll} we use the theory introduced in the previous chapter to build many new recollements in the setting of exact subcategories of the category of cochain complexes on a Grothendieck category. This chapter is based on the second half of the joint work \cite{bazzoni2018recollements}.
  \item In Chapter~\ref{ch:nakaoka} we introduce and study Nakaoka contexts and their hearts in the setting of additive categories, with the objective of finding an axiomatization that would guarantee the existence of an abelian heart. We examine in particular Nakaoka contexts in abelian and triangulated categories. This chapter is based on the joint work (in preparation) \cite{mstv2018} with Manuel Saor\'in, Simone Virili, and Octavio Mendoza.
\end{itemize}

Next, we will give a detailed summary of each chapter of the thesis.

\section*{Chapter \ref{ch:acycexact}: Cotorsion pairs and model structures in exact categories}

In this chapter we give a summary of the main results regarding cotorsion pairs and model structures in exact categories.

In Section \ref{sec:preliminaries} we give the relevant definitions of cotorsion pairs and model structure and state Hovey's correspondence explicitly in Theorem~\ref{T:correspondence}.
Then, in Definition \ref{def:recoll} we define what a recollement is and we state the main tools we will use in order to find recollements, namely Gillespie's Theorem~3.4 and Corollary~4.5 in \cite{G7} (numbered Theorem~\ref{thm:gil_inj_recoll} and Corollary~\ref{cor:gill_local_recoll} respectively in this work).

First, recall the definition of injective cotorsion pair:
\begin{nonlisting_def}
  A complete cotorsion pair $(\W,\R)$ is \emph{injective} if $\W$ is thick and contains the injective objects.

  If $(\W,\R)$ is an injective cotorsion pair in $\C$, it induces a model structure $(\C,\W,\R)$.
\end{nonlisting_def}

\begin{nonlisting_thm}[\ref{thm:gil_inj_recoll}]
  Let $\C$ be a weakly idempotent complete (WIC) exact category with enough injective and suppose we have three injective cotorsion pairs
  \[
    \M_1=(\W_1,\R_1)\text{,\hfill}\M_2=(\W_2,\R_2)\text{,\hfill}\M_3=(\W_3,\R_3)
  \]
  such that $\R_2,\R_3\subseteq \R_1$. If $\W_3\cap\R_1=\R_2$ (or equivalently $\W_2\cap\W_3=\W_1$, and $\R_2\subseteq \W_3$), then there exists a recollement
  \begin{equation*}
    \begin{tikzcd}
      {\R_2/\sim}\arrow{r}{I}
      & {\R_1/\sim}\arrow[bend left=50]{l}\arrow[bend right=50]{l}\arrow{r}{Q}
      & \C/\W_3\arrow[bend left=50]{l}{\rho}\arrow[bend right=50]{l}[']{\lambda}
    \end{tikzcd}
  \end{equation*}
  where the functor $I$ is simply the inclusion and $Q$ is the quotient functor of Lemma~\ref{lemma:gill_quotient_functor}. Moreover, $\lambda$ has essential image $(\W\cap\R_1)/\sim$, $\rho$ has essential image ${\R_3/\sim}$, and they provide an equivalence \[\lambda\colon{\R_3/\sim} \longleftrightarrow (\W_2\cap\R_1)/\sim\colon\rho.\]
\end{nonlisting_thm}

In the case of a Frobenius category, an injective cotorsion pair is called \emph{localizing}. If $(\X,\Y)$ and $(\Y,\Z)$ are both localizing cotorsion pairs in a Frobenius category, we call $(\X,\Y,\Z)$ a \emph{localizing cotorsion triple}.

\begin{nonlisting_cor}[\ref{cor:gill_local_recoll}]
  Let $(\X,\Y,\Z)$ be a localizing cotorsion triple in a WIC Frobenius category $\C$. Then, there is a recollement
  \begin{equation*}
    \recdiagram{\Y/\sim}{\C/\sim}{\C/\Y}
  \end{equation*}
  where $\C/\sim$ is the stable category.
\end{nonlisting_cor}

In Section \ref{sec:more_cot} we give the tools needed to construct the cotorsion pairs we will use in the next chapter. We will see that starting from a completely hereditary cotorsion pair $(\A,\B)$ in a Grothendieck category we can find several complete hereditary cotorsion pairs in $\Ch(\G)$, namely $({^\perp dw\B},dw \B)$, $(dw \A,{dw \A^\perp})$, $({^\perp ex\B},ex \B)$, $(ex\A, {ex\A^\perp})$, $(\wac{A},dg \B)$, and $(dg \A,\wac{\B})$ (see Notation \ref{N:notation} for the meaning of these symbols).

We conclude the chapter by stating several restriction propositions, namely \ref{P:prop7.3}, \ref{P:prop7.3-dual}, \ref{P:prop7.2}, and \ref{P:7.2-Groth}. Given a complete hereditary cotorsion pair $(\A,\B)$ in a Grothendieck category $\G$, these results give conditions under which it is possible to restrict a cotorsion pair in $\Ch(\G)$ to one in $\Ch(\B)$ (or, dually, $\Ch(\A)$).

\section*{Chapter \ref{ch:recoll}: Recollements from cotorsion pairs}

We consider the case of a complete hereditary cotorsion pair $(\A,\B)$ in a Grothendieck category $\G$, sometimes requiring that $\A$ contains a generator of finite projective dimension. It is well known that, if $\G$ contains a projective generator, there is a recollement
\begin{equation*}
  \recdiagram{K_{ac}(\G)}{K(\G)}{D(\G)}.
\end{equation*}

In this chapter our objective is to find similar recollements, involving the classes $K(\B)$ and $D(\B)$ (and, dually, for $\A$). The main tools are the aforementioned \cite[Theorem~3.4 and Corollary~4.5]{G7}

In Theorem~\ref{T:derived-B} we prove that there is a recollement
\begin{equation*}
  \recdiagram{\dfrac{ex\B}{\sim}}{D(\B)}{D(\G)}
\end{equation*}
where $D(\B)$ is the derived category of $\B$ in Neeman's sense, and the homotopy category $ex\B/\sim$ on the left hand side is the full subcategory of $D(\B)$ consisting of the exact complexes with terms in $\B$.

Then, by constructing a localizing cotorsion triple on $\Ch(\B)$ of the form $(-,ex\B,-)$, where $ex\B$ is the class of exact complexes in $\Ch(\G)$ with terms in $\B$, we use Corollary~\ref{cor:gill_local_recoll} to find in Theorem~\ref{T:triple-B} a recollement
\begin{equation*}
  \recdiagram{\dfrac{ex\B}{\sim}}{K(\B)}{D(\G)}.
\end{equation*}

It would be interesting to find cases where the previous recollement ends in the derived category $D(\B)$. To this aim, we observe that when $ex\B=\wac{\B}$, where $\wac{\B}$ is the class of complexes $B^\bullet\in ex\B$ such that $Z^n(B^\bullet)\in\B$ for all $n\in\mathbb{Z}$,  the previous recollement becomes
\begin{equation*}
  \recdiagram{\dfrac{\wac{\B}}{\sim}}{K(\B)}{D(\B)}
\end{equation*}
and in Section~\ref{S:tildeB} we describe several cases where this happens. A first example is that of a cotorsion pair $(\A,\B)$ where all objects in $\A$ have finite projective dimension, thus including the case of a tilting cotorsion pair, however this condition is far from necessary. In fact it was proved in \cite{BCIE} that the cotorsion modules satisfy $ex\Cot=\wac{\Cot}$.

Moreover, it is possible to find the previous recollement for cotorsion pairs that do not satisfy $ex\B=\wac{\B}$, e.g. the complete hereditary cotorsion pair $(^\perp\FpInj,\FpInj)$ generated by the class of finitely presented modules over a coherent ring, provided by \sto\  in \cite{Stopurity}.

We also study the dual case, and find that when $ex\A=\wac{\A}$ we can find the dual recollement. Again, the condition $ex\A=\wac{\A}$ is satisfied when $\B$ consists of objects of finite injective dimension, this is true for example for a cotilting cotorsion pair.

Finally, one example in the case of $R$-modules of a cotorsion pair $(\A,\B)$ with $\wac{\A}\subsetneq ex\A$ can be found when $\A=\Flat$ is the class of flat modules and follows by Neeman's~\cite{Nee08}.

\section*{Chapter \ref{ch:nakaoka}: Nakaoka contexts with abelian hearts}

In this chapter we will give a quick overview of the notion of torsion pair in an additive category. The difference with the usual notion of torsion pair is that, in lack of short exact sequences, we will resort to require the existence of pseudokernels and pseudocokernels. This will not be a problem, since our notion of torsion pair will coincide with the usual concept in Abelian categories, and t-structures in triangulated category will be (up to shift) torsion pairs.

\begin{nonlisting_def}
  Let $\T,\F$ be subclasses of the additive category $\mathscr{C}$, $(\T,\F)$ is a \emph{torsion pair} if:
  \begin{enumerate}
    \item $\F=\{X|\Hom_\mathscr{C}(T,X)=0\text{ for all }T\in\T\}$;
    \item $\T=\{X|\Hom_\mathscr{C}(X,F)=0\text{ for all }F\in\F\}$;
    \item for any $X\in\mathscr{C}$ there are two maps $\varepsilon_X$ and $\lambda_X$
      \[
        T_X\nto{\varepsilon_X}X\nto{\lambda_X}F^X
      \]
      such $T_X\in\T$, $F^X\in\F$, $\varepsilon_X$ is a pseudokernel of $\lambda_X$, and $\lambda_X$ is a pseudocokernel of $\varepsilon_X$.
  \end{enumerate}

  The torsion pair $(\T,\F)$ is called left (resp. right) functorial if the inclusion functor $i:\T\to \mathscr{C}$ (resp. $j:\F\to\mathscr{C}$) has a right (resp. left) adjoint $t:\mathscr{C}\to \T$ (resp. $f:\mathscr{C}\to \F$).
\end{nonlisting_def}

The central notion of this chapter is the following:
\begin{nonlisting_def}
A (pre-Abelian) Nakaoka context in an additive category $\mathscr{C}$ is a couple $\t=(\t_1,\t_2)$ of torsion pairs $\t_i=(\T_i,\F_i)$ satisfying the following axioms:

\begin{enumerate}
  \item[(CT.1)] $\t_1=(\T_1,\F_1)$ and $\t_2=(\T_2,\F_2)$ are respectively a left functorial and a right functorial torsion pair;
  \item[(CT.2)] $\T_2\subseteq \T_1$ (equiv. $\F_1\subseteq\F_2$);
  \item[(CT.3)] any  $g\colon H\to H'$ in $\clH:=\T_1\cap\F_2$ admits a pseudocokernel $g^C\colon H'\to C$ in $\T_1$, such that
    \begin{equation*}
      \begin{tikzcd}
        0\arrow{r} &(C,-)_{|\F_2}\arrow{r}{(g^C,-)} & (H',-)_{|\F_2}\arrow{r}{(g,-)}& (H,-)_{|\F_2}
      \end{tikzcd}
    \end{equation*}
    is an exact sequence in $\Func(\F_2,\Ab)$;
  \item[(CT.3)$^\ast$] dual of \textbf{(CT.3)}.
\end{enumerate}
\end{nonlisting_def}

Axioms \textbf{(CT.1)} and \textbf{(CT.2)} are simply setting the stage of two torsion pairs with the appropriate inclusions and functoriality. The fundamental axioms are \textbf{(CT.3)} and its dual, since they allow us to construct kernels and cokernels in the heart $\clH:=\T_1\cap\F_2$ with the following procedure: given a morphism $f\in\clH$, take a pseudocokernel $C\in\T_1$ satisfying \textbf{(CT.3)}, then the kernel of $f$ in $\clH$ is the torsion-free part $f_2(C)$ of $C$ with respect to $\F_2$. This result is stated in Theorem~\ref{pre_abelian_theorem}.

Observe that \textbf{(CT.3)} is essentially stating that there is a special pseudocokernel in $\T_1$, so we can regard pre-Abelian Nakaoka contexts as couples of torsion pairs with pseudokernels and pseudocokernels satisfying additional properties. In particular, since we already have a pre-Abelian heart $\clH$, we want to give additional requirements for the pseudokernels and pseudocokernels in such a way that $\clH$ becomes abelian.

In Theorem~\ref{thm:2.6} we show that the abelianity of $\clH$ is equivalent to the following axioms:
\begin{enumerate}
  \item[(CT.4)] given a morphism $f\colon H\to H'$ in $\clH$ that admits a pseudo-kernel $f^K\colon H''\to H$ in $\F_2$, such that $H''\in \F_1$, and the commutative diagram
    \begin{equation*}
      \xymatrix{
        & H\ar[d]^{a}\ar[r]^{f} & H'\ar@{=}[d]\ar[r]^{f^C} & T_1\ar[d]^{\lambda_{2,T_1}}\\
        t_1F_2\ar@{.>}[ur]^{b}\ar[r]^{\varepsilon_{1,F_2}} & F_2\ar[r]^{g^K} & H'\ar[r]^{g} & f_2T_1
      }
    \end{equation*}
    where $f^C$  is a pseudo-cokernel of $f$ in $\T_1$ and $g^K$ is a pseudo-kernel of $g$ in $\F_2$, there exists a morphism $b\colon t_1F_2\to H_1$ such that $ab=\varepsilon_{1,F_2}$;
  \item[(CT.4)$^\ast$] dual to \textbf{(CT.4)}.
\end{enumerate}

After laying the theory, we study pre-Abelian Nakaoka contexts in special categories, namely Abelian and triangulated categories.

In the setting of Abelian categories we prove the following:

\begin{nonlisting_thm}[\ref{thm_2_4}]
  Let $\mathbbm{t}=(\mathbbm{t}_1,\mathbbm{t}_2)$ Nakaoka context in an abelian category
  $\mathscr{A}$. Then, for
  $\mathcal{H}:=\mathcal{T}_1\cap\mathcal{F}_2$ the following statements are equivalent:
  \begin{enumerate}[label=(\alph*)]
    \item $\mathcal{H}$ is an abelian category.
    \item The following conditions hold:
      \begin{enumerate}[label=(\alph{enumi}\arabic*)]
        \item For any $f:H\to H'$ in $\mathcal{H}$, with $\Ker(f)\in\mathcal{F}_1$,
        we have that $\Ker(f)=0$.
        \item For any $f:H\to H'$ in $\mathcal{H}$, with $\Coker(f)\in\mathcal{T}_2$,
        we have that $\Coker(f)=0$.
        \item $\mathcal{H}$ is closed under kernels (resp. cokernels) of epimorphisms
        (resp. monomorphisms) in $\mathscr{A}$.
      \end{enumerate}
    \item $\mathcal{H}$ is closed under kernels and cokernels in $\mathscr{A}$.
  \end{enumerate}
\end{nonlisting_thm}

In triangulated categories we restrict our attention to special Nakaoka contexts built from t-structures, that we call \emph{related pairs}:
\begin{nonlisting_def}[\ref{def:related_pair}]
  Let $\t_1=(\T_1,\F_2)$ and $\t_2=(\T_2,\F_2)$ be two torsion pairs in a triangulated category $\cat{T}$.
  We will say that $\t$ is a \emph{related} pair if $(\T_1,\F_1[1])$ and $(\T_2,\F_2[1])$ are t-structures and
  $\mathcal{T}_1[1]\subseteq \mathcal{T}_2\subseteq\mathcal{T}_1$.
\end{nonlisting_def}

We show in Proposition~\ref{prop:2.5} that any related pair is a pre-Abelian Nakaoka context. The reason why we restrict to related pairs is that by requiring that our torsion pairs are in fact t-structures we can compute the pseudokernels and pseudocokernels using cones and cocones. This allows us to give a much easier formulation of the axioms \textbf{(CT.4)} and \textbf{(CT.4)$^\ast$}, in particular Lemma~\ref{lemma_rst_equiv} states that they are equivalent to the following:

\begin{nonlisting_def}[\ref{def:strongly_related}]
  A related pair $\mathbbm{t}=(\mathbbm{t}_1,\mathbbm{t}_2)$ in the triangulated category
  $\cat{T}$ is \emph{strong} if for any morphism $f:H_1\to H_2$, in $\mathcal{H}:=\mathcal{T}_1\cap\mathcal{F}_2$,
  and a distinguished triangle \[ V\to H_1\nto{f}H_2\to V[1],\] the following conditions
  hold true:
  \begin{relatedtorsion}
  \item $V\in\mathcal{F}_1$ implies $V\in\mathcal{F}_2[-1]$;
  \item $V\in\mathcal{T}_2$ implies $V\in\mathcal{T}_1[1]$.
  \end{relatedtorsion}

  We will call such pairs \emph{strongly related}.
\end{nonlisting_def}

We have an analogous of Theorem~\ref{thm_2_4}:
\begin{nonlisting_thm}[\ref{thm:2.9}]
  Let $\mathbbm{t}=(\mathbbm{t}_1,\mathbbm{t}_2)$ be a related pair in a triangulated
  category $\mathcal{T}$. Then, the following statements are equivalent.
  \begin{enumerate}[label=(\alph*)]
    \item \ref{ax:rst1} holds.
    \item For any monomorphism $\alpha:H_1\into H_2$, in the abelian category
      $\mathcal{H}_1:=\mathcal{T}_1\cap\mathcal{F}_1[1]$, with $H_1,H_2\in\mathcal{H}$,
      we have that $\Coker_{\mathcal{H}_1}(\alpha)\in\mathcal{H}$.
    \item $\mathcal{H}$ is closed under kernels and cokernels in the abelian category
      $\mathcal{H}_1$
    \item $\mathcal{H}$ is an abelian category.
    \item For any epimorphism $H\onto X$ in $\mathcal{H}_1$, with $H\in\mathcal{H}$,
      we have that $X\in\mathcal{H}$ (i.e. $\mathcal{H}$ is closed under quotients in $\mathcal{H}_1$).
  \end{enumerate}
\end{nonlisting_thm}

From the Theorem above and the related proofs, it is clear that there is some relation between the heart
$\clH$ of a related pair $\t=(\t_1,\t_2)$, and the heart $\clH_1=\T_1\cap\F_1[1]$ of the t-structure induced by $\t_1$.
This relation is made clear in Theorem~\ref{thm:2.11}, where we show that there is a bijection between the following two classes:

\[RtAb(\mathcal{T}):=\{\text{related pairs }\mathbbm{t}=(\mathbbm{t}_1,\mathbbm{t}_2)
\text{ in }\mathcal{T}\text{ s.t. }\mathcal{H}_\mathbbm{t}\text{ is abelian}\};\]
\begin{equation*}
  \mathrm{t-}stCoh(\mathcal{T}) :=
  \left\{
    \begin{array}{c}
      \text{pairs }(\overline{\mathbbm{t}}_1,\tau)\text{ s.t. }
      \overline{\mathbbm{t}}_1\text{ is a t-structure in }\mathcal{T}\text{ and }
      \tau\text{ is a}\\
      \text{ cohereditary torsion pair in the abelian category } \\
      \mathcal{H}_1:=\T_1\cap\F_1[1] \text{, where }\overline{\t}_1=(\T_1,\F_1[1])
    \end{array}
  \right\}.
\end{equation*}


%*****************************************
%*****************************************
%*****************************************
%*****************************************
%*****************************************
