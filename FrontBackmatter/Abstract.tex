%*******************************************************
% Abstract
%*******************************************************
%\renewcommand{\abstractname}{Abstract}
\pdfbookmark[1]{Abstract}{Abstract}
\begingroup
\let\clearpage\relax
\let\cleardoublepage\relax
\let\cleardoublepage\relax

\chapter*{Abstract}
We consider a complete hereditary cotorsion pair $(\A,\B)$ in a Grothendieck category $\G$ such that $\A$ contains a generator of finite projective dimension. The derived category $\D(\B)$ of the exact category $\B$ is defined as the quotient of the category $\Ch(\B)$, of unbounded complexes with terms in $\B$, modulo the subcategory $\wac{\B}$ consisting of the acyclic complexes with terms in $\B$ and cycles in $\B$.

We prove that there are recollements
\begin{equation*}
  \begin{tikzcd}
    \frac{ex\B}{\sim}\arrow{r}{inc}
    & \D(\B)\arrow[bend left=50]{l}\arrow[bend right=50]{l}\arrow{r}{Q}
    & \D(\G)\arrow[bend left=50]{l}\arrow[bend right=50]{l}
  \end{tikzcd}
\end{equation*}
and
\begin{equation*}
  \begin{tikzcd}
    \frac{ex\B}{\sim}\arrow{r}{inc}
    & K(\B)\arrow[bend left=50]{l}\arrow[bend right=50]{l}\arrow{r}{Q}
    & \D(\G).\arrow[bend left=50]{l}\arrow[bend right=50]{l}
  \end{tikzcd}
\end{equation*}

Then, we restrict our attention to the cotorsion pairs such that $\wac{\B}$ coincide with the class $ex\B$ of the acyclic complexes of $\Ch(\G)$ with terms in $\B$. In this case the derived category $\D(\B)$ fits into a recollement
\begin{equation*}
  \begin{tikzcd}
    \frac{ex\B}{\sim}\arrow{r}{inc}
    & K(\B)\arrow[bend left=50]{l}\arrow[bend right=50]{l}\arrow{r}{Q}
    & \D(\B).\arrow[bend left=50]{l}\arrow[bend right=50]{l}
  \end{tikzcd}
\end{equation*}

We will explore the conditions under which $ex\B=\wac{\B}$ and provide some examples. Symmetrically, we prove analogous results for the exact category $\A$.

We also introduce the notion of Nakaoka context in additive categories as couples $\t_i=(\T_i,\F_i)$ for $i=1,2$ of torsion pairs such that $\T_2\subseteq \T_1$. We give a set of axioms for a Nakaoka context in order to ensure that the heart $\clH:=\T_1\cap \F_2$ is Abelian. Then, we inspect the properties of Nakaoka contexts in Abelian and triangulated categories. In particular, we find a bijection between the t-structures $(\T_1,\F_1[1]), (\T_2,\F_2[1])$ such that $\T_1[1]\subseteq\T_2\subseteq\T_1$ whose heart $\clH:=\T_1\cap\F_2$ is Abelian and the cohereditary torsion pairs in $\clH_1:=\T_1\cap\F_1[1]$.

\endgroup

\cleardoublepage
\pdfbookmark[1]{Riassunto}{Riassunto}
\begingroup
\let\clearpage\relax
\let\cleardoublepage\relax
\let\cleardoublepage\relax


\chapter*{Riassunto}

Consideriamo un cotorsion pair completo ed ereditario $(\A,\B)$ in una categoria di Grothendieck $\G$ tale che $\A$ contenga un generatore di dimensione proiettiva finita. La categoria derivata $\D(\B)$ della categoria esatta $\B$ \`e definita come il quoziente fra la categoria $\Ch(\B)$ dei complessi illimitati a termini in $\B$ modulo la sottocategoria $\wac{\B}$ dei complessi aciclici a termini in $\B$ e cicli in $\B$.

Dimostriamo che esistono i recollement
\begin{equation*}
  \begin{tikzcd}
    \frac{ex\B}{\sim}\arrow{r}{inc}
    & \D(\B)\arrow[bend left=50]{l}\arrow[bend right=50]{l}\arrow{r}{Q}
    & \D(\G)\arrow[bend left=50]{l}\arrow[bend right=50]{l}
  \end{tikzcd}
\end{equation*}
e
\begin{equation*}
  \begin{tikzcd}
    \frac{ex\B}{\sim}\arrow{r}{inc}
    & K(\B)\arrow[bend left=50]{l}\arrow[bend right=50]{l}\arrow{r}{Q}
    & \D(\G).\arrow[bend left=50]{l}\arrow[bend right=50]{l}
  \end{tikzcd}
\end{equation*}

Successivamente restringiamo la nostra attenzione ai cotorsion pair tali che $\wac{\B}$ coincida con la classe $ex\B$ dei complessi aciclici di $\Ch(\G)$ con termini in $\B$. In questo caso la categoria derivata $\D(\B)$ appartiene a un recollement
\begin{equation*}
  \begin{tikzcd}
    \frac{ex\B}{\sim}\arrow{r}{inc}
    & K(\B)\arrow[bend left=50]{l}\arrow[bend right=50]{l}\arrow{r}{Q}
    & \D(\B).\arrow[bend left=50]{l}\arrow[bend right=50]{l}
  \end{tikzcd}
\end{equation*}

Studieremo le condizioni per cui $ex\B=\wac{\B}$ e mostreremo alcuni esempi. Simmetricamente dimostriamo risultati analoghi per la categoria esatta $\A$.

Inoltre, introduciamo la nozione di Nakaoka context in categorie additive come coppie di torsion pair $\t_i=(\T_i,\F_i)$ per $i=1,2$ tali che $\T_2\subseteq \T_1$. Daremo un insieme di assiomi per un Nakaoka context che garantisca l'abelianit\`a del cuore $\clH:=\T_1\cap\F_2$. Successivamente studieremo le propriet\`a dei Nakaoka context in categorie Abeliane e triangolate. In particolare troviamo una biezione tra le t-strutture $(\T_1,\F_1[1]), (\T_2,\F_2[1])$ tali che $\T_1[1]\subseteq\T_2\subseteq\T_1$ il cui cuore $\clH:=\T_1\cap\F_2$ sia abeliano e le torsion pair coereditarie in $\clH_1:=\T_1\cap\F_[1]$.


\endgroup


\vfill
