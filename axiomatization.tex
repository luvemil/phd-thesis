\clearpage

\section{Axiomatization}

\begin{definition}
  Let $\mathcal{X}$ be an additive category with pseudokernels and pseudocokernels, a \emph{compatible}
  torsion pair $\mathbb{t}=(\mathbb{t}_1,\mathbb{t}_2)$ in $\mathcal{X}$ consists of the two pairs
  $\mathbb{t}_1=(\mathcal{T}_1,\mathcal{F}_1)$ and $\mathbb{t}_2=(\mathcal{T}_2,\mathcal{F}_2)$ of full subcategories
  of $\mathcal{X}$ satisfying the following axioms:

  \begin{torsionaxioms}
    \item\label{ax:ct1} $\mathbb{t}_1=(\mathcal{T}_1,\mathcal{F}_1)$ and $\mathbb{t}_2=(\mathcal{T}_2,\mathcal{F}_2)$ are respectively a left functorial and a
    right functorial torsion pair
    \item\label{ax:ct2} $\mathcal{T}_2\subseteq \mathcal{T}_1$ (equiv. $\mathcal{F}_1\subseteq\mathcal{F}_2$)
    \item\label{ax:ct3} Any morphism $g:T_1\to T_1'$ in $\mathcal{T}_1$ iadmits a pseudocokernel $g^C:T_1'\to\PCok_\mathcal{X}(g)$,
    with $T_1'':=\PCok_\mathcal{X}(g)\in\mathcal{T}_1$, such that
    \begin{equation*}
      \begin{tikzcd}
        0\arrow{r} &(T_1'',-)_{|\mathcal{F}_2}\arrow{r}{(g^C,-)}
        & (T_1',-)_{|\mathcal{F}_2}\arrow{r}{(g,-)}& (T_1,-)_{|\mathcal{F}_2}
      \end{tikzcd}
    \end{equation*}
    is an exact sequence of functors.
    \varitem{^\ast}\label{ax:ct3op} Dual of \ref{ax:ct3}. \ref{ax:ct3op}
  \end{torsionaxioms}
\end{definition}

\begin{notation}
  In the case of a compatible torsion pair $\mathbb{t}=(\mathbb{t}_1,\mathbb{t}_2)$ in $\mathcal{X}$,
  we have the adjoint pairs
  \begin{equation*}
    \begin{tikzcd}
      (i_1,t_1):\mathcal{T}_1\arrow[hook, shift left]{r}{i_1}
        & \mathcal{X}\arrow[shift left]{l}{t_1}
        & \mathrm{and}
        & (t_2,j_2):\mathcal{F}_2\arrow[hook, shift right]{r}[']{j_2}
          & \mathcal{X}\arrow[shift right]{l}[']{t_2}
    \end{tikzcd}
  \end{equation*}

  In this case there are also the counit $\varepsilon_{1,M}:t_1(M)\to M$ and the
  unit $\lambda_{2,M}:M\to f_2(M)$. The heart of $\mathbb{t}$ is defined as
  $\mathcal{H}=\mathcal{H}_\mathbb{t}:=\mathcal{T}_1\cap\mathcal{F}_2$.
\end{notation}

\begin{lemma}\label{lem:1.2}
  Let $\mathbb{t}=(\mathbb{t}_1,\mathbb{t}_2)$ be a compatible torsion pair in $\mathcal{X}$.
  Then the following statements hold true.
  \begin{enumerate}[label=(\alph*)]
    \item $\mathcal{F}_1\cap\mathcal{T}_2=0$,
    \item $f_2(\mathcal{T}_1)\subseteq\mathcal{H}$ and $t_1(\mathcal{F}_2)\subseteq\mathcal{H}$.
  \end{enumerate}
\end{lemma}

\begin{proof}
  \begin{enumerate}[label=(\alph*)]
    \item Since $\mathcal{T}_2\subseteq\mathcal{T}_1$, we have $\mathcal{F}_1\cap\mathcal{T}_2\subseteq \mathcal{F}_1\cap\mathcal{T}_1=0$.
    \item Let $T_2\in\mathcal{T}_2$ and $F_2\in\mathcal{F}_2$. Then,
    \begin{equation*}
      \myhom{\mathcal{X}}{T_2,t_1(F_2)}\cong \myhom{\mathcal{X}}{T_2,F_2}=0.
    \end{equation*}
    Hence, $t_1(F_2)\in\mathcal{T}_2^\perp = \mathcal{F}_2$. An analogous proof shows that $f_2(\mathcal{T}_1)\subseteq\mathcal{T}_1$.
  \end{enumerate}
\end{proof}

\begin{prop}\label{prop:1.3}
  Let $\mathbb{t}=(\mathbb{t}_1,\mathbb{t}_2)$ be a compatible torsion pair in $\mathcal{X}$ and
  $f:H_1\to H_2$ be a morphism in $\mathcal{H}$. Then
  \begin{enumerate}[label=(\alph*)]
    \item\label{prop:1.3:a} if $f^C:H_2\to T_1$ is the pseudocokernel of $f$, given by \ref{ax:ct3}
    and $\lambda_{2,T_1}:T_1\to f_2(T_1)$, then
    \begin{equation*}
      \Coker(H_1\nto{f}H_2) = (H_2\nto{\tilde{f}^C}f_1(T_1))
    \end{equation*}
    in $\mathcal{H}$, where $\tilde{f}^C:=\lambda_{2,T_1}\circ f^C$;
    \item\label{prop:1.3:b} if $f^K:F_2\to H_1$ is the pseudokernel of $f$, given by \ref{ax:ct3}$^\ast$
    and $\varepsilon_{1,F_2}:t_1(F_2)\to F_2$, then
    \begin{equation*}
      \Ker(H_1\nto{f} H_2) = (t_1(F_2)\nto{\tilde{f}^K}H_1)
    \end{equation*}
    in $\mathcal{H}$, where $\tilde{f}^K=f^K\circ \varepsilon_{1,F_2}$.
  \end{enumerate}
\end{prop}

\begin{proof}
  \begin{enumerate}[label=(\alph*)]
    \item Let $g:H_2\to H$ in $\mathcal{H}$ such that $gf=0$. And consider the solid part of
    the following diagram
    \begin{equation*}
      \begin{tikzcd}
        H_1\arrow{r}{f}
          & H_2\arrow{r}{f^C}\arrow{d}{g}
            & T_1\arrow[dashed]{dl}[description]{f'}\arrow{r}{\lambda_{2,T_1}}
              & f_2(T_1)\arrow[dashed]{dll}[description]{f''}\\
          & H
            & &
      \end{tikzcd}
    \end{equation*}
    with $T_1\in\mathcal{T}_1$ and $f_2(T_1)\in\mathcal{H}$ (by \cref{lem:1.2}).
    Since $H\in\mathcal{H}\subseteq \mathcal{F}_2$, by \ref{ax:ct3} there is a
    unique $f':T_1\to H$ such that $f'f^C=g$. Since $\mathbb{t}_2$ is right functorial,
    there is a $f'':f_2(T_1)\to H$ making the diagram commute. Hence, $g=f''\circ \tilde{f}^C$.

    As for unicity, let $r:f_2(T_1)\to H$, such that $g=r\circ \tilde{f}^C$. Then,
    $(f''\circ\lambda_{2,T_1})\circ f^C = (r\circ \lambda_{2,T_1})\circ f^C$,
    so $f''\circ \lambda_{2,T_1}=r\circ \lambda_{2,T_1}$ by $\ref{ax:ct3}$, and $f''=r$.
    \item Dual.
  \end{enumerate}
\end{proof}

\begin{prop}\label{prop:1.4}
  Let $\mathbb{t}=(\mathbb{t}_1,\mathbb{t}_2)$ be a compatible torsion pair in $\mathcal{X}$. Then, for any $f:H\to H'$
  in $\mathcal{H}$ we have that:
  \begin{enumerate}[label=(\alph*)]
    \item\label{prop:1.4:a} $f$ is a monomorphism in $\mathcal{H}$ if and only if there is a pseudokernel
    $\PKer_{\mathcal{F}_2}(f)\in\mathcal{F}_1$;
    \item\label{prop:1.4:b} $f$ is an epimorphism in $\mathcal{H}$ if and only if there is a pseudocokernel
    $\PCok_{\mathcal{T}_1}(f)\in\mathcal{T}_2$.
  \end{enumerate}
\end{prop}

\begin{proof}
  \begin{enumerate}[label=(\alph*)]
    \item Let $f:H\to H'$ be a morphism in $\mathcal{H}$. By \cref{prop:1.3:b} there is a diagram
    \begin{equation*}
      \begin{tikzcd}
        F_2\arrow{r}{f^K}
          & H\arrow{r}{f}
            & H'\\
        t_1{F_2}\arrow{u}{\varepsilon_{1,F_2}}\arrow{ur}{\tilde{f}^K}
      \end{tikzcd}
    \end{equation*}
    with $\Ker(H\nto{f} H') = (t_1(F_2)\nto{\tilde{f}^K}H)$.

    Assume that $f$ is a monomorphism, then $t_1(F_2)=0$, so $0=\myhom{\mathcal{X}}{Z,t_1(F_2)}\cong
    \myhom{\mathcal{X}}{Z,F_2}$ for all $Z\in\mathcal{T}_1$. Hence, $F_2\in\mathcal{F}_1$.

    Conversely, let $f':F_1\to H$ be a pseudokernel of $f$ in $\mathcal{F}_2$, with $F_1\in\mathcal{F}_1$. Consider the
    solid part of the diagram
    \begin{equation*}
      \begin{tikzcd}
        t_1(F_2)\arrow{d}{\varepsilon_{1,F_2}}\arrow{dr}{\tilde{f}^K}
        & &\\
      F_2\arrow[dashed,shift right]{d}[']{u}\arrow{r}{f^K}
        & H\arrow{r}{f}
          & H'\\
      F_1\arrow[dashed,shift right]{u}[']{v}\arrow{ur}{f'}
        & &.
      \end{tikzcd}
    \end{equation*}
    Since $f^K$ and $f'$ are pseudokernel of $f$ in $\mathcal{F}_2$, there exist $u$ and $v$ such
    that $f'=f^K\circ v$ and $f^K=f'\circ u$. Therefore, $f^K=f^K\circ v\circ u$, and so
    $\tilde{f}^K=f^K\circ\varepsilon_{1,F_2}=f^K\circ v\circ u\circ\varepsilon_{1,F_2}$. Notice
    that $u\circ \varepsilon_{1,F_2}=0$ since $\myhom{\mathcal{X}}{\mathcal{T}_1,\mathcal{F}_1}=0$.
    Thus, $f$ has the zero morphism as its kernel in $\mathcal{H}$, i.e. $f$ is a monomorphism.
    \item Dual.
  \end{enumerate}
\end{proof}

\begin{definition}
  A compatible torsion pair $\mathbb{t}=(\mathbb{t}_1,\mathbb{t}_2)$ in $\mathcal{X}$
  is strong if the following axioms hold:
  \begin{torsionaxioms}
    \setcounter{enumi}{3}
    \item Let $f:H_1\to H_2$ in $\mathcal{H}$ be such that there is a pseudokernel
    $\PKer_{\mathcal{F}_2}(f)\in\mathcal{F}_1$. Then, for the commutative diagram
    \begin{equation*}
      \begin{tikzcd}
        & H_1\arrow{d}{a}\arrow{r}{f}
          & H_2\arrow[equal]{d}\arrow{r}{f^C}
            & T_1:=\PCok_\mathcal{X}(f)\arrow{d}{\lambda_{2,T_1}}\\
        t_1(F)_2\arrow{r}{\varepsilon_{1,F_2}}
        & F_2:=\PKer_\mathcal{X}(g)\arrow{r}{g^K}
          & H_2\arrow{r}{g}
            & f_2(T_1)
      \end{tikzcd}
    \end{equation*}
    there exists a morphism $b:t_1(F_2)\to H_1$ such that $ab=\varepsilon_{1,F_2}$.
    \varitem{^\ast} Dual
  \end{torsionaxioms}
\end{definition}

With these axioms we can prove that the heart has kernels and cokernels.

\begin{thm}
  Let $\mathbb{t}=(\mathbb{t}_1,\mathbb{t}_2)$ be a compatible torsion pair in $\mathcal{X}$. Then, the following are
  equivalent:
  \begin{enumerate}[ref=(a)]
    \item $\mathbb{t}$ is strong,
    \item $\mathcal{H}$ is an abelian category.
  \end{enumerate}
\end{thm}

\subsection{The case of an abelian category}

Let's consider the case $\mathcal{X}=\mathcal{A}$ of an Abelian category with
two torsion pairs $\mathbb{t}_i=(\mathcal{T}_i,\mathcal{F}_i)$ for $i=1,2$.
Consider $\mathbb{t}=(\mathbb{t}_1,\mathbb{t}_2)$.

\begin{rmk}\label{rmk:2.2}
  In the case of an Abelian category $\mathcal{X}=\mathcal{A}$, we have that
  $\mathbb{t}$ is compatible if and only if $\mathcal{T}_2\subseteq\mathcal{T}_1$.
\end{rmk}

\begin{proof}
  Let $\mathcal{T}_2\subseteq\mathcal{T}_1$, we need to show that \cref{ax:ct1},
  \cref{ax:ct3} and \cref{ax:ct3op} hold.
  \begin{torsionaxioms}
    \item It is well known that any torsion pair in an abelian category is functorial.
    \setcounter{enumi}{2}
    \item Let $g:T_1\to T_1'$ be a morphism in $\mathcal{T}_1$. Consider the cokernel morphism
    of $g$ in $\mathcal{A}$
    \begin{equation*}
      \Coker_\mathcal{A}(T_1\nto{g}T_1')=(T_1'\nto{c_g}\Coker(g)).
    \end{equation*}
    Since $\mathcal{T}_1$ is closed under quotient objects, we get that $\Coker(g)\in\mathcal{T}_1$.
    Therefore, we can choose $c_g:T_1'\to\Coker(g)$ as $g^C:T_1'\to\PCok_\mathcal{A}(g)$.\todo{Is there an explicit choice
    that we made somewhere before when we talk about $\PCok_\mathcal{A}$?}
    \varitem{^\ast} Anologous to the previous.
  \end{torsionaxioms}
\end{proof}

\begin{corollary}\label{cor:2.3}
  Let $\mathbb{t}=(\mathbb{t}_1,\mathbb{t}_2)$ be a torsion pair in $\mathcal{A}$ with $\mathcal{T}_2\subseteq\mathcal{T}_1$.
  Then, for $f:H_1\to H_2$ in $\mathcal{H}$, the following statements hold:
  \begin{enumerate}[label=(\alph*),ref=(\alph*)]
    \item\label{cor:2.3:a} the cokernel of $f$ in $\mathcal{H}$ is the composition of the morphisms
      \begin{equation*}
        \begin{tikzcd}
          H_2\arrow{r}{c_f}&\Coker(f)\arrow{r}{\lambda_{2,\Coker(f)}}&f_2(\Coker(f));
        \end{tikzcd}
      \end{equation*}
    \item\label{cor:2.3:b} the kernel of $f$ in $\mathcal{H}$ is the composition of the morphisms
      \begin{equation*}
        \begin{tikzcd}
          t_1(\Ker(f))\arrow{r}{\varepsilon_{1,\Ker(f)}}&\Ker(f)\arrow{r}{k_f}&H_1;
        \end{tikzcd}
      \end{equation*}
    \item\label{cor:2.3:c} $f$ is an epimorphism in $\mathcal{H}$ if and only if $\Coker(f)\in\mathcal{T}_2$;
    \item\label{cor:2.3:d} $f$ is a monomorphism in $\mathcal{H}$ if and only if $\Ker(f)\in\mathcal{F}_1$.
  \end{enumerate}
\end{corollary}

\begin{proof}
  \ref{cor:2.3:a} and \ref{cor:2.3:b} follow from the proof of \cref{rmk:2.2}\todo{this should be a remark, fix it}
  and \cref{prop:1.3}.

  \ref{cor:2.3:c}
  \begin{enumerate}
    \item[$\Leftarrow$] is trivial.
    \item[$\Rightarrow$] By \cref{prop:1.4} \ref{prop:1.4:b} there exists $f^C:H_2\to T_2$, where
      $T_2=\Coker_{T_1}(f)\in\mathcal{T}_2$. Then, we have

      \begin{minipage}[b]{0.45\linewidth}
        \begin{equation*}
          \begin{tikzcd}[column sep=tiny]
            H_2\arrow{rr}{f^C}\arrow{dr}[']{c_f}
              & & T_2 \arrow[dashed, shift right]{ld}[']{u}\\
              & \Coker(f)\arrow[dashed, shift right]{ur}[']{v}
          \end{tikzcd}
        \end{equation*}
      \end{minipage}
      \begin{minipage}[b]{0.45\linewidth}
        \begin{equation*}
          \text{such that }
          \left\{
          \begin{array}{c}
            uf^C = c_f,\\
            vc_f = f^C.
          \end{array}
          \right.
        \end{equation*}
      \end{minipage}

      Hence, $uvc_f=c_f$, but $c_f$ is epi, therefore $uv=1$. Hence, $\Coker(f)$ is a direct
      summand of $T_2\in\mathcal{T}_2$ so $\Coker(f)\in\mathcal{T}_2$.
  \end{enumerate}

  \ref{cor:2.3:d} Similar to the previous proof.
\end{proof}

\begin{thm}
  Let $\mathbb{t}_i=(\mathcal{T}_i,\mathcal{F}_i)$ be a torsion pair in an abelian category
  $\mathcal{A}$, for $i=1,2$, such that $\mathcal{T}_2\subseteq \mathcal{T}_1$. Then, for
  $\mathcal{H}:=\mathcal{T}_1\cap\mathcal{F}_2$ the following statements are equivalent:
  \begin{enumerate}[label=(\alph*)]
    \item $\mathcal{H}$ is an abelian category.
    \item The following conditions hold:
      \begin{enumerate}[label=(\alph{enumi}\arabic*)]
        \item For any $f:H\to H'$ in $\mathcal{H}$, with $\Ker(f)\in\mathcal{F}_1$,
        we have that $\Ker(f)=0$.
        \item For any $f:H\to H'$ in $\mathcal{H}$, with $\Coker(f)\in\mathcal{T}_2$,
        we have that $\Coker(f)=0$.
        \item $\mathcal{H}$ is closed under kernels (resp. cokernels) of epimorphisms
        (resp. monomorphisms) in $\mathcal{A}$.
      \end{enumerate}
    \item $\mathcal{H}$ is closed under kernels and cokernels in $\mathcal{A}$.
  \end{enumerate}
\end{thm}

\subsection{Related torsion pairs in triangulated categories}

Let $\mathcal{X}=\mathcal{T}$ be a triangulated category on which idempotents split.
We start by recalling the definition of a t-structure in $\mathcal{T}$.

\begin{definition}
  A pair $(\mathcal{A},\mathcal{B})$ of full subcategories of $\mathcal{T}$ is a t-structure
  in $\mathcal{T}$ if
  \begin{enumerate}[label=(\alph*)]
    \item $\mathbb{t}=(\mathcal{A},\mathcal{B}[-1])$ is a torsion pair in $\mathcal{T}$, and
    \item $\mathcal{A}[1]\subseteq \mathcal{A}$.
  \end{enumerate}
\end{definition}

\begin{rmk}
  It is well known that any t-structure $(\mathcal{A},\mathcal{B})$ in $\mathcal{T}$
  gives a functional torsion pair $\mathbb{t}=(\mathcal{A},\mathcal[-1])$ and
  $\mathbb{B}[-1]\subseteq\mathcal{B}$. Furthermore, $\mathcal{A}$ and
  $\mathcal{B}$ are closed under extensions and direct summands. Note that the t-structure
  $(\mathcal{A},\mathcal{B})$ depends only on $\mathcal{A}$, since $\mathcal{B}=\mathcal{A}^\perp[1]$.
\end{rmk}

\begin{definition}
  A \emph{related} torsion pair $\mathbb{t}=(\mathbb{t}_1,\mathbb{t}_2)$ in triangulated category
  $\mathcal{T}$ consists of the torsion pairs $\mathbb{t}_1=(\mathcal{T}_1,\mathcal{F}_1)$
  and $\mathbb{t}=(\mathcal{T}_2,\mathcal{F}_2)$ in $\mathcal{T}$ such that
  $\mathcal{T}_1[1]\subseteq \mathcal{T}_2\subseteq\mathcal{T}_1$.
\end{definition}

\begin{prop}
  Let $\mathbb{t}=(\mathbb{t}_1,\mathbb{t}_2)$ be a related torsion pair in $\mathcal{T}$. Then
  \begin{enumerate}[label=(\alph*)]
    \item $(\mathcal{T}_1,\mathcal{F}_1[1])$ and $(\mathcal{T}_2,\mathcal{F}_2[1])$
    are t-structures in $\mathcal{T}$;
    \item $\mathbb{t}$ is a compatible torsion pair in $\mathcal{T}$;
    \item the heart $\mathcal{H}_\mathbb{t} := \mathcal{T}_1\cap\mathcal{F}_2[1]$ is a preabelian category.
  \end{enumerate}
\end{prop}

\begin{definition}
  A related torsion pair $\mathbb{t}=(\mathbb{t}_1,\mathbb{t}_2)$ in the triangulated category
  $\mathcal{T}$ is \emph{strong} if for any morphism $f:H_1\to H_2$, in $\mathcal{H}:=\mathcal{T}_1\cap\mathcal{F}_2$,
  and a distinguished triangle $Z\to H_1\nto{f}H_2\to Z[1]$, the following conditions
  hold true
  \begin{relatedtorsion}
    \item $Z\in\mathcal{F}_1$ if and only if $Z\in\mathcal{F}_2[-1]$;
    \item $Z[1]\in\mathcal{T}_2$ if and only if $Z\in\mathcal{T}_1$.
  \end{relatedtorsion}
\end{definition}

\begin{thm}\label{thm:2.6}
  Let $\mathbb{t}=(\mathbb{t}_1,\mathbb{t}_2)$ be a strongly related torsion pair in
  the triangulated category $\mathcal{T}$. Then, the heart $\mathcal{H}=\mathcal{H}_\mathbb{t}$
  is an abelian category.
\end{thm}

\begin{example}
  Let $(\mathcal{A},\mathcal{B})$ be a t-structure in $\mathcal{T}$. Consider
  $\mathbb{t}_1:=(\mathcal{A},\mathcal{B}[-1])$ and $\mathbb{t}_2 :=(\mathcal{A}[1],\mathcal{B})$.
  It is not hard to see that $\mathbb{t}=(\mathbb{t}_1,\mathbb{t}_2)$ is a strongly related
  torsion pair in $\mathcal{T}$. In this case, by \cref{thm:2.6}, we get that $\mathcal{H}=\mathcal{A}\cap\mathcal{B}$
  is an abelian category (BBD theorem).
\end{example}

\begin{example}
  Let $R$ be any (associative with 1) ring. Consider the triangulated category $\mathcal{T}:=\mathcal{D}(R)$.
  The derived category $\mathcal{D}(R)$ has the so called natural t-structure
  $(\mathcal{D}^{\leq 0}(R),\mathcal{D}^{\geq}(R))$ where
  \begin{align*}
    \mathcal{D}^{\leq 0}(R) &:= \{ X\in\mathcal{D}(R) \,|\,H^i(X)=0\text{ for } i>0\},\\
    \mathcal{D}^{\geq 0}(R) &:= \{ X\in\mathcal{D}(R) \,|\,H^i(x)=0\text{ for } i<0\}.
  \end{align*}

  For any ideal $I\trianglelefteq R$, we have the TTF-triple $(\mathcal{C}_I,\mathcal{T}_I,\mathcal{F}_I)$
  associated to $I$, where
  \begin{align*}
    \mathcal{C}_I &:= \{M\in\Mod{R}\,|\,IM=M\},\\
    \mathcal{T}_I &:= \{M\in\Mod{R}\,|\,IM=0\}\cong \Mod{\frac{R}{I}},\\
    \mathcal{F}_I &:= \{M\in\Mod{R}\,|\,Ix=0\text{ and }x\in M\Rightarrow x=0\}.
  \end{align*}

  Consider the t-structure (Happel-Reiten-Smalo) $(\mathcal{D}^{\leq 0}_{t_I}(R), \mathcal{D}^{\geq 0}_{t_I}(R))$
  associated to the torsion pair $t_I=(\mathcal{C}_I,\mathcal{T}_I)$, where
  \begin{align*}
    \mathcal{D}^{\leq 0}_{t_I}(R) &:= \{ X\in\mathcal{D}^{\leq 0}(R)\,|\, H^0(X)\in\mathcal{C}_I \},\\
    \mathcal{D}^{\geq 0}_{t_I}(R) &:= \{ X\in\mathcal{D}^{\geq 0}(R)\,|\, H^0(X)\in\mathcal{T}_I \}.
  \end{align*}

  It can be seen that $\mathbb{t}=(\mathbb{t}_1,\mathbb{t}_2)$ where
  $\mathbb{t}_1:=(\mathcal{D}^{\leq 0}(R),\mathcal{D}^{\geq 1}(R))$ and
  $\mathbb{T}_2:=(\mathcal{D}^{\leq 0}_{t_I}(R),\mathcal{D}^{\geq 1}_{t_I}(R))$, is a strongly
  related torsion pair in $\mathcal{T}=\mathcal{D}(R)$.
\end{example}

We recall the following bijection given by A. Polishchuk, and in order to do that,
for a t-structure $(\mathcal{U}_1,\mathcal{U}_1^\perp[1])$ in $\mathcal{T}$, we have the cohomological
functor $H^0_1:\mathcal{T}\to \mathcal{H}_1:=\mathcal{U}_1\cap\mathcal{U}_1^\perp[1]$
($\mathcal{H}_1$ is an abelian category).

\begin{prop}[Polishchuk]\label{prop:2.7}
  Let $(\mathcal{U}_1,\mathcal{U}_1^\perp[1])$ be a t-structure in a triangulated category.
  Then we have a bijection (Polishchuk's bijection)
  \begin{equation*}
    \begin{tikzcd}
      \left\{
      \begin{array}{c}
        \text{torsion pairs in} \\
        \mathcal{H}_1=\mathcal{U}_1\cap \mathcal{U}_1^\perp[1]
      \end{array}
      \right\}
      \arrow[leftrightarrow]{r}{\mathrm{Pol}_{\mathcal{H}_1}}
        &
        \left\{
          \begin{array}{c}
            \text{t-structures }
            (\mathcal{U}_2,\mathcal{U}_2^\perp) \\
            \text{ in } \mathcal{D}
            \text{ satisfying } \\ \mathcal{U}_1[1]\subseteq \mathcal{U}_2\subseteq\mathcal{U}_1
          \end{array}
        \right\}\\
      (\mathcal{X},\mathcal{Y})\arrow[mapsto]{r}
        & (\mathcal{U}_2,\mathcal{U}_2^\perp[1])\\
      (\mathcal{U}_2\cap\mathcal{H}_1,\mathcal{U}_2^\perp\cap\mathcal{H}_1)\arrow[mapsfrom]{r}
        & (\mathcal{U}_2,\mathcal{U}_2^\perp[1])
    \end{tikzcd}
  \end{equation*}
  where
  \begin{align*}
    \mathcal{U}_2 = \{X\in\mathcal{U}_1\,|\,H^0_1(X)\in\mathcal{X}\}\\
    \mathcal{U}_2^\perp = \{Y\in\mathcal{U}_1^\perp\,|\,H^0_1(Y)\in\mathcal{Y}\}.
  \end{align*}
\end{prop}

\begin{rmk}\label{rmk:2.8}
  \begin{enumerate}[label=(\arabic*)]
    \item\label{rmk:2.8:1} Note that $\mathrm{Pol}^{-1}_{\mathcal{H}_1}(\mathcal{U}_2,\mathcal{U}^\perp_2[1])=
      (\mathcal{U}_2\cap\mathcal{U}_1^\perp[1],\mathcal{H})$, where
      $\mathcal{H}:=\mathcal{U}_1\cap\mathcal{U}_2^\perp$.

    \item By \ref{rmk:2.8:1}, it follows that $\mathcal{H}$ is a torsion free class in the abelian category
      $\mathcal{H}_1:=\mathcal{U}_1\cap\mathcal{U}_1^\perp[1]$.
  \end{enumerate}
\end{rmk}

\begin{thm}\label{thm:2.9}
  Let $\mathbb{t}=(\mathbb{t}_1,\mathbb{t}_2)$ be a related torsion pair in a triangulated
  category $\mathcal{T}$. Then, the following statements are equivalent.
  \begin{enumerate}[label=(\alph*)]
    \item For any distinguished triangle $V\to H_1\nto{f}H_2\nto{+}$, with
      $f$ a morphism in $\mathcal{H}=\mathcal{H}_\mathbb{t}:=\mathcal{T}_1\cap\mathcal{F}_2$,
      we have that
      \begin{equation*}
        V\in\mathcal{F}_1 \Rightarrow V[1]\in\mathcal{F}_2.
      \end{equation*}
    \item For any monomorphism $\alpha:H_1\into H_2$, in the abelian category
      $\mathcal{H}_1:=\mathcal{T}_1\cap\mathcal{F}_1[1]$, with $H_1,H_2\in\mathcal{H}$,
      we have that $\Coker_{\mathcal{H}_1}(\alpha)\in\mathcal{H}$.
    \item $\mathcal{H}$ is closed under kernels and cokernels in the abelian category
      $\mathcal{H}_1$
    \item $\mathcal{H}$ is an abelian category.
    \item For any epimorphism $H\onto X$ in $\mathcal{H}_1$, with $H\in\mathcal{H}$,
      we have that $X\in\mathcal{H}$ (i.e. $\mathcal{H}$ is closed under quotients in $\mathcal{H}_1$).
  \end{enumerate}
\end{thm}

Let $t=(\mathcal{A},\mathcal{B})$ be a pair of full subcategories of the triangulated
category $\mathcal{T}$. We will use the following notation:
\begin{align*}
  t[1] &:= (\mathcal{A}[1],\mathcal{B}[1]),\\
  \overline{t} &:= (\mathcal{A},\mathcal{B}[1]).
\end{align*}

Note that $\overline{t}$ is a t-structure in $\mathcal{T}$ if and only if $t$ is a torsion
pair $\mathcal{T}$ such that $\mathcal{A}[1]\subseteq\mathcal{A}$.

\begin{rmk}
  Consider $\mathbb{t}=(\mathbb{t}_1,\mathbb{t}_2)$, where $\mathbb{t}_i:=(\mathcal{U}_i,\mathcal{U}_i^\perp)$ for
  $i=1,2$. We have
  \begin{enumerate}
    \item $\mathcal{H}_\mathbb{t}:=\mathcal{U}_1\cap\mathcal{U}_2^\perp$,
      $\mathcal{H}_i:=\mathcal{U}_i\cap\mathcal{U}_i^\perp[1]$,
    \item $\mathbb{t}' :=(\mathbb{t}_2,\mathbb{t}_1[1])$
  \end{enumerate}

  Note that
  \begin{enumerate}
    \setcounter{enumi}{2}
    \item $\mathbb{t}=(\mathbb{t}_1,\mathbb{t}_2)$ is a related torsion pair in $\mathcal{T}$
      \begin{align*}
        &\Leftrightarrow \mathcal{U}_1[1]\subseteq\mathcal{U}_2\subseteq\mathcal{U}_1\\
        &\Leftrightarrow \mathcal{U}_2[1]\subseteq\mathcal{U}_1[1]\subseteq\mathcal{U}_2\\
        &\Leftrightarrow \mathbb{t}'=(\mathbb{t}_2,\mathbb{t}_1[1]) \text{ is a related torsion pair in }\mathcal{T}.
      \end{align*}
    \item Let $\mathbb{t}=(\mathbb{t}_1,\mathbb{t}_2)$ is a related torsion pair in $\mathcal{T}$. In
      this case, we have
      \begin{align*}
        &\mathcal{H}_\mathbb{t} = \mathcal{U}_1\cap\mathcal{U}_2^\perp,~
        \mathcal{H}_{\mathbb{t}'}=\mathcal{U}_2\cap \mathcal{U}_1^\perp[1],\\
        & \mathrm{Pol}_{\mathcal{H}_1}^{-1}(\overline{\mathbb{t}}_2)=
        \mathrm{Pol}_{\mathcal{H}_1}^{-1}(\mathcal{U}_2,\mathcal{U}_2^\perp[1])=
        (\mathcal{H}_{\mathbb{t}'},\mathcal{H}_{\mathbb{t}}),\\
        &\mathrm{Pol}_{\mathcal{H}_2}^{-1}(\overline{\mathbb{t}}_1[1])=
        \mathrm{Pol}_{\mathcal{H}_1}^{-1}(\mathcal{U}_1[1],\mathcal{U}_1^\perp[2])=
        (\mathcal{H}_{\mathbb{t}}[1],\mathcal{H}_{\mathbb{t}'}).
      \end{align*}

      Thus, $(\mathcal{H}_{\mathbb{t}'},\mathcal{H}_\mathbb{t})$ is a torsion pair in the abelian
      category $\mathcal{H}_1$,
      $(\mathcal{H}_\mathbb{t}[1],\mathcal{H}_{\mathbb{t}'})$ is a torsion pair in the abelian category
      $\mathcal{H}_2$.
  \end{enumerate}
\end{rmk}

\begin{corollary}
  Let $\mathbb{t}=(\mathbb{t}_1,\mathbb{t}_2)$ be a related torsion pair in a triangulated category
  $\mathcal{T}$. Then, the following statements are equivalent:
  \begin{enumerate}[label=(\alph*)]
    \item For any distinguished triangle $V\to H_1\nto{f}H_2\nto{+}$, with
    $f$ a morphism in $\mathcal{H}_{\mathbb{t}'}=\mathcal{T}_2\cap\mathcal{F}_1[1]$,
    we have that $V\in\mathcal{F}_2$ implies $V\in\mathcal{F}_1$.
    \item For any monomorphism $\alpha:H_1\into H_2$, in the abelian category
    $\mathcal{H}_2:=\mathcal{T}_2\cap\mathcal{F}_2[1]$, with $H_1,H_2\in\mathcal{H}_{\mathbb{t}'}$,
    we have that $\Coker_{\mathcal{H}_2}(\alpha)\in\mathcal{H}_{\mathbb{t}'}$.
    \item $\mathcal{H}_{\mathbb{t}'}$ is closed under kernels and cokernels in
    the abelian category $\mathcal{H}_2$.
    \item $\mathcal{H}_{\mathbb{t}'}$ is an abelian category.
    \item $\mathcal{H}_{\mathbb{t}'}$ is closed under quotients in $\mathcal{H}_2$.
  \end{enumerate}
\end{corollary}

We recall that a torsion pair $(\mathcal{T},\mathcal{F})$ in an abelian category
$\mathcal{A}$ is cohereditary if the class $\mathcal{F}$ is closed under quotients
in $\mathcal{A}$.

\begin{definition}
  For a triangulated category $\mathcal{T}$, we consider the following classes:
  \begin{enumerate}
    \item $RtAb(\mathcal{T}):=\{\text{related torsion pairs }\mathbb{t}=(\mathbb{t}_1,\mathbb{t}_2)
      \text{ in }\mathcal{T}\text{ s.t. }\mathcal{H}_\mathbb{t}\text{ is abelian}\}$;
    \item
      \begin{align*}
        \mathrm{t-}stCoh(\mathcal{T}) &:=
        \left\{
        \begin{array}{c}
          \text{pairs }(\overline{\mathbb{t}}_1,\tau)\text{ s.t. }
          \overline{\mathbb{t}}_1\text{ is a t-structure in }\mathcal{T}\text{ and }
          \tau\text{ is a}\\
          \text{ cohereditary torsion pair in the abelian category } \\
          \mathcal{H}_1:=\mathcal{U}_1\cap\mathcal{U}_1^\perp[1]
        \end{array}
        \right\};
      \end{align*}
    \item[1\rlap{$^\prime$}.] $RtAb^\prime(\mathcal{T}):=\{\text{related torsion pairs }\mathbb{t}=(\mathbb{t}_1,\mathbb{t}_2)
      \text{ in }\mathcal{T}\text{ s.t. }\mathcal{H}_{\mathbb{t}'}\text{ is abelian}\}$;
    \item[2\rlap{$^\prime$}.]
      \begin{align*}
        \mathrm{t-}stCoh^\prime(\mathcal{T}) &:=
        \left\{
        \begin{array}{c}
          \text{pairs }(\overline{\mathbb{t}}_2,\tau)\text{ s.t. }
          \overline{\mathbb{t}}_2\text{ is a t-structure in }\mathcal{T}\text{ and }
          \tau\text{ is a}\\
          \text{ cohereditary torsion pair in the abelian category } \\
          \mathcal{H}_2:=\mathcal{U}_2\cap\mathcal{U}_2^\perp[1]
        \end{array}
        \right\}.
      \end{align*}
  \end{enumerate}
\end{definition}

\begin{thm}\label{thm:2.11}
  For a triangulated category $\mathcal{T}$, the following statements hold true.
  \begin{enumerate}[label=(\alph*)]
    \item There is a bijective correspondence
      \begin{equation*}
        \begin{tikzcd}[row sep=tiny]
          RtAb(\mathcal{T})\arrow[leftrightarrow]{r}{\alpha}
            & \mathrm{t}-stCoh(\mathcal{T})\\
          \mathbb{t}\arrow[mapsto]{r}
            & (\overline{\mathbb{t}}_1,\mathrm{Pol}^{-1}_{\mathcal{H}_1}(\overline{\mathbb{t}}_2))\\
          (\mathbb{t}_1,\mathbb{t}_2)\arrow[mapsfrom]{r}
            & (\overline{\mathbb{t}}_1,\tau)
        \end{tikzcd}
      \end{equation*}
      where $\overline{\mathbb{t}}_2=\mathrm{Pol}_{\mathcal{H}_1}(\tau)$.
    \item There is a bijective correspondence
      \begin{equation*}
        \begin{tikzcd}
          RtAb^\prime(\mathcal{T})\arrow[leftrightarrow]{r}{\alpha'}
            & \mathrm{t}-stCoh^\prime(\mathcal{T})\\
          \mathbb{t}\arrow[mapsto]{r}
            & (\overline{\mathbb{t}}_2,\mathrm{Pol}^{-1}_{\mathcal{H}_2}(\overline{\mathbb{t}}_1[1]))\\
          (\mathbb{t}_1,\mathbb{t}_2)\arrow[mapsfrom]{r}
            & (\overline{\mathbb{t}}_2,\tau)
        \end{tikzcd}
      \end{equation*}
      where $\overline{\mathbb{t}}_1=\mathrm{Pol}_{\mathcal{H}_2}(\tau)[-1]$.
  \end{enumerate}
\end{thm}

% \begin{lemma}\label{sec2:lem6}
%   The heart $\mathcal{H}=\mathcal{T}_1\cap\mathcal{F}_2$ has kernels and cokernels.
% \end{lemma}
%
% \begin{rmk}
%   If $\mathcal{X}$ is abelian, then:
%   \begin{enumerate}
%     \item $f:H\to H'$ is mono in $\mathcal{H}$ iff $\Ker_\mathcal{X}(f)\in\mathcal{F}_1$
%     \item $f:H\to H'$ is epi in $\mathcal{H}$ iff $\Coker_\mathcal{X}(f)\in\mathcal{T}_2$
%   \end{enumerate}
% \end{rmk}
%
% \begin{prop}\label{sec2:prop1}
%   Let $(\mathcal{T}_1,\mathcal{F}_1)$ and $(\mathcal{T}_2,\mathcal{F}_2)$ be two torsion pairs
%   in the abelian category $\mathcal{A}$. TFAE:
%   \begin{enumerate}
%     \item $\mathcal{H}$ is abelian.
%     \item We have:
%     \begin{enumerate}
%       \item If $f:H\to H'$ is a morphism in $\mathcal{H}$ such that
%       $\Ker_\mathcal{A}(f)\in\mathcal{F}_1$, then $\Ker_\mathcal{A}(f)=0$.
%       \item If $f:H\to H'$ is a morphism in $\mathcal{H}$ such that
%       $\Coker_\mathcal{A}(f)\in\mathcal{T}_2$, then $\Coker_\mathcal{A}(f)=0$.
%       \item $\mathcal{H}$ is closed by kernels of epics and cokernels of monics.
%     \end{enumerate}
%     \item $\mathcal{H}$ is an exact abelian subcategory of $\mathcal{A}$.\todo{?}
%   \end{enumerate}
% \end{prop}
