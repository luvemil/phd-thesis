\clearpage

\section{Introduction}

Let $\mathcal{A}$ be a \emph{good} category
(abelian/exact/triangulated). The precise meaning of this will have to
be clarified later (probably, the recent Nakaoka-Palu's paper is the
right setting). But, whatever the choice, two things should happen.
First, idempotents should split in $\mathcal{A}$. Secondly,  each
torsion pair considered  in $\mathcal{A}$ should be functorial on both
sides. If $(\mathcal{T},\mathcal{F})$ is such a torsion pair, we will
denote by $t:\mathcal{A}\longrightarrow\mathcal{T}$ (resp.
$f:\mathcal{A}\longrightarrow\mathcal{F}$) the right (resp. left)
adjoint of the inclusion functor and, also, the composition
$\mathcal{A}\stackrel{t}{\longrightarrow}\mathcal{T}\stackrel{i}{\hookrightarrow}\mathcal{A}$
(resp.
$\mathcal{A}\stackrel{f}{\longrightarrow}\mathcal{F}\stackrel{j}{\hookrightarrow}\mathcal{A}$),
where $\mathcal{T}\stackrel{i}{\hookrightarrow}\mathcal{A}$ (resp.
$\mathcal{F}\stackrel{j}{\hookrightarrow}\mathcal{A}$) is the inclusion
functor. The functoriality should then give rise to an admissible
sequence $t(M)\longrightarrow M\longrightarrow f(M)$, for each object
$M\in\mathcal{A}$ (e.g. if $\mathcal{A}$ is abelian, that sequence
should be short exact, if $\mathcal{A}$ is exact it should be a
conflation, if $A$ is triangulated it should be a triangle).



$\mathcal{W}$ a full subcategory of $\mathcal{A}$ closed by direct
summands and extensions, and consider the category
$\underline{\mathcal{A}}=\dfrac{\mathcal{A}}{\mathcal{W}}$.

Let $(\mathcal{X},\mathcal{Y})$ be a orthogonal pair in
$\underline{\mathcal{A}}$ and consider the following classes in
$\mathcal{A}$:
\begin{align*}
  \mathcal{T} &= \{ T\in\mathcal{A} | \underline{T}\in\mathcal{X} \} \\
  \mathcal{F} &= \{ F\in\mathcal{A} | \underline{F}\in\mathcal{Y} \}.
\end{align*}

\begin{lemma}
  [empty]
\end{lemma}

If $\mathbb{t}=(\mathcal{T},\mathcal{F})$ is a orthogonal pair in a
cocomplete and locally small abelian category $\mathcal{A}$, then
$\mathbb{t}$ is a torsion pair. Indeed, if $M$ is any object and we
consider the set $\mathcal{T}_M$ of subobjects of $M$ which are in
$\mathcal{T}$, then $t(M):=\sum_{T\in\mathcal{T}_M}T$ is subobject of
$M$ which is an epimorphic image of $\coprod_{T\in\mathcal{T}_M}$ and,
hence, we have that $t(M)\in\mathcal{T}_M$. If we had a nonzero morphism
$f:T'\longrightarrow M/t(M)$, where $T'\in\mathcal{T}$, then we would
have that $\text{Im}(f)=\tilde{T}/t(M)$ is a nonzero submodule of
$M/t(M)$ which is in $\mathcal{T}$. Since $\mathcal{T}$ is closed under
extensions and we have an exact sequence $0\rightarrow
t(M)\longrightarrow\tilde{T}\longrightarrow \tilde{T}/t(M)\rightarrow
0$, we conclude that $\tilde{T}\in\mathcal{T}$. But then we have that
$\tilde{T}\in\mathcal{T}_M$, which is a contradiction since $t(M)$
contains all subobjects in $\mathcal{T}_M$.

\sepline

\begin{lemma}
  Let $(\mathcal{T}_1,\mathcal{F}_1)$ and
$(\mathcal{T}_2,\mathcal{F}_2)$ be torsion pairs in $\mathcal{A}$, with
associated radical functors $t_i$ and coradical functors $f_i$
($i=1,w$), respectively. Suppose that they satisfy the following conditions:
    \begin{enumerate}
    \item[a)] $\mathcal{T}_2\subseteq\mathcal{T}_1$ (equivalently,
$\mathcal{F}_1\subseteq\mathcal{F}_2$)
    \item[b)] $\mathcal{T}_2\cap\mathcal{F}_1=0$.
    \item[c)] $(\underline{\mathcal{T}}_1,\underline{\mathcal{F}}_2)$ is an orthogonal pair in $\underline{\mathcal{A}}:=\mathcal{A}/\mathcal{W}$,
where  $\mathcal{W}=\mathcal{T}_1\cap\mathcal{F}_2$.
    \end{enumerate}
    Then the
following assertions hold:

    \begin{enumerate}
    \item $\mathcal{T}_1$ consists of those objects $X\in\mathcal{A}$
such that $f_2(X)\in\mathcal{W}$. We will write
$\mathcal{T}_1=\mathcal{T}_2\star\mathcal{W}$.
    \item $\mathcal{F}_2$ consists of those objects $Y\in\mathcal{A}$
such that $t_1(Y)\in\mathcal{W}$. We will write
$\mathcal{F}_2=\mathcal{W}\star\mathcal{F}_1$.
    \end{enumerate}

\end{lemma}

\begin{proof}
We just prove assertion 1, and assertion 2 will follow by duality.
Let us take $X\in\mathcal{T}_2\star\mathcal{W}$. Since we have an
admissible sequence $t_2(X)\longrightarrow X\longrightarrow f_2(X)$
whose outer terms are in $\mathcal{T}_2$ and $\mathcal{W}$,
respectively, and these two classes are contained in $\mathcal{T}_1$ we
conclude that $\mathcal{T}'_1\subseteq\mathcal{T}_1$, because
$\mathcal{T}_1$ is closed under taking extensions in $\mathcal{A}$.



  Let $T_1$ be in $\mathcal{T}_1$ and consider its canonical admissible
sequence

  \begin{equation}
    t_2(T_1)\to T_1\nto{f} f_2(T_1).
  \end{equation}
    Note that $\underline{f}=0$ because of condition c) in the
statement. It follows that $f$ decomposes  in the form
$f:T_1\stackrel{\gamma}{\longrightarrow}W\stackrel{\phi}{\longrightarrow}f_2(T_1)$,
where $W\in\mathcal{W}$. We then consider the following admissible
pullback diagram
  \begin{equation}
    \begin{tikzcd}
      t_2(T_1)\arrow{r}\arrow[equal]{d} &
        \widehat{T}_1\arrow{r}\arrow{d} &
          W\arrow{d}{\phi}\\
      t_2(T_1)\arrow{r}&
        T_1\arrow{r}{f}&
          f_2(T_1)
    \end{tikzcd}
  \end{equation}


  Then, there exist a (non necessarely unique) $\eta:T_1\to
\widehat{T}_1$ making the following diagram commute.

  \begin{equation}
    \begin{tikzcd}
      T_1\arrow[bend
right]{ddr}[']{1}\arrow[dashed]{dr}{\eta}\arrow[bend left]{rrd}{\gamma}&&\\
      &
        \widehat{T}_1\arrow{r}\arrow{d}&
          W\arrow{d}{\phi}\\
      &
        T_1\arrow{r}{f}&
          f_2(T_1)
    \end{tikzcd}
  \end{equation}

  Hence, $T_1 <_\oplus \widehat{T}_1\in \mathcal{T}_2  \ast
\mathcal{W}$. This implies that  $\mathcal{T}_1\subseteq
\mathrm{add}(\mathcal{T}_2\ast \mathcal{W})$. The proof will be finished
once we check that $\mathcal{T}_2\ast \mathcal{W}$ is closed under
direct summands. But this is a direct consequence of the functoriality
of the torsion pair. Indeed if we have admissible torsion sequences
$t_2(M)\longrightarrow M\longrightarrow f_2(M)$ and
$t_2(N)\longrightarrow N\longrightarrow f_2(N)$, then the coproduct
sequence $t_2(M)\oplus t_2(N)\longrightarrow M\oplus N\longrightarrow
f_2(M)\oplus f_2(N)$ is the admissible torsion sequence for $M\oplus N$.
The fact that $M\oplus N\in\mathcal{T}_2\ast \mathcal{W}$ is then
equivalent to the fact that $f_2(M)\oplus f_2(N)\in\mathcal{W}$. Since
$\mathcal{W}$ is closed under direct summands, we conclude that
$f_2(M)\in\mathcal{W}$ and, hence, that $M\in\mathcal{T}_2\ast\mathcal{W}$.

\end{proof}
