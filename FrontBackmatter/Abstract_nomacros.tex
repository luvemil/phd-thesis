%*******************************************************
% Abstract
%*******************************************************
%\renewcommand{\abstractname}{Abstract}
\pdfbookmark[1]{Abstract}{Abstract}
\begingroup
\let\clearpage\relax
\let\cleardoublepage\relax
\let\cleardoublepage\relax

\chapter*{Abstract}
We consider a complete hereditary cotorsion pair $(\mathcal{A},\mathcal{B})$ in a Grothendieck category $\mathcal{G}$ such that $\mathcal{A}$ contains a generator of finite projective dimension. The derived category $D(\mathcal{B})$ of the exact category $\mathcal{B}$ is defined as the quotient of the category $Ch(\mathcal{B})$, of unbounded complexes with terms in $\mathcal{B}$, modulo the subcategory $\widetilde{\mathcal{B}}$ consisting of the acyclic complexes with terms in $\mathcal{B}$ and cycles in $\mathcal{B}$.

We prove that there are recollements
\begin{equation*}
  \begin{tikzcd}
    \frac{ex\mathcal{B}}{\sim}\arrow{r}{inc}
    & D(\mathcal{B})\arrow[bend left=50]{l}\arrow[bend right=50]{l}\arrow{r}{Q}
    & D(\mathcal{G})\arrow[bend left=50]{l}\arrow[bend right=50]{l}
  \end{tikzcd}
\end{equation*}
and
\begin{equation*}
  \begin{tikzcd}
    \frac{ex\mathcal{B}}{\sim}\arrow{r}{inc}
    & K(\mathcal{B})\arrow[bend left=50]{l}\arrow[bend right=50]{l}\arrow{r}{Q}
    & D(\mathcal{G}).\arrow[bend left=50]{l}\arrow[bend right=50]{l}
  \end{tikzcd}
\end{equation*}

Then, we restrict our attention to the cotorsion pairs such that $\widetilde{\mathcal{B}}$ coincide with the class $ex\mathcal{B}$ of the acyclic complexes of $Ch(\mathcal{G})$ with terms in $\mathcal{B}$. In this case the derived category $D(\mathcal{B})$ fits into a recollement
\begin{equation*}
  \begin{tikzcd}
    \frac{ex\mathcal{B}}{\sim}\arrow{r}{inc}
    & K(\mathcal{B})\arrow[bend left=50]{l}\arrow[bend right=50]{l}\arrow{r}{Q}
    & D(\mathcal{B}).\arrow[bend left=50]{l}\arrow[bend right=50]{l}
  \end{tikzcd}
\end{equation*}

We will explore the conditions under which $ex\mathcal{B}=\widetilde{\mathcal{B}}$ and provide some examples. Symmetrically, we prove analogous results for the exact category $\mathcal{A}$.

We also introduce the notion of Nakaoka context in additive categories as couples $t_i=(\mathcal{T}_i,\mathcal{F}_i)$ for $i=1,2$ of torsion pairs such that $\mathcal{T}_2\subseteq \mathcal{T}_1$. We give a set of axioms for a Nakaoka context in order to ensure that the heart $\mathcal{H}:=\mathcal{T}_1\cap \mathcal{F}_2$ is Abelian. Then, we inspect the properties of Nakaoka contexts in Abelian and triangulated categories. In particular, we find a bijection between the t-structures $(\mathcal{T}_1,\mathcal{F}_1[1]), (\mathcal{T}_2,\mathcal{F}_2[1])$ such that $\mathcal{T}_1[1]\subseteq\mathcal{T}_2\subseteq\mathcal{T}_1$ whose heart $\mathcal{H}:=\mathcal{T}_1\cap\mathcal{F}_2$ is Abelian and the cohereditary torsion pairs in $\mathcal{H}_1:=\mathcal{T}_1\cap\mathcal{F}_1[1]$.

\endgroup

\cleardoublepage
\pdfbookmark[1]{Riassunto}{Riassunto}
\begingroup
\let\clearpage\relax
\let\cleardoublepage\relax
\let\cleardoublepage\relax


\chapter*{Riassunto}

Consideriamo un cotorsion pair completo ed ereditario $(\mathcal{A},\mathcal{B})$ in una categoria di Grothendieck $\mathcal{G}$ tale che $\mathcal{A}$ contenga un generatore di dimensione proiettiva finita. La categoria derivata $D(\mathcal{B})$ della categoria esatta $\mathcal{B}$ \`e definita come il quoziente fra la categoria $Ch(\mathcal{B})$ dei complessi illimitati a termini in $\mathcal{B}$ modulo la sottocategoria $\widetilde{\mathcal{B}}$ dei complessi aciclici a termini in $\mathcal{B}$ e cicli in $\mathcal{B}$.

Dimostriamo che esistono i recollement
\begin{equation*}
  \begin{tikzcd}
    \frac{ex\mathcal{B}}{\sim}\arrow{r}{inc}
    & D(\mathcal{B})\arrow[bend left=50]{l}\arrow[bend right=50]{l}\arrow{r}{Q}
    & D(\mathcal{G})\arrow[bend left=50]{l}\arrow[bend right=50]{l}
  \end{tikzcd}
\end{equation*}
e
\begin{equation*}
  \begin{tikzcd}
    \frac{ex\mathcal{B}}{\sim}\arrow{r}{inc}
    & K(\mathcal{B})\arrow[bend left=50]{l}\arrow[bend right=50]{l}\arrow{r}{Q}
    & D(\mathcal{G}).\arrow[bend left=50]{l}\arrow[bend right=50]{l}
  \end{tikzcd}
\end{equation*}

Successivamente restringiamo la nostra attenzione ai cotorsion pair tali che $\widetilde{\mathcal{B}}$ coincida con la classe $ex\mathcal{B}$ dei complessi aciclici di $Ch(\mathcal{G})$ con termini in $\mathcal{B}$. In questo caso la categoria derivata $D(\mathcal{B})$ appartiene a un recollement
\begin{equation*}
  \begin{tikzcd}
    \frac{ex\mathcal{B}}{\sim}\arrow{r}{inc}
    & K(\mathcal{B})\arrow[bend left=50]{l}\arrow[bend right=50]{l}\arrow{r}{Q}
    & D(\mathcal{B}).\arrow[bend left=50]{l}\arrow[bend right=50]{l}
  \end{tikzcd}
\end{equation*}

Studieremo le condizioni per cui $ex\mathcal{B}=\widetilde{\mathcal{B}}$ e mostreremo alcuni esempi. Simmetricamente dimostriamo risultati analoghi per la categoria esatta $\mathcal{A}$.

Inoltre, introduciamo la nozione di Nakaoka context in categorie additive come coppie di torsion pair $t_i=(\mathcal{T}_i,\mathcal{F}_i)$ per $i=1,2$ tali che $\mathcal{T}_2\subseteq \mathcal{T}_1$. Daremo un insieme di assiomi per un Nakaoka context che garantisca l'abelianit\`a del cuore $\mathcal{H}:=\mathcal{T}_1\cap\mathcal{F}_2$. Successivamente studieremo le propriet\`a dei Nakaoka context in categorie Abeliane e triangolate. In particolare troviamo una biezione tra le t-strutture $(\mathcal{T}_1,\mathcal{F}_1[1]), (\mathcal{T}_2,\mathcal{F}_2[1])$ tali che $\mathcal{T}_1[1]\subseteq\mathcal{T}_2\subseteq\mathcal{T}_1$ il cui cuore $\mathcal{H}:=\mathcal{T}_1\cap\mathcal{F}_2$ sia abeliano e le torsion pair coereditarie in $\mathcal{H}_1:=\mathcal{T}_1\cap\mathcal{F}_[1]$.


\endgroup


\vfill
