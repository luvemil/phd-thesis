%\section{Projective cotorsion pairs in the exact category $\Ch(\B)$}\label{S:B}
%======================
\chapter{Recollements from cotorsion pairs}\label{ch:recoll}
%======================

In \cite{Nee08} Neemann described the homotopy category of the projective modules as a localization of the homotopy category of flat modules and he obtained a recollement  with middle  term the homotopy category of flat modules. His recollement can be compared with the classical one having the homotopy category of a ring $R$ as middle term, the derived category of $R$ as right term and the category of acyclic complexes modulo the homotopy relation as left term.


This chapter is extracted from the joint work \cite{bazzoni2018recollements} where we exhibit many other examples of recollements of analogous type.

Our results are strongly based on the two papers ~\cite{G6} and \cite{G7} by Gillespie and also inspired by Becker's idea in \cite{Beck12} to consider triples of injective cotorsion pairs giving rise to model structures and to the corresponding recollements.


Starting from a complete hereditary cotorsion pair $(\A, \B)$ in a Grothendieck category, we consider triples of examples of injective and projective cotorsion pairs on the categories of unbounded complexes with components in the exact categories $\A$ or $\B$.
The examples are constructed in order that the associated model structures on the categories of complexes satisfy the assumptions allowing to build the relevant recollements.

Our aim is mainly to describe the homotopy categories $K(\B)$ and $K(\A)$ as well as the derived categories $\D(\B)$ and $\D(\A)$.

Imposing some mild conditions on a Grothendieck category $\G$ (which are always satisfied by module categories), Theorem~\ref{T:triple-B} gives the recollement % and Theorem~\ref{T:triple-A} give recollements
\vskip0.7cm
\[
  \xymatrix{ (\ast) \quad \dfrac{ex\B}{\sim} \ar[rr]^{inc} &&{K(\B)} \ar@/^2pc/[ll]\ar@/_2pc/[ll] \ar[rr]^{Q}
                                           &&{\dfrac{\Ch(\B)}{ex\B }}\ar@/^2pc/ [ll] \ar@/_2pc/ [ll] },
\]
\vskip0.7cm
 where for every subcategory $\C$ of $\G$, $ex{\C}$ denotes the class of acyclic unbounded complexes with terms in $\C$.
The dual is given by Theorem~\ref{T:triple-A}.

%\[
%\xymatrix{\dfrac{ex\A}{\sim} \ar[rr]^{inc} &&{K(\A)} \ar@/^2pc/[ll]\ar@/_2pc/[ll] \ar[rr]^{Q}
%&&{\dfrac{\Ch(\A)}{ex\A }}\ar@/^2pc/ [ll] \ar@/_2pc/ [ll] }
%\]
%\vskip0.7cm

 The  recollement $(\ast)$ generalizes the recollement obtained by Krause (~\cite{Kr05}) where the middle term is the homotopy category of the injective objects.

 %

For a complete hereditary cotorsion pair $(\A, \B)$ the term $\dfrac{\Ch(\B)}{ex\B }$ is equivalent to the derived category $\D(\G)$ of the Grothendieck category, essentially because $\Ch(\B)$ contains the dg-injective complexes. Analogously, if $\G$ has enough projectives, $\dfrac{\Ch(\A)}{ex\A}$ is equivalent to the derived category $\D(\G)$ of the Grothendieck category, since $\Ch(\A)$ contains the dg-projective complexes.

Concerning the derived categories $\D(\B)$ and $\D(\A)$, recall that
by Neeman~\cite{Nee90} the derived category of an idempotent complete exact category $\C$ is defined as the quotient $\dfrac{\Ch(\C)}{\widetilde{\C}}$, where $\widetilde{\C}$ denotes the class of unbounded complexes acyclic in $\C$, meaning that the differentials factor through short exact sequences in $\C$. By Theorem~\ref{T:derived-B} we get the recollement

\vskip0.7cm
\[
\xymatrix{\dfrac{ex\B}{\sim} \ar[rr]^{inc} &&{\D(\B)} \ar@/^2pc/[ll]\ar@/_2pc/[ll] \ar[rr]^{Q}
                                           &&{\D(\G)}\ar@/^2pc/ [ll] \ar@/_2pc/ [ll] }
\]
\vskip0.7cm
and its dual in Theorem~\ref{T:derived-A}.
% \label{T:derived-B}

It would be important to get recollements analogous to $(\ast)$, but with right term $\D(\B)$ and $\D(\A)$, that is the derived categories of the exact categories $\B$ or $\A$. Of course, if $\widetilde{\B}=ex\B$ or $\widetilde{\A}=ex{\A}$, the recollement $(\ast)$ degenerates into
\vskip0.7cm
\[
  \xymatrix{(\ast \ast)\quad \dfrac{\widetilde{\B}}{\sim} \ar[rr]^{inc} &&{K(\B)} \ar@/^2pc/[ll]\ar@/_2pc/[ll] \ar[rr]^{Q}
&&{\dfrac{\Ch(\B)}{\widetilde{\B} }}\ar@/^2pc/ [ll] \ar@/_2pc/ [ll] }
\]
\vskip0.7cm
(and dually for $\A$).

%\[
%\xymatrix{\dfrac{\widetilde{\A}}{\sim} \ar[rr]^{inc} &&{K(\A)} \ar@/^2pc/[ll]\ar@/_2pc/[ll] \ar[rr]^{Q}
%&&{\dfrac{\Ch(\A)}{\widetilde{\A} }}\ar@/^2pc/ [ll] \ar@/_2pc/ [ll] }
%.\]
%\vskip0.7cm

The condition $\widetilde{\B}=ex\B$ is very strong, hence it would be interesting to find other examples of $(\ast \ast)$ for the case $\widetilde{\B}\subsetneq ex\B$.

The only non degenerate example of this type of which we are aware is given by the cotorsion pair $(\A, \FpInj)$ over a coherent ring, where $\FpInj$ denotes the class of Fp-injective modules, that is the right Ext-orthogonal to the class of finitely presented modules. This follows by \sto's results in \cite{Sto14} which we are able to slightly generalize in Proposition~\ref{P:B-inj-n}.

Symmetrically, it seems there are very few non degenerate examples of such recollements for the case $\widetilde{\A}\subsetneq ex\A$. The more important one follows by the celebrated Neeman's result in \cite{Nee08} and it is the case when $\A$ is class of flat modules. We show a slight generalization of this situation in Proposition~\ref{P:A-flat}.%

From the results in Section~\ref{S:2} and the results in a recent paper~\cite{BCIE} we obtain examples of cotorsion pairs $(\A, \B)$ in module categories satisfying the condition $\widetilde{\B}=ex\B$. These include tilting and cotilting cotorsion pairs, the closure of the cotorsion pair generated by the compact objects of finite projective dimension and the cotorsion pair $(\F, \C)$ of the flat and cotorsion modules.


\section{Projective cotorsion pairs in the exact category \texorpdfstring{$\Ch(\B)$}{Ch(B)}}\label{S:B}



For every complete hereditary cotorsion pair $(\A, \B)$ in a Grothendieck category $\G$ we look for cotorsion pairs on the exact category $\Ch(\B)$ of unbounded complexes with terms in $\B$ in order to describe the derived category $\D( \B)$ and also recollements linking it to the derived category of $\G$.

We start by choosing projective cotorsion pairs in $\Ch(\G)$ satisfying the assumptions of Proposition~\ref{P:prop7.3-dual}.
When needed we assume some extra conditions on the Grothendieck category $\G$, like in example (3) below.


\begin{expl}\label{E:examples-proj} Let $(\A, \B)$ be a complete hereditary cotorsion pair in a  Grothendieck category $\G$.
 \begin{enumerate}
%
%
\item The complete hereditary cotorsion pair $(dg \A, \widetilde{\B})$ in $\Ch(\G)$ satisfies the conditions in  Proposition~\ref{P:prop7.3-dual}, hence we have the projective cotorsion pair:
%
 \[\M_1=\Big(dg{\A}\cap dw \B, \widetilde{\B}\Big)\] in $\Ch(\B)$ and the localizing cotorsion pair \[\Big([dg{\A}\cap dw \B]_K,  \widetilde{\B}\Big)\] in $\Ch(\B)_{dw}.$
\item The complete hereditary cotorsion pair $(\widetilde{\A}, dg \B)$ in $\Ch(\G)$ satisfies the conditions in  Proposition~\ref{P:prop7.3-dual}, hence we have the projective cotorsion pair:
%
 \[\M_2=\Big(\widetilde{\A}\cap dw \B, dg \B\Big)\] in $\Ch(\B)$ and the localizing cotorsion pair \[\Big([\widetilde{\A}\cap dw \B]_K, dg (\B)\Big)\] in $\Ch(\B)_{dw}.$

%
%
%

\item If $\A$ contains a generator of finite projective dimension, then by Proposition~\ref{P:complete?}~(3), $({}^\perp{} ex\B, ex\B)$ is a complete hereditary cotorsion pair in $\Ch(\G)$ and it satisfies the conditions in  Proposition~\ref{P:prop7.3-dual}, hence  we have the projective cotorsion pair:
 \[\M_3=({}^\perp{} ex\B\cap dw \B, ex\B)\] in $\Ch(\B)$ and the localizing cotorsion pair \[\Big([{}^\perp{} ex\B\cap dw \B]_K, ex\B\Big)\] in $\Ch(\B)_{dw}.$

 \end{enumerate}
 \end{expl}
 \begin{rem}\label{R:thick-1} The three examples above satisfy Proposition~\ref{P:prop7.3-dual} since $\tilde{\B}$, $dg\B$ and $ex\B$  are thick in $\Ch(\B)$ by Lemma~\ref{L:HOM} and they clearly contain the contractible complexes with terms in $\B$.
 \end{rem}
 %

\begin{thm}\label{T:recollement-proj} Let $(\A, \B)$ be a complete hereditary cotorsion pair in a  Grothendieck category $\G$ such that $\A$ contains a generator of finite projective dimension.

The three projective cotorsion pairs in Example~\ref{E:examples-proj} satisfy the conditions of \cite[Theorem 3.5]{G7}, so that we get the recollement:

%
%
%
%
%
%
%
%
%
\[
\xymatrix{\dfrac{\widetilde{\A}\cap dw \B}{\sim} \ar[rr]^{inc} &&\dfrac{dg{\A}\cap dw \B}{\sim} \ar@/^2pc/
[ll]\ar@/_2pc/[ll] \ar[rr]^{Q}
&&{\Ch(\B)/ex\B}\ar@/^2pc/ [ll] \ar@/_2pc/ [ll] }
\]
\vskip 0.7cm

where $\sim$ denotes the homotopy relation associated to the corresponding model structure and coincides with the chain homotopy relation; moreover, $inc$ is the inclusion and $Q$ is the quotient functor.
\end{thm}
%
%
%
%
\begin{rem}\label{R:proj-recoll} In the above examples write $\M_i=(\C_i, \W_i)$, for every $i=1,2,3$.  We have that $\C_i\cap\W_i=\widetilde{\A}\cap\widetilde{\B}$. Moreover, dually to \cite[Proposition 3.2]{G7} $\C_i$ is a Frobenius category with the projective-injective objects being exactly the complexes in $\widetilde{\A}\cap\widetilde{\B}$. Thus $(\widetilde{\A}\cap dw\B)/\sim$ and $(dg{\A}\cap dw\B)/\sim$ are the stable categories and they are also equivalent to the homotopy categories $K(\widetilde{\A}\cap dw \B)$ and $K(dg{\A}\cap dw \B)$. Moreover,  all the three terms in the recollement are equivalent to the homotopy categories of the three model structures on $\Ch(\B)$ corresponding to the projective cotorsion pairs $\M_1, \M_2, \M_3$. Furthermore, $\Ch(\B)/ex\B$ is equivalent to the derived category of $\G$ as we will see more explicitly later in Remark~\ref{R:derived-of-R}.
\end{rem}
\vskip 0.5cm
By \cite{Nee90}, the derived category of $\Ch(\B)$ is the quotient of $\Ch(\B)$ modulo the acyclic complexes in  $\Ch(\B)$, that is the complexes in $\widetilde{\B}$. Thus we need an exact model structure on $\Ch(\B)$ with $\widetilde{\B}$ as the class of trivial objects. This is provided by Example~\ref{E:examples-proj}~(1).
%
\begin{thm}\label{T:derived-B} In the setting of Example~\ref{E:examples-proj}~(1), \[\M_1=\Big(dg{\A}\cap dw \B, \widetilde{\B}, dw\B\Big)\] is an exact model structure in the category $\Ch(\B)$.
 In particular, we can define the derived category $\D(\B)$ as the quotient  $\Ch(\B)/\widetilde{\B}$.

 Moreover, we have the following triangle equivalences between the derived category of $\Ch(\B)$ and the homotopy category of the model structure $\M_1$:
 \[\D(\B)=\Ho(\M_1)\cong \dfrac{dg{\A}\cap dw \B}{\widetilde{\A}\cap\widetilde{\B}}\]
 and in the assumptions of Theorem~\ref{T:recollement-proj} there is also a recollement:
\vskip0.7cm
\[
\xymatrix{\dfrac{ex\B}{\sim} \ar[rr]^{inc} &&{\D(\B)} \ar@/^2pc/[ll]\ar@/_2pc/[ll] \ar[rr]^{Q}
                                           &&{\D(\G)}\ar@/^2pc/ [ll] \ar@/_2pc/ [ll] }
\]
\vskip0.7cm
where $ex\B/\sim$ is the full subcategory of $\D(\B)$ consisting of exact complexes (in $\Ch(\G)$).

 \end{thm}
 %
 \begin{proof} The projective cotorsion pair $\Big(dg{\A}\cap dw \B, \widetilde{\B}\Big)$ in $\Ch(\B)$ of  Example~\ref{E:examples-proj}~(1) corresponds to  the exact model structure
   $\Big(dg{\A}\cap dw \B, \widetilde{\B}, dw \B\Big)$. The equivalences follow from general properties of model categories (see the discussion in \cite[Section~3.1]{G7}), hence the recollement follows by Theorem~\ref{T:recollement-proj}.
 \end{proof}
 %
 %
 Another way to obtain the exact model structure of Theorem~\ref{T:derived-B} is to use results by Gillespie in \cite{G4}, \cite{G8} and \cite{G9}.

 \begin{thm}\label{T:Gill-B} Let $(\A, \B)$ be a complete hereditary cotorsion pair generated by a set of objects
in a Grothendieck category $\G$. The two complete hereditary cotorsion pairs
 $({}^\perp{} dw\B, dw\B)$ and $(dg\A, \tilde\B)$ in $\Ch(\G)$  give rise to a cofibrantly generated model structure \[\M=\Big(dg\A, \V, dw\B\Big)\] in $\Ch(\G )$ satisfying $\V\cap dw\B=\tilde\B$ and $\V\cap dg\A={}^\perp{} dw\B$ whose restriction in $\Ch(\B)$ is the exact model structure \[\M_1=\Big(dg{\A}\cap dw \B, \widetilde{\B}, dw\B\Big)\] of Theorem~\ref{T:derived-B}.

 Moreover, if $(\A, \B)$ is generated by a set of finitely presented objects then the model structure $\M=\Big(dg\A, \V, dw\B\Big)$ in $\Ch(\G)$ is finitely generated hence its homotopy category is compactly generated.

\end{thm}

 \begin{proof} By  \cite[Proposition 4.3 and Proposition 4.4]{G4} the cotorsion pairs $({}^\perp{} dw\B, dw\B)$ and $(dg\A, \tilde\B)$ are small and they are hereditary since $(\A, \B)$ is hereditary. The existence of the model structure $\M$ in $ \Ch(\G )$ follows by \cite[Theorem 1.1]{G9}. The fact that the model structure is cofibrantly generated follows by \cite[Section 7.4]{Hov02}.
 The last statement follows also by  \cite[Section 7.4]{Hov02}.
\end{proof}
  Combining Theorem~\ref{T:derived-B} with Theorem~\ref{T:Gill-B} we obtain the following consequence:
 \begin{cor}\label{C:compactly generated} Let $(\A, \B)$ be a hereditary cotorsion pair in $\Modr R$ generated by a set of finitely presented modules. Then the derived category $\D(\B)\cong\Ch(\B)/\tilde\B$ is compactly generated. In particular, if $R$ is a coherent ring and $(\A,\FpInj)$ is the complete cotorsion pair generated by all finitely presented modules, then $\D(\FpInj)$ is compactly generated.

 \end{cor}
 %
\begin{proof} Only the second statement needs a comment. If $R$ is a coherent ring, then the complete cotorsion pair $(\A, \FpInj)$ is hereditary. \end{proof}

 %\section{Injective cotorsion pairs in the exact category $\Ch(\B)$}\label{S:B-2}
\section{Injective cotorsion pairs in the exact category \texorpdfstring{$\Ch(\B)$}{Ch(B)}}\label{S:B-2}

 In this section we want to investigate models for $\K(\B)$ and recollements linking it to $\D(\G)$, in particular we look for localizing cotorsion triple in $\Ch(\B)_{dw}$ whose middle term is $ex \B$.


%

We exhibit three examples of injective cotorsion pairs in $\Ch(\G)$ satisfying the assumptions of Proposition~\ref{P:prop7.2}.

When needed, we assume some extra conditions on $\G$ like in (2) below.
 \begin{expl}\label{E:examples-inj} Let $(\A, \B)$ be a complete hereditary cotorsion pair in a Grothendieck  category $\G$.
 \begin{enumerate}
\item  By Proposition~\ref{P:complete?}~(2) we have that $({}^\perp{} dw \Inj, dw \Inj)$ is an injective cotorsion pair  in $\Ch(\G)$ (notice that $(\G, \Inj)$ is generated by a set).
  %
 Hence by Proposition~\ref{P:prop7.2} and \cite[Theorem 6.3]{G7} we obtain the injective cotorsion pair in $\Ch(\B)$:
  \[\N_1=\Big({}^\perp{} dw \Inj\cap dw \B, dw \Inj\Big)\] and the localizing cotorsion pair \[\Big({}^\perp{} dw \Inj\cap dw \B, [dw \Inj]_K\Big)\] in $\Ch(\B)_{dw}.$
\item If $\G$ has a generator of finite projective dimension, by Proposition \ref{P:complete?} $({}^\perp{} ex \Inj, ex \Inj)$ is an injective cotorsion pair  in $\Ch(\G)$. Hence, by Proposition~\ref{P:prop7.2} and \cite[Theorem 6.3]{G7} we obtain the injective cotorsion pair in $\Ch(\B)$:
 \[\N_2=\Big({}^\perp{} ex \Inj\cap dw \B, ex \Inj\Big)\] and the localizing cotorsion pair \[\Big({}^\perp{} ex \Inj\cap dw \B, [ex \Inj]_K\Big)\] in $\Ch(\B)_{dw}.$
 \todo{Fix the references in item 2 too}
 \item
By Proposition~\ref{P:description-tilde}~(3) $(\E, dg\Inj)$ is a complete hereditary cotorsion pair in $\Ch(\G)$.
 Hence, by Proposition~\ref{P:prop7.2} and \cite[Theorem 6.3]{G7} we obtain the injective cotorsion pair in $\Ch(\B)$:
 \[\N_3=\Big(ex\B,dg \Inj\Big)\] and the localizing cotorsion pair \[\Big(ex\B, [dg \Inj]_K\Big)\] in $\Ch(\B)_{dw}.$

\end{enumerate}
\end{expl}

 In the above examples we  write $\N_i=(\W_i, \R_i)$, for every $i=1,2,3$. Then, $\R_2\subseteq \W_3$ and
$\W_2\cap\W_3=\W_1$. In fact, by the analogous of \cite[Theorem 4.7]{G4} in a Grothendieck category,  ${}^\perp{} ex \Inj\cap \E={}^\perp{} dw\Inj$. Thus, they satisfy the conditions of  \cite[Theorem 3.4]{G7} and allow to build a recollement which, as we will point out in Remark~\ref{R:inj-recoll}, is nothing else than Krause's recollement~\cite{Kr05}.
%

\begin{thm}\label{T:recollement-inj} Let $\G$ be a Grothendieck category with a generator of finite projective dimension and let $(\A, \B)$ be a complete hereditary cotorsion pair in $\G$.The three injective cotorsion pairs in Example~\ref{E:examples-inj} satisfy the conditions of \cite[Theorem 3.4]{G7}, so that we get the recollement:


\[
\xymatrix{\dfrac{ex \Inj}{\sim} \ar[rr]^{inc} &&\dfrac{dw\Inj}{\sim} \ar@/^2pc/
[ll]\ar@/_2pc/[ll] \ar[rr]^{Q}
&&{\Ch(\B)/ex\B}\ar@/^2pc/ [ll] \ar@/_2pc/ [ll] }
\]
\vskip 0.7cm

where $\sim$ denotes the homotopy relation associated to the corresponding model structure and coincides with the chain homotopy relation; moreover, $inc$ is the inclusion and $Q$ is the quotient functor.
\end{thm}


\begin{rem}\label{R:inj-recoll}
\begin{enumerate}
{\,}
\item Writing $\N_i=(\W_i, \R_i)$, for every $i=1,2,3$, \cite[Proposition 3.2]{G7} implies that $\R_i$ is a Frobenius category with the projective-injective object being exactly  the injective objects in $\Ch(\G)$ or, equivalently, in $\Ch(\B)$. Note that, for every $i=1,2,3$,  $\R_i\cap\W_i$ is the class of injective objects in $\Ch(\B)$. Thus, $\R_1/\sim$ is equivalent to the  homotopy category $K(\Inj)$ of the complexes with injective terms and $\R_2 /\sim$ is equivalent to $K(ex\Inj)$ the full subcategory of $K(\Inj)$ consisting of exact complexes of injectives. Moreover, $\Ch(\B)/ex\B$ is equivalent to the derived category $\D(R)$, as it will be clear from Theorem~\ref{T:triple-B}.

 That is, Theorem~\ref{T:recollement-inj} is yet another instance of Krause's recollement \cite{Kr05}, which was recovered also in \cite{Beck14}.

\item The complexes in ${}^\perp{} dw \Inj$ are called coacyclic in \cite{Pos} (see also \cite{Stopurity} and \cite{Beck14}). By \cite[Proposition 6.9]{Stopurity} the homotopy category of the injective cotorsion pair $\N_1$ is equivalent to $K(\Inj)$ and called the coderived category of $\G$. Thus the central term of the above recollement is equivalent to the coderived category of $\G$.
\end{enumerate}
\end{rem}

Combining the above example with an example from Section~\ref{S:B} we can state the following:
\begin{thm}\label{T:triple-B}  Let $(\A, \B)$ be a complete hereditary cotorsion pair in a  Grothendieck category $\G$ such that $\A$ contains a generator of finite projective dimension.
The triple $\Big([{}^\perp{} ex\B\cap dw \B]_K, ex\B, [dg \Inj]_K\Big)$ is a localizing cotorsion triple in $\Ch(\B)_{dw}$. Then, there are equivalences of triangulated categories:
\[\frac{[{}^\perp{} ex\B\cap dw \B]_K}{\sim} \cong \frac{\Ch(\B)}{ex\B} \cong  \frac{[dg \Inj]_K}{\sim}\]
where $\sim$ is the chain homotopy equivalence
and a recollement:
\vskip0.7cm
\[
\xymatrix{\dfrac{ex\B}{\sim} \ar[rr]^{inc} &&{K(\B)} \ar@/^2pc/[ll]\ar@/_2pc/[ll] \ar[rr]^{Q}
                                           &&{\dfrac{\Ch(\B)}{ex\B }\cong \D(\G)}\ar@/^2pc/ [ll] \ar@/_2pc/ [ll] }
\]
\vskip0.7cm
where the middle term is  the homotopy category $K(\B)$ of the complexes with terms in $\B$ modulo the chain homotopy equivalence.

\end{thm}
%
\begin{proof} By Example~\ref{E:examples-proj}~(3)  and Example~\ref{E:examples-inj}, we have two localizing cotorsion pairs $\Big([{}^\perp{} ex\B\cap dw \B]_K, ex\B\Big)$ and $\Big(ex\B,  [dg \Inj]_K\Big)$ in $\Ch(\B)_{dw}$. Let $\X=[{}^\perp{} ex\B\cap dw \B]_K$, $\Y=ex\B$ and $\Z=[dg \Inj]_K$, then $(\X, \Y, \Z)$ is a localizing cotorsion triple in $\Ch(\B)_{dw}$.

Then, the conclusion follows by  \cite[Corollary 4.5]{G7} .
\end{proof}


\begin{rem}\label{R:derived-of-R} From the equivalence $\dfrac{\Ch(\B)}{ex\B} \cong \dfrac{ [dg \Inj]_K}{\sim}$ we see that  $\dfrac{\Ch(\B)}{ex\B}$  is equivalent to the usual derived category $\D(\G)$.
\end{rem}

%\section{When is $\tilde{\B}$ the central term of a localizing cotorsion triple in $\Ch(\B)_{dw}$?}\label{S:tildeB}
\section{When is \texorpdfstring{$\tilde{\B}$}{tilde B} the central term of a localizing cotorsion triple in \texorpdfstring{$\Ch(\B)_{dw}$}{Ch(B)_{dw}}?}\label{S:tildeB}


%
In Example~\ref{E:examples-inj}~(3) we have shown that there is an injective cotorsion pair $\Ch(\B)$ with $ex\B$
 as left term and Example~\ref{E:examples-proj}~(1) provides a projective cotorsion pair in $\Ch(\B)$ with right component $ \widetilde{\B}$.

 Our aim will be to find cotorsion pairs $(\A, \B)$ for which there exist an injective cotorsion pair $(\widetilde{\B}, \R)$ in $\Ch(\B)$ with $
\R\subseteq dw\Inj$ in order to obtain a localizing cotorsion triple in $\Ch(\B)_{dw}$ with $\widetilde{\B}$ as central term.

 Section~\ref{S:2} provides examples of cotorsion pairs $(\A, \B)$ such that $ex\B=\widetilde{\B}$.

A first case appears in Proposition~\ref{P:finite-proj-dim}.
\begin{prop}\label{P:new-triple}
Let $(\A, \B)$ be a complete hereditary cotorsion pair in $\G$ with $\A\subseteq \clP$ ($\clP$ the class of objects with finite projective dimension). Then $ex\B=\tilde\B$, hence $(\tilde\B, dg\Inj)$ is an injective cotorsion pair in $\Ch(\B)$ and there is a recollement as in Theorem~\ref{T:triple-B} with $ex \B$ replaced by $\tilde\B$.

In particular, the derived category $\D(\B)$ of $\B$ is equivalent to the usual derived category of $\G$.
\end{prop}
%

\begin{cor}\label{C:recoll-tilting} Let $(\A, \T)$ be an $n$-tilting cotorsion pair in $\Modr R$. For the tilting class $\T$ we have a recollement:
\vskip0.7cm
\[
\xymatrix{\dfrac{\widetilde{\T}}{\sim} \ar[rr]^{inc} &&{K(\T)} \ar@/^2pc/[ll]\ar@/_2pc/[ll] \ar[rr]^{Q}
&&{\dfrac{\Ch(\T)}{\widetilde{\T} }}\ar@/^2pc/ [ll] \ar@/_2pc/ [ll] }
.\]
\vskip0.7cm
\end{cor}

\begin{proof} By Proposition~\ref{P:tilt-cotil} $ex\T=\widetilde{\T}$, hence the conclusion follows by Proposition~\ref{P:new-triple}.
\end{proof}

%
%
%
%
%
%
%
%
%
%
%
%
%
%
%
%
 \begin{prop}\label{P:ex-Cot} The cotorsion pair $(\Flat, \Cot)$ in $\Modr R$ satisfies $ex\Cot=\widetilde{\Cot}$,  hence it induces a recollement:
%
%
\vskip0.7cm
\[
\xymatrix{\dfrac{\widetilde{\Cot}}{\sim} \ar[rr]^{inc} &&{K(\Cot)} \ar@/^2pc/[ll]\ar@/_2pc/[ll] \ar[rr]^{Q}
&&{\dfrac{\Ch(\Cot)}{\widetilde{\Cot} }}\ar@/^2pc/ [ll] \ar@/_2pc/ [ll] }
.\]
\vskip0.7cm
\end{prop}
%
\begin{proof} The fact that $ex\Cot=\widetilde{\Cot}$ in $\Ch(R)$ is proved in \cite[Theorem 4.1~(2)]{BCIE}.
  Hence, the conclusion follows  by Theorem~\ref{T:triple-B}.\end{proof}

%
%
%
%
%
%
%
%
%
%
%
%

 \begin{prop}\label{P:B-infty} The cotorsion pair $(\A^{\infty}, \B_{\infty})$ from Notation~
 \ref{N:notations}~(5) satisfies $ex\B_{\infty}=\widetilde{\B_{\infty}}$,  hence it induces a recollement:
 \vskip0.7cm
\[
\xymatrix{\dfrac{\widetilde{\B_{\infty}}}{\sim} \ar[rr]^{inc} &&{K(\B_{\infty})} \ar@/^2pc/[ll]\ar@/_2pc/[ll] \ar[rr]^{Q}
&&{\dfrac{\Ch(\B_{\infty})}{\widetilde{\B_{\infty}} }}\ar@/^2pc/ [ll] \ar@/_2pc/ [ll] }
.\]
\vskip0.7cm
\end{prop}
%
%
%
%
%
%

%
\begin{proof} By Proposition~\ref{P:ex-B-infty} we have $ex\B_{\infty}=\widetilde{\B_{\infty}}$, hence the conclusion follows again by Theorem~\ref{T:triple-B}.\end{proof}

In view of Lemma~\ref{L:Cot-inj} we have the following characterization.

\begin{prop}\label{P:no-tilde} Let $(\A, \B)$ be a complete hereditary cotorsion pair in $\Modr R$ with $\B\supseteq  \B_{\infty}$.

Then in $\Ch(\B)$ there exists an injective cotorsion pair $(\widetilde{\B}, \R)$ with $
\R\subseteq dw\Inj$ \iff $ex\B=\tilde\B$.\end{prop}
%

\begin{proof}
Assume that in $\Ch(\B)$ there is an injective cotorsion pair $(\tilde \B, \R) $ with $
\R\subseteq dw\Inj$.  This means that $\R=\tilde\B{}^\perp{} \cap dw \B$. By Lemma~\ref{L:Cot-inj}~(3) and (4), $\R= dg\Inj$ and ${}^\perp{} \R\cap dw\B=ex\B$. Then, $ex\B=\tilde\B$.

Conversely, if $ex\B=\tilde\B$ then $(\tilde\B, dg\Inj)$ is an injective cotorsion pair in $\Ch(\B)$ by Example~\ref{E:examples-inj}~(3).
\end{proof}
%
\begin{rem}\label{R:many-B} Note that complete hereditary cotorsion pairs $(\A, \B)$ satisfying $\B\supseteq  \B_{\infty}$ may be abundant, since  $ \B_{\infty}$ may be rather small.

For instance, if the little finitistic dimension of $R$  is finite (e.g. $R$ is  semihereditary), then $ \B_{\infty}$ coincides with the class of injective modules (see Proposition~\ref{P:semihereditary}).
\end{rem}
%



A positive answer to the question in the title of this section is provided by \sto\  in \cite{Stopurity} for  the cotorsion pair $(\A, \FpInj)$ generated by the class of finitely presented modules over a coherent ring $R$.
In view of Example~\ref{E:examples-proj}~(1) and Example~\ref{E:examples-inj}~(1), we restate \Sto's theorem in our notations.
 %
 \begin{prop}\label{P:fp-inj}(\cite[Proposition 6.11, Theorem 6.12]{Stopurity})  Let $R$ be a coherent ring and let $(\A, \FpInj)$ be the complete hereditary cotorsion pair generated by the class of finitely presented modules. Then:
 \[{}^\perp{} dw\Inj\cap dw \FpInj=\widetilde{\FpInj}\]
  hence $\Big([dg\A\cap dw\FpInj]_K, \widetilde{\FpInj}, [dw \Inj]_K\Big)$ is a localizing cotorsion triple in $\Ch(\FpInj)_{dw}$. There are equivalences:
 %
\[\frac{[dg\A\cap dw\FpInj]_K}{\sim} \cong \frac{\Ch(\FpInj)}{ \widetilde{\FpInj}} \cong  \frac{[dw \Inj]_K}{\sim}\]
where $\sim$ is the chain homotopy equivalence
and a recollement:
\vskip0.7cm
\[
\xymatrix{\dfrac{ \widetilde{\FpInj}}{\sim} \ar[rr]^{inc} &&{K(\FpInj)} \ar@/^2pc/[ll]\ar@/_2pc/[ll] \ar[rr]^{Q}
&&{\dfrac{\Ch(\FpInj)}{ \widetilde{\FpInj}}\cong \D(\FpInj)}\ar@/^2pc/ [ll] \ar@/_2pc/ [ll] }
.\]
\vskip0.7cm
\end{prop}
%
%
%
%
%
%
%

We exhibit now another case of cotorsion pairs giving rise to a result analogous to Proposition~\ref{P:fp-inj}

%
%
%
%
%
%
%
%
%
%
%
%

\begin{prop}\label{P:B-inj-n} Let $R$ be a coherent ring and let $(\A, \B)$ be a complete hereditary cotorsion pair in $\Modr R$. Assume that $\B\subseteq \FpInj$ and that $\B\subseteq \I_n$ (the class of modules of injective dimension at most $n$).

 Then, every Fp-injective  $\B$-periodic module belongs to $\B$. Thus
 $\widetilde{\FpInj}\cap dw \B=\tilde{\B}$, $ ^\perp dw\Inj \cap dw\B=\tilde{\B}$ and we have a recollement:
%
\vskip0.7cm
\[
\xymatrix{\dfrac{ \widetilde{\B}}{\sim} \ar[rr]^{inc} &&{K(\B)} \ar@/^2pc/[ll]\ar@/_2pc/[ll] \ar[rr]^{Q}
&&{\dfrac{\Ch(\B)}{ \widetilde{\B}}\cong \D(\B)}\ar@/^2pc/ [ll] \ar@/_2pc/ [ll] }
.\]
\vskip0.7cm

%
%
\end{prop}
 %
 \begin{proof} Let $(\ast)\quad 0\to M\to B\to M\to 0$ be an exact sequence with $M$ Fp-injective and $B\in \B$. Let $ 0\to M\to E\to M_1\to 0$ be an exact sequence with $E$ injective; then, $M_1$ is Fp-injective. An application of the horseshoe lemma gives the following commutative diagram:
 \[\xymatrix{
&0 \ar[d]  & 0 \ar[d] & 0 \ar[d] \\
0 \ar[r] & M \ar[d] \ar[r] &B\ar[r] \ar[d] & M\ar[d]  \ar[r] & 0 \\
0 \ar[r] & E \ar[d] \ar[r] &E\oplus E\ar[r] \ar[d] & E\ar[d]  \ar[r] & 0  \\
0 \ar[r] & M_1 \ar[d]\ar[r] &D\ar[r] \ar[d] & M_1\ar[d]  \ar[r] & 0 \\
&0 & 0 & 0
}
\]
where  $D\in \B$, since $\B$ is coresolving. We have \[ \mathrm{inj.dim} D= \mathrm{inj.dim} B-1, \] hence w.l.o.g. we can assume that in our starting sequence  $(\ast)$ $B$ has injective dimension at most $1$. Thus, in the above diagram we have that $D$ is injective and, by Fact~\ref{F:periodic}~(1), we conclude that $M_1$ is injective. The latter implies that inj.dim $M\leq 1$.
Let $A\in \A.$ Then \[0=\Ext^1_R(A, B)\to \Ext^1_R(A, M)\to \Ext^2_R(A, M)=0,\] hence $M\in \B$ and
$\widetilde{\FpInj}\cap dw \B=\tilde{\B}$  by Fact~\ref{F:periodic}~(2).  Hence, the equality $ ^\perp dw\Inj \cap dw\B=\tilde{\B}$ is obtained by intersecting with $dw\B$ the equality $^\perp{} dw\Inj\cap dw \FpInj=\widetilde{\FpInj}$ from \cite[Proposition 6.11 ]{Stopurity}.

The existence of a recollement as in the statement follows by the same arguments as in the proof of Proposition~\ref{P:fp-inj} applied to the cotorsion pair $(\A, \B)$ in the assumptions.\end{proof}



 %\section{Injective cotorsion pairs in the exact category $\Ch(\A)$}
\section{Injective cotorsion pairs in the exact category \texorpdfstring{$\Ch(\A)$}{Ch(A)}}

 In this section we state results dual to the ones in Section~\ref{S:B}. Their proofs are obtained  by dual arguments.

%
Starting with a complete hereditary cotorsion pair $(\A, \B)$ in a Grothendieck  category $\G$, we exhibit three examples of injective cotorsion pairs in $\Ch(\G)$ satisfying the assumptions of Proposition~\ref{P:prop7.3}. Note that the examples below satisfy Proposition~\ref{P:prop7.3} since $\tilde{\A}$, $dg\A$ and $ex\A$  are thick in $\Ch(\A)$ by Lemma~\ref{L:HOM} and they clearly contain the contractible complexes with terms in $\A$.




 \begin{expl}\label{E:exam-A-inj}
 \begin{enumerate}
   \begin{sloppypar}
   \item The complete hereditary cotorsion pair ${( \widetilde{\A}, dg\B)}$ in $\Ch(\G)$ satisfies the conditions in  Proposition~\ref{P:prop7.3}, hence we have the injective cotorsion pair:
%
\[\Delta_1=\Big(\tilde\A, dg{\B}\cap dw\A\Big)\] in $\Ch(\A)$ and the localizing cotorsion pair \[\Big(\tilde\A,  [dg{\B}\cap dw\A]_K\Big)\] in $\Ch(\A)_{dw}.$

\item The complete hereditary cotorsion pair $(dg\A, \widetilde{\B})$ in $\Ch(\G)$ satisfies the conditions in  Proposition~\ref{P:prop7.3},  hence we have the injective cotorsion pair:
 \[\Delta_2=\Big(dg\A, \tilde\B\cap dw\A\Big)\] in $\Ch(\A)$ and the localizing cotorsion pair \[\Big(dg\A,  [\tilde\B\cap dw\A]_K\Big)\] in $\Ch(\A)_{dw}.$

\item If $\A$ is deconstructible, then by Proposition~\ref{P:complete?}~(5) $(ex\A, ex\A{}^\perp{})$ is a complete hereditary cotorsion pair in $\Ch(\G)$ and it satisfies the conditions in Proposition~\ref{P:prop7.3}, hence we have the injective cotorsion pair
 \[\Delta_3=\Big(ex\A, ex\A{}^\perp{}\cap dw\A\Big)\] in $\Ch(\A)$ and the localizing cotorsion pair \[\Big(ex\A,  [ex{}^\perp{}\A\cap dw\B]_K\Big)\] in $\Ch(\A)_{dw}.$
 \end{sloppypar}

 \end{enumerate}
 \end{expl}
 %
 %
The three injective cotorsion pairs $\Delta_1, \Delta_2, \Delta_3$ of Example~\ref{E:exam-A-inj} satisfy the conditions of \cite[Theorem 3.4]{G7}, hence we have:
%
\begin{thm}\label{T:recollement-inj-A} Let $(\A, \B)$ be a complete hereditary cotorsion pair in a  Grothendieck category $\G$ such that $\A$ is deconstructible. Then, there is a recollement:

%
%
%
%
%
%
\vskip 0.7cm
\[
\xymatrix{\dfrac{\widetilde{\B}\cap dw \A}{\sim} \ar[rr]^{inc} &&\dfrac{dg{\B}\cap dw \A}{\sim} \ar@/^2pc/
[ll]\ar@/_2pc/[ll] \ar[rr]^{Q}
&&{\Ch(\A)/ex\A}\ar@/^2pc/ [ll] \ar@/_2pc/ [ll] }
\]
\vskip 0.7cm

where $\sim$ denotes the homotopy relation associated to the corresponding model structure and coincides with the chain homotopy relation; moreover, $inc$ is the inclusion and $Q$ is the quotient functor.
\end{thm}
%
%
\begin{thm}\label{T:derived-A}  In the setting of Example~\ref{E:exam-A-inj}~(1), \[\Delta_1=\Big(dw{\A}, \widetilde{\A}, dg\B\cap dw \A\Big)\] is an exact model structure in the category $\Ch(\A)$.
 In particular, we can define the derived category $\D(\A)$ as the quotient  $\Ch(\A)/\widetilde{\A}$.

 Moreover, we have the following triangle equivalences between the derived category of $\D(\A)$ and the homotopy category of the model structure $\Delta_1$:
 \[\D(\A)=\Ho(\Delta_1)\cong \dfrac{dg{\B}\cap dw \A}{\widetilde{\A}\cap\widetilde{\B}}\]
 and in the assumptions of Theorem~\ref{T:recollement-inj-A} there is also a recollement:
\vskip0.7cm
\[
\xymatrix{\dfrac{ex\A}{\sim} \ar[rr]^{inc} &&{\D(\A)} \ar@/^2pc/[ll]\ar@/_2pc/[ll] \ar[rr]^{Q}
                                           &&{\D(\G)}\ar@/^2pc/ [ll] \ar@/_2pc/ [ll] }
\]
\vskip0.7cm
where $ex\A/\sim$ is the full subcategory of $\D(\A)$ consisting of exact complexes (in $\Ch(\G)$).


 \end{thm}
 %
 \begin{proof} Dual of Theorem~\ref{T:derived-B}.
 \end{proof}
 %
 %
 Another way to obtain the exact model structure of Theorem~\ref{T:derived-A} is to use results by Gillespie in \cite{G4}, \cite{G8} and \cite{G9}.

 \begin{thm}\label{T:Gill-A}
   \begin{sloppypar}
   Let $(\A, \B)$ be a complete hereditary cotorsion pair in $\G$ such that $\A$ is deconstructible.  The two cotorsion pairs
   ${(dw\A, dw\A{}^\perp{} )}$ and $(\tilde\A, dg\B)$ in $\Ch(\G)$ are hereditary and complete and give rise to a cofibrantly generated model structure ${\N=\Big(dw\A, \W, dg\B\Big)}$ in $\Ch(\G)$ satisfying ${\W\cap dw\A=\tilde\A}$ and ${\W\cap dg\B=dw\A{}^\perp{} }$ whose restriction in $\Ch(\A)$ is the exact model structure $\N_1=\Big(dw{\A}, \widetilde{\A},  dw \A\cap dg\B\Big)$ of Theorem~\ref{T:derived-A}.
 \end{sloppypar}

%
%
\end{thm}

 \begin{proof} The smallness of the cotorsion pairs $(dw\A, dw\A{}^\perp{} )$ and $(\tilde\A, dg\B)$ follow by the fact that $dw\A$ and $\tilde\A$ are deconstructible in $\Ch(\G)$ (see  Proposition~\ref{P:complete?}~(3) and ~\ref{P:description-tilde}~(3))  and they are hereditary since $(\A, \B)$ is hereditary. The existence of the model structure $\N$ in $ \Ch(R)$ follows by \cite[Theorem 1.1]{G9}. The fact that the model structure is cofibrantly generated follows by \cite[Section 7.4]{Hov02}.
 %
\end{proof}
%
%
%
 %
%
  %\section{Projective cotorsion pairs in the exact category $\Ch(\A)$}
\section{Projective cotorsion pairs in the exact category \texorpdfstring{$\Ch(\A)$}{Ch(A)}}
   In this section we state results dual to the ones in Section~\ref{S:B-2}.
 %
 %
\begin{expl}\label{E:exam-A-proj}

Starting with a complete hereditary cotorsion pair $(\A, \B)$ in a Grothendieck  category $\G$ with enough projective objects, we exhibit three examples of projective cotorsion pairs in $\Ch(\G)$ satisfying the assumptions of Proposition~\ref{P:7.2-Groth}.

%
 \begin{enumerate}
\item  By Proposition~\ref{P:complete?}~(3), $(dw\Proj, dw \Proj{}^\perp{})$ is a complete cotorsion pair  in $\Ch(\G)$, and it is a projective cotorsion pair. By Proposition~\ref{P:7.2-Groth},  we have the projective cotorsion pair:
 \[\Gamma_1=\Big(dw\Proj, dw \Proj{}^\perp{}\cap dw\A\Big)\] in $\Ch(\A)$ and the localizing cotorsion pair \[\Big([dw\Proj]_K, dw \Proj{}^\perp{}\cap\A\Big)\] in $\Ch(\A)_{dw}.$
 \item By Proposition~\ref{P:complete?}~(6), $(ex\Proj, ex\Proj{}^\perp{})$ is a projective cotorsion pair in $\Ch(\G)$ and by Proposition~\ref{P:7.2-Groth} we have the projective cotorsion pair:
  \[\Gamma_2=\Big(ex\Proj, ex \Proj{}^\perp{}\cap dw\A\Big)\] in $\Ch(\A)$ and the localizing cotorsion pair \[\Big([ex\Proj]_K, ex \Proj{}^\perp{}\cap\A\Big)\] in $\Ch(\A)_{dw}.$
  \item Since $(dg\Proj, \E )$ is a projective cotorsion pair in $\Ch(\G)$, by Proposition~\ref{P:7.2-Groth} we have the projective cotorsion pair:
  \[\Gamma_3=\Big(dg\Proj, ex\A\Big)\]  in $\Ch(\A)$ and the localizing cotorsion pair \[\Big([dg\Proj]_K, ex\A\Big)\] in $\Ch(\A)_{dw}.$
\end{enumerate}
\end{expl}

 The above three examples $\Gamma_1, \Gamma_2, \Gamma_3$ of projective cotorsion pairs  in $\Ch(\A)$ satisfy the assumptions of \cite[Theorem 3.5]{G7}. Hence we obtain:
  %
  \begin{thm}\label{T:recollement-proj-A} If $\G$ is a Grothendieck category with enough projective objects and $(\A, \B)$ is a complete hereditary cotorsion pair in $\G$, there is a recollement %

\vskip 0.7cm
\[
\xymatrix{\dfrac{ex \Proj}{\sim} \ar[rr]^{inc} &&\dfrac{dw\Proj}{\sim} \ar@/^2pc/
[ll]\ar@/_2pc/[ll] \ar[rr]^{Q}
&&{\Ch(\A)/ex\A}\ar@/^2pc/ [ll] \ar@/_2pc/ [ll] }
\]
\vskip 0.7cm

where $\sim$ denotes the homotopy relation associated to the corresponding model structure and coincides also with the chain homotopy relation; moreover, $inc$ is the inclusion, $Q$ is the quotient functor.  In particular, the central term is the chain homotopy category $K(\Proj)$ of the complexes with projective components and the right hand term is equivalent to the derived category of $\G$.
%
%
\end{thm}

Moreover, we have:

\begin{thm}\label{T:triple-A} If $\G$ is a Grothendieck category with enough projective objects and $(\A, \B)$ is a complete hereditary cotorsion pair in $\G$ such that $\A$ is deconstructible, the triple
\[\Big([dg\Proj]_K, ex\A, [ex\A{}^\perp{}\cap dw\A]_K\Big)\] is a localizing cotorsion triple in $\Ch(\A)_{dw}$ and there are equivalences of triangulated categories:
\[\frac{[dg\Proj]_K}{\sim} \cong \frac{\Ch(\A)}{ex\A} \cong  \frac{[ex\A{}^\perp{}\cap dw\A]_K}{\sim}\]
where $\sim$ is the chain homotopy equivalence
and a recollement:
\vskip0.7cm
\[
\xymatrix{\dfrac{ex\A}{\sim} \ar[rr]^{inc} &&{K(\A)} \ar@/^2pc/[ll]\ar@/_2pc/[ll] \ar[rr]^{Q}
                                           &&{\dfrac{\Ch(\A)}{ex\A }=\D(\A)}\ar@/^2pc/ [ll] \ar@/_2pc/ [ll] }
.\]
\vskip0.7cm
\end{thm}
\begin{proof}
  \begin{sloppypar}
  By Examples~\ref{E:exam-A-inj}~(3) and Examples~\ref{E:exam-A-proj}~(3)  we have two localizing cotorsion pairs ${\Big([dg\Proj]_K, ex\A\Big)}$ and ${\Big(ex\A, [ex\A{}^\perp{}\cap dw\A]_K\Big)}$ in $\Ch(\A)_{dw}$.  Thus, the statement follows by arguing as in the proof of Theorem~\ref{T:triple-B}.
\end{sloppypar}
\end{proof}


%\section{When is $\tilde{\A}$ the central term of a localizing cotorsion triple in $\Ch(\A)_{dw}$?}
\section{When is \texorpdfstring{$\tilde{\A}$}{tilde A} the central term of a localizing cotorsion triple in \texorpdfstring{$\Ch(\A)_{dw}$}{Ch(A)_{dw}}?}
%
%

By Example~\ref{E:exam-A-inj}~(1) we have shown that in $\Ch(\A)$ there is an injective cotorsion pair  with $\tilde \A$
 as left term and Example~\ref{E:exam-A-proj}~(3) provides a projective cotorsion pair in $\Ch(\A)$ with right component $ex{\A}$.
%
 Our aim will be to find cotorsion pairs $(\A, \B)$ for which there exists a projective cotorsion pair $(\C, \widetilde{\A})$ in $\Ch(\A)$ with $
\C\subseteq dw\Proj$ in order to obtain a localizing cotorsion triple in $\Ch(\A)_{dw}$ with $\widetilde{\A}$ as central term.

%

The most famous example of this situation is provided by the cotorsion pair $(\Flat, \Cot)$. In fact, by \cite[Theorem 8.6]{Nee08} $dw(\Proj)^\perp\cap dw{\Flat}=\widetilde{\Flat}$. Hence, as noted by Gillespie in \cite{G7}, Example~\ref{E:exam-A-proj}~(1) provides the wanted example and induces Neeman's recollement, that is the recollement as in Theorem~\ref{T:triple-A}:
\vskip0.7cm
\[
\xymatrix{(a)\quad\dfrac{\widetilde{\Flat} }{\sim} \ar[rr]^{inc} &&{K(\Flat)} \ar@/^2pc/[ll]\ar@/_2pc/[ll] \ar[rr]^{Q}
&&{\dfrac{\Ch(\Flat)}{\widetilde{\Flat} }}\ar@/^2pc/ [ll] \ar@/_2pc/ [ll] }
.\]
\vskip0.7cm

Another case of cotorsion pairs giving rise to a result analogous to the previous one is provided by the following:
\begin{prop}\label{P:A-flat}
  \begin{sloppypar}
  Let $(\A, \B)$ be a complete hereditary cotorsion pair in $\Modr R$. Assume that $\A\subseteq \Flat$ and that $\A\subseteq \clP_n$, where $\clP_n$ is the class of modules of projective dimension at most $n$.
  Then, every flat $\A$-periodic module belongs to $\A$. Thus ${\widetilde{\Flat}\cap dw \A=\tilde{\A}}$, ${ (dw\Proj)^\perp\cap dw\A=\tilde{\A}}$ and there is a recollement:
\end{sloppypar}
\[
\xymatrix{\dfrac{\widetilde{\A} }{\sim} \ar[rr]^{inc} &&{K(\A)} \ar@/^2pc/[ll]\ar@/_2pc/[ll] \ar[rr]^{Q}
&&{\dfrac{\Ch(\A)}{\widetilde{\A} }}\ar@/^2pc/ [ll] \ar@/_2pc/ [ll] }
.\]
\vskip0.7cm
\end{prop}

 %
 \begin{proof} Let $0\to F\to A\to F\to 0$ be an exact sequence with $F$ flat and $A\in \A$. Let $0\to F_1\to P\to F\to 0$ be an exact sequence with $P$ projective; then $F_1$ is flat. An application of the horseshoe lemma gives the following commutative diagram:
 \[\xymatrix{
&0 \ar[d]  & 0 \ar[d] & 0 \ar[d] \\
0 \ar[r] & F_1 \ar[d] \ar[r] &Q\ar[r] \ar[d] & F_1\ar[d]  \ar[r] & 0 \\
0 \ar[r] & P \ar[d] \ar[r] &P\oplus P\ar[r] \ar[d] & P\ar[d]  \ar[r] & 0  \\
0 \ar[r] & F \ar[d]\ar[r] &A\ar[r] \ar[d] & F\ar[d]  \ar[r] & 0 \\
&0 & 0 & 0
}
\]
where  $Q\in \A$, since $\A$ is resolving. We have p.dim $Q=$ p.dim $A-1$, hence w.l.o.g. we can assume that in our starting sequence  $0\to F\to A\to F\to 0$ $A$ has projective dimension at most $1$. Thus, in the above diagram we have that $Q$ is projective and by \cite{BG}, $F_1$ is projective. The latter implies that p.dim $F\leq 1$.
Let $B\in \B.$ Then \[0=\Ext^1_R(A, B)\to \Ext^1_R(F, B)\to \Ext^2_R(F, B)=0,\] hence $F\in \A$ and
$\widetilde{\Flat}\cap dw \A=\tilde{\A}$ by Fact~\ref{F:periodic}~(2).  Now the equality $ (dw\Proj)^\perp\cap dw\A=\tilde{\A}$ is obtained by intersecting with $dw\A$ the equality $dw(\Proj)^\perp\cap dw{\Flat}=\widetilde{\Flat}$ from \cite[Theorem 8.6]{Nee08}.

The arguments used above to obtain the recollement $(a)$ for the cotorsion pair $(\Flat, \Cot)$ can be repeated for the case of the cotorsion pair $(\A, \B)$ in our assumption to obtain the stated recollement.\end{proof}
%
 \begin{rem}
   The above proposition applies, for example, to the case of the cotorsion pair $(\A,\B)$ generated by the localizations $\{R[s^{-1}]~\mid~s\in R\}$ of a commutative ring $R$. The class $\A$ was introduced by Positselski in \cite{Pos12} and called the class of very flat modules. Clearly $\A\subseteq \Flat \cap \clP_1$.
 \end{rem}

Another situation is provided by Proposition~\ref{P:finite-proj-dim}:
\begin{prop}\label{P:new-triple-A}
Let $\G$ be a Grothendieck category with enough projective objects. Let $(\A, \B)$ be a complete hereditary cotorsion pair in $\G$ with $\B\subseteq \I$ ($\I$ the class of objects with finite injective dimension). Then $ex\A=\tilde\A$, hence $(dg\Proj, \tilde{\A})$ is a projective cotorsion pair in $\Ch(\A)$ and there is a recollement as in Theorem~\ref{T:triple-A} with $ex \A$ replaced by $\tilde\A$.
 In particular, the derived category $\D(\A)$ of $\A$ is equivalent to the usual derived category of $\G$.
 \end{prop}
 %
\begin{cor}\label{C:recoll-cotilting} Let $(\C, \B)$ be an $n$-cotilting cotorsion pair in $\Modr R$. For the cotilting class $\C$ we have a recollement:
\vskip0.7cm

\[
\xymatrix{\dfrac{\widetilde{\C}}{\sim} \ar[rr]^{inc} &&{K(\C)} \ar@/^2pc/[ll]\ar@/_2pc/[ll] \ar[rr]^{Q}
&&{\dfrac{\Ch(\C)}{\widetilde{\C} }}\ar@/^2pc/ [ll] \ar@/_2pc/ [ll] }
.\]
\vskip0.7cm
\end{cor}
\begin{proof} By Proposition~\ref{P:tilt-cotil}, $ex\C=\widetilde{\C}$, hence the conclusion follows by Proposition~\ref{P:new-triple-A}.
\end{proof}
