%************************************************
\chapter{Cotorsion pairs and model structures in exact categories}\label{ch:acycexact} % $\mathbb{ZNR}$
%************************************************
%\section*{Introduction}
The notion of cotorsion pairs goes back to the seventies when it was introduced by Salce~\cite{Sal} in the case of abelian groups.
It got an enormous impulse thanks to the discovery by Hovey~\cite{Hov07} of the bijective correspondence between abelian model structures and cotorsion pairs in abelian categories. Many examples of cotorsion pairs and the corresponding model structures have been illustrated by Gillespie \cite{G5} who also extended the notion to the case of exact categories.

A famous example of cotorsion pair is given by the pair $(\F, \C)$ where $\F$ is the class of flat objects. It gave rise to the celebrated Flat Cover Conjecture by Enochs and solved in \cite{BEE} in the case of module categories and in ~\cite{ElB} for Grothendieck categories. It is particularly important in categories with no nonzero projective objects like for instance the categories of coherent sheaves.

This chapter is based on the joint work \cite{bazzoni2018recollements}. We will give an overview of the theory of cotorsion pairs and exact model structures in exact categories.

\section{Preliminaries}\label{sec:preliminaries}

\subsection{Cotorsion pairs}\label{S:cot-pair}

The notion of an \emph{exact category} was introduced by Quillen in \cite{Q}.
An exact category is an additive category  $\C$ endowed with  a collection $\Phi$ of kernel-cokernel pairs  satisfying some axioms which allow to work with the sequences  in $\Phi$ as if they were exact sequences in an abelian category. An element $E\in \Phi$ is denoted by $0\to A\overset{i}\to B\overset{d}\to C\to 0$ and is called a \emph{conflation or short exact sequence}. The map $i$ is called \emph{inflation} or \emph{admissible monomorphism} and $d$ is called \emph{deflation} or \emph{admissible epimorphism}.
In an exact category pushouts (pullbacks) of inflations (deflations) exist and inflations (deflations) are stable under pushouts (pullbacks).

The axioms on conflations allow to define the Yoneda functor $\Ext^i_{\C}(M,N)$ for every pair of objects $M,N$
 in $\C$.
For more details see \cite{Kel90} or \cite{Bu}.

We will deal with weakly idempotent complete (WIC) additive categories, that is categories such that every section has a cokernel or, equivalently, every retraction has a kernel.

Given a class $\X$ of objects in an exact category $\C$, the right orthogonal class $\X^{\perp}$ consists of the objects $Y$ such that $\Ext^1_{\C}(X,Y)=0$ for each object $X\in \X$. Similarly, the left orthogonal class $^{\perp}\X $ consists of the objects $Y$ such that $\Ext^1_{\C}(Y,X)=0$ for each object $X\in \X$.

\begin{defn}\label{D:cotorsion-pair} A pair of classes $(\A, \B)$ in an exact category $\C$ is called a cotorsion pair if
\begin{enumerate}
\item $\A^{\perp}=\B$ and $^{\perp} \B=\A$.
\item A cotorsion pair is \emph{generated (cogenerated)} by a class $\X$ of objects if $\B=\X^{\perp} (\A={^{\perp}\X})$.
\item A cotorsion pair $(\A, \B)$ has \emph{enough projectives} if every object $C\in \C$ has a special $\A$-precover, that is there is a short exact sequence $0\to B\to A\to C\to 0$ in $\C$ with $A\in \A$ and $\B\in B$. Dually, we say that  $(\A, \B)$ has \emph{enough injective} if every object $C\in \C$ has a special $\B$-preenvelope, that is there is a short exact sequence $0\to C\to B\to A\to 0$ in $\C$ with $A\in \A$ and $\B\in B$.
\item A cotorsion pair is \emph{complete} when it has \emph{enough injectives} and \emph{enough projectives}.
\item A cotorsion pair is called \emph{hereditary} if $\A$ is generating, $\B$ is cogenerating, and
  \[ \Ext^i_{\C}(A,B)=0 \text{ for all }A\in \A\text{, }B\in \B\text{, and } i\geq 1. \]
\end{enumerate}
\end{defn}
A class $\C$ of objects in an exact category is \emph{deconstructible} and denoted by $\F ilt\ \clS$, if there is a set $\clS$ of objects such that every object of $\C$ is a transfinite extension of objects of $\clS$ (for more details see \cite[Definition 3.7 and 3.10]{Sto13}).

It is possible to prove, using the so called Small Object Argument, that any cotorsion pair $(\A,\B)$ generated by a set in a category of modules is complete (see \cite{Q} or \cite{ET01}). The argument can be actually extended to Grothendieck categories, provided that $\A$ is generating. We give a precise statement in the following lemma.

\begin{lem}\label{lem1_2} Let $(\A,\B)$ be a cotorsion pair in a Grothendieck category such that $\A$ is generating. Then:
  \begin{enumerate}
    \item $\A$ is generated by a set if and only if it is deconstructible.
    \item If the equivalent conditions in (1) hold, then $(\A,\B)$ is complete.
  \end{enumerate}
\end{lem}

\begin{proof}
  (1) If $\A$ is deconstructible, call $\clS$ the set such that $\A = \F ilt\,\clS$. Then, $\F ilt\,\clS \subseteq {^\perp(\clS^\perp)}$ by Eklof's lemma, but ${^\perp(\clS^\perp)}\subseteq\A$ so they are actually equal, i.e. $(\A,\B)$ is generated by $\clS$.
  Conversely, if $(\A,\B)$ is generated by a set $\clS$, it is also generated by $\clS'=\clS\cup\{G\}$, where $G\in\A$ is a generator. Then, by \cite[Theorem~5.16]{Sto13} $\A$ consists of retracts of $\F ilt \clS$, and by \cite[Proposition~2.9(1)]{St10-deconstr} it is deconstructible.

  (2) \cite[Theorem~5.16]{Sto13} actually gives a proof of this statement.
\end{proof}

%


We will mostly deal with  hereditary cotorsion pairs and
in order to characterize them we recall the following definition.
\begin{defn}\label{D:thick} Let $\C'$ be  a full subcategory of a WIC exact category $\C$.
\begin{enumerate}
\item $\C'$ is \emph{thick} if it is closed under direct summands and has the 2 out of 3 property on short exact sequences.
\item $\C'$ is \emph{resolving} in $\C$ if $A\in \C'$ for every exact sequence $0\to A\to B\to C\to 0$ in $\C$ with $B, C\in \C'$.
\item $\C'$ is \emph{coresolving} in $\C$ if $C\in \C'$ for every exact sequence $0\to A\to B\to C\to 0$ in $\C$ with $A, B\in \C'$.
\end{enumerate}
\end{defn}

It can be shown that a complete cotorsion pair $(\A, \B)$ is hereditary \iff $\A$ is resolving or, equivalently, \iff $\B$ is coresolving:
\begin{lem}{\cite[Lemma~6.17]{Sto13}}
  Let $\C$ be a WIC exact category and $(\A,\B)$ be a cotorsion pair in $\C$ such that $\A$ is generating and $\B$ is cogenerating (e.g. if $(\A,\B)$ is complete). Then the following are equivalent:
  \begin{enumerate}
    \item $\Ext^n_\C(A,B)=0$ for each $A\in\A$, $B\in\B$, and $n\geq 1$.
    \item $\Ext^2_\C(A,B)=0$ for each $A\in\A$ and $B\in\B$.
    \item $\A$ is resolving.
    \item $\B$ is coresolving.
  \end{enumerate}
\end{lem}

%
%
%

\subsection{Model structures}\label{S:model}
The notion of model structures on categories with finite limits and colimits was introduced by Quillen in \cite{QHtp}.
For our purposes we refer to the book by Hovey \cite{Hov99} or to the survey \cite{Sto13}.

We recall that a \emph{model structure} on a category $\C$ consists
of three classes of morphisms Cof, W, Fib called \emph{cofibrations}, \emph{weak equivalences} and \emph{fibrations}, respectively, satisfying certain axioms.

\begin{defn}
  Let Cof, W, Fib be a model structure on a category $\C$.
  An object $X\in \C$ is \emph{cofibrant} if $0\to X$ is a cofibration,  \emph{fibrant} if $X\to 0$ is a fibration and it is \emph{trivial} if $0\to X$ is a weak equivalence.
\end{defn}

In particular, the class $\W$ of trivial objects has the 2-out-of-3 property.
%A \emph{model category} $\C$ is an abelian cocomplete category with a model structure.
 The \emph{homotopy category} $\Ho \C$ is obtained by  formally inverting all morphisms in W.

A tremendous impulse to the theory was given by  Hovey who discovered in~\cite{Hov07} a bijective correspondence between abelian model structures and cotorsion pairs in abelian categories.
%
In \cite{G5} Gillespie extended the notion of model structures on exact categories and proved the analogous of Hovey's correspondence in this more general setting.
 We recall the basic notions and results.
 \begin{defn}\label{D:exact-model} An \emph{exact model structure} on an exact category $\C$ is a model structure such that cofibrations (fibrations) are the inflations (deflations) with cofibrant (fibrant) cokernels (kernels).

   An \emph{abelian model structures} is an exact model structure on an abelian category considered as an exact category with the exact structure given precisely by the class of short exact sequences.
\end{defn}
%

We explicitly state the aforementioned Hovey's correspondence, in the setting of WIC exact categories.

\begin{thm}\label{T:correspondence} (\cite{Hov07}, \cite{G5}) Let $\C$ be a WIC exact category with an exact model structure. Let $\clQ$ be the class of cofibrant objects, $\R$ the class of fibrant objects and $\W$ the class of trivial objects. Then $\W$ is a thick subcategory of $\C$, and $(\clQ, \R\cap \W)$ and $(\clQ\cap\W, \R)$ are complete cotorsion pairs in $\C$.
Conversely, given three classes $\W, \clQ, \R$ such that $\W$ is thick in $\C$, $(\clQ, \R\cap \W)$ and $(\clQ\cap\W, \R)$ are complete cotorsion pairs in $\C$, then there is an exact model structure on $\C$ where $\clQ$ are the cofibrant objects, $\R$ are the fibrant objects and $\W$ the trivial objects.
\end{thm}

Given an exact model structure on $\C$, let $\clQ$, $\W$, and $\R$ the three classes from the Theorem, we will follow the usual convention and indicate the model structure via $\M=(\clQ,\W,\R)$ and call it a \emph{Hovey triple}.

The main reason to use model structure is to be able to do homotopy theory, i.e. define good homotopy relations between maps and study the equivalence classes of morphisms modulo homotopy. The usual setting is that of a bicomplete category, however it is not necessary to assume that our category has all limits and colimits, and in fact an exact category has enough limits and colimits to be able to construct left and right homotopies from a model structure.

The following Proposition characterizes left and right homotopies in terms of cotorsion pairs.

\begin{prop}{\cite[Proposition~2.5]{G7}}\label{prop:gil_homotopy}
  Let $\C$ be an exact category with an exact model structure. Let $(\clQ,\W\cap\R$ and $(\clQ\cap\W,\R)$ be the corresponding complete cotorsion pairs from Theorem \ref{T:correspondence}.
  \begin{enumerate}[label=(\arabic*)]
    \item Two maps $f,g:X\to Y$ in $\C$ are \emph{right homotopic} \iff $g-f$ factors through a trivially cofibrant object, i.e. one in $\clQ\cap\W$.
    \item Two maps $f,g:X\to Y$ in $\C$ are \emph{left homotopic} \iff $g-f$ factors through a trivially fibrant object, i.e. one in $\W\cap\R$.
    \item Suppose $Y$ is fibrant, i.e. $Y\in \R$. Then two maps $f,g:X\to Y$ in $\C$ are right homotopic \iff $g-f$ factors through an object of $\clQ\cap\W\cap\R$.
    \item Suppose $X$ is cofibrant, i.e. $X\in\clQ$. Then two maps $f,g:X\to Y$ in $\C$ are left homotopic \iff $g-f$ factors through an object of $\clQ\cap\W\cap\R$.
      \item\label{prop:gil_homotopy.5} Suppose $X$ is cofibrant and $Y$ is fibrant. Then two maps $f,g:X\to Y$ in $\C$ are homotopic \iff $g-f$ factors through an object of $\clQ\cap\W\cap\R$ \iff $g-f$ factors through an object of $\clQ\cap\W$ \iff $g-f$ factors through an object of $\R\cap\W$.
  \end{enumerate}
\end{prop}

For a WIC exact category $\C$, it is possible to show that the homotopy category of an exact model structure satisfies a universal property, namely that it is the triangulated localization of $\C$ with respect to the class $\W$ of trivial objects. This was done explicitly in \cite{G7} in the context of injective model structure.

First, let's observe that the class of weak equivalence, and hence the homotopy category, depend only on the trivial objects.

\begin{lem}{\cite[Lemma~3.1]{G7}}
  Let $\C$ be a WIC exact category with an exact model structure, and let $\W$ denote the class of trivial objects. Then, a map $f$ is a weak equivalence \iff it factors as an admissible monomorphism with cokernel in $\W$ followed by an admissible epimorphism with kernel in $\W$.
\end{lem}

Hence, whenever we have a model structure $\M=(\clQ,\W,\R)$ on $\C$ its homotopy category depends only on $\W$ and will be denoted as $\C/\W$, or, when it is important to remember the rest of the model structure, by $\Ho(\M)$.

To be more explicit, we denote by $\gamma\colon\C\to\C/\W$ the canonical functor to the category $\C/\W$ where we formally inverted weak equivalences. It is a fundamental result of the theory of model categories that
\[ \C/\W=\Ho(\M)\cong (\clQ\cap\R)/\sim \]
where $\sim$ denotes the equivalence relation in Proposition~\ref{prop:gil_homotopy}\ref{prop:gil_homotopy.5}.

We will deal mainly with injective or projective model structure, so we will introduce them.
The notions of injective and projective Hovey triples and of injective and projective cotorsion pairs have been introduced in \cite{G7} following the analogous concepts defined in \cite{Beck14} and \cite{G6}.



%
\begin{defn}  Assume that a WIC exact category $\C$ has enough injective objects.
%
A complete cotorsion pair $(\A, \B)$ in $\C$ is an \emph{injective cotorsion pair} if $\A$ is thick and contains the injective objects.
Symmetrically, assume that $\C$ has enough projective objects.
%
A complete cotorsion pair $(\A, \B)$ in $\C$ is a \emph{projective cotorsion pair} if $\B$ is thick and contains the projective objects.


\end{defn}
%
Thus, an injective cotorsion pair $(\A, \B)$ corresponds to the model structure $(\C, \A, \B)$ where all objects are cofibrant and a projective cotorsion pair $(\A, \B)$ corresponds to the model structure $ (\A, \B, \C)$ where all objects are fibrant.

%
%
Finally, we collect some important information about the localization functor $\gamma$ in the following Proposition.

\begin{prop}{\cite[Proposition~3.2]{G7}}
  Let $\M=(\W,\R)$ be an injective cotorsion pair in a WIC exact category $\C$ with enough injectives. Then:
  \begin{enumerate}
    \item $\R$ naturally inherits the structure of a Frobenius category with the projective-injective objects being precisely the injectives from $\C$.
    \item The functor $\gamma\colon\C\to\C/\W=\Ho(\M)\cong {\R/\sim}$ is exact in the sense that it takes short exact sequences in $\A$ to exact triangles in $\Ho(\M)$.
    \item $\gamma$ is universal among triangulated categories $\T$ which "kill" $\W$. That is, given another exact functor $F\colon \C\to \T$ with $F(\W)=0$, it factors uniquely through $\gamma$.
  \end{enumerate}
\end{prop}

\subsection{Recollements}

A recollement is a functor diagram among triangulated categories summarizing many properties of the functors involved. They were introduced in the seminal paper \cite{bbd82} by Beilinson, Bernstein, and Deligne.

\begin{defn}\label{def:recoll}
  Let $\mathscr{X}$, $\mathscr{Y}$, and $\mathscr{Z}$ be triangulated categories. A \emph{recollement} is a diagram of functors
  \begin{equation*}
    \begin{tikzcd}
      \mathscr{X}\arrow{r}{i_\ast}
      & \mathscr{Y}\arrow[bend left=50]{l}{i^!}\arrow[bend right=50]{l}[']{i^\ast}\arrow{r}{j^\ast}
      & \mathscr{Z}\arrow[bend left=50]{l}{j_\ast}\arrow[bend right=50]{l}[']{j_!}
    \end{tikzcd}
  \end{equation*}
  such that
  \begin{enumerate}[label=(\arabic*)]
    \item $(i^\ast,i_\ast)$, $(i_\ast,i^!)$, $(j_!,j^\ast)$, and $(j^\ast,j_\ast)$ are adjoint pairs;
    \item $i_\ast$, $j_\ast$, and $j_!$ are full embeddings;
    \item $i^!\circ j_\ast$ (hence, $j^\ast\circ i_\ast=0$ and $i^\ast\circ j_!=0$ too);
    \item for each $Y\in\mathscr{Y}$ there are triangles
      \[
        i_\ast i^! Y\to Y\to j_\ast j^\ast Y\to i_\ast i^! Y[1]
      \]
      \[
        j_\ast j^\ast Y\to Y\to i_\ast i^\ast Y\to j_\ast j^\ast Y[1]
      \]
  \end{enumerate}
\end{defn}

In the case of multiple injective cotorsion pairs on a WIC exact category it is possible to define functors that give rise to a recollement.

\begin{lem}{\cite[Lemma~3.3]{G7}}\label{lemma:gill_quotient_functor}
  Let $\C$ be a WIC exact category with enough injectives and suppose we have injective cotorsion pairs $\M=(\W,\R)$ and $\M'=(\W',\R')$ with $\R'\subseteq \R$, and indicate with $\gamma_\W$ and $\gamma_{\W'}$ the localization functors associated with the respective model structures. Then, the quotient functor $Q:{\R/\sim}\to \C/\W'$ defined by $Q([f])=\gamma_{\W'}(f)$ is well defined
\end{lem}

We have the following \emph{Injective Recollement Theorem}:

\begin{thm}{\cite[Theorem~3.4]{G7}}\label{thm:gil_inj_recoll}
  Let $\C$ be a WIC exact category with enough injective and suppose we have three injective cotorsion pairs
  \[
    \M_1=(\W_1,\R_1)\text{, }\M_2=(\W_2,\R_2)\text{, }\M_3=(\W_3,\R_3)
  \]
  such that $\R_2,\R_3\subseteq \R_1$. If $\W_3\cap\R_1=\R_2$ (or equivalently $\W_2\cap\W_3=\W_1$, and $\R_2\subseteq \W_3$), then there exists a recollement
  \begin{equation*}
    \begin{tikzcd}
      {\R_2/\sim}\arrow{r}{I}
      & {\R_1/\sim}\arrow[bend left=50]{l}\arrow[bend right=50]{l}\arrow{r}{Q}
      & \C/\W_3\arrow[bend left=50]{l}{\rho}\arrow[bend right=50]{l}[']{\lambda}
    \end{tikzcd}
  \end{equation*}
  where the functor $I$ is simply the inclusion and $Q$ is the quotient functor of Lemma~\ref{lemma:gill_quotient_functor}. Moreover, $\lambda$ has essential image $(\W\cap\R_1)/\sim$, $\rho$ has essential image ${\R_3/\sim}$, and they provide an equivalence \[\lambda\colon{\R_3/\sim} \longleftrightarrow (\W_2\cap\R_1)/\sim\colon\rho.\]
\end{thm}

Injective (and projective) cotorsion pairs will turn out to be extremely powerful in the case of Frobenius categories, especially in order to find recollements as we will see at the end of this section.

\begin{defn} (\cite[Definition 4.3]{G7}) Let $\C$ be a WIC Frobenius category.
An injective complete cotorsion pair $(\A, \B)$ in $\C$ is called a \emph{localizing cotorsion pair}. If $(\A, \B)$ and $(\B, \D)$ are injective cotorsion pairs in $\C$, then $(\A, \B, \D)$ is called a \emph{localizing cotorsion triple} in $\C$.
\end{defn}

Of course, the fact that $\C$ is Frobenius can be used to find several descriptions of a localizing cotorsion pair. These are listed in \cite[Proposition 4.2]{G7}, that we state here for the reader's convenience, but first we recall the definition of syzygy and cosyzygy.

\begin{defn}
  In any Frobenius category $\C$, the formal suspension of an object $X\in\C$ is an object $\Sigma X$ such that there is a short exact sequence
  \[ 0\to X\to W\to \Sigma X\to 0 \]
  where $W$ is injective. $\Sigma X$ is called the cosyzygy of $X$ and is unique up to a canonical isomorphism in the stable category.
\end{defn}

\begin{prop}{ \cite[Proposition 4.2]{G7}}\label{prop:gil_localizing_pair}
  Let $(\A,\B)$ be a cotorsion pair in a WIC Frobenius category $\C$. Then, the following are equivalent:
  \begin{enumerate}
    \item $(\A,\B)$ is hereditary with $\A$ cosyzygy closed and $\B$ syzygy closed,
    \item $\A$ is both syzygy and cosyzygy closed,
    \item $\B$ is both syzygy and cosyzygy closed,
    \item $\A$ is thick,
    \item $\B$ is thick.
  \end{enumerate}

  Moreover, if $(\A,\B)$ is complete then the conditions above are also equivalent to:
  \begin{enumerate}\setcounter{enumi}{5}
    \item $(\A,\B)$ is an injective cotorsion pair,
    \item $(\A,\B)$ is a projective cotorsion pair.
  \end{enumerate}
\end{prop}

\begin{rem}
  In any WIC Frobenius category there is a canonical cotorsion pair $\M=(\W,\C)$, where $\W$ is the class of projective-injective objects. $\M$ is localizing, and when regarded as a projective cotorsion pair it yields the trivial model structure $(\W,\C,\C)$. However, it can also be considered an injective cotorsion pair, yielding a model structure $(\C,\W,\C)$ whose homotopy category is the stable category ${\C/\sim}$ of $\C$.
\end{rem}

Finally, if we have a localizing cotorsion triple in a WIC Frobenius categories we get a recollement via the following:

\begin{cor}{\cite[Corollary~4.5]{G7}}\label{cor:gill_local_recoll}
  Let $(\X,\Y,\Z)$ be a localizing cotorsion triple in a WIC Frobenius category $\C$. Then, there is an equivalence of triangulated categories:
  \[ {\X/\sim} \cong {\C/\Y}  \cong {\Z/\sim} \]
  where $\X/\sim$ and $\Z/\sim$ are the images of $\X$ and $\Z$ in the stable category $\C/\sim$.

  Moreover, applying Theorem~\ref{thm:gil_inj_recoll} to the injective cotorsion pairs
  \[ \M_1=(\W,\C)\text{, }\M_2=(\X,\Y)\text{, and }\M_3=(\Y,\X) \]
  where $\M_1$ is the canonical localizing cotorsion pair, yields a recollement
  \begin{equation*}
    \recdiagram{\Y/\sim}{\C/\sim}{\C/\Y}.
  \end{equation*}
\end{cor}


\section{More on Cotorsion Pairs}\label{sec:more_cot}

We denote by $\Ch(\C)$ the category of cochain complexes $X$ with component $X^n\in \C$ in degree $n$ and with differential $d^n_X\colon X^n\to X^{n+1}$ for every $n\in \bbZ$. The morphisms in $\Ch(\C)$ are the usual cochain maps. The suspension is denoted by $[-]$.
 If $\C$ is an exact category, then $\Ch(C)$ is equipped with the exact structure where the short exact sequences are the sequences which are exact in each degree. We can also consider the exact structure on $\Ch(\C)$ where the short exact sequences are degreewise splitting. $\Ext_{dw}(X, Y)$ denotes the Yoneda group of these degreewise splitting sequences.

 For every object $C\in \C$, $S^n(C)$ denotes the complex with entries $0$ for every $i\neq n$ and with $C$ in degree $n$; $D^n(C)$ denotes the complex with $C$ in degrees $n$ and $n+1$ and $0$ elsewhere and with differential $d^n$ being the identity on $C$. The homotopy category $\K(\C)$ has the same objects as $\Ch(\C)$ and the equivalence classes of cochain maps under the homotopy relation as morphisms.

Given two complexes $X$ and $Y$, the complex $\HOM(X, Y)$ is defined as the complex of abelian groups having $\prod\limits_{p\in \bbZ}\Hom_{\C}(X^p, Y^{n+p})$ in degree $n$  and with differential $d_H(f)=d_Y\circ f -(-1)^nf\circ d_X$. The $n^{\text th}$-cohomology of  $\HOM(X, Y)$ is given by $\Hom_{\K(C)} (X, Y[n])$.


We recall the useful and important formula
\[(\ast)\quad \Ext^1_{dw}(X, Y) \cong \Hom_{\K(\C)}(X,Y[1]).\] %
%
%
\begin{nota}\label{N:notation} (Following Gillespie's notations) Let $\A$ be a class of objects in an abelian category $\C$.
 Define the following classes of cochain
  complexes in $\Ch(\C)$:

  \begin{itemize}
    \item
      $dw\A$ is the class of all complexes $X\in\Ch(\C)$
      such that $X^n\in\A$ for all $n\in\bbZ$.  $\Ch(\A)$ will denote the full subcategory of  $\Ch(\C)$ with objects in $dw \A$.
       \item
      $ex\A$ is the class of all acyclic complexes in $dw\A$.
       \item
      $\tilde{\A}$ is the class of all complexes $X$ in $
      ex\A$ with the cycles $Z^n(X)$ in $\A$ for all $n\in \bbZ$.
   $\Ch_{ac}(\A)$ will denote the full subcategory of  $\Ch(\C)$ with objects in $\tilde\A$.


    \item  If $(\A, \B)$ is a cotorsion pair in $\C$, then:

      $dg\A$ is the class of all complexes $X\in dw\A$
      such that  any morphism
      $f:X\to Y$ with $Y\in\tilde {\B}$ is null homotopic. Since $\Ext^1_{\C}(A^n, B^n)=0$ for every $n\in \bbZ$ formula $(\ast)$ shows that $dg\A={}^\perp{} \tilde {\B}$.

      Similarly, $dg\B$ is the class of all complexes $Y \in dw\B$
      such that  any morphism
      $f:X\to Y$ with $X\in\tilde{\A}$  is null homotopic. Hence $dg\B=\tilde{\A}{}^\perp{}$.
  \end{itemize}
\end{nota}
%
%
%
%
%
\begin{lem}\label{L:HOM} Let $\C$ be an abelian category and let $0\to X\to Y\to Z\to 0$ be a short exact sequence of complexes in $\Ch(\C)$ with the degreewise exact structure. For every $A\in \Ch(\C)$ the sequence:
\[0\to \HOM(A, X)\to  \HOM(A, Y)\to \HOM(A, Z)\]
is an exact sequence of complexes in $\Ch(\bbZ)$ and it is also right exact provided that $\Ext_{\C}(A^n, X^n)=0$ for all $n\in \bbZ$.
 Dually, for every $B\in \Ch(\C)$ the sequence:
\[0\to \HOM(Z, B)\to  \HOM(Y, B)\to \HOM(X, B)\]
is an exact sequence of complexes in $\Ch(\bbZ)$ and it is also right exact provided that $\Ext_{\C}(Z^n, B^n)=0$ for all $n\in \bbZ$.
\end{lem}
\begin{proof} Immediate from the definition of the complex $\HOM$.
\end{proof}
%
%
\subsection{Hereditary cotorsion pairs in Grothendieck categories}

We recall some results which will be used throughout. Their proof can be found in \cite{St10-deconstr}, \cite{Sto13},  \cite{G4}, \cite{G6}.

 \begin{prop}\label{P:description-tilde} (\cite[Proposition 7.13, 7.14]{Sto13} Let $(\A, \B)$ be a complete cotorsion pair in an abelian category $\C$. The following hold true
  \begin{enumerate}
  \item A complex $Y$ belongs to $\tilde\B$ \iff $\Ext^1_{\Ch(\C)}(S^n(A), Y)=0$ for every $n\in \bbZ$ and every $A\in \A$.
  \item A complex $X$ belongs to $\tilde\A$ \iff $\Ext^1_{\Ch(\C)}(X, S^n(B))=0$ for every $n\in \bbZ$ and every $B\in \B$.
  \item If $\C$ is a Grothendieck category and $(\A, \B)$ is a complete hereditary cotorsion pair, then $(\tilde\A, dg\B)$ and $(dg \A, \tilde \B)$ are complete hereditary cotorsion pairs in $\Ch(\C)$.

  %
    \item $(dg\A, \E, dg\B)$ is a model structure on $\Ch(\C)$ with the acyclic complexes $\E$ as trivial objects. In particular, $dg\A\cap\E=\tilde \A$ and $dg\B\cap\E=\tilde \B$.
  \end{enumerate}
  \end{prop}
  \begin{proof}
 (1) and (2) are proved in \cite[Lemma 7.13]{Sto13}. (3) is proved in \cite[Proposition 7.14]{Sto13}.
  (4) follows by (3) and by Hovey's correspondence (see Theorem~\ref{T:correspondence}).
  \end{proof}
  %
   \begin{prop}\label{P:complete?} Let $(\A, \B)$ be a complete cotorsion pair in an Grothendieck category $\G$.
 \begin{enumerate}
  \item $(dw\A, dw\A^\perp)$ and $(^\perp dw\B, dw\B)$ are cotorsion pairs in $\Ch(\G)$.
  \item If $(\A, \B)$ is generated by a set, then so is $(^\perp dw\B, dw\B)$.
 %
  \item If $(\A, \B)$ is generated by a set, then so is $(dw\A, dw\A^\perp)$.

  \item If $\A$ contains a generator of $\G$ with finite projective dimension,
  %
  then $(^\perp ex\B, ex\B)$ is a cotorsion pair in $\Ch(\G)$. If moreover, $(\A, \B)$ is generated by a set, then so is $(^\perp ex\B, ex\B)$.
          %
  \item $(ex\A, ex\A^\perp)$ is a cotorsion pair. Moreover, if $(\A, \B)$ is generated by a set, then so is $(ex\A, ex\A^\perp)$.
   %


   \end{enumerate}
  \end{prop}
  %
 \begin{proof}
  (1) is proved in \cite[Proposition 3.2]{G4},

  (2) is proved in \cite[Proposition 4.4]{G4}.

  (3) is proved as follows: by Lemma~\ref{lem1_2}, $\A$ is deconstructible and by \cite[Theorem 4.2]{St10-deconstr} so is $dw\A$. Moreover, $dw\A$ contains a generator, so $(dw\A,dw\A^\perp)$ is generated by a set by Lemma~\ref{lem1_2}.

  The first part of (4) is proved in \cite[Proposition 3.3]{G4}; the second part in \cite[Proposition 4.6]{G4}.

  The first part of (5) is again proved in \cite[Proposition 3.3]{G4}; for the second part we argue as in the proof of \cite[Proposition 7.3]{G6}. $ex\A=dw\A\cap \E$, where $\E$ is the class of acyclic complexes. By \cite[Theorem 4.2]{St10-deconstr} $\E$ and $dw \A$ are deconstructible, hence $ex\A$ is deconstructible by \cite[Proposition 2.9]{St10-deconstr}.
  Moreover, $ex\A$ contains a generator, so $(ex\A,ex\A^\perp)$ is generated by a set by Lemma~\ref{lem1_2}.
  \end{proof}

  %
%
%
%
%
%
%
%
%
%
%
%
 %

%
%
%
%
%

\begin{rem} If $(\A, \B)$ is a complete hereditary cotorsion pair in an abelian category, then the complete cotorsion pairs defined in the above proposition are hereditary, too.
\end{rem}

\begin{lem}\label{L:ex-tilde} Let $(\A, \B)$ be a complete hereditary cotorsion pairs in a Grothendieck category $\G$. Then $ex\B=\tilde \B$ if and only if $dw\B=dg\B$.
Dually $ex\A= \tilde \A$ if and only if $dw\A=dg\A$
 \end{lem}
 %
  \begin{proof} Assume that $ex\B= \tilde \B$ and let $Y\in dw\B$. We have to show that $\Ext^1_{\Ch(\C)}(X, Y)=0$ for every $X\in \tilde \A$.  Equivalently we have to show that the complex $\HOM(X, Y)$ is exact  for every $X\in \tilde \A$. Since, $(\tilde\A, dg\B)$ is a complete cotorsion pair in $\Ch(\C)$  there is a short exact sequence
\[ 0\to Y\to Z\to V\to 0\]
with $Z\in dg\B$ and $V\in \tilde\A$. Now, $\B$ is coresolving,  hence $V\in \tilde\A\cap dw\B=\tilde\A\cap ex\B$ and the last is $\tilde\A\cap \tilde\B$ by  assumption. Thus, $V$ is contractible, hence null homotopic.
By Lemma~\ref{L:HOM} we have a short exact sequence
\[0\to \HOM(X, Y)\to  \HOM(X, Z)\to \HOM(X, V)\to 0\]
for every $X\in \tilde \A$. The second and the third nonzero terms are exact, hence also $ \HOM(X, Y)$ is exact.

Conversely, assume that $dw\B=dg\B$ and let $Y\in ex\B$. Then $Y\in dw\B\cap\E=dg\B\cap \E$ and by Proposition~\ref{P:description-tilde}~(4), $Y\in \tilde\B$.
%
%
%
%
%
%

The dual statement is proved in similar ways.
\end{proof}

%\section{Cotorsion pairs $(\A, \B)$ satisfying $ex\B=\tilde{\B}$}\label{S:2}
\subsection{Cotorsion pairs \texorpdfstring{$(\A, \B)$}{(A,B)} satisfying \texorpdfstring{$ex\B=\tilde{\B}$}{ex B = tilde B} }\label{S:2}
We are interested in describing cotorsion pairs $(\A, \B)$ such that $ex\B=\tilde \B$ or $ex\A=\tilde \A$, since in these cases we have the following important consequences on the corresponding model structures.
%
\begin{cor}\label{C:cofibrant-fibrant}  Let $(\A, \B)$ be a complete hereditary cotorsion pairs in a Grothendieck category $\G$. The following hold true:
\begin{enumerate}
\item  If $ex\B=\tilde \B$, then $(dg\A, \E, dw\B)$ is a model structure in $\Ch(\G)$ for which the fibrant objects are exactly the complexes with components in $\B$.
\item If $ex\A=\tilde \A$, then $(dw\A, \E, dg\B)$ is a model structure in $\Ch(\G)$ for which the cofibrant objects are exactly the complexes with components in $\A$.
\end{enumerate}
\end{cor}
%
\begin{proof} Follows by Proposition~\ref{P:description-tilde}~(4) and by Lemma~\ref{L:ex-tilde}.
\end{proof}

  %
We say that an object $M$ in a Grothendieck category $\G$ has projective dimension at most $n$ if $\Ext_{\G}^i(M, -)$ vanishes for every $i>n$ and we denote by $\clP_n$ the class of objects of projective dimension at most $n$.
 Analogously, $M$ has injective dimension at most $n$ if $\Ext_{\G}^i(-, M)$ vanishes for every $i>n$ and we denote by $\I_n$ the class of objects of  injective dimension at most $n$. We denote by $\clP=\bigcup_n\clP_n$ the class of objects with finite projective dimension and by and $\I=\bigcup_n\I_n$ the class of objects with finite injective dimension.

  \begin{prop}\label{P:finite-proj-dim} Let $(\A, \B)$ be a complete hereditary cotorsion pair in a Grothendieck category $\G$ and let $Y$ be an acyclic complex with terms in $\B$. The following hold true:
  \begin{enumerate}
  \item If $M$ is an object in $\A$ with finite projective dimension, then the cycles $Z^j(Y)$ of $Y$ belong to $M{}^\perp{}$.
  \item If $\A\subseteq \clP$, then $Y\in \tilde \B$, hence $ex\B=\tilde \B$.

    In particular, in the abelian model structure corresponding to the cotorsion pair $(\A, \B)$ by Corollary~\ref{C:cofibrant-fibrant}~(1), $dw \B$ is the class of fibrant objects.
  \end{enumerate}

   Dually, let $X$ be an acyclic complex with terms in $\A$. Then:
   \begin{enumerate}
  \item[(3)] If $N$ is an object in $\B$ with finite injective dimension, then the cycles $Z^j(X)$ of $X$ belong to ${^\perp N}$.
  \item[(4)] If $\B\subseteq \I$, then $X\in \tilde \A$, hence $ex\A=\tilde \A$.

    In particular, in the abelian model structure corresponding to the cotorsion pair $(\A, \B)$ by Corollary~\ref{C:cofibrant-fibrant}~(2), $dw \A$ is the class of cofibrant objects.

  \end{enumerate}

  \end{prop}
  %

  \begin{proof} (1) Clearly it is enough to verify that the $0$-cycle $Z^0$ of $Y$ is in $M{}^\perp{}$.  Consider the exact complex
  \[\dots Y^{-n}\to\dots  \to Y^{-2}\to Y^{-1}\to Z^0\to 0.\]
 If $M$ is in $ \A$, then $\Ext^j_{\G}(M, Y^n)=0$ for every $n\in \bbZ$ and every $j\geq 1$. A dimension shifting argument gives $\Ext^i_{\G}(M, Z^0)\cong \Ext^{i+k}_{\G}(M, Z^{-k})$, for every $k\geq 1$. Hence by the finiteness of the projective dimension of $M$   we conclude that $\Ext^i_{\G}(M, Z^0)=0$ for every $i\geq 1$.

 (2) The first statement  follows by~(1). The second statement follows by Corollary~\ref{C:cofibrant-fibrant}.

 The proof of the dual statement is obtained by considering the acyclic complex:
 \[0\to Z^0\to X^0\to X^1\to \dots\to X^n\to \dots\]
 and using dimension shifting for the functor $\Hom_{\G}(- ,N)$.

\end{proof}
We consider now the particular case of a module category and
 we exhibit some situations in which the assumptions of the previous proposition are satisfied.

 Recall that $T$ is an $n$-tilting $R$-module if it has projective dimension at most $n$, $\Ext^i_R(T, T^{(\lambda)})=0$ for every cardinal $\lambda$ and every $i\geq 0$, and  the ring $R$ has a finite coresolution with terms in $\Add T$, where $\Add T$ denotes the class of direct summands of direct sums of copies of $T$.
 The cotorsion pair generated by $T$ is called $n$-tilting cotorsion pair.

 Dually, an $R$-module $C$ is $n$-cotilting if it has injective dimension at most $n$, $\Ext^i_R(C^{\lambda}, C)=0$ for every cardinal $\lambda$ and every $i\geq 0$, and  an injective cogenerator has a finite resolution with terms in $\Prod C$, where $\Prod C$ denotes the class of direct summands of direct products of copies of $C$.
 The cotorsion pair cogenerated by $C$ is called $n$-cotilting cotorsion pair.

  \begin{prop}\label{P:tilt-cotil} If $(\A, \B)$ is an $n$-tilting cotorsion pair in $\Modr R$, then $ex\B= \tilde \B$ and $dw\B=dg\B$. Hence there is a model structure in $\Ch(R)$ in which the fibrant objects are the complexes with components in the  $n$-tilting class $\B$ and the trivial objects are the acyclic complexes.

 Dually, if $(\A, \B)$ is an $n$-cotilting cotorsion pair in $\Modr R$, then $ex\A= \tilde \A$ and $dw\A=dg\A$.
 Hence there is a model structure in $\Ch(R)$ in which the cofibrant objects are the complexes with components in the  $n$-cotilting class $\A$ and the trivial objects are the acyclic complexes.
 \end{prop}

%
\begin{proof} If $(\A, \B)$ is a tilting (cotilting) cotorsion pair, then $\A\subseteq \clP_n$ ($\B\subseteq \I_n$), by \cite[Lemmas 13.10, 15.4]{GT12}. Hence the conclusion follows by Proposition~\ref{P:finite-proj-dim}.
\end{proof}
%
%
%
To exhibit other examples of cotorsion pairs $(\A, \B)$ satisfying the condition $ex\B=\tilde \B$ we use the notion of the closure of a cotorsion pair.

 Recall that a cotorsion pair  $(\A, \B)$ is \emph{closed} if $\A$ is closed under direct limits. Consider the lattice of cotorsion pairs, with respect to inclusion on the left component.
Since the cotorsion pair $(\Modr R, \Inj)$ is closed and the meet of closed cotorsion pairs is closed (see e.g. \cite{AT} or \cite[Lemma 6.1]{G6}), every cotorsion pair $(\A, \B)$ is contained in a smallest closed cotorsion pair, called the closure of $(\A, \B)$
\begin{nota}\label{N:notations} Let $R$ be a ring.
\begin{enumerate}
\item We denote by $\modr R$ the class of modules $M$ admitting a projective resolution of the form
\[\dots\to P_i\to P_{i-1}\to \dots\to P_1\to P_0\to M\to 0,\]
with $P_j$ finitely generated  for every $j \geq 0$.
\item For every $n\geq 0$, denote by $\clP_n(\modr R)$  the class $\clP_n\cap\modr R$ and by $\clP(\modr R)$  the class $ \clP\cap\modr R$.
%
\item The \emph{little finitistic dimension} of $R$ is the supremum of the projective dimension of modules in $\modr R$ having finite projective dimension.
\item The \emph{big projective (flat) finitistic dimension}  of $R$ is  the supremum of the projective (flat) dimension of modules having finite projective (flat) dimension.
\item Denote by $(\A^{\omega}, \B_{\omega})$ the complete hereditary cotorsion pair generated by $\clP(\modr R)$. By \cite[Theorem 2.3, Corollary 2.4]{AT}, its closure
\[(\A^{\infty}, \B_{\infty})
\]  is a complete cotorsion pair cogenerated by the class of pure injective modules belonging to $\B_{\omega}$, hence it is hereditary, since cosyzygies of pure injective modules of $\B_{\omega}$ are in $\B_{\omega}$. Moreover, $\A^{\infty}=\varinjlim \A^{\omega}=\varinjlim \clP(\modr R)$ and it is closed under pure epimorphic images.
\end{enumerate}
\end{nota}

\begin{rem}\label{R:sup-flat-if-cot} \begin{enumerate}\item Since $\varinjlim \clP_0(\modr R)$ is the class of flat modules, $\A^{\infty}$ contains all flat modules and it coincides with the class of flat modules if and only if every module in $\clP_n(\modr R)$ is projective, i.e. if the little finitistic dimension of $R$ is $0$.
\item By part (1), $\B_{\infty}$ is contained in the class of cotorsion modules and it is properly contained in it whenever the little finitistic dimension of $R$ is greater than $0$.
\item Moreover,  $\clP_1\subseteq \varinjlim \clP_1(\modr R)$, hence $\clP_1\subseteq \A^{\infty}$.
\item By \cite[Theorem 6.7~(vi)]{BH09}, if $R$ has a classical ring of quotients $Q$ such that $Q$ is Von Neumann regular or has big finitistic flat dimension $0$, then $\varinjlim \clP_1$ coincides with the class $\F_1$ of  modules of flat dimension at most $1$.
Hence $\A^{\infty}$ contains $\F_1$ and $ \B_{\infty}$ is contained in the class $\F_1{}^\perp{}$ which is also called the class of weakly injective modules (see \cite{FuLee1} and \cite{FuLee2}).
 In particular, this applies to any commutative ring such that the total quotient ring is a perfect ring or a Von Neumann regular ring.
\end{enumerate}
\end{rem}
%
  \begin{prop}\label{P:semihereditary} Let $R$ be a (coherent) ring. The class $\B_{\infty}$ coincides with the class of injective right  $R$-modules if and only if every module in $\modr R$ (every finitely presented module) has finite projective dimension. In particular, this applies to rings with finite little finitistic dimension and thus to right semihereditary rings.
  \end{prop}

 \begin{proof}  $ \B_{\infty}$ coincides with the class of injectives if and only if $\A^{\infty}=\Modr R$. If every module in $\modr R$  has finite projective dimension, then $\A^{\infty}=\Modr R$, since $\A^{\infty}$ is closed under direct limits. Conversely, if $\A^{\infty}=\Modr R$, then every finitely presented right module $X$ belongs to $\varinjlim \clP(\modr R)$, hence it is a summand of a finite direct sum of modules in $ \clP(\modr R)$. Thus $X$ has finite projective dimension and so does every module in $ \modr R$.

 The last statement follows easily. In particular,  if $R$ is right semihereditary, then every finitely presented right $R$-module has projective dimension at most one.\end{proof}
  %

  We show now that $ex\B_{\infty}=\widetilde{\B_{\infty}}$. To this aim we apply the results proved in a recent paper ~\cite{BCIE} about periodic modules.
   Recall that a module $M$ is periodic with respect to a class $\C$ if there exists a short exact sequence $0\to M\to C\to M\to 0$ with $C\in \C$. A module $M$ is Fp-injective if $\Ext^1_R(X, M)=0$ for every finitely presented module $X$.

\begin{fact}\label{F:periodic}
 \begin{enumerate}
\item \cite[Proposition 3.8~(1)]{BCIE} every Fp-injective $\Inj$-periodic module is injective.
 \item \cite{EFI} If $\C$ is  a class closed under direct sums or direct products and  $\D$ is a class closed under direct summands, then the following are equivalent:
\begin{enumerate}
\item[(a)] Every cycle of an acyclic complex with components in $\C$ belongs to $\D$.
\item[(b)]  Every $\C$-periodic module belongs to $\D$.
\end{enumerate}

\end{enumerate}
\end{fact}

%
%
%
%
%
%
%
%
%
%
%
%

 \begin{prop}\label{P:ex-B-infty} The cotorsion pair $(\A^{\infty}, \B_{\infty})$ from Notation~
 \ref{N:notations}~(4) satisfies $ex\B_{\infty}=\widetilde{\B_{\infty}}$.
%
%
%
%
%
%
\end{prop}
%
\begin{proof}  Let $M$ be a $\B_{\infty}$-periodic module. By \cite[Lemma 3.4]{BCIE} ${}^\perp M\supseteq \clP(\modR)$. As mentioned in Notation~
 \ref{N:notations}~(4), the class  $\A^{\infty}$ coincides with $\varinjlim \clP(\modR)$ and is closed under pure epimorphic images. By \cite[Theorem 3.7]{BCIE} ${}^\perp M\supseteq \A^{\infty}$, hence $M\in\B_{\infty}$. By Fact~\ref{F:periodic}~(2), $ex\B_{\infty}=\widetilde{\B_{\infty}}$ in $\Ch(R)$.
  %
  \end{proof}
%

As a corollary we get  an improvement of ~\cite[Corollary 5.9]{Stopurity} in the case of a module category, since $ \B_{\infty}$ is in general properly contained in the class of cotorsion modules.
\begin{cor}\label{C:exInj}
Let $Y$ be an acyclic complex with injective components. Then every cycle of $Y$ belongs to $ \B_{\infty}$, hence $Y\in \widetilde{ \B_{\infty}}$.
\end{cor}
%
\begin{proof} By assumption $Y\in ex\B_{\infty}$, hence the conclusion follows by Proposition~\ref{P:ex-B-infty}.
%
%
 \end{proof}
%
%
%
%
 The next properties will be used in Section~\ref{S:tildeB}.

  \begin{lem}\label{L:Cot-inj} Let $(\A,\B)$ be a complete hereditary cotorsion pair in a Grothendieck category $\G$.
Let $\Inj$ denote the class of injective objects of $\G$. The following hold true:
\begin{enumerate}
\item[(1)] $\B{}^\perp{}\cap \B\subseteq \Inj$ and $\widetilde{\B}{}^\perp{}\cap dw \B\subseteq dw \Inj$.
\item[(2)]  $\widetilde{\B}{}^\perp{}\cap dw \B=dw \Inj$ \iff $\tilde\B\subseteq {}^\perp{} dw\Inj$.
\end{enumerate}
Moreover, if $\G=\Modr R$ and $\B$ contains the class $ \B_{\infty}$ defined in Notation~\ref{N:notations}~(5) then
\begin{enumerate}
  \item[(3)]\label{L:Cot-inj.3} $\widetilde{\B}{}^\perp{}\cap dw \B=dg \Inj$.
\item[(4)]\label{L:Cot-inj.4} ${}^\perp{} dg\Inj\cap dw\B=ex \B$
\end{enumerate}

%
%
%
%
%

%
\end{lem}
%
\begin{proof}
(1) Let $B\in \B{}^\perp{}\cap \B$ and consider an exact sequence $0\to B\to I\to I/B\to 0$ with $I\in \Inj$. Then $I/B\in \B$, since $\B$ is coresolving, hence the sequence splits and $B$ is injective.

If $B\in \B$, then $D^n(B)\in \widetilde{\B}$ for every $n\in \bbZ$ and by \cite[Lemma 3.1]{G3}, $\Ext^1_{\Ch(\G)}(D^n(B), Y)\cong \Ext^1_{\G}(B, Y^n)$, for every complex $Y$. Thus if $Y\in \widetilde{\B}{}^\perp{}\cap dw \B$, then $Y^n\in \B{}^\perp{}\cap\B$ for every $n\in \bbZ$. By the above we conclude that $Y\in dw \Inj$.

(2) If $\tilde\B\subseteq {}^\perp{} dw\Inj$, then $\tilde\B{}^\perp{}\supseteq  ({}^\perp{} dw\Inj){}^\perp{}=dw\Inj$, by \cite[Proposition 4.4]{G4}, hence by part (1) $\tilde{\B}{}^\perp{}\cap dw \B=dw \Inj$.

Conversely, if $\widetilde{\B}{}^\perp{}\cap dw \B=dw \Inj$, then $dw\Inj\subseteq \tilde\B^\perp$, hence $\tilde\B\subseteq {}^\perp{}(\tilde\B{}^\perp{})\subseteq ^\perp{}dw\Inj.$

(3) We show the inclusion $\widetilde{\B}{}^\perp{}\cap dw \B\subseteq dg \Inj$. Let $Y\in \widetilde{ \B}{}^\perp{}\cap dw \B$; using the complete cotorsion pair $(\E, dg \Inj)$ in $\Ch(R)$ we can consider a  short exact sequence $(\ast)\quad 0\to Y\to dg I\to E\to 0$ with $dg I\in dg \Inj$ and $E$ an exact complex. By part (1) the sequence is degreewise splitting hence $E^n$ is an injective module for every $n\in \bbZ$ which means that $E\in ex \Inj$.
 By  Corollary~\ref{C:exInj}, $ex \Inj\subseteq \widetilde{ \B_{\infty}}\subseteq\widetilde{ \B}$, hence the sequence $(\ast)$ splits showing that $Y\in dg \Inj$.

 The other inclusion is  obvious since  ${}^\perp{} dg \Inj$ in $\Ch(R)$ is the class of acyclic complexes $\E$ and $\E\supseteq \widetilde{ \B}$.

 (4) Obvious, since ${}^\perp{} dg \Inj=\E$.  \end{proof}
%
%
%
%
%
%
%
%
%
\begin{rem} If  $\G$ has enough projective objects, then the dual of the statements in Lemma~\ref{L:Cot-inj}~(1) and (2) hold  substituting the right orthogonal with the left orthogonal and $\Inj$ with $\Proj$.
\end{rem}

Points \ref{L:Cot-inj.3} and \ref{L:Cot-inj.4} of the previous Lemma can be generalized, with essentially the same proof, to a complete hereditary cotorsion pair $(\A,\B)$ in a Grothendieck category $\G$ such that $ex \Inj\subseteq \wac{\B}$.

%
%
%
%
%
%
%
%
%
%
%
%
%
%
%
%


 \subsection{Cotorsion pairs in exact categories}

 We state a result valid in general for cotorsion pairs in exact categories.
 %
 \begin{prop}\label{P:inducing} Let $(\A, \B)$ be a (hereditary) complete cotorsion pair in an exact category $\C$ and let $\D$ be an extension closed subcategory of $\C$ with the exact structure induced by that of $\C$. If $\D$ contains $\A$ and is resolving in $\C$ or if $\D$ contains $\B$ and is coresolving in $\C$, (see Definition~\ref{D:thick}), then $(\A\cap\D, \ \B\cap\D)$ is a (hereditary) complete cotorsion pair in the exact category $\D$.

 \end{prop}
 %
 %
 \begin{proof} We prove the statement in  case $\D\supseteq \B$, the other case being similar. First we show that $(\A\cap\D, \B)$ is a cotorsion pair in $\D$.
 Clearly ${}^\perp{}\B= \A\cap \D$ in $\D$ and also $(\A\cap \D){}^\perp{} \supseteq \B$. We show that $(\A\cap \D){}^\perp{} =\B$ in $\D$.
 Let $D\in \D$ be such that $\Ext^1(X, D)=0$ for every $X\in \A\cap \D$. Since $(\A, \B)$ is complete, there is an exact sequence $0\to D\to B\to A\to 0$ in $\C$, with $B\in \B$ and $A\in \A$. Since $\D$ is coresolving  in $\C$ and contains $\B$, we have that $A\in \D$, hence $A\in A\cap\D$ showing that the exact sequence splits, thus $D\in \B$.

To show that $(\A\cap\D, \B)$ is complete, let $(\ast)\quad 0\to B\to A\to D\to 0$ be a special $\A $-precover of an object $D\in \D$, then $A\in \A\cap\D$, since $\D$ is extension closed, hence $(\ast)$ is a special $\A\cap \D $-precover of $D$. If $ (\ast\ast):\quad 0\to D\to B\to A\to 0$ is a special $\B $-preenvelope of $D\in \D$, then $A\in \D$ since $\D$ is coresolving, hence $(\ast\ast)$ is  special $\B$-preenvelope of $D$ \wrt to  $(\A\cap\D, \B)$. \end{proof}


%
%
%
%
%
%
%


%

%
From now on $\G$ will be a Grothendieck category.

For every complete cotorsion pair $(\A, \B)$ in a Grothendieck category $\G$ the classes $\A$ and $\B$ are extension closed subcategories of $\G$, hence they inherit the exact structure from the abelian structure of $\G$.

Moreover, it is obvious that they are idempotent complete.

 It is well known that a Grothendieck category has enough injectives.
When needed we will assume that $\G$ has enough projectives and enough flat objects.

We will denote by $\Inj$ and $\Proj$ the classes of injective and projective objects, respectively; by $\Flat$ the class of flat objects and by $\Cot$ the class of cotorsion objects.  We have the complete hereditary cotorsion pairs $(\Proj, \G)$, $(\G, \Inj)$ and $(\Flat, \Cot)$, hence the four classes defined above are exact subcategories of $\G$.

We first collect some well known facts.
\begin{fact}\label{F:proj-inj} Let $(\A, \B)$ be a complete hereditary cotorsion pair in $\G$. The following hold true:
\begin{enumerate}
\item $\A$ has enough injectives and projectives: the projectives are the same as in $\G$ and the injectives are the objects in $\A\cap\B$.
\item $\B$ has enough injectives and projectives: the injectives are the same as in $\G$ and the projectives are the objects in $\A\cap\B$.
\item (\cite[Corollary 2.9]{G7} $\Ch(\A)$ has enough injectives and projectives: the projectives are the same as in $\Ch(\G)$ and the injectives are the contractible complexes with components in $\A\cap\B$.
\item (\cite[Corollary 2.9]{G7} $\Ch(\B)$ has enough injectives and projectives: the injective are the same as in $\Ch(\G)$ and the projectives are the contractible complexes with components in $\A\cap\B$.
\item (\cite[Corollary 2.8]{G7} $\Ch(\A)_{dw}$ and $\Ch(\B)_{dw}$ are Frobenius exact categories with the  projective-injective objects being the contractible complexes with terms in $\A$ or $\B$ respectively.
\end{enumerate}
\end{fact}

%
%
%

\begin{rem}\label{R:induced} If $(\A, \B)$ is a complete hereditary cotorsion pair in a Grothendieck category $\G$, Proposition~\ref{P:inducing} tells us that $(\B, \Inj)$ and $(\A\cap \B, \B)$  are complete hereditary cotorsion pairs in the exact category $\B$; $(\Proj, \A)$ and $(\A, \A\cap \B)$ are complete hereditary cotorsion pairs  in the exact category $\A$.
\end{rem}


%

We conclude this chapter by giving several propositions that allow us to restrict a cotorsion pair in $\Ch(G)$ to the exact subcategories $\Ch(\A)$ or $(\Ch(\B)$ and the Frobenius categories $\Ch(\A)_{dw}$ and $\Ch(\B)_{dw}$.

 The following Proposition~\ref{P:prop7.3} is a generalization of \cite[Proposition 7.3]{G7} which was formulated for the case of the cotorsion pair $(\F, \C)$ in a module category. %

Moreover, in Proposition~\ref{P:prop7.3-dual} we state a generalization of the dual of \cite[Proposition 7.3]{G7}.
%
 %


\begin{prop}\label{P:prop7.3} Let $(\A, B)$ be a complete hereditary cotorsion pair in a Grothendieck category $\G$ and let $(\hat\A, \hat\B)$ be a complete cotorsion pair in $\Ch(\G)$ with $\hat\A\subseteq dw\A$. Assume that $\hat\A$ is thick in the exact category $\Ch(\A)$ and that it contains the contractible complexes with terms in $\A$.
Then, \[\Big(\hat\A, \hat\B\cap dw \A\Big)\] is an injective cotorsion pair in $\Ch(\A)$.
Moreover,  \[\Big(\hat\A, [\hat\B\cap dw \A]_K\Big)\] is a localizing cotorsion pair in the Frobenius category $\Ch(\A)_{dw}$, where a complex $X\in \Ch(\A)$ belongs to $[\hat\B\cap dw \A]_K$ if and only if it is chain homotopy equivalent to a complex in $\hat\B\cap dw \A$.
\end{prop}
%
\begin{proof} The fact that $\Big(\hat\A, \hat\B\cap dw \A\Big)$ is a complete cotorsion pair follows by Proposition~\ref{P:inducing} and it is an injective cotorsion pair by definition and by the assumptions on $\hat\A$. Moreover, $\hat\B\subseteq dw \B$. In fact, for every $n\in \bbZ$ and every  $A\in \A$  the contractible complex $D^n(A)$ is in $\hat\A$, hence $\Ext^1_{\Ch}(D^n(A),  B)=0$, for every $B\in\hat\B$ and then $B^n$ belongs to $\B$, by \cite[Lemma 3.1]{G1}. Hence, a short exact sequence $0\to Y\to Z\to X\to 0$ in $\Ch(\A)$ with $Y\in\hat\B\cap dw \A$ and $X\in \hat\A$ is degreewise splitting. The second statement follows by \cite[Theorem 6.3, Proposition 6.4]{G7}.
\end{proof}
%

%
\begin{prop}\label{P:prop7.3-dual} Let $(\A, B)$ be a complete hereditary cotorsion pair in $\G$ and let $(\hat\A, \hat\B)$ be a complete cotorsion pair in $\Ch(\G)$ with $\hat\B\subseteq dw\B$. Assume that $\hat\B$ is thick in the exact category $\Ch(\B)$ and contains the contractible complexes with terms in $\B$.
Then, \[\Big(\hat\A\cap dw \B, \hat\B\Big)\] is a projective cotorsion pair in $\Ch(\B)$.
Moreover,  \[\Big([\hat\A\cap dw \B]_K, \hat\B\Big)\] is a localizing cotorsion pair  in the Frobenius category $\Ch(\B)_{dw}$,  where a complex $X\in \Ch(\B)$ belongs to $[\hat\A\cap dw \B]_K$ if and only if it is chain homotopy equivalent to a complex in  $\hat\A\cap dw \B$.
\end{prop}
%
\begin{proof}
  Dual of \ref{P:prop7.3}.
Note that Theorem 6.3 and Proposition 6.4 in \cite{G7} have  obvious dual statements for projective cotorsion pairs from which the second statement of our proposition follows.\end{proof}

The next proposition is the analogue of \cite[Proposition 7.2 ]{G7} and its dual is stated in \ref{P:prop7.2}.
 %

 \begin{prop}\label{P:7.2-Groth} Let $\G$ be a Grothendieck category with enough projective objects and let $(\clP, \W)$ be a projective cotorsion pair in $\Ch(\G)$ with $\clP\subseteq dw\Proj$.
 Let $(\A, \B)$ be a complete hereditary cotorsion pair in $\G$. Then,
 \[(\clP, \W\cap dw\A)\] is a projective cotorsion pair in $\Ch(\A)$ and \[\Big([\clP]_K, \W\cap dw \A\Big)\] is a localizing cotorsion pair in the Frobenius category $\Ch(\A)_{dw}$. A complex $X\in \Ch(\A)$ is in $[\clP]_K$ if and only if it is chain homotopy equivalent to a complex in $P\in \clP$.
 \end{prop}
 \begin{proof} $(\clP, \W\cap dw\A)$ is a complete cotorsion pair by Proposition~\ref{P:inducing} and it is automatically a projective cotorsion pair. The second statement follows by the dual of \cite[Theorem 6.3, Proposition 6.4]{G7}.
\end{proof}


\begin{prop}\label{P:prop7.2} Let $\G$ be a Grothendieck category and let $(\W, \I)$ be an injective cotorsion pair in $\Ch(\G)$ with $\I\subseteq dw \Inj$ and let $(\A, \B)$ be a complete hereditary cotorsion pair in $\G$. Then \[\Big(\W\cap dw \B, \I\Big)\] is an injective cotorsion pair in $\Ch(\B)$ and \[\Big(\W\cap dw \B, [\I]_K\Big)\] is a localizing cotorsion pair in the Frobenius category $\Ch(\B)_{dw}$, where a complex $X\in \Ch(\B)$ belongs to $[\I]_K$ if and only if it is chain homotopy equivalent to a complex in $\I$.
\end{prop}

%
\begin{proof} The first statement follows by Proposition~\ref{P:inducing}. The second statement follows by \cite[Theorem 6.3]{G7}.
\end{proof}



%
%
%


