\clearpage

\section{Nakaoka contexts in additive categories}

Let $\mathcal{X}$ be a pointed category. Recall that, given a map $\phi\colon X\to Y$, a map $\psi\colon Y\to Z$ such that $\psi\phi=0$ is said to be a {\em pseudocokernel} for $\phi$ if any other morphism $\psi'\colon Y\to Z'$ such that $\psi'\phi=0$ factors (not necessarily uniquely) through $\psi$. Pseudokernels can be defined dually.

\medskip\noindent
{\bf Convention.} We fix  trough this section an additive category $\mathcal{X}$ where  idempotents split, and in which any morphism has at least a pseudokernel and a pseudocokernel.

\medskip
In this section, after some generalities about torsion pairs, we introduce the notion of a ``Nakaoka context" in $\mathcal X$ and we associate to it a full subcategory $\mathcal H$ of $\mathcal X$, called the heart. We then give conditions on a given Nakaoka context for the heart to be pre-Abelian, integral (?) and Abelian. 

\subsection{Generalities on torsion pairs}

For a class of objects $\mathcal A\subseteq \mathcal X$ we introduce the following notations
\[
\mathcal A^\perp = \{X\in\mathcal{X}:\mathcal{X}(\mathcal A,X)=0\}\quad \text{ and }\quad {^\perp\mathcal A} = \{X\in\mathcal{X}:\mathcal{X}(X,\mathcal A)=0\}.
\]
In the following definition we recall the concept of torsion pair in $\mathcal X$, which is just an adaptation of the notion of torsion pair in Abelian categories to our broader context:

\begin{definition}
A pair $\mathbb{t}=(\mathcal{T},\mathcal{F})$ of full subcategories of $\mathcal{X}$ is  a {\em torsion pair} if:
\begin{enumerate}
\item[\rm (TP.1)] $\mathcal{F}=\mathcal{T}^\perp$ and $\mathcal{T}={^\perp\mathcal{F}}$;
\item[\rm (TP.2)] given $X\in\mathcal{X}$ there is a pseudokernel-pseudocokernel sequence
\begin{equation*}
\begin{tikzcd}
T_X\arrow{r}{\varepsilon_X}& X\arrow{r}{\lambda_X}&F^X,
\end{tikzcd}
\end{equation*}
where $T_X\in\mathcal{T}$ and $F^X\in\mathcal{F}$.
\end{enumerate}
\end{definition}

\begin{definition}
A torsion pair $\mathbb{t}=(\mathcal{T},\mathcal{F})$  is said to be {\em left} (resp. {\em right}) {\em functorial} if $\T$ (resp. $\F$) is a coreflective (resp. reflective) subcategory of $\X$. 
%
% the assignment $X\mapsto t(X) :=T_X$ (resp. $X\mapsto f(X):=F^X)$ is functorial and defines an adjoint pair $(i,t)$ (resp. $(f,j)$), where $i\colon\mathcal{T}\into \mathcal{X}$ (resp. $j\colon\mathcal{F}\into \mathcal{X}$) is the inclusion functor.
%In such a case, we denote by $\varepsilon$ (resp. $\lambda$) the counit (resp. unit) of the given adjoint pair.\\
Furthermore, we say that $\mathbb{t}$ is {\em functorial} if it is both left and right functorial.
\end{definition}

It is a classical result that torsion pairs in Abelian categories are automatically functorial. Similarly, $t$-structures in triangulated categories are examples of functorial torsion pairs. 
  
In the following lemma we collect two basic observations about left functorial torsion pairs. The proof is omitted as it is essentially an exercise on the definitions.

\begin{lemma}\label{rmk:1.1}
Let $\mathbb{t}=(\mathcal{T},\mathcal{F})$ be a left functorial torsion pair in $\mathcal{X}$. Then
\begin{enumerate}[label=(\alph*)]
\item\label{rmk:1.1a} for any $M\in\mathcal{X}$, $T'\in\mathcal{T}$ and $\alpha\in\mathcal{X}(T',M)$ there is a unique $\alpha'\in\mathcal{X}(T',t(M))$ such that $\varepsilon_M\circ\alpha'=\alpha$;
%    , i.e.
%    \begin{equation*}
%      \begin{tikzcd}
%        & T'\arrow[dashed]{ld}[']{\exists!\,\alpha'}\arrow{d}{\alpha}\\
%        t(M)\arrow{r}{\varepsilon_M}&M
%      \end{tikzcd}
%    \end{equation*}
\item\label{rmk:1.1b} for a morphis $g:T_1\to T_2$ in $\mathcal{T}$, any pseudocokernel $g^C:T_2\to C$ in $\mathcal T$ is also a pseudocokernel of $g$ in $\mathcal{X}$.
\end{enumerate}
\end{lemma}
%\begin{proof}
%(a) Since $(i,t)$ is an adjoint pair, we have a functorial isomorphism
%\begin{equation*}
%\Theta:\myhom{\mathcal{T}}{T',t(M)}\nto{\sim}\myhom{\mathcal{X}}{i(T'),M}=\myhom{\mathcal{X}}{T',M}.
%\end{equation*}
%Let $\alpha':=\Theta^{-1}(\alpha)$. Then, $\varepsilon_M\circ \alpha'=\Theta(\alpha')=\alpha$.
%
%\smallskip\noindent
%(b) Let $X\in\mathcal{X}$ and $h:T_2\to X$ such that $hg=0$. Consider the commutative diagram
%\begin{equation*}
%\begin{tikzcd}
%T_1\arrow{r}{g} & T_2\arrow{r}{g^C}\arrow{d}{h}\arrow[dashed]{dr}{\exists h'} & \PCok_\mathcal{T}(g)\arrow[dashed]{d}{\exists h''}\\
%  & X  & t(X)\arrow{l}{\varepsilon_X}.
%\end{tikzcd}
%\end{equation*}
%Then, $h'\circ g=0$. In fact, $0=h\circ g=\varepsilon_X \circ h' \circ g$ and $\varepsilon_X\circ 0=0$, so, by \ref{rmk:1.1:proof:a}, $h'\circ g=0$. Since $h'\circ g=0$, it follows that there is a map $h'':\PCok_\mathcal{T}(g)\to t(X)$ such that $h''\circ g^C=h'$. Finally, $h=(\varepsilon_X\circ h'')\circ g^C$.
%\end{proof}

\subsection{Nakaoka contexts and the heart construction}

We start this subsection giving the main definition of the paper:

\begin{definition}
A \emph{Nakaoka context} is a pair $\mathbb{t}=(\mathbb{t}_1,\mathbb{t}_2)$ of torsion pairs in $\mathcal{X}$, satisfying the following axioms:
\begin{torsionaxioms}
\item\label{ax:ct1} $\mathbb{t}_1=(\mathcal{T}_1,\mathcal{F}_1)$ and $\mathbb{t}_2=(\mathcal{T}_2,\mathcal{F}_2)$ are respectively a left functorial and a right functorial torsion pair;
\item\label{ax:ct2} $\mathcal{T}_2\subseteq \mathcal{T}_1$ (equiv. $\mathcal{F}_1\subseteq\mathcal{F}_2$).
\end{torsionaxioms}
\end {definition}


\begin{notation}
For a Nakaoka context $\mathbb{t}=(\mathbb{t}_1,\mathbb{t}_2)$ in $\mathcal{X}$, we will always take $\t_1=(\T_1,\F_1)$ and $\t_2=(\T_2,\F_2)$. Furthermore, we will use the following notation for the corresponding coreflection and reflection:
\begin{equation*}
\begin{tikzcd}
(i_1,t_1)\colon\mathcal{T}_1\arrow[hook, shift left]{r}{i_1}
& \mathcal{X}\arrow[shift left]{l}{t_1}
& \mathrm{and}
& (t_2,j_2)\colon\mathcal{F}_2\arrow[hook, shift right]{r}[']{j_2}
& \mathcal{X}\arrow[shift right]{l}[']{f_2}.
\end{tikzcd}
\end{equation*}
The counit of the first adjunction will be denoted by $\varepsilon_{1}\colon t_1i_1\to \id_\X$, while the unit of the second will be denoted by  $\lambda_{2}\colon \id_\X \to f_2i_2$.
\end{notation}
 
 
\begin{definition}
The {\em heart} of a Nakaoka context $\mathbb{t}=(\t_1,\t_2)$ is $\mathcal{H}=\mathcal{H}_\mathbb{t}:=\mathcal{T}_1\cap\mathcal{F}_2$. 
\end{definition}

In the following lemma, we collect two general observations about Nakaoka contexts.

\begin{lemma}\label{lem:1.2}
The following statements hold true for a Nakaoka context $\mathbb{t}=(\mathbb{t}_1,\mathbb{t}_2)$:
\begin{enumerate}[label=(\alph*)]
\item $\mathcal{F}_1\cap\mathcal{T}_2=0$;
\item $f_2(\mathcal{T}_1)\subseteq\mathcal{H}$ and $t_1(\mathcal{F}_2)\subseteq\mathcal{H}$.
\end{enumerate}
\end{lemma}
\begin{proof}
(a) Since $\mathcal{T}_2\subseteq\mathcal{T}_1$, we have $\mathcal{F}_1\cap\mathcal{T}_2\subseteq \mathcal{F}_1\cap\mathcal{T}_1=0$.

\smallskip\noindent
(b) Let $T_2\in\mathcal{T}_2$ and $F_2\in\mathcal{F}_2$. Then, $\mathcal{X}(T_2,t_1(F_2))\cong \mathcal{X}(T_2,F_2)=0$.
Hence, $t_1(F_2)\in\mathcal{T}_2^\perp = \mathcal{F}_2$. An analogous proof shows that $f_2(\mathcal{T}_1)\subseteq\mathcal{T}_1$.
\end{proof}

In the following lemma we introduce a technical condition under which we can easily construct kernels of morphisms in the heart of a given Nakaoka context:

\begin{lemma}\label{lemma_contruction_kernels}
Let $\mathbb{t}=(\mathbb{t}_1,\mathbb{t}_2)$ be a Nakaoka context and let $f\colon H\to H'$ be a morphism in the heart $\H=\H_\t$. If there is a morphism $f^K\colon K\to H$, with $K\in \F_2$, such that the following sequence is exact in $\Func(\T_1, \Ab)$
\[
(*)\qquad 0\to (-,K)\restriction_{\T_1}\to (-,H)\restriction_{\T_1}\to (-,H')\restriction_{\T_1},
\]
then the composition $f^K\circ \varepsilon_{1,K}\colon t_1K\to K\to H$ is a kernel for $f$ in $\H$.
\end{lemma}
\begin{proof}
Consider the exact sequence in $(*)$ and notice that it gives, by restriction of the functors, an exact sequence of the form:
\[
0\to (-,K)\restriction_{\H}\to (-,H)\restriction_{\H}\to (-,H')\restriction_{\H}.
\]
The map $\varepsilon_{1,K}\colon t_1K\to K$, induces a natural isomorphism $(-,t_1K)\restriction_\H \to (-,K)\restriction_\H$, so we get an exact sequence 
\[
\xymatrix@C=13pt{
0\ar[rr]&& (-,t_1K)\restriction_{\H}\ar[rr]^{k\circ \varepsilon_{1,K}\circ -}&& (-,H)\restriction_{\H}\ar[rr]&& (-,H')\restriction_{\H}.
}
\]
To conclude one notices that, since $t_1K\in\H$ by Lemma \ref{lem:1.2}, the above exact sequence means exactly that $f^K\circ \varepsilon_{1,K}$ is a kernel of $f$.
\end{proof}

\subsection{Pre-abelian Nakaoka contexts}

Let us start recalling that an additive category is  {\em pre-Abelian} if any of its morphisms has a kernel and a cokernel. In view of  Lemma \ref{lemma_contruction_kernels}, it is natural to introduce the following definition:

\begin{definition}
A Nakaoka context is said to be {\em pre-Abelian} if it satisfies the following axioms:
\begin{torsionaxioms}\setcounter{enumi}{2}
%\item\label{ax:ct1} $\mathbb{t}_1=(\mathcal{T}_1,\mathcal{F}_1)$ and $\mathbb{t}_2=(\mathcal{T}_2,\mathcal{F}_2)$ are respectively a left functorial and a right functorial torsion pair
%\item\label{ax:ct2} $\mathcal{T}_2\subseteq \mathcal{T}_1$ (equiv. $\mathcal{F}_1\subseteq\mathcal{F}_2$)
\item\label{ax:ct3} any  $g\colon H\to H'$ in $\H(\subseteq \T_1)$ admits a pseudocokernel $g^C\colon H'\to C$ in $\T_1$, such that
\begin{equation*}
\begin{tikzcd}
0\arrow{r} &(C,-)_{|\mathcal{F}_2}\arrow{r}{(g^C,-)} & (H',-)_{|\mathcal{F}_2}\arrow{r}{(g,-)}& (H,-)_{|\mathcal{F}_2}
\end{tikzcd}
\end{equation*}
is an exact sequence in $\Func(\F_2,\Ab)$. 
\varitem{^\ast}\label{ax:ct3op} Dual of \ref{ax:ct3}.
\end{torsionaxioms}
\end{definition}



\begin{thm}\label{pre_abelian_theorem}
For a pre-Abelian Nakaoka context $\t=(\t_1,\t_2)$, the heart $\H=\H_\t$ is a pre-Abelian category.
\end{thm}
\begin{proof}
This is a consequence of the axioms, Lemma \ref{lemma_contruction_kernels} and its dual.
\end{proof}

In the following proposition we give a characterization of those morphisms that are monomorphisms in the heart. For this, remember that, in a pre-Abelian category, a morphism is mono if and only if its kernel is trivial. 

\begin{prop}\label{prop:1.4}
The following are equivalent for a morphism $f\colon H\to H'$ in the heart $\H=\H_\t$ of a pre-Abelian Nakaoka context $\mathbb{t}=(\mathbb{t}_1,\mathbb{t}_2)$:
\begin{enumerate}[label=(\alph*)]
\item $f$ is a monomorphism (in $\mathcal{H}$);
\item there is a pseudokernel $f^K\colon K\to H$ of $f$ in $\F_2$ such that $K\in \F_1$. 
\end{enumerate}
\end{prop}
\begin{proof}
For any morphism $f\colon H\to H'$ in $\H$, by the axiom (CT.$3^*$), we can consider a diagram as follows
\begin{equation*}
\begin{tikzcd}
F_2\arrow{r}{f^K} & H\arrow{r}{f} & H'\\
t_1{F_2}\arrow{u}{\varepsilon_{1,F_2}}\arrow{ur}[']{\tilde{f}^K}
\end{tikzcd}
\end{equation*}
where $F_2\in \F_2$ is a pseudo-kernel of $f$ in $\F_2$ and, by Lemma \ref{lemma_contruction_kernels}, $t_1F_2\to H$ is the kernel of $f$ in $\H$.

\smallskip\noindent
(a)$\Rightarrow$(b). Since $f$ is a monomorphism in $\H$, its kernel is trivial, that is, $t_1F_2=0$. Hence, $0=\mathcal{X}(Z,t_1(F_2))\cong \mathcal{X}(Z,F_2)$ for all $Z\in\mathcal{T}_1$, that is, $F_2\in\T_1^{\perp}=\mathcal{F}_1$.

\smallskip\noindent
(b)$\Rightarrow$(a). If  $f^K\colon K\to H$ is a pseudokernel of $f$ in $\F_2$ such that $K\in \F_1$, then the kernel $0=t_1(K)\to H$ of $f$
 in $\H$ is trivial, that is, $f$ is a monomorphism.
 %Conversely, let $f':F_1\to H$ be a pseudokernel of $f$ in $\mathcal{F}_2$, with $F_1\in\mathcal{F}_1$. Consider the solid part of the diagram
%\begin{equation*}
%\begin{tikzcd}
%t_1(F_2)\arrow{d}{\varepsilon_{1,F_2}}\arrow{dr}{\tilde{f}^K} & &\\
%F_2\arrow[dashed,shift right]{d}[']{u}\arrow{r}{f^K} & H\arrow{r}{f} & H'\\
%F_1\arrow[dashed,shift right]{u}[']{v}\arrow{ur}{f'} & &.
%\end{tikzcd}
%\end{equation*}
%Since $f^K$ and $f'$ are pseudokernel of $f$ in $\mathcal{F}_2$, there exist $u$ and $v$ such that $f'=f^K\circ v$ and $f^K=f'\circ u$. Therefore, $f^K=f^K\circ v\circ u$, and so $\tilde{f}^K=f^K\circ\varepsilon_{1,F_2}=f^K\circ v\circ u\circ\varepsilon_{1,F_2}$. Notice that $u\circ \varepsilon_{1,F_2}=0$ since ${\mathcal{X}}(\mathcal{T}_1,\mathcal{F}_1)=0$. Thus, $f$ has the zero morphism as its kernel in $\mathcal{H}$, i.e. $f$ is a monomorphism.
\end{proof}






%\begin{prop}\label{prop:1.3}
%  Let $\mathbb{t}=(\mathbb{t}_1,\mathbb{t}_2)$ be a compatible torsion pair in $\mathcal{X}$ and
%  $f:H_1\to H_2$ be a morphism in $\mathcal{H}$. Then
%  \begin{enumerate}[label=(\alph*)]
%    \item\label{prop:1.3:a} if $f^C:H_2\to T_1$ is the pseudocokernel of $f$, given by \ref{ax:ct3}
%    and $\lambda_{2,T_1}:T_1\to f_2(T_1)$, then
%    \begin{equation*}
%      \Coker(H_1\nto{f}H_2) = (H_2\nto{\tilde{f}^C}f_1(T_1))
%    \end{equation*}
%    in $\mathcal{H}$, where $\tilde{f}^C:=\lambda_{2,T_1}\circ f^C$;
%    \item\label{prop:1.3:b} if $f^K:F_2\to H_1$ is the pseudokernel of $f$, given by \ref{ax:ct3op}%$^\ast$
%    and $\varepsilon_{1,F_2}:t_1(F_2)\to F_2$, then
%    \begin{equation*}
%      \Ker(H_1\nto{f} H_2) = (t_1(F_2)\nto{\tilde{f}^K}H_1)
%    \end{equation*}
%    in $\mathcal{H}$, where $\tilde{f}^K=f^K\circ \varepsilon_{1,F_2}$.
%  \end{enumerate}
%\end{prop}

%\begin{proof}
%  \begin{enumerate}[label=(\alph*)]
%    \item Let $g:H_2\to H$ in $\mathcal{H}$ such that $gf=0$. And consider the solid part of
%    the following diagram
%    \begin{equation*}
%      \begin{tikzcd}
%        H_1\arrow{r}{f}
%          & H_2\arrow{r}{f^C}\arrow{d}{g}
%            & T_1\arrow[dashed]{dl}[description]{f'}\arrow{r}{\lambda_{2,T_1}}
%              & f_2(T_1)\arrow[dashed]{dll}[description]{f''}\\
%          & H
%            & &
%      \end{tikzcd}
%    \end{equation*}
%    with $T_1\in\mathcal{T}_1$ and $f_2(T_1)\in\mathcal{H}$ (by \cref{lem:1.2}).
%    Since $H\in\mathcal{H}\subseteq \mathcal{F}_2$, by \ref{ax:ct3} there is a
%    unique $f':T_1\to H$ such that $f'f^C=g$. By \cref{rmk:1.1}\ref{rmk:1.1a},
%    there is a $f'':f_2(T_1)\to H$ making the diagram commute. Hence, $g=f''\circ \tilde{f}^C$.
%
%    As for unicity, let $r:f_2(T_1)\to H$, such that $g=r\circ \tilde{f}^C$. Then,
%    $(f''\circ\lambda_{2,T_1})\circ f^C = (r\circ \lambda_{2,T_1})\circ f^C$,
%    so $f''\circ \lambda_{2,T_1}=r\circ \lambda_{2,T_1}$ by $\ref{ax:ct3}$, and $f''=r$ by \cref{rmk:1.1}\ref{rmk:1.1a}
%    \item Dual.
%  \end{enumerate}
%\end{proof}





\subsection{Abelian Nakaoka contexts}


\begin{definition}
  A compatible torsion pair $\mathbb{t}=(\mathbb{t}_1,\mathbb{t}_2)$ in $\mathcal{X}$
  is strong if the following axioms hold:
  \begin{torsionaxioms}
    \setcounter{enumi}{3}
    \item\label{ax:ct4} Let $f:H_1\to H_2$ in $\mathcal{H}$ be such that there is a pseudokernel
    $\PKer_{\mathcal{F}_2}(f)\in\mathcal{F}_1$. Then, for the commutative diagram
    \begin{equation*}
      \begin{tikzcd}
        & H_1\arrow{d}{a}\arrow{r}{f}
          & H_2\arrow[equal]{d}\arrow{r}{f^C}
            & T_1:=\PCok_\mathcal{X}(f)\arrow{d}{\lambda_{2,T_1}}\\
        t_1(F_2)\arrow{r}{\varepsilon_{1,F_2}}
        & F_2:=\PKer_\mathcal{X}(g)\arrow{r}{g^K}
          & H_2\arrow{r}{g}
            & f_2(T_1)
      \end{tikzcd}
    \end{equation*}
    there exists a morphism $b:t_1(F_2)\to H_1$ such that $ab=\varepsilon_{1,F_2}$.
    \varitem{^\ast}\label{ax:ct4op} Dual to \ref{ax:ct4}.
  \end{torsionaxioms}
\end{definition}

With these axioms we can prove that the heart has kernels and cokernels.

\begin{thm}
TFAE for a pre-Abelian Nakaoka context $\mathbb{t}=(\mathbb{t}_1,\mathbb{t}_2)$:
\begin{enumerate}[label=(\alph*)]
\item $\mathbb{t}$ is Abelian;
\item the heart $\mathcal{H}=\H_\t$ is an Abelian category.
\end{enumerate}
\end{thm}

\begin{proof}
(a)$\Rightarrow$(b). By Theorem \ref{pre_abelian_theorem}, $\mathcal{H}$ is pre-Abelian. To prove that $\mathcal{H}$ is Abelian, we just need to show that any monomorphism (resp. epimorphism) is a kernel (resp. cokernel).
Hence, let $f\colon H_1\to H_2$ be a monomorphism in $\mathcal{H}$ and consider the cokernel $f^C\colon H_2\to C$ in $\H$. We want to prove that $f$ is the kernel of $f^C$. Since we already know that $f$ is mono, it is the kernel of any morphism for which it is a pseudokernel, hence we have just to prove that $f$ is a pseudokernel of $f^C$. Let $a\colon A\to H_2$ be a morphism such that $f^C\circ a=0$. 

%, that is, we need to prove that the following sequence of functors is exact:
%\[
%0\to (-,H_1)\restriction_{\H}\to (-,H_2)\restriction_{\H}\to (-,C)\restriction_{\H}
%\]





By \ref{ax:ct3} and \ref{ax:ct3op} we can consider the solid part of the commutative diagram
\begin{equation*}
\begin{tikzcd}
 & H_1\arrow{r}{f}\arrow{d}{a} & H_2\arrow{r}{f^C}\arrow[equal]{d} & T_1 = \PCok_\mathcal{X}(f)\arrow{d}{\lambda_{2,T_1}}\\
t_1(F_2)\arrow[dashed]{ur}{b}\arrow{r}{\varepsilon_{1,F_2}} & F_2=\PKer_\mathcal{X}(g)\arrow{r}{g^K} & H_2\arrow{r}{g} & f_2(T_1)
\end{tikzcd}
\end{equation*}
where $gf = \lambda_{2,T_1}(f^C f)=0$, so there exists $a:H_1\to F_2$. Note that $\exists \PKer_{\mathcal{F}_2}(f)\in\mathcal{F}_1$ since $f$ is a monomorphism in $\mathcal{H}$ (by \cref{prop:1.4}\ref{prop:1.4:a}). So by \ref{ax:ct4} \todo{Add proper reference} there is a map $b:t_1(F_2)\to H_1$ making the diagram commute.

We claim that $\Ker g = f$. Let $\alpha:H\to H_2$ be a morphism such that $g\alpha =0$. Since $F_2 = \PKer_\mathcal{X}(g)$, there is a morphism $\alpha':H\to F_2$ such that $g^K\alpha'=\alpha$. By remark \ref{rmk:1.1}, $\alpha'$ factors as $\varepsilon_{1,F_2}\alpha''$, as in the diagram:
\begin{equation*}
\begin{tikzcd}
& H_1\arrow{r}{f}\arrow{d}{a} & H_2\arrow{r}{f^C}\arrow[equal]{d}& T_1 = \PCok_\mathcal{X}(f)\arrow{d}{\lambda_{2,T_1}}\\
t_1(F_2)\arrow{ur}{b}\arrow{r}{\varepsilon_{1,F_2}} & F_2=\PKer_\mathcal{X}(g)\arrow{r}{g^K} & H_2\arrow{r}{g} & f_2(T_1)\\
& H.\arrow{ur}{\alpha}\arrow[dashed]{u}[description]{\alpha'}\arrow[dashed]{ul}[description]{\alpha''} & &
\end{tikzcd}
\end{equation*}
By setting $\alpha''' := b\alpha'':H\to H_1$, we get $f\alpha''' = g^K ab\alpha'' = g^K\varepsilon_{1,F_2}\alpha''=g^K\alpha' = \alpha$. Thus, any morphism $\alpha:H \to H_2$ such that $g\alpha =0$ factors through $f$. To conclude that $f$ is a kernel of $g$, it suffice to observe that $f$ is a monomorphism, so $\alpha'''$ must be unique.

\bigskip\bigskip\bigskip\bigskip\noindent
(b)$\Rightarrow$(a). Let $\mathcal{H}$ be an abelian category. We only check \ref{ax:ct4} \todo{Fix reference} since the proof of \ref{ax:ct4op} is analogous.

Consider the solid part of the commutative diagram
\begin{equation*}
\begin{tikzcd}
 & H_1\arrow{r}{f}\arrow{d}{a}\arrow[dashed]{dl}[']{\beta} & H_2\arrow{r}{f^C}\arrow[equal]{d} & T_1=\PCok_\mathcal{X}(f)\arrow{d}{\lambda_{2,T_1}}\\
t_1(F^2)\arrow{r}{\varepsilon_{1,F_2}} & F_2=\PKer_\mathcal{X}(g)\arrow{r}{g^K} & H_2\arrow{r}{g} &f_2(T_1)
\end{tikzcd}
\end{equation*}
where $f:H_1\to H_2$ is a morphism in $\mathcal{H}$ such that $\PKer_{\mathcal{F}_2}(f)\in\mathcal{F}_1$. Since $f$ is a monomorphism in $\mathcal{H}$ by \ref{prop:1.4}$(a)$ \todo{Fix reference} and $\mathcal{H}$ is abelian, we have that $f=\Ker \Coker (f)$. On the other hand, by \ref{rmk:1.1} $(a)$ there is $\beta:H_1\to t_1(F_2)$ such that $\varepsilon_{1,F_2}\beta = a$ completing the diagram above. Since $f=\Ker \Coker(f)$, it follows that $\beta$ is an isomorphism. Therefore, \ref{ax:ct4} follows by setting $b:=\beta^{-1}$.
\end{proof}





