\subsection{Abelian categories}

Now we work in an abelian category with two torsion pairs $(\mathcal{T}_1,\mathcal{F}_1)$
and $(\mathcal{T}_2,\mathcal{F}_2)$ such that $t_2(\mathcal{F}_1)\subseteq \mathcal{F}_1$ and
$f_1(\mathcal{T}_2)\subseteq \mathcal{T}_2$ and let $\mathcal{W}=\mathcal{T}_2\cap\mathcal{F}_1$.

Recall that $(\underline{\mathcal{T}_1\ast\mathcal{W}},\underline{\mathcal{F}_1})$
(resp. $(\underline{\mathcal{T}_2},\underline{\mathcal{W}\ast\mathcal{F}_2})$) is a left (resp. right)
functorial torsion pair in $\underline{\mathcal{A}} = \dfrac{\mathcal{A}}{\mathcal{W}}$. Moreover, they
satisfy $TC1-3,3^*$.

\begin{lemma}
  The inclusion $i:\mathcal{T}_1\ast \mathcal{W}\hookrightarrow \mathcal{A}$ admits a right adjoint
  $\widehat{t}$.
\end{lemma}

\begin{proof}
  For $M\in\mathcal{A}$ consider the exact sequence
  \begin{equation*}
    0\rightarrow T_1\rightarrow M\rightarrow f_1(M) \rightarrow 0
  \end{equation*}
  with $T_1\in\mathcal{T}_1$ and $f_1(M)\in\mathcal{F}$. Take $t_2f_1(M)\hookrightarrow f_1(M)$ and observe
  that $t_2f_1(M)\in \mathcal{W}$. Call it $W_M$ and take the pullback diagram
  \begin{equation*}
    \begin{tikzcd}
      \widehat{t}(M)\arrow{r}\arrow{d}
        & W_M\arrow[tail]{d}\\
      M \arrow[two heads]{r}
        & f_1(M)
    \end{tikzcd}
  \end{equation*}
  then $\widehat{t}(M)\in\mathcal{T}_1\ast\mathcal{W}$.

  Now for any morphism $\widehat{T}\to M$ with $\widehat{T}\in\widehat{\mathcal{T}}$ the solid part
  of the following diagram commutes
  \todo{$\widehat{T}\to M$ is mono by Buhler prop. 2.14: pullback of monic along epic is monic}
  \begin{equation*}
    \begin{tikzcd}
      T\arrow[tail]{r}\arrow[equal]{d}
        & \widehat{t}(M)\arrow[two heads]{r}\arrow{d}\arrow[from=dd,bend left=50,dotted]
          \arrow[phantom,"\ulcorner",very near start]{dr}
          & W\arrow[tail]{d}\\
      T \arrow[tail,crossing over]{r}
        & M\arrow[two heads]{r}
          & f_1(M)\\
      T_1\arrow[tail]{r}
        & \widehat{T}\arrow[two heads]{r}\arrow{u}
          & W_1\arrow[dashed]{u}\arrow[bend right=50,dashed]{uu}
    \end{tikzcd}
  \end{equation*}
  Since the composition $T_1\to \widehat{T}\to M\to f_1(M)$ is zero, there exists the dashed
  morphism $W_1\to f_1(M)$, which lifts to the morphism $W_1\to W$ (since $W\to f_1(M)$ is a $\mathcal{W}$-precover).
  Hence, there is a morphism $\widehat{T}\to \widehat{t}(M)$ making the diagram commutative. This means that
  \begin{equation*}
    \mathcal{A}(\widehat{T},\widehat{t}(M))\nto{\mathcal{A}(\widehat{T},\widehat{t}(M)\to M)}
      \mathcal{A}(\widehat{T},M)
  \end{equation*}
  is surjective. But it is also injective, since $\Ker (\widehat{t}(M)\to M)=0$. Hence, it is an iso and $\widehat{t}$ is
  right adjoint to $i$.\todo{functoriality should follow immediately}
\end{proof}

\begin{lemma}
  Let $\widehat{T}_1\in\mathcal{T}_1\ast\mathcal{W}$, i.e. there is an exact sequence
  \begin{equation*}
    0\to t_1(\widehat{T}_1)\to \widehat{T}_1\to W_1\to 0.
  \end{equation*}
  If
  \begin{equation*}
    \begin{tikzcd}
      \widehat{T}_1\arrow[two heads]{r}{p}\arrow{d}{g}
        & W_1\arrow{d}{g'} \\
      \widehat{T}'_1\arrow{r}{q}&N
    \end{tikzcd}
  \end{equation*}
  is a pushout diagram, then $N\in\mathcal{T}_1\ast\mathcal{W}$.
\end{lemma}

\begin{proof}
  Since it is a pushout, $\widehat{T}'_1 \to N$ is epi, then consider $\widehat{t}(N)$
  and the following commutative diagram
  \begin{equation*}
    \begin{tikzcd}
      \widehat{T}'_1\arrow[two heads]{rr}{q} \arrow[dashed]{dr}{\rho}
        & & N\\
        & \widehat{t}(N)\arrow[tail]{ur}{\varepsilon}
          &
    \end{tikzcd}
  \end{equation*}
  where the map $\widehat{T}'_1\to \widehat{t}(N)$ is given by the adjunction $(i,\widehat{t})$.
  Since $q=\varepsilon\circ\rho$ is epi, then $\varepsilon$ is epi. But it is mono, so it is an isomorphism,
  hence $N\in\mathcal{T}_1\ast\mathcal{W}$.
\end{proof}

\begin{lemma}
  In the same notation as the previous lemma, if $\varphi:N\to P$ is any map s.t. $\underline{\varphi}\circ \underline{q} = \underline{0}$ in
  $\underline{\mathcal{A}}$, then $\varphi$ factors through $\mathcal{W}$.
\end{lemma}

\begin{proof}
  Since $\underline{\varphi}\circ\underline{q}=\underline{0}$ it means that $\varphi\circ q$ factors through $\mathcal{W}$, hence we
  have that the solid part of the following diagram is commutative.
  \begin{equation*}
    \begin{tikzcd}
      0\arrow{r}
        & t_1(T'_1)\arrow{r}\arrow{d}
          & \widehat{T}_1\arrow{r}\arrow{d}
            & W_1\arrow{r}\arrow{d}\arrow[bend left, dashed]{dd}
              & 0\\
      0\arrow{r}
        & X\arrow{r}
          & \widehat{T}'_1\arrow{r}\arrow[equal]{d}
            & N\arrow[crossing over]{dr}{\varphi}\arrow[dotted]{d}
              &\\
        & & \widehat{T}'_1\arrow{r}
            & W\arrow{r}
              & P
    \end{tikzcd}
  \end{equation*}
  Since $t_1(T'_1)\to \widehat{T}_1\to \widehat{T}'_1\to W$ is zero, there is the dashed morphism $W_1\to W$ making the
  diagram commute. Since the square on the right is a pushout there is a map $N\to W$, and again the diagram commutes.
  Hence $\varphi$ factors through $\mathcal{W}$.
\end{proof}


\begin{lemma}
  If $\mathcal{H}$ is balanced (i.e. mono and epi implies iso), then whenever $f:H_1\to H_2$
  is mono and epi in $\mathcal{H}$, there are bicartesian squares in $\mathcal{A}$
  \begin{equation*}
    \begin{tikzcd}
      F_1\arrow{r}\arrow{d}\arrow[phantom,"\ulcorner",very near start]{dr}
        \arrow[phantom,"\lrcorner",very near end]{dr}
        & H_1\arrow{r}\arrow{d}{f}\arrow[phantom,"\ulcorner",very near start]{dr}
          \arrow[phantom,"\lrcorner",very near end]{dr}
          & W_1\arrow{d}\\
      W_2\arrow{r}
        & H_2\arrow{r}
          & T_2
    \end{tikzcd}
  \end{equation*}
  where $W_1=f_1(H_1)$ and $W_2=t_2(H_2)$. In particular there is an exact sequence
  \begin{equation*}
    0\to F_1\to W_1\oplus W_2 \to T_2\to 0.
  \end{equation*}
\end{lemma}

\begin{proof}
  We can build the pullback on the left and the pushout on the right as usual
  \begin{equation}\label{lem12:eq1}
    \begin{tikzcd}
      F_1\arrow{r}{f^K}\arrow{d}{u}\arrow[phantom,"\ulcorner",very near start]{dr}
        & H_1\arrow{r}{r}\arrow{d}{f}\arrow[phantom,"\lrcorner",very near end]{dr}
          & W_1\arrow{d}{s}\\
      W_2\arrow{r}{v}
        & H_2\arrow{r}{f^C}
          & T_2
    \end{tikzcd}
  \end{equation}

  We will only prove that the square on the right hand side is a pullback, since the proof
  that the left square is a pushout is dual. The statetment that the square on the right is
  a pushout is equivalent to saying that there is an exact sequence
  \begin{equation}\label{lem12:eq2}
    \begin{tikzcd}
      H_1\arrow{r}{
        \begin{psmallmatrix}
          f \\ r
        \end{psmallmatrix}
      }
        & H_2\oplus W_1 \arrow{r}{
          \begin{psmallmatrix}
            f^C \amsamp s
          \end{psmallmatrix}
        }
          & T_2\arrow{r}
            & 0
    \end{tikzcd}
  \end{equation}
  \todo{$H_1\oplus T_2\cong H_2\cong W_1$
  $\Rightarrow$ $T_2\in\mathcal{W}\ast\mathcal{F}$
  $W'\into T_2\overset{0}{\onto} F'$, with
  $W'\in\mathcal{W},T_2\in\mathcal{T}_2, F'\in\mathcal{F}_2$, hence $T_2\in\mathcal{W}$.}

  Since $f$ is both a mono and an epi in $\mathcal{H}$, then it is an iso and hence
  both a section and a retraction. Consider $g:H_2\to H_1$ such that
  $\underline{g}\circ\underline{f} = \underline{1_{H_1}}$, that is there are maps $\alpha:H_1\to W$ and
  $\beta:W\to H_1$ such that
  \begin{equation*}
    \begin{tikzcd}
      H_1\arrow{r}{
        \begin{psmallmatrix}
          f\\ \alpha
        \end{psmallmatrix}
      }\arrow[bend right]{rr}[']{1_{H_1}}
      & H_2\oplus W \arrow{r}{
        \begin{psmallmatrix}
          g\amsamp \beta
        \end{psmallmatrix}
        }
        & H_1
    \end{tikzcd}
  \end{equation*}
  is commutative in $\mathcal{A}$, and hence $H_1$ is a direct summand of $H_2\oplus W$. We can actually choose $W=W_1$,
  in fact consider the commutative diagram
  \begin{equation*}
    \begin{tikzcd}
      H_1\arrow[equal]{d}\arrow{r}{
        \begin{psmallmatrix}
          f \\ r
        \end{psmallmatrix}
      }
        & H_2\oplus W_1 \arrow{r}{
          \begin{psmallmatrix}
            f^C \amsamp -s
          \end{psmallmatrix}
        }\arrow{d}{
          \begin{psmallmatrix}
            1 \amsamp 0\\
            0 \amsamp \rho
          \end{psmallmatrix}
        }
          & T_2
      \\
      H_1\arrow{r}{
        \begin{psmallmatrix}
          f\\ \alpha
        \end{psmallmatrix}
      }\arrow[bend right]{rr}[']{1_{H_1}}
      & H_2\oplus W \arrow{r}{
        \begin{psmallmatrix}
          g\amsamp \beta
        \end{psmallmatrix}
        }
        & H_1
    \end{tikzcd}
  \end{equation*}
  where $\rho:W_1\to W$ comes from the fact that $H_1\to W_1$ is a $\mathcal{W}$-preenvelope. Hence,
  $
    \begin{psmallmatrix}
      g\amsamp \beta
    \end{psmallmatrix}
    \circ
    \begin{psmallmatrix}
      1 \amsamp 0\\
      0 \amsamp \rho
    \end{psmallmatrix}
    \circ
    \begin{psmallmatrix}
      f \\ r
    \end{psmallmatrix}
    = 1_{H_1}
  $, that is $H_1$ is a direct summand of $H_2\oplus W_1$. Moreover, it means that
  $
    \begin{psmallmatrix}
      f \\ r
    \end{psmallmatrix}
  $ is a section, that is the sequence in \eqref{lem12:eq2} is also exact on the left and
  the corresponding square in \eqref{lem12:eq1} is a pullback diagram.

  Since both squares in \eqref{lem12:eq1} are bicartesian, it follows that the square
  \begin{equation*}
    \begin{tikzcd}
      F_1\arrow{r}\arrow{d}
        \arrow[phantom,"\ulcorner",very near start]{dr}
        \arrow[phantom,"\lrcorner",very near end]{dr}
        & W_1\arrow{d}\\
      W_2\arrow{r}& T_2
    \end{tikzcd}
  \end{equation*}
  is bicartesian as well.
\end{proof}
