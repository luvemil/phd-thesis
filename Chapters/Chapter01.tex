%************************************************
\chapter{Introduction}\label{ch:introduction}
%************************************************

Model categories were introduced by Quillen for the first time in \cite{Q} as a formal way of dealing with the notion of homotopy in a category. The main problem is the following: we have a class of morphisms, called weak equivalences, and we want to treat them as if they were isomorphisms. Formally inverting these weak equivalences does not necessarily produce nice results, since we have to resort to some kind of calculus of fractions to construct the morphisms in the quotient category, and these methods are not guaranteed to yield a \emph{set} of homomorphisms. This, however, happens whenever the weak equivalences are part of a model structure on a model category.

There are different examples of model categories. In \cite{Q}, Quillen constructs model structures for simplicial categories, while in \cite{Hov99} Hovey adds, among the others, the category of modules over a Frobenius ring and the category of cochain complexes of modules over any ring.

In \cite{Hov02} Hovey proved that there is a bijective correspondence between abelian model structures on abelian categories and certain couple of complete cotorsion pairs. Namely, if $\clQ$, $\W$, and $\R$ are the classes of cofibrant, trivial, and fibrant objects respectively, then $(\clQ,\W\cap\R)$ and $(\clQ\cap\W,\R)$ are complete cotorsion pairs, and conversely given three such classes forming two complete cotorsion pair, there is an abelian model structure, indicated by $(\clQ,\W,\R)$, whose cofibrant, trivial and fibrant objects are exactly those classes.

Two elementary but very important examples of model structures on the category of unbounded cochain complexes of $R$-modules are the following:
\begin{itemize}
  \item $(\Ch(R),\E,dg \Inj(R))$, where:
    \begin{itemize}
      \item all cochain complexes are cofibrant;
      \item the class of trivial objects $\E$ is the class of all acyclic complexes;
      \item the class of fibrant objects $dg \Inj(R)$ is the class of dg-injective complexes, i.e. those complexes $I^\bullet$ with $I^n\in\Inj(R)$ and such that any map $E^\bullet \to I^\bullet$ with $E^\bullet\in\E$ is null homotopic.
    \end{itemize}
  \item $(\Ch(R),\W,dw \Inj(R))$, where:
    \begin{itemize}
      \item all cochain complexes are cofibrant;
      \item $\W$ is some class of modules making this into a model structure, its existence is guaranteed by \cite{G4};
      \item the class of fibrant objects $dw \Inj(R)$ is the class of all complexes with injective components.
    \end{itemize}
\end{itemize}

The quotient categories, called homotopy categories, of these model structures are respectively the derived category of the ring $D(R)$ and the cochain homotopy category $K(R)$.

This turns model categories into powerful tools for studying localizations of abelian categories, especially since it is possible to prove the existence of a recollement via the techniques described in \cite{G7}. A recollement is a diagram of functors
\begin{equation*}
  \begin{tikzcd}
    \mathscr{X}\arrow{r}{i_\ast}
    & \mathscr{Y}\arrow[bend left=50]{l}{i^!}\arrow[bend right=50]{l}[']{i^\ast}\arrow{r}{j^\ast}
    & \mathscr{Z}\arrow[bend left=50]{l}{j_\ast}\arrow[bend right=50]{l}[']{j_!}
  \end{tikzcd}
\end{equation*}
between triangulated categories satisfying, among other conditions, the requirement that $(i^\ast,i_\ast)$, $(i_\ast,i^!)$, $(j_!,j^\ast)$, and $(j^\ast,j_\ast)$ are adjoint pairs. Beilinson, Bernstein and Deligne introduced them in \cite{bbd82} and developed techniques to glue t-structures along recollements.

A classical example of a recollement can be found in the setting of the category of cochain complexes using the cotorsion pairs mentioned above. The recollement is:
\begin{equation*}
\recdiagram{K_{ac}(R)}{K(R)}{D(R)},
\end{equation*}
where $K_{ac}(R)$ is the full subcategory of $K(R)$ consisting of acyclic cochain complexes.

This thesis is divided into two parts, the first based on the joint work \cite{bazzoni2018recollements} with Silvana Bazzoni and the second on a work in preparation with Manuel Saor\'in, Simone Virili and Octavio Mendoza.

\smallskip

In the first part our objective will be to construct model structures in exact subcategories of $\Ch(\ModR)$ in order to prove the existence of interesting recollements between their homotopy categories. We will consider a complete hereditary cotorsion pair $(\A,\B)$ in $\ModR$ and find the following recollements ending in $D(R)$ involving respectively the derived category of $\B$ and the cochain homotopy category of $\B$:

\begin{equation*}
  \recdiagram{\frac{ex \B}{\sim}}{D(\B)}{D(R)},
\end{equation*}

\begin{equation*}
  \recdiagram{\frac{ex \B}{\sim}}{K(\B)}{D(R)}.
\end{equation*}

Furthermore, we address the problem of finding a recollement ending in $D(\B)$. We show that, assuming $ex(\B)=\wac{\B}$, it is possible to find a recollement:

\begin{equation*}
  \recdiagram{\frac{ex \B}{\sim}}{K(\B)}{D(\B)}.
\end{equation*}

The condition $ex(\B)=\wac{\B}$ is satisfied whenever all objects of $\A$ have finite projective dimension. This condition however is not necessary, in fact it was shown in \cite{BCIE} that the class $\Cot(R)$ of cotorsion modules satisfies $ex\Cot(R)=\wac{\Cot(R)}$. The problem of finding a necessary condition for a cotorsion pair $(\A,\B)$ in $\ModR$, or more generally in a Grothendieck category with a projective generator, remains open.

\smallskip

In \cite{bbd82} Beilinson, Bernstein and Deligne introduced the notion of t-structure in a triangulated in order to build the category of perverse sheaves. They also proved that to any t-structures is associated an abelian category.

%*****************************************
%*****************************************
%*****************************************
%*****************************************
%*****************************************
